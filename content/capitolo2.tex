\chapter{Regole della teoria dei tipi}
\label{cap:regole-teoria-dei-tipi}
%%dalla lezione 7 alla lezione 11
Lo scopo della teoria dei tipi \`e offrire un sistema formale in cui derivare, tramite regole e assiomi, giudizi nella forma:
\[ A \hspace{0.1cm} type[\Gamma] \qquad
A = B \hspace{0.1cm} type[\Gamma] \qquad
a \in A\hspace{0.1cm} [\Gamma] \qquad
a=b \in A \hspace{0.1cm}[\Gamma]
\]
\[
+ \hspace{0.1cm} ausiliaria \quad \Gamma \hspace{0.1cm} cont
\]
L'ultimo giudizio non \`e necessario, serve esclusivamente per imparare.\\
Quando si formula una nuova teoria dei tipi \`e bene impiegare il minor numero possibile di regole strutturali e di formazione di tipi e termini. Tali regole devono essere rivolte all'ottimizzazione e correttezza della teoria. Alcune di queste, come quelle di indebolimento e sostituzione in \S\ref{subsec:indebolimento-sostituzione}, sono irrinunciabili, la cui validit\`a \`e sempre garantita e utilizzate nella derivazione di ogni teoria.\\\\
Se la teoria dei tipi \`e dipendente si ha bisogno di tutti i giudizi. Invece in una teoria dei tipi non dipendente, come quella dei tipi semplici, il giudizio $A = B \hspace{0.1cm} type[\Gamma]$ pu\`o venire omesso.\\

\section{Regole strutturali}
\label{sec:regole-strutturali}
\textbf{Assioma unico: [\hspace{0.1cm}] cont}\\\\
Nel calcolo dei sequenti, in Logica classica, le derivazioni di giudizio, valide in una teoria dei tipi con solo le regole singoletto, diventano derivazioni di sequenti nella forma $\Gamma$ $\dashv$ A e unico assioma $\varphi$ $\dashv$ $\varphi$.
\\\\
Di seguito illustro le principali regole di contesto comuni a tutte le teorie dei tipi.
\subsection{Regole di formazione dei contesti}
\label{subsec:formazione-contesti}
\begin{center} [\hspace{0.1cm}] cont \quad dove \quad [\hspace{0.1cm}] $=$ $\varnothing$ \end{center}
\begin{prooftree}
\AxiomC{A type[$\Gamma$]}
\LeftLabel{F-c)}\RightLabel{(x $\in$ A) $\notin$ $\Gamma$}
\UnaryInfC{$\Gamma$, x $\in$ A}
\end{prooftree}
\subsection{Regole di assunzione delle variabili}
\label{subsec:assunzione-variabili}
\begin{prooftree}
\AxiomC{$\Gamma$, x $\in$ A, $\Delta$ cont}
\LeftLabel{var)}
\UnaryInfC{x $\in$ A[$\Gamma$,x $\in$ A, $\Delta$]}
\end{prooftree}
\subsection{Regole strutturali addizionali sull'uguaglianza}
\label{subsec:uguaglianza}
L'uguaglianza, in una teoria dei tipi, consiste in una relazione di equivalenza sia fra tipi che fra termini. Sono perci\`o valide le seguenti regole di uguaglianza tra tipi:
\[ ref) \quad \frac{A\hspace{0.1cm}type[\Gamma]}{A=A\hspace{0.1cm}type[\Gamma]} \qquad sym) \quad \frac{A=B\hspace{0.1cm}type[\Gamma]}{B=A\hspace{0.1cm}type[\Gamma]} \]
\[ tra) \quad \frac{A=B\hspace{0.1cm}type[\Gamma] \qquad B=C\hspace{0.1cm}type[\Gamma]}{A=C\hspace{0.1cm}type[\Gamma]} \]
E allo stesso modo anche le regole di uguaglianza definizionale/computazionale tra termini:
\[ ref) \quad \frac{a \in A\hspace{0.1cm}[\Gamma]}{a=a \in A\hspace{0.1cm}[\Gamma]} \qquad sym) \quad \frac{a=b\in A \hspace{0.1cm}[\Gamma]}{b=a \in A\hspace{0.1cm}[\Gamma]} \]
\[ tra) \quad \frac{a=b \in A\hspace{0.1cm}[\Gamma] \qquad b=c \in A\hspace{0.1cm}[\Gamma]}{a=c \in A\hspace{0.1cm}[\Gamma]} \]
\subsection{Regole di conversione dell'uguaglianza per tipi uguali}
\label{subsec:conversione-uguaglianza}
L'appartenenza si conserva con l'uguaglianza di termini e tipi. Le regole da aggiungere, in una teoria dei tipi, per garantirlo sono:
\[ conv) \quad \frac{a \in A\hspace{0.1cm}[\Gamma] \quad A=B\hspace{0.1cm}type[\Gamma]}{a \in B\hspace{0.1cm}[\Gamma]} \]
\[ conv-eq) \quad \frac{a=b \in A\hspace{0.1cm}[\Gamma] \qquad A=B\hspace{0.1cm}type[\Gamma]}{a=b \in B\hspace{0.1cm}[\Gamma]} \]
\subsection{Regole di indebolimento e di sostituzione}
\label{subsec:indebolimento-sostituzione}
\subsubsection{Indebolimento}
\label{subsec:indebolimento}
\[ ind-ty) \quad \frac{A\hspace{0.1cm}type[\Gamma] \quad \Gamma,\Delta\hspace{0.1cm}cont}{A\hspace{0.1cm}type[\Gamma,\Delta]} \quad ind-ty-eq) \quad \frac{A=B\hspace{0.1cm}type[\Gamma] \quad \Gamma,\Delta\hspace{0.1cm}cont}{A=B\hspace{0.1cm}type[\Gamma,\Delta]} \]
\[ ind-te) \quad \frac{a \in A\hspace{0.1cm}[\Gamma] \quad \Gamma,\Delta\hspace{0.1cm}cont}{a \in A\hspace{0.1cm}[\Gamma,\Delta]} \quad ind-te-eq) \quad \frac{a=b \in A\hspace{0.1cm}
[\Gamma] \quad \Gamma,\Delta\hspace{0.1cm}cont}{a=b \in A\hspace{0.1cm}[\Gamma,\Delta]} \]
\subsubsection{Sostituzione}
\label{subsec:sostituzione}
\[
\begin{split}
C(x_1,...,x_n)\hspace{0.1cm}type[\Gamma, x_1 \in A_1,...,x_n \in A_n(x_1,...,x_{n-1})] \\ sub-typ) \quad \frac{a_1 \in A_1\hspace{0.1cm}[\Gamma]\hspace{0.1cm}...a_n \in A_n(a_1,...,a_{n-1})\hspace{0.1cm}[\Gamma]}{ C(a_1,...,a_n)\hspace{0.1cm}type[\Gamma]}
\end{split}
\]

\[
\begin{split}
C(x_1,...,x_n)\hspace{0.1cm}type[\Gamma, x_1 \in A_1,...,x_n \in A_n(x_1,...,x_{n-1})] \\ sub-eq-typ) \quad \frac{a_1 = b_1  \in A_1\hspace{0.1cm}[\Gamma]\hspace{0.1cm}...a_n = b_n \in A_n(a_1,...,a_{n-1})\hspace{0.1cm}[\Gamma]}{C(a_1,...,a_n) = C(b_1,...,b_n) \hspace{0.1cm}type[\Gamma]}
\end{split}
\]

\[
\begin{split}
C(x_1,...,x_n) = D(x_1,...,x_n) \hspace{0.1cm}type[\Gamma, x_1 \in A_1,...,x_n \in A_n(x_1,...,x_{n-1})] \\ sub-Eqtyp) \quad \frac{a_1 \in A_1\hspace{0.1cm}[\Gamma]\hspace{0.1cm}...a_n \in A_n(a_1,...,a_{n-1})\hspace{0.1cm}[\Gamma]}{C(a_1,...,a_n) = D(a_1,...,a_n)\hspace{0.1cm}type[\Gamma]}
\end{split}
\]

\[
\begin{split}
C(x_1,...,x_n) = D(x_1,...,x_n) \hspace{0.1cm}type[\Gamma, x_1 \in A_1,...,x_n \in A_n(x_1,...,x_{n-1})] \\ sub-eq-Eqtyp) \quad \frac{a_1 = b_1\in A_1\hspace{0.1cm}[\Gamma]\hspace{0.1cm}...a_n = b_n \in A_n(a_1,...,a_{n-1})\hspace{0.1cm}[\Gamma]}{C(a_1,...,a_n) = D(b_1,...,b_n)\hspace{0.1cm}type[\Gamma]}
\end{split}
\]

\[
\begin{split}
c(x_1,...,x_n) \in C(x_1,...,x_n) \hspace{0.1cm}[\Gamma, x_1 \in A_1,...,x_n \in A_n(x_1,...,x_{n-1})] \\ sub-ter) \quad \frac{a_1 \in A_1\hspace{0.1cm}type[\Gamma]\hspace{0.1cm}...a_n \in A_n(a_1,...,a_{n-1})\hspace{0.1cm}[\Gamma]}{ c(a_1,...,a_n) \in C(a_1,...,a_n)\hspace{0.1cm}type[\Gamma]}
\end{split}
\]

\[
\begin{split}
c(x_1,...,x_n) = d(x_1,...,x_n)  \in C(x_1,...,x_n) \hspace{0.1cm}[\Gamma, x_1 \in A_1,...,x_n \in A_n(x_1,...,x_{n-1})] \\ sub-eqter) \quad \frac{a_1 \in A_1\hspace{0.1cm}[\Gamma]\hspace{0.1cm}...a_n \in A_n(a_1,...,a_{n-1})\hspace{0.1cm}[\Gamma]}{ c(a_1,...,a_n) = d(a_1,...,a_n) \in C(a_1,...,a_n)\hspace{0.1cm}[\Gamma]}
\end{split}
\]

\[
\begin{split}
c(x_1,...,x_n) \in C(x_1,...,x_n) \hspace{0.1cm}[\Gamma, x_1 \in A_1,...,x_n \in A_n(x_1,...,x_{n-1})] \\ sub-eq-ter) \quad \frac{a_1 = b_1 \in A_1\hspace{0.1cm}[\Gamma]\hspace{0.1cm}...a_n = b_n \in A_n(a_1,...,a_{n-1})\hspace{0.1cm}[\Gamma]}{ c(a_1,...,a_n) = c(b_1,...,b_n) \in C(a_1,...,a_n)\hspace{0.1cm}[\Gamma]}
\end{split}
\]

\[
\begin{split}
c(x_1,...,x_n) = d(x_1,...,x_n) \in C(x_1,...,x_n) \hspace{0.1cm}[\Gamma, x_1 \in A_1,...,x_n \in A_n(x_1,...,x_{n-1})] \\ sub-eq-eqter) \quad \frac{a_1 = b_1 \in A_1\hspace{0.1cm}[\Gamma]\hspace{0.1cm}...a_n = b_n \in A_n(a_1,...,a_{n-1})\hspace{0.1cm}[\Gamma]}{ c(a_1,...,a_n) = d(b_1,...,b_n) \in C(a_1,...,a_n)\hspace{0.1cm}[\Gamma]}
\end{split}
\]

\subsection{Regole proprie e derivabili}
\label{subsec:regole-proprie-derivabili}
In una teoria formale ci sono due tipi di regole:
\begin{itemize}
\item \textbf{regole proprie del calcolo}, come lo sono le regole strutturali e quelle del singoletto;
\item \textbf{regole derivabili}, come le regole di sostituzione, utili per abbreviare le derivazioni.
\end{itemize}
\noindent
Una regola r) $\frac{J_1,...,J_n}{J}$ \`e ammissibile in un calcolo t sse i giudizi derivabili in $t+r$ sono gli stessi dei giudizi derivabili in t. Ci\`o comporta che l'aggiunta di una regola rt non cambia i giudizi che ne possono derivare.\\ Quando un assioma \`e ammissibili e derivabile questo coincide con un giudizio derivabile.

\subsection{Nozione di contesto telescopico}
\label{subsec:contesto-telescopico}
Un giudizio, in teoria dei tipi dipendenti si esprime nella forma
\begin{center}\textbf{A(x$_1$,...,x$_n$)[x$_1$ $\in$ B$_1$,...,x$_n$ $\in$ B$_n$]}\end{center}
e prende il nome di \textbf{contesto telescopico}.
Questi presenta una dipendenza continua, esemplificata nel seguente giudizio
\[A(x_1,x_2,x_3)\hspace{0.1cm}type[x \in C_1, x_2 \in C(x_1), x_3 \in C(x_1 x_2)..)\]
Si parla di contesti rigidi, ovvero senza possibilit\`a di scambio. Come appare dall'esempio sotto. \\
$[x \in Nat, y \in Nat, z \in Mat(x,y)]$ cont  $\Rightarrow$ \textbf{\`e derivabile}\\
$[y \in Nat, x \in Nat, z \in Mat(x,y)]$ cont $\Rightarrow$ \textbf{\`e derivabile}\\
$[y \in Nat, z \in Mat(x,y), x \in Nat]$ $\Rightarrow$ \textbf{non \`e un contesto}. \\\\ Per cui non esiste lo scambio arbitrario, e si deve porre attenzione alle dipendenza delle assunzioni, che provoca una sostituzione rigida.
\subsection{Esempi di applicazione delle regole di sostituzione}
\label{subsec:esempi-di-applicazione}
\textit{Nota: attenzione all'ordine di sostituzione, si deve partire sempre da quello con meno dipendenze.}
\begin{prooftree}
\AxiomC{c $\in$ [C, $\Gamma$]}
\AxiomC{b $\in$ [B(c), $\Gamma]$}
\AxiomC{A(x,y) type[x $\in$ C, y $\in$ B(x)]}
\LeftLabel{sost}
\TrinaryInfC{A(c,b) type$[\Gamma]$}
\end{prooftree}

\begin{prooftree}
\AxiomC{c $\in$ [C,$\Gamma$]}
\AxiomC{b $\in$ [B(c),$\Gamma]$}
\AxiomC{a(x,y) $\in$ A(x,y)[x $\in$ C, y $\in$ B(x)]}
\LeftLabel{sost}
\TrinaryInfC{a(c,b) $\in$ A(c,b) $[\Gamma]$}
\end{prooftree}
Se si ha un giudizio di tipo pu\`o venire usato il giudizio di uguaglianza tra termini e la sostituzione.
\begin{prooftree}
\AxiomC{A(x) type[$\Gamma$, x $\in$ C]}
\AxiomC{c $=$ e $\in$ C[$\Gamma$]}
\LeftLabel{sost-eq}
\BinaryInfC{A(c) $=$ A(e) type[$\Gamma$]}
\end{prooftree}
\noindent
\`E dunque fondamentale, tra tipi dipendenti, il concetto di uguaglianza fra tipi. \\Allo stesso modo posso considerare il caso in cui si ha un elemento

\begin{prooftree}
\AxiomC{a(x) $\in$ A(x) [$\Gamma$, x $\in$ C]}
\AxiomC{c $=$ e $\in$ C[$\Gamma$]}
\LeftLabel{sost-eq}
\BinaryInfC{a(c) $=$ a(e) $\in$ A(c) [$\Gamma$]}
\end{prooftree}
\noindent
Per poter affermare che A(e) $=$ A(c) devo poterlo dedurre. Per farlo mi sono indispensabili le \textbf{regole di Conversione dell'uguaglianza del tipaggio (\S\ref{subsec:regole-tipaggio})}.

\subsection{Regole di tipaggio}
\label{subsec:regole-tipaggio}

\subsubsection{Regole di Conversione}
\label{subsubsec:regole-di-conversione}
\[conv)\frac{c \in C[\Gamma] \qquad C = D\hspace{0.1cm}type[\Gamma]}{c \in D\hspace{0.1cm}[\Gamma]}\]
Se due tipi sono uguali allora hanno gli stessi termini: $C=D \Rightarrow (c \in C \Leftrightarrow c \in D)$. L'uguaglianza fra tipi \`e per questo simmetrica.\\
Tuttavia non sempre l'unicit\`a del tipaggio di un termine (il $\Leftrightarrow$) \`e garantito per ogni teoria. Nei casi trattati dal corso s\`i, in quanto verr\`a inteso che $C = D\hspace{0.1cm}type[\Gamma]$ sse due tipi hanno gli stessi elementi (come gi\`a accade in \textit{set theory}), ma pu\`o non essere sempre vero.

\subsubsection{Regole di Conversione dell'uguaglianza}
\label{subsubsec:regole-di-conversione-uguaglianza}
\[conv-eq)\frac{c=d \in C[\Gamma] \qquad C = D\hspace{0.1cm}type[\Gamma]}{c = d \in D \hspace{0.1cm} [\Gamma]}\]
Questa regola permette di convertire le uguaglianze nel tipaggio di un termine.


\section{Il tipo singoletto}
\label{sec:tipo-singoletto}
Il tipo singoletto risulta essere paradigmatico per gli altri tipi. Per definirlo impiegher\`o i giudizi nella forma $A\hspace{0.1cm}type[\Gamma]$, $a \in A\hspace{0.1cm}[\Gamma]$ e $a = b \in A\hspace{0.1cm}[\Gamma]$. L'uguaglianza, invece, non verr\`a coinvolta, in quanto non pu\`o essere impiegata per definire un nuovo tipo, \`e difatti usata solo nelle derivazioni.\\\\
Innanzitutto come gi\`a visto in \S\ref{subsec:formazione-contesti} ogni derivazione parte sempre dal contesto vuoto ([ ]).
\begin{prooftree}
\AxiomC{A type[$\Gamma$]}
\LeftLabel{F-c)}\RightLabel{(x $\in$ A) $\notin$ $\Gamma$}
\UnaryInfC{$\Gamma$, x $\in$ A}
\end{prooftree}

\subsection{Regola di Formazione}
\label{subsec:formazione-singoletto}
\begin{prooftree}
\AxiomC{$\Gamma$ cont}
\LeftLabel{F-S)}
\UnaryInfC{$N_1$ type[$\Gamma]$}
\end{prooftree}
\textit{La regola (F-S) permette di derivare vari giudizi e di dire che cosa \`e un tipo.}\\\\
Con l'impiego solo delle regole \textit{(F-c)} e \textit{(F-S)} si possono derivare [$x_1$ $\in$ $N_1$... $x_n$ $\in$ $N_1$] cont e ottenere, cos\`i, contesti di una lista arbitraria di variabili diverse, appartenenti a $N_1$, come si vede dall'esempio seguente.
\begin{prooftree}
\AxiomC{[ ] cont}
\LeftLabel{F-S}
\LeftLabel{F-c}\RightLabel{($x_1$ $\in$ $N_1$) $\notin$ [ ]}
\UnaryInfC{[$x_1$ $\in$ $N_1$] cont}
\LeftLabel{F-S}
\UnaryInfC{$N_1$ type [$x_1$ $\in$ $N_1$]}
\LeftLabel{F-c}\RightLabel{($x_2$ $\in$ $N_1$) $\notin$ $x_1$ $\in$ $N_1$}
\UnaryInfC{[$x_1$ $\in$ $N_1$, $x_2$ $\in$ $N_1$] cont}
\end{prooftree}

\subsection{Regole di Introduzione}
\label{subsec:introduzione-singoletto}
\begin{prooftree}
\AxiomC{$\Gamma$ cont}
\LeftLabel{I-S)}
\UnaryInfC{$\ast$ $\in$ $N_1$ [$\Gamma]$}
\end{prooftree}
\textit{Sia $N_1$ in ogni contesto $\Gamma$, partendo da contesto [ ], la regola (I-S) permette di formare i termini, per mezzo dell'introduzione di un elemento costante $\ast$ in $N_1$.}\\\\
Un esempio diretto della sua applicazione \`e
\begin{prooftree}
\AxiomC{$x_1$ $\in$ $N_1$...$x_n$ $\in$ $N_1$ cont}
\LeftLabel{I-S)}
\UnaryInfC{$\ast$ $\in$ $N_1$ [$x_1$ $\in$ $N_1$...$x_n$ $\in$ $N_1$]}
\end{prooftree}

\subsection{Regole di Eliminazione}
\label{subsec:eliminazione-singoletto}
\begin{prooftree}
\AxiomC{t $\in$ $N_1$ [$\Gamma$]}
\AxiomC{M(z) type[$\Gamma$, z $\in$ $N_1$]}
\AxiomC{c  $\in$ M($\ast$)[$\Gamma$]}
\LeftLabel{E-S)}
\TrinaryInfC{$El_{N1}$(t, c) $\in$ M($\ast$)[$\Gamma$]}
\end{prooftree}
\noindent
\textit{El trattasi di costruttore di funzioni e $M[t] = M(z)[\frac{z}{t}]$.}
\subsection{Regole di Conversione}
\label{subsec:conversione-singoletto}
\begin{prooftree}
\AxiomC{M(z) type[$\Gamma$, z $\in$ $N_1$]}
\AxiomC{c $\in$ M($\ast$)[$\Gamma$]}
\LeftLabel{C-S)}
\BinaryInfC{$El_{N1}$($\ast$, c) $=$ c $\in$ M($\ast$)[$\Gamma$]}
\end{prooftree}
\textit{La conversione rende possibile l'applicazione della regola di eliminazione (E-S) introducendo delle uguaglianze.}

\subsection{Regole di Uguaglianza}
\label{subsec:uguaglianza-singoletto}
\begin{prooftree}
\AxiomC{M(z) type[$\Gamma$, z $\in$ \textit{N$_1$}]}
\AxiomC{t$_1$ $=$ t$_2$ $\in$ \textit{N$_1$}[$\Gamma$]}
\AxiomC{c$_1$ $=$ c$_2$ $\in$ M($\ast$)[$\Gamma$]}
\LeftLabel{eq-E-N$_1$)}
\TrinaryInfC{El$_{N1}$(t$_1$, c$_1$) $=$ El$_{N1}$(t$_2$, c$_2$) $\in$ M($t_1$)[$\Gamma$]}
\end{prooftree}
\noindent
\\\\
Le regole \textit{(S), (I-S), (E-S)} e \textit{(C-S)}  hanno una spiegazione computazionale, e riguardano la compatibilit\`a tra tipi, ma non da caratterizzare il tipo dei tipi.\\ Inoltre il tipo singoletto non \`e dipendente.

\subsection{Eliminatore dipendente}
\label{subsec:eliminatore dipendente-singoletto}
La regola di eliminazione si pu\`o equivalentemente scrivere in un altro modo
\begin{prooftree}
\AxiomC{M(z) type[$\Gamma$, z $\in$ $N_1$]}
\AxiomC{c  $\in$ M($\ast$)[$\Gamma$]}
\LeftLabel{E-S$_{dip}$)}
\BinaryInfC{$El_{N1}$(z, c) $\in$ M(z)[$\Gamma$, z $\in$ $N_1$]}
\end{prooftree}
Le regole \textit{(E-S$_{dip}$)} + \textit{(le regole di sostituzione)} + \textit{(F-S)} + \textit{(I-S)} permettono di verificare la validit\`a di \textit{(E-S)}.
\begin{prooftree}
\AxiomC{t $\in$ $N_1$[$\Gamma$]}
\AxiomC{M(z) type[$\Gamma$, z $\in$ $N_1$]}
\AxiomC{c $\in$ M($\ast$)[$\Gamma$]}
\LeftLabel{E-S$_{dip}$)}
\BinaryInfC{$El_{N1}$(z, c) $\in$ M(z)[$\Gamma$]}
\LeftLabel{sost)}
\BinaryInfC{$El_{N1}$(t, c) $\in$ M(t)[$\Gamma$]}
\end{prooftree}
\noindent
Inoltre vale anche il viceversa, da \textit{(E-S)} si riesce a ottenere \textit{(E-S$_{dip}$)}.
\subsection{Osservazioni sul tipo singoletto}
\label{subsec:osservazioni-singoletto}
L'eliminatore El$_{N1}$(z, c) rappresenta una funzione definita per ricorsione su N$_1$, difatti in \textit{(C-S)} si ha che El$_{N1}$(z, c)[$\frac{z}{\ast}$] $=$ El$_{N1}$($\ast$, c).\\
Supposto che se $\ast$ $\in$ N$_1$[$\Gamma$] in \textit{(E-S)}, allora per la singola conversione vale che El$_{N1}$($\ast$,c) $=$ c $\in$ M($\ast$).
Dunque El$_{N1}$(z, c) rappresenta un programa funzionale per ricorsione. Questi \`e definito su N$_1$, a partire da c $\in$ M($\ast$), perci\`o El$_{N1}$($\ast$, c) $=$ c.\\
La regola di eliminazione permette di definire un programma funzionale da N$_1$ a M(z) esclusivamente con c $\in$ M($\ast$), ovvero definendo $\ast$ come \textbf{elemento canonico}. Inoltre non svolge solo il compito di ricorsione, ma anche d'induzione.
\\\\
In \begin{prooftree}
\AxiomC{t $\in$ N$_1$[$\Gamma$]}
\UnaryInfC{t $=$ $\ast$ $\in$ N$_1$[$\Gamma$]}
\end{prooftree}
\textit{risulta vera l'uguaglianza definizionale?}\\
No, non \`e vera. La regola di eliminazione consente di dare un valore al termine canonico, permettendo cos\`i di attribuire un valore a tutti i possibili termini del singoletto. Ma in generale questo non vale all'interno della teoria. Difatti l'uguaglianza definizionale \`e diversa da quella matematica e va intesa come computazionale e non proposizionale (come definito in \S \ref{subsec:uguaglianza-computazionale-proposizionale}).

\section{Sanitary checks rules}
\label{sec:sanitary-checks}
Le \textit{Sanitary checks} sono regole strutturali utili per abbreviare le derivazioni. Queste sono derivabili solo se la teoria dei tipi \`e corretta.\\\\
\noindent
Assumendo \textit{T}, come una teoria dei tipi di riferimento, le \textit{Sanitary checks} sono le seguenti:
\begin{prooftree}
\AxiomC{[$\Gamma$, $\Delta$] cont}
\UnaryInfC{$\Gamma$ cont}
\end{prooftree}
Se \textbf{[$\Gamma$, $\Delta$] cont} \`e derivabile in \textit{T} allora anche \textbf{[$\Gamma$] cont} \`e derivabile in \textit{T}.
\begin{prooftree}
\AxiomC{J$_1$,...,J$_n$}
\UnaryInfC{J}
\end{prooftree}
Se \textbf{J$_i$ con i $=$ 1,...,n} in \textit{T} sono derivabili allora anche \textbf{J} \`e derivabile in \textit{T}.
\begin{prooftree}
\AxiomC{A type [$\Gamma$]}
\UnaryInfC{$\Gamma$ cont}
\end{prooftree}
Se \textbf{A type [$\Gamma$]} \`e derivabile in \textit{T} allora anche \textbf{$\Gamma$ cont} \`e derivabile in \textit{T}.
\begin{center}
\AxiomC{A $=$ B type [$\Gamma$]}
\UnaryInfC{A type[$\Gamma$]}
\DisplayProof
\AxiomC{A $=$ B type [$\Gamma$]}
\UnaryInfC{B type[$\Gamma$]}
\DisplayProof
\end{center}
Se \textbf{A $=$ B type [$\Gamma$]} \`e derivabile in \textit{T} allora anche \textbf{A type[$\Gamma$]} e \textbf{B type[$\Gamma$]} sono derivabili in \textit{T}.
\begin{prooftree}
\AxiomC{a $\in$ A[$\Gamma$]}
\UnaryInfC{A type[$\Gamma$]}
\end{prooftree}
Se \textbf{a $\in$ A[$\Gamma$]} \`e derivabile in \textit{T} allora anche \textbf{A type[$\Gamma$]} \`e derivabile in \textit{T}.
\begin{center}
\AxiomC{a $=$ b $\in$ A[$\Gamma$]}
\UnaryInfC{a $\in$ A[$\Gamma$]}
\DisplayProof
\AxiomC{a $=$ b $\in$ A[$\Gamma$]}
\UnaryInfC{b $\in$ A[$\Gamma$]}
\DisplayProof
\end{center}
Se \textbf{a $=$ b $\in$ A[$\Gamma$]} \`e derivabile in \textit{T} allora anche \textbf{a $\in$ A[$\Gamma$]} e \textbf{b $\in$ A[$\Gamma$]} sono  derivabili in \textit{T}.

\section{Schema generale}
\label{sec: schema-generale}
Di seguito illustro uno schema generale, valido per ogni teoria dei tipi, di produzione di regole definenti un tipo, i suoi termini e l'uguaglianza.
\begin{enumerate}
\item \textbf{Regole di Formazione:}
identificate con il preambolo F-\textit{T}, con \textit{T} teoria dei tipi in esame e t$^\backprime$ elemento canonico.\\Tali regole rispettano la forma 
\AxiomC{[$\Gamma$] cont}
\UnaryInfC{\textit{T} type[$\Gamma$]}
\DisplayProof
\item \textbf{Regole di Introduzione:}
identificate con il preambolo I-\textit{T}, con \textit{T} teoria dei tipi in esame e t$^\backprime$ elemento canonico.\\Tali regole consistono nella forma introduttiva degli elementi canonici di \textit{T} e rispettano la forma
\AxiomC{[$\Gamma$] cont}
\UnaryInfC{t$^\backprime$ $\in$ \textit{T}[$\Gamma$]}
\DisplayProof
\item \textbf{Regole di Eliminazione:}
identificate con il preambolo E-\textit{T} con \textit{T} teoria dei tipi in esame e t$^\backprime$ elemento canonico.\\Tali regole sono definenti E$_T$, a partire dagli elementi di \textit{T} a valori in un tipo M(z) type[$\Gamma$, z $\in$\textit{T}]. L'ipotesi valida \`e che siano dati degli elementi in M(z) sui valori canonici di \textit{T}.\vspace{0.2cm}
Tali regole rispettano la forma\\
\AxiomC{t $\in$ \textit{T}[$\Gamma$]}
\AxiomC{M(z) type[$\Gamma$, z $\in$ \textit{T}]}
\AxiomC{c  $\in$ M(t$^\backprime$)[$\Gamma$]}
\TrinaryInfC{$El_{\textit{T}}$(t, c) $\in$ M(t)[$\Gamma$]}
\DisplayProof
\item \textbf{Regole di Conversione:}
identificate con il preambolo C-\textit{T}, con \textit{T} teoria dei tipi in esame e t$^\backprime$ elemento canonico.\\, che stabiliscono che gli eliminatori in (3) sono stabiliti per ricorsione a partire dalle ipotesi.\vspace{0.2cm}
Tali regole rispettano la forma\\
\AxiomC{M(z) type[$\Gamma$, z $\in$ \textit{T}]}
\AxiomC{c $\in$ M(t$^\backprime$)[$\Gamma$]}
\BinaryInfC{El$_{\textit{T}}$(t$^\backprime$, c) $=$ c $\in$ M(t$^\backprime$)[$\Gamma$]}
\DisplayProof
\item \textbf{Regole di Uguaglianza:}
identificate con il preambolo eq-E-\textit{T}, con \textit{T} teoria dei tipi in esame e t$^\backprime$ elemento canonico.\\
Tali regole stabiliscono che i costruttori di \textit{T} in (2) e (3) permettono l'uguaglianza definizionale dei termini da cui dipendono.\vspace{0.2cm}
Tali regole rispettano la forma\\
\AxiomC{M(z) type[$\Gamma$, z $\in$ \textit{T}]}
\AxiomC{t $=$ s $\in$ \textit{T}[$\Gamma$]}
\AxiomC{c $=$ d $\in$ M(t$^\backprime$)[$\Gamma$]}
\TrinaryInfC{El$_{\textit{T}}$(t, c) $=$ El$_{\textit{T}}$(s, d) $\in$ M(t)[$\Gamma$]}
\DisplayProof
\end{enumerate}
\noindent
Le regole (5) sono implicite in \textit{T}, ma non ovvie dal punto di vista computazionale.

\section{Uguaglianza definizionale}
\label{sec: uguaglianza-definizionale}
\textit{Definizione}: se P$_1$ e P$_2$ programmi\\\\
\textit{$P_1$, $P_2$: Nat$^m$  $\rightarrow$ Nat \quad allora \quad
$P_1$ $=$ $P_2$ sse $\forall$ n$_1$...n$_m$ e P$_1$(n$_1$...n$_m$) $=$ P$_2$(n$_1$...n$_m$)}. Ma questo non \`e decidibile.\\\\
Ovvero le funzioni ricorsive per P$_1$ e P$_2$ non sono decidibili. A seguito di ci\`o non esiste un algoritmo in grado di decidere se due programmi P$_1$ e P$_2$ sono estensionalmente (proposizionalmente) uguali o meno.\\\\
Risulta per\`o vero il concetto di \textbf{uguaglianza definizionale}/computazionale (in teoria dei tipi intensionali). \\Dati a $\in$ A[$\Gamma$] e b $\in$ A[$\Gamma$] derivabili, nella nostra teoria \textit{T} di \textit{Martin-L$\ddot{o}$f}, esiste un algoritmo \textit{H} (a $\in$ A[$\Gamma$], b $\in$ A[$\Gamma$]) $=$
$
\begin{cases}
\text{\textbf{sì} sse a \text{=} b} \in A[\Gamma]  \text{ è derivabile in \textit{T}} \\
\text{\textbf{no} sse a \text{=} b} \in A[\Gamma] \text{ non ha derivazione in \textit{T}}
\end{cases}
$
\\
Il giudizio a $=$ b $\in$ A[$\Gamma$] \`e decidibile, anche con \textit{J} giudizio, in teoria dei tipi di \textit{Martin-L$\ddot{o}$f}, derivabile. Per cui con H si scrive: Giudizi di \textit{T} $\rightarrow$ \{0,1\}\\
H[\textit{J}] $=$
$
\begin{cases}
\textbf{1} \text{ sse \textit{J} è derivabile in \textit{T}}  \\
\textbf{0} \text{ sse \textit{J} non è derivabile in \textit{T}}
\end{cases}
$
\\
H lavora come un \textit{proof-assistant} (esempio \textit{Coq}).\\\\
\noindent

\subsection{Applicazione dell'uguaglianza definizionale tra termini}
\label{subsec: applicazione-uguaglianza-definizionale-tra-termini}
\textit{Definizione}: i termini \textit{untyped} sono
\begin{center}t $=$ x $|$ $\ast$ $|$ El$_{N1}$(t$_1$, t$_2$)\end{center}
\textit{Definizione}: relazione di riduzione
\begin{center}$\rightarrow$ $_1$ $\subseteq$ term x term\end{center}
t$_1$ $\rightarrow$ $_1$ t$_2$ $\equiv$ t$_1$ si riduce in un passo di computazione a t$_2$. Ecco che esiste una relazione che computa t$_1$ con t$_2$.\\
Le \textbf{relazioni di computazione}, dell'uguaglianza definizionale, le ho descritte in \S\ref{sec: semantica-operazionale-singoletto}.
\\\\
\textit{Definizione}: t termine \textit{untyped} \`e in forma normale (in NF) sse non esiste termine s tale che t $\rightarrow$ $_1$ s. Ovvero non \`e pi\`u riducibile a nulla. \\  Le forme normali sono i valori assumibili dai programmi.\\
\\\\
\textit{Definizione}:\textbf{Teoria di Validit\`a}\\ Dati t $\in$ A[$\Gamma$] e s $\in$ A[$\Gamma$] in \textit{T$_1$} allora, t $\rightarrow$ $_1$ s $\Rightarrow$ t $=$ s $\in$ A[$\Gamma$] derivabile in \textit{T}.


\begin{table}[H]
\centering
\begin{tabularx}{\textwidth}{XXX}
\hline 
\rowcolor{amethyst}
{\color[HTML]{FFFFFF}\textbf{termini}} & {\color[HTML]{FFFFFF} \textbf{termini in forma canonica}} & {\color[HTML]{FFFFFF}\textbf{termini non in forma canonica o introduttiva}}\\
\hline\hline
termini in NF & $\ast$ & x, El$_{N1}$(x, $\ast$)\\
termini non in NF & $\diagup$ & El$_{N1}$($\ast$, x)\\
\end{tabularx}
\caption{\label{tab:termini-NF-nonNF-N1}Termini a $\rightarrow$ $_1$ di N$_1$.} 
\end{table}
\noindent
I termini non in forma canonica  derivano dalle regole di introduzione; invece quelli non in forma canonica vengono introdotte dall'eliminatore.\\\\
La chiusura riflessiva, simmetrica e transitiva delle derivazioni \`e proprio l'uguaglianza definizionale. Tale propriet\`a non vale esclusivamente per $\rightarrow_1$ ma per qualsiasi combinazione delle riduzioni in \$\ref{sec: semantica-operazionale-singoletto}, con  variazione per\`o delle forma normali (che non sono altro che i risultati dei nostri programmi, derivanti dai diversi cammini di computazione).\\
Un esempio significativo di applicazione di strategie di computazione l'ho riportato in \S\ref{sec:esercizi-cap2}, esercizio 4. Utile per comprendere a cosa serve la relazione $\rightarrow_1$   e che ogni teoria \textit{T} $\rightarrow$ $_1$ non \`e deterministica.
\\\\
Le definizioni seguenti sono definite sulla regole di semantica operazionale.\\\\
\textit{Definizione} \textbf{Riducibilit\`a}\\
Dati i termini t e s allora \textbf{t \textit{Red$_{NF}$} s} sse s \`e in NF ed esistono h$_1$...h$_n$ (n$>=$1) tale che h$_1$ $\equiv$ t, h$_n$ $\equiv$ s e se n$>$1 h$_i$ $\rightarrow$ h$_{i+1}$ per i $=$ 1 a n-1.\\\\
\textbf{t \textit{Red$_{NF}$} s} $=$
$
\begin{cases}
\text{\textbf{t}} \equiv \text{s e t è in NF} \\
\text{esiste n>1, esistono } h_1\text{...}h_n \text{ termini tale che \textbf{t}} \equiv h_1 \rightarrow _1 h_2 \text{...}h_n
\end{cases}
$
\\\\
\noindent
\textit{Definizione} \textbf{Teorema di Confluenza}$\rightarrow _1$ per \textit{T} computabile\\
Dato t (termine) e s$_1$ e s$_2$ (in NF) tale che \textbf{t \textit{Red$_{NF}$ }s$_1$} e \textbf{t \textit{Red$_{NF}$} s$_2$} allora s$_1$ $\equiv$ s$_2$ (coincide a meno di rinomina di variabile  vincolante).\\
Quando t si riduce s$_1$ e in s$_2$ c'\`e l'unicit\`a della forma normale.
\\\\
\noindent
\textit{Definizione} \textbf{Teorema della forma normale (debole)}\\
Dato t termine della grammatica, esiste s termine in NF, tale che \textbf{t \textit{Red$_{NF}$ }s}. Allora esistono t $\equiv$ h$_1$ $\rightarrow$ $_1$...$\rightarrow$ $_1$ h$_n$ con n>1 se t non \`e gi\`a in NF; oppure t $\equiv$ s se t \`e gi\`a in NF.\\
Questo significa che \`e sempre possibile rendere un programma convergente. Ma si pu\`o dire di pi\`u: $\nexists$ programmi che divergono. 
\\\\
\noindent
\textit{Definizione} \textbf{Teorema della forma normale (forte)}\\
Per ogni termine t, l'albero dei cammini di riduzione di t \`e ben formato (ovvero $\nexists$ un cammino di riduzioni $\rightarrow$ $_1$ infinito).\\
In questo modo ogni strategia deterministica \`e convergente.\\
Con quanto appena enunciato sopra possiamo definire quanto segue.\\
Dato t $\in$ A[$\Gamma$] derivabile in \textit{T}\\
NF(t$_1$) $\equiv$
$
\begin{cases}
\text{\textbf{t} se t è in NF} \\
\text{\textbf{s} se \textit{\textbf{t Red}}}_{NF} \text{\textit{\textbf{ s}}}
\end{cases}
$
\\
Dunque se t non \'e in NF, per il teorema normale, comunque esiste una riduzione in NF.\\
Sono cos\`i in grado di dimostrare che, dati a $\in$ A[$\Gamma$] e b $\in$ A[$\Gamma$], giudizi derivabili in \textit{T$_i$} allora a $=$ b $\in$ A[$\Gamma$] sse NF(a) $\equiv$ NF(b) sse
\begin{enumerate}
\item a e b sono in NF e quindi a $\equiv$ b
\item a non in NF, b in NF e \textbf{a \textit{Red$_{NF}$ }b}
\item a in NF, b non in NF e \textbf{b \textit{Red$_{NF}$ }a}
\item n\'e a n\'e b sono in NF esiste s in NF tale che \textbf{a \textit{Red$_{NF}$ }s} e \textbf{b \textit{Red$_{NF}$ }s}
\end{enumerate}
Per i punti elencanti sopra trova validit\`a la relazione \textbf{a \textit{Red$_{NF}$ }NF(a)} $\Rightarrow$ a $=$ NF(a) $\in$ A[$\Gamma$] \`e derivabile (la forma normale \`e uguale al termine stesso).\\ Questo rende l'uguaglianza computabile, si \`e difatti in grado di dimostrare che esiste P programma tale che P(a) $=$ NF(a), per ogni a termine \textit{untyped} in \textit{T}(incluso \textit{T$_1$}).\\\\
In conclusione la computabilit\`a dell'uguaglianza (definizionale) tra due termini, si riduce a computare le forme normali del primo termine con quelle del secondo e a verificare se sono identicamente la stessa (a meno di rinomina di variabili).

\section{Semantica operazionale del singoletto}
\label{sec: semantica-operazionale-singoletto}
La relazione $\rightarrow_1$ viene definita all'interno dei termini con l'uso delle seguenti regole di riduzione:
\begin{itemize}
\item $\beta_{N1}$-red) El$_{N1}$($\ast$, t) $\rightarrow_1$ t
\item \AxiomC{t$_1$ $\rightarrow_1$ t$_2$}
\LeftLabel{red$_I$)}
\UnaryInfC{El$_{N1}$(t$_1$, c) $\rightarrow_1$ El$_{N1}$(t$_2$, c)}
\DisplayProof \qquad
\AxiomC{c$_1$ $\rightarrow_1$ c$_2$}
\LeftLabel{red$_{II}$)}
\UnaryInfC{El$_{N1}$(t, c$_1$) $\rightarrow_1$ El$_{N1}$(t, c$_2$)}
\DisplayProof 
\item red$_{I}$ e red$_{II}$ possono venire simulate da un'unica regola
\AxiomC{t$_1$ $\rightarrow_1$ t$_2$}
\AxiomC{c$_1$ $\rightarrow_1$ c$_2$}
\BinaryInfC{El$_{N1}$(t$_1$, c$_1$) $\rightarrow_1$ El$_{N1}$(t$_2$, c$_2$)}
\DisplayProof
\end{itemize}
\noindent
\textit{$\beta_{N1}$-red} risulta valida per \textit{(C-S)}, le regole di riduzione red$_{I}$ e red$_{II}$ per \textit{(eq-E-S)}.\\\\


\section{Esercizi}
\label{sec:esercizi-cap2}
\paragraph{1)} 
\textbf{Data}
\begin{prooftree}
\AxiomC{M(w) type [$\Gamma$, w $\in$ N$_1$]}
\AxiomC{d $\in$ M($\ast$) type[$\Gamma$]}
\LeftLabel{E-N$_{1prog}$)}
\BinaryInfC{El$_{N1}$(w, d) $\in$ M(w) type[$\Gamma$, w $\in$ N$_1$]}
\end{prooftree}
\textbf{dimostrare che in \textit{T$_1$} la regola \textit{E-N$_{1prog}$} \`e derivabile. Al fine di ci\`o basta mostrare che se i giudizi premessa sono derivabili, allora lo \`e anche il giudizio di conclusione.}\\\\
\textbf{Soluzione}\\\\
\textit{Per una maggiore comprensione delle derivazioni, ho ritenuto opportuno, ove necessario, spezzare l'albero in pi\`u parti.}\\\\
\noindent 
In E-$N_{1prog}$ il contesto \`e $\Sigma$ $\equiv$ $\Gamma$, w $\in$ N$_1$ che permette l'applicazione di E-S.

\begin{adjustwidth}{-4em}{}
\begin{prooftree}
\AxiomC{}
\UnaryInfC{M(w) type[$\Gamma$, w $\in$ N$_1$]}
\LeftLabel{s-checks}
\UnaryInfC{[$\Gamma$, w $\in$ N$_1$] cont}
\LeftLabel{var}
\UnaryInfC{w $\in$ N$_1$[$\Gamma$, w $\in$ N$_1$]}
\AxiomC{\textbf{1}}
\UnaryInfC{M(z) type[$\Gamma$, w $\in$ N$_1$, z $\in$ N$_1$]}
\AxiomC{}
\UnaryInfC{d $\in$ M($\ast$)[$\Gamma$, w $\in$ N$_1$]}
\LeftLabel{E-S}
\TrinaryInfC{El$_{N1}$(w, d) $\in$ M(w)[$\Gamma$, w $\in$ N$_1$]}
\end{prooftree}
\end{adjustwidth}

\vspace{0.5cm}
\textbf{1}
\scriptsize{
\begin{adjustwidth}{-12em}{}
\begin{prooftree}
\AxiomC{}
\UnaryInfC{M(w) type[$\Gamma$, w $\in$ N$_1$]}
\LeftLabel{s-checks}
\UnaryInfC{$\Gamma$, w $\in$ N$_1$ cont}
\LeftLabel{F-S}
\UnaryInfC{N$_1$ type [$\Gamma$, w $\in$ N$_1$]}
\LeftLabel{F-c}\RightLabel{\begin{tabular}[c]{cc}(z $\in$ N$_1$) $\notin$\\ ($\Gamma$, w $\in$ N$_1$)\end{tabular}}
\UnaryInfC{$\Gamma$, w $\in$ N$_1$, z $\in$ N$_1$ cont}
\LeftLabel{var}
\UnaryInfC{z $\in$ $N_1$[$\Gamma$, w $\in$ $N_1$, z $\in$ N$_1$]}

\AxiomC{}
\UnaryInfC{M(w) type[$\Gamma$, w $\in$ N$_1$]}
\AxiomC{}
\UnaryInfC{M(w) type[$\Gamma$, w $\in$ N$_1$]}
\LeftLabel{s-checks}
\UnaryInfC{$\Gamma$, w $\in$ N$_1$ cont}
\LeftLabel{F-S}
\UnaryInfC{N$_1$ type[$\Gamma$, w $\in$ N$_1$]}
\LeftLabel{F-c}\RightLabel{\begin{tabular}[c]{cc}(z $\in$ N$_1$) $\notin$\\ ($\Gamma$, w $\in$ N$_1$)\end{tabular}}
\UnaryInfC{$\Gamma$, w $\in$ N$_1$, z $\in$ $N_1$ cont}
\LeftLabel{ind-ty}
\BinaryInfC{M(w) type[$\Gamma$, w $\in$ N$_1$,  z$\in$ $N_1$]}

\LeftLabel{sub-typ}
\BinaryInfC{M(z) type[$\Gamma$, w $\in$ N$_1$, z $\in$ N$_1$]}
\end{prooftree}
\end{adjustwidth}
}
\noindent
\normalsize{Assumo che le premesse di \textit{E-N$_{1prog}$} (\textit{M(w) type [$\Gamma$,w $\in$ N$_1$]} e \textit{d $\in$ M($\ast$)[$\Gamma$]}) siano valide, perci\`o \`e valido, dalla prova sopra, anche il giudizio di conclusione \textit{El$_{N1}$(w, d) $\in$ M(w)[$\Gamma$, w $\in$ N$_1$]}, di conseguenza derivabile in \textit{T$_1$}.}


\paragraph{2)} 
\textbf{Dimostrare che la regola \textit{E-S} \`e derivabile in una teoria dei tipi \textit{T$_1$}, in cui si \`e rimpiazzata la regola di eliminazione \textit{E-S} con la regola \textit{E-N$_{1prog}$}, aggiungendovi le regole di indebolimento, sostituzione e di \textit{Sanitary checks}}.
\begin{prooftree}
\AxiomC{M(w) type [$\Gamma$, w $\in$ N$_1$]}
\AxiomC{d $\in$ M($\ast$)[$\Gamma$]}
\LeftLabel{E-N$_{1prog}$)}
\BinaryInfC{El$_{N1}$(w, d) $\in$ M(w)[$\Gamma$, w $\in$ N$_1$]}
\end{prooftree}
\begin{prooftree}
\AxiomC{t$^\backprime$ $\in$ N$_1$[$\Gamma$] }
\AxiomC{M(w) type [$\Gamma$, w $\in$ N$_1$]}
\AxiomC{d $\in$ M($\ast$)[$\Gamma$] }
\LeftLabel{E-S)}
\TrinaryInfC{El$_{N1}$(t$^\backprime$, d) $\in$ M(t$^\backprime$)[$\Gamma$]}
\end{prooftree}
\textbf{Soluzione}\\\\
\textit{Idea: parto dalla regola di eliminazione E-S, vi applico la regola di sostituzione sub-typ giungendo cos\`i alle premesse di E-N$_{1prog}$}
\begin{prooftree}
\AxiomC{}
\UnaryInfC{t$^\backprime$ $\in$ N$_1$[$\Gamma$]}
\AxiomC{}
\UnaryInfC{M(w) type [$\Gamma$, w $\in$ N$_1$]}
\AxiomC{}
\UnaryInfC{d $\in$ D($\ast$)[$\Gamma$]}
\LeftLabel{E-N$_{1prog}$}
\BinaryInfC{El$_{N1}$(w, d) $\in$ M(w)[$\Gamma$, w $\in$ N$_1$]}
\LeftLabel{sub-typ}
\BinaryInfC{El$_{N1}$(t$^\backprime$, d) $\in$ M(t$^\backprime$)[$\Gamma$]}
\end{prooftree}
Assumo che siano valide per costruzione le premesse di \textit{E-N$_{1prog}$} (come dimostro nell'esercizio 1) e di \textit{E-S}.

\paragraph{3)} \textbf{Sia \textit{T$_1$} la teoria dei tipi definita del tipo singoletto con le regole strutturali, definite in questo capitolo, incluse quelle di sostituzione e indebolimento. Allora stabilire se i seguenti termini sono tipabili come termini del tipo singoletto, secondo \textit{T$_1$} e quali sono uguali definizionalmente.}
\begin{itemize}
\item El$_{N1}$($\ast$, $\ast$)
\item El$_{N1}$(x, $\ast$)
\item El$_{N1}$($\ast$, y)
\item El$_{N1}$(x, y)
\item El$_{N1}$(El$_{N1}$($\ast$, y), El$_{N1}$(x, $\ast$))
\end{itemize}
\noindent
\\
\textbf{Soluzione}\\\\
\textit{Per una maggiore comprensione delle derivazioni, ho ritenuto opportuno, ove necessario, spezzare l'albero in pi\`u parti.}\\
\noindent
\textbf{1}\\
\small
\begin{prooftree}
\AxiomC{[ ] cont}
\LeftLabel{I-S}
\UnaryInfC{$\ast$ $\in$ N$1$[ ]}
\AxiomC{[ ] cont}
\LeftLabel{F-S}
\UnaryInfC{N$_1$ type[ ]}
\LeftLabel{F-c}\RightLabel{(z $\in$ N$_1$) $\notin$ [ ]}
\UnaryInfC{z $\in$ N$_1$ cont}
\LeftLabel{F-S}
\UnaryInfC{N$_1$ type[z $\in$ N$_1$]}
\AxiomC{[ ] cont}
\LeftLabel{I-S}
\UnaryInfC{$\ast$ $\in$ N$_1$[ ]}
\LeftLabel{E-S}
\TrinaryInfC{El$_{N1}$($\ast$, $\ast$) $\in$ $N_1$[ ]}
\end{prooftree}
\normalsize
Applicando la \textit{$\beta$ N$_1$-red} allora El$_{N1}$($\ast$, $\ast$) $\rightarrow_1$ $\ast$ \\
El$_{N1}$($\ast$, $\ast$) \`e uguale definizionalmente.
\\\\\\
\noindent
\textbf{2}\\
\scriptsize
\begin{prooftree}
\AxiomC{[ ] cont}
\LeftLabel{F-S}
\UnaryInfC{N$_1$ type[ ]}
\LeftLabel{F-c}\RightLabel{(x $\in$ N$_1$) $\notin$ [ ]}
\UnaryInfC{x $\in$ N$_1$ cont}
\LeftLabel{var}
\UnaryInfC{x $\in$ N$_1$[x $\in$ N$_1$]}
\AxiomC{[ ] cont}
\LeftLabel{F-S}
\UnaryInfC{N$_1$ type [ ]}
\LeftLabel{F-c}\RightLabel{(x $\in$ N$_1$) $\notin$ [ ]}
\UnaryInfC{x $\in$ N$_1$ cont}
\LeftLabel{F-S}
\UnaryInfC{N$_1$ type[x $\in$ N$_1$]}
\LeftLabel{F-c}\RightLabel{(z $\in$ N$_1$) $\notin$ x $\in$ N$_1$}
\UnaryInfC{x $\in$ N$_1$, z $\in$ N$_1$ cont}
\LeftLabel{F-S}
\UnaryInfC{N$_1$ type[x $\in$ N$_1$, z $\in$ N$_1$]}
\AxiomC{[ ] cont}
\LeftLabel{F-S}
\UnaryInfC{N$_1$ type[ ]}
\LeftLabel{F-c}\RightLabel{(x $\in$ N$_1$) $\notin$ [ ]}
\UnaryInfC{x $\in$ N$_1$ cont}
\LeftLabel{I-S}
\UnaryInfC{$\ast$ $\in$ N$_1$[x $\in$ N$_1$]}
\LeftLabel{E-S}
\TrinaryInfC{El$_{N1}$(x, $\ast$) $\in$ N$_1$[x $\in$ N$_1$]}
\end{prooftree}

\noindent
\normalsize
Applicando la \textit{$\beta$ N$_1$-red} allora El$_{N1}$(x, $\ast$) $\nrightarrow_1$ \\
El$_{N1}$(x, $\ast$) non \`e uguale definizionalmente.
\\\\\\
\noindent
\textbf{3}\\
\scriptsize
\begin{prooftree}
\AxiomC{[ ] cont}
\LeftLabel{F-S}
\UnaryInfC{N$_1$ type[ ]}
\LeftLabel{F-c}\RightLabel{(y $\in$ N$_1$) $\notin$ [ ]}
\UnaryInfC{y $\in$ N$_1$ cont}
\LeftLabel{I-S}
\UnaryInfC{$\ast$ $\in$ N$_1$[y $\in$ N$_1$]}
\AxiomC{[ ] cont}
\LeftLabel{F-S}
\UnaryInfC{N$_1$ type[ ]}
\LeftLabel{F-c}\RightLabel{(y $\in$ N$_1$) $\notin$ [ ]}
\UnaryInfC{y $\in$ N$_1$  cont}
\LeftLabel{F-S}
\UnaryInfC{N$_1$ type[y $\in$ N$_1$]}
\LeftLabel{F-c}\RightLabel{(z $\in$ N$_1$) $\notin$ y $\in$ N$_1$} 
\UnaryInfC{y $\in$ N$_1$, z $\in$ N$_1$ cont}
\LeftLabel{F-S}
\UnaryInfC{N$_1$ type[y $\in$ N$_1$, z $\in$ N$_1$]}
\AxiomC{[ ] cont}
\LeftLabel{F-S}
\UnaryInfC{N$_1$ type[ ]}
\LeftLabel{F-c}\RightLabel{(y $\in$ N$_1$) $\notin$ [ ]}
\UnaryInfC{y $\in$ N$_1$ cont}
\LeftLabel{var}
\UnaryInfC{y $\in$ N$_1$[y $\in$ N$_1$]}
\LeftLabel{E-S}
\TrinaryInfC{El$_{N1}$($\ast$, y) $\in$ N$_1$[y $\in$ N$_1$]}
\end{prooftree}

\noindent
\normalsize
Applicando la \textit{$\beta$ N$_1$-red} allora El$_{N1}$($\ast$, y) $\rightarrow_1$ y \\
El$_{N1}$($\ast$, y) \`e uguale definizionalmente.\\\\
\noindent
\textbf{4}\\\\
\scriptsize
\begin{adjustwidth}{-14em}{}
\begin{prooftree}
\AxiomC{[ ] cont}
\LeftLabel{F-S}
\UnaryInfC{N$_1$ type [ ]}
\LeftLabel{F-c}\RightLabel{(x $\in$ N$_1$) $\notin$ [ ]}
\UnaryInfC{x $\in$ N$_1$ cont}
\LeftLabel{F-S}
\UnaryInfC{N$_1$ type[x $\in$ N$_1$]}
\LeftLabel{F-c}\RightLabel{\begin{tabular}[c]{cc}(y $\in$ N$_1$) \\ $\notin$ x $\in$ N$_1$\end{tabular}}
\UnaryInfC{x $\in$ N$_1$, y $\in$ N$_1$ cont}
\LeftLabel{var}
\UnaryInfC{x $\in$ N$_1$[x $\in$ N$_1$, y $\in$ N$_1$]}
\AxiomC{[ ] cont}
\LeftLabel{F-S}
\UnaryInfC{N$_1$ type[ ]}
\LeftLabel{F-c}\RightLabel{(x $\in$ N$_1$) $\notin$ [ ]}
\UnaryInfC{x $\in$ N$_1$ cont}
\LeftLabel{F-S}
\UnaryInfC{N$_1$ type[x $\in$ N$_1$]}
\LeftLabel{F-c}\RightLabel{\begin{tabular}[c]{cc}(y $\in$ N$_1$) \\ $\notin$ (x $\in$ N$_1$)\end{tabular}}
\UnaryInfC{x $\in$ N$_1$, y $\in$ N$_1$ cont}
\LeftLabel{F-S}
\UnaryInfC{N$_1$ type[x $\in$ N$_1$, y $\in$ N$_1$]}
\LeftLabel{F-c}\RightLabel{\begin{tabular}[c]{cc}(z $\in$ N$_1$) $\notin$ \\ (x $\in$ N$_1$, y $\in$ N$_1$)\end{tabular}}
\UnaryInfC{x $\in$ N$_1$, y $\in$ N$_1$, z $\in$ N$_1$ cont}
\LeftLabel{F-S}
\UnaryInfC{N$_1$ type[x $\in$ N$_1$, y $\in$ N$_1$, z $\in$ N$_1$]}
\AxiomC{[ ] cont}
\LeftLabel{F-S}
\UnaryInfC{N$_1$ type [ ]}
\LeftLabel{F-c}\RightLabel{(x $\in$ N$_1$) $\notin$ [ ]}
\UnaryInfC{x $\in$ N$_1$ cont}
\LeftLabel{F-S}
\UnaryInfC{N$_1$ type[x $\in$ N$_1$]}
\LeftLabel{F-c}\RightLabel{\begin{tabular}[c]{cc}(y $\in$ N$_1$) \\ $\notin$ x $\in$ N$_1$\end{tabular}}
\UnaryInfC{x $\in$ N$_1$, y $\in$ N$_1$ cont}
\LeftLabel{var}
\UnaryInfC{y $\in$ N$_1$[x $\in$ N$_1$, y $\in$ N$_1$]}
\LeftLabel{E-S}
\TrinaryInfC{El$_{N1}$(x, y) $\in$ N$_1$[x $\in$ N$_1$, y $\in$ N$_1$]}
\end{prooftree}
\end{adjustwidth}
\noindent
\\
\normalsize
Applicando la \textit{$\beta$ N$_1$-red} allora El$_{N1}$(x, y)$\nrightarrow_1$ \\
El$_{N1}$(x, y) non \`e uguale definizionalmente.
\\\\\\
\noindent
\textbf{5}\\
\small
\begin{adjustwidth}{-9em}{}
\begin{prooftree}
\AxiomC{\textbf{5$_A$}}
\UnaryInfC{El$_{N1}$($\ast$, y) $\in$ N$_1$[y $\in$ N$_1$, x $\in$ N$_1$]}
\AxiomC{\textbf{5$_B$}}
\UnaryInfC{N$_1$ type[y $\in$ N$_1$, x $\in$ N$_1$, z $\in$ N$_1$]}
\AxiomC{\textbf{5$_C$}}
\UnaryInfC{El$_{N1}$(x, $\ast$) $\in$ N$_1$[y $\in$ N$_1$, x $\in$ N$_1$]}
\LeftLabel{E-S}
\TrinaryInfC{El$_{N1}$(El$_{N1}$($\ast$, y), El$_{N1}$(x, $\ast$)) $\in $ N$_1$[y $\in$ N$_1$, x $\in$ N$_1$]}
\end{prooftree}
\end{adjustwidth}

\vspace{0.5cm}
\textbf{5$_A$}
\small
\begin{prooftree}
\AxiomC{}
\UnaryInfC{El$_{N1}$($\ast$, y) $\in$ N$_1$[y $\in$ N$_1$]}
\AxiomC{[ ] cont}
\LeftLabel{F-S}
\UnaryInfC{N$_1$ type[ ]}
\LeftLabel{F-c}\RightLabel{(y $\in$ N$_1$) $\notin$ [ ]}
\UnaryInfC{y $\in$ N$_1$ cont}
\LeftLabel{F-S}
\UnaryInfC{N$_1$ type[y $\in$ N$_1$]}
\LeftLabel{F-c}\RightLabel{(x $\in$ N$_1$) $\notin$ y $\in$ N$_1$}
\UnaryInfC{y $\in$ N$_1$, x $\in$ N$_1$ cont}
\LeftLabel{ind-te}
\BinaryInfC{El$_{N1}$($\ast$, y) $\in$ N$_1$[y $\in$ N$_1$, x $\in$ N$_1$]}
\end{prooftree}

\vspace{0.5cm}
\textbf{5$_B$}
\small
\begin{prooftree}
\AxiomC{[ ] cont}
\LeftLabel{F-S}
\UnaryInfC{N$_1$ type[ ]}
\LeftLabel{F-c}\RightLabel{(y $\in$ N$_1$) $\notin$ [ ]}
\UnaryInfC{y $\in$ N$_1$ cont}
\LeftLabel{F-S}
\UnaryInfC{N$_1$ type[y $\in$ N$_1$]}
\LeftLabel{F-c}\RightLabel{(x $\in$ N$_1$) $\notin$ y $\in$ N$_1$}
\UnaryInfC{y $\in$ N$_1$, x $\in$ N$_1$ cont}
\LeftLabel{F-S}
\UnaryInfC{N$_1$ type[y $\in$ N$_1$, x $\in$ N$_1$]}
\LeftLabel{F-c}\RightLabel{(z $\in$ N$_1$) $\notin$ (y $\in$ N$_1$, x $\in$ N$_1$)}
\UnaryInfC{y $\in$ N$_1$, x $\in$ N$_1$, z $\in$ N$_1$ cont}
\LeftLabel{F-S}
\UnaryInfC{N$_1$ type[y $\in$ N$_1$, x $\in$ N$_1$, z $\in$ N$_1$]}
\end{prooftree}

\newpage
\textbf{5$_C$}
\scriptsize
\begin{adjustwidth}{-3em}{}
\begin{prooftree}
\AxiomC{}
\UnaryInfC{El$_{N1}$(x, $\ast$) $\in$ N$_1$[x $\in$ N$_1$]}
\AxiomC{[ ] cont}
\LeftLabel{F-S}
\UnaryInfC{N$_1$ type [ ]}
\LeftLabel{F-c}\RightLabel{(x $\in$ N$_1$) $\notin$ [ ]}
\UnaryInfC{x $\in$ N$_1$ cont}
\LeftLabel{F-S}
\UnaryInfC{N$_1$ type[x $\in$ N$_1$]}
\LeftLabel{F-c}\RightLabel{(y $\in$ N$_1$) $\notin$ x $\in$ N$_1$}
\UnaryInfC{x $\in$ N$_1$, y $\in$ N$_1$ cont}
\LeftLabel{ind-te}
\BinaryInfC{El$_{N1}$(x, $\ast$) $\in$ N$_1$[x $\in$ N$_1$, y $\in$ N$_1$]}
\AxiomC{[ ] cont}
\LeftLabel{F-S}
\UnaryInfC{N$_1$ type [ ]}
\LeftLabel{F-c}\RightLabel{(y $\in$ N$_1$) $\notin$ [ ]}
\UnaryInfC{y $\in$ N$_1$ count}
\LeftLabel{F-S}
\UnaryInfC{N$_1$ type[y $\in$ N$_1$]}
\LeftLabel{F-c}\RightLabel{(x $\in$ N$_1$) $\notin$ y $\in$ N$_1$}
\UnaryInfC{y $\in$ N$_1$, x $\in$ N$_1$ cont}
\LeftLabel{ex-te}
\BinaryInfC{El$_{N1}$(x, $\ast$) $\in$ N$_1$[y $\in$ N$_1$, x $\in$ N$_1$]}
\end{prooftree}
\end{adjustwidth}
\noindent
\\
\normalsize Per le premesse \textit{El$_{N1}$($\ast$, y) $\in$ N$_1$[y $\in$ N$_1$]} e \textit{El$_{N1}$(x, $\ast$) $\in$ N$_1$[x $\in$ N$_1$]} ho gi\`a dimostrato sopra (in 3 e 2) la loro tipabilit\`a per il tipo singoletto.

\noindent
\normalsize
Applicando la \textit{red$_I$} e \textit{$\beta$ N$_1$-red} allora El$_{N1}$(El$_{N1}$($\ast$, y), El$_{N1}$(x, $\ast$)) $\rightarrow_1$ El$_{N1}$(y, El$_{N1}$(x, $\ast$)). \\
Pi\`u nel dettaglio la riduzione \`e la seguente
\small
\begin{prooftree}
\AxiomC{}
\LeftLabel{$\beta$ N$_1$-red}
\UnaryInfC{El$_{N1}$($\ast$, y) $\rightarrow_1$ y}
\LeftLabel{red$_I$}
\UnaryInfC{El$_{N1}$(El$_{N1}$($\ast$, y), El$_{N1}$(x, $\ast$))$\rightarrow_1$ El$_{N1}$(y, El$_{N1}$(x, $\ast$))}
\end{prooftree}
\normalsize El$_{N1}$(El$_{N1}$($\ast$, y), El$_{N1}$(x, $\ast$)) \`e uguale definizionalmente.

\paragraph{4)} \textbf{Dati i termini definiti dalla seguente grammatica relativa ai termini del tipo singoletto
\begin{center} t $\equiv$ v $|$ $\ast$ $|$ El$_{N1}$(t$_1$, t$_2$)\end{center}
con v $\in$ \{x,y,w,z\} $\bigcup$ \{x$_i$ $|$ i $\in$ Nat\}, ovvero considerando come variabili le ultime lettere dell'alfabeto inglese e poi tutte le variabili ottenute ponendo alla variabili x un indice che varia nei numeri naturali.\\
Sia $\rightarrow_1$ una relazione binaria tra questi termini \textit{untyped} definita a partire dalle seguenti regole
\begin{center} $\beta_{N1})$ red \qquad El$_{Nat}$($\ast$,c) $\rightarrow_1$ c\end{center} 
}
\textbf{
\AxiomC{t$_1$ $\rightarrow$ t$_2$}
\LeftLabel{red$_I$)}
\UnaryInfC{El$_{N1}$(t$_1$, c) $\rightarrow$ $_1$ El$_{N1}$(t$_2$, c)}
\DisplayProof \qquad
\AxiomC{c$_1$ $\rightarrow$ c$_2$}
\LeftLabel{red$_{II}$)}
\UnaryInfC{El$_{N1}$(t, c$_1$) $\rightarrow$ $_1$ El$_{N1}$(t, c$_2$)}
\DisplayProof
\noindent
\begin{itemize}
\item Costruire l'albero dei cammini (ovvero sequenze) di passi di riduzione possibili fino a un termine in forma normale, ovvero non ulteriormente riducibile rispetto alla relazione $\rightarrow_1$ del termine
\begin{center}El$_{N_1}$(El$_{N1}$($\ast$, $\ast$), El$_{N1}$($\ast$, x)) \end{center}
\item Produrre un infinit\`a di termini del tipo singoletto che non sono riducibili secondo la relazione di un passo di riduzione$\rightarrow_1$.\\
Dati due di questi termini , si riesce a dire che sono definionalmente uguali secondo le regole del tipo singoletto?
\end{itemize}
}  

\noindent
\textbf{Soluzione}\\\\
\textit{Idea: uso un albero di derivazione per mostrare ogni passo di derivazione di ogni cammino.}\\
Se w $=$ El$_{N1}$(El$_{N1}$($\ast$, $\ast$), El$_{N1}$($\ast$, x)) combino il lambda termine w  con l'applicazione della strategia deterministica di riduzione ($\rightarrow_1$), con la quale il termine si riduce eventualmente a forma normale (implicando la definizione di riducibilit\`a).\\
$\beta_{N1}$-red:\\
\textit{El$_{N1}$($\ast$, $\ast$) $\rightarrow_1$ $\ast$}\\
\textit{El$_{N1}$($\ast$, x) $\rightarrow_1$ x}

\begin{prooftree}
\AxiomC{x}
\LeftLabel{$\beta$-N$_1$ red}
\UnaryInfC{El$_{N1}$($\ast$, x)}
\AxiomC{x}
\RightLabel{$\beta$-N$_1$ red}
\UnaryInfC{El$_{N1}$($\ast$, x)}
\LeftLabel{$\beta$-N$_1$ red}\RightLabel{red$_{II}$}
\BinaryInfC{El$_{NI}$($\ast$, El$_{N1}$($\ast$, x))}
\AxiomC{x}
\RightLabel{$\beta$-N$_1$ red}
\UnaryInfC{El$_{N1}$($\ast$, x)}
\RightLabel{red$_I$}
\UnaryInfC{El$_{N1}$(El$_{N1}$($\ast$,$\ast$), x)}
\LeftLabel{red$_{I}$}\RightLabel{red$_{II}$}
\BinaryInfC{El$_{N1}$(El$_{N1}$($\ast$, $\ast$), El$_{N1}$($\ast$, x))}
\end{prooftree}
\noindent
 $\Rightarrow$ (w, (red$_{II}$, red$_I$, $\beta_{N1}$-red)) rappresenta un programma.\\\\
\noindent
Termine t non pi\`u riducibile significa che \`e un termine \textit{untyped} che \`e in forma normale, perch\`e non esiste alcun altro termine s tale che t $\rightarrow_1$ s. Dunque l'infinit\`a di termini singoletto, non pi\`u riducibili rispettano la definizione data sopra corrisponde a\\
\begin{center}
t $\equiv$
$
\begin{cases}
v \\
\ast \\
El_{N1}(t_1, t_2)
\end{cases}
$
\end{center}
\noindent Dati due termini t$^\backprime$ e t$^{\backprime\backprime}$ termini \textit{untyped} non riesco a dire che sono definizionalmente uguali perch\`e gi\`a in forma normale. Difatti,  per il teorema della forma normale forte, vale $\rightarrow_0$.






















