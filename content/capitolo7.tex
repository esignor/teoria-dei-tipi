\chapter{Tipo somma indiciata forte}
\label{cap: indexed-sum-type}
%17 inclusa
\normalsize
La somma indiciata forte \`e il potenziamento indiciato della somma disgiunta binaria (\S\ref{cap: tipo-somma-disgiunta}). Tipo induttivo, ovvero generato con il principio d'induzione della regola di eliminazione, di tipi dipendente.\\\\
\noindent
\textit{Definizione set-teorica}\\
\noindent
\textbf{
\begin{center}$\bigcup\limits_{x \epsilon B\hspace{0.1cm}set}^.$C(x) \qquad (C(x) set) x $\in$ B \end{center}
$\neq$ \quad $\bigcup$ \{y : $\exists$ x $\in$ B \quad y $\in$ C(x)\}\\
\begin{center}${\overset{\mathit{def}}{=}}$ \{(b,c)$:$ b $\in$ B e c $\in$ C(b)\}\end{center}
}
\`E l'unione disgiunta di una famiglia di insiemi, definito dalle regole seguenti.

\section{Regole di Formazione}
\label{subsec: formazione-indexed-sum-type}
\begin{prooftree}
\AxiomC{C(x) type [$\Gamma$, x $\in$ B]}
\LeftLabel{F-$\Sigma$)}
\UnaryInfC{$\sum\limits_{x \in B}$ C(x) type[$\Gamma$]}
\end{prooftree}

\section{Regole di Introduzione}
\label{subsec: introduzione-indexed-sum-type}
\begin{prooftree}
\AxiomC{b $\in$ B[$\Gamma$]}
\AxiomC{c $\in$ C(b)[$\Gamma$]}
\AxiomC{$\sum\limits_{x \in B}$ C(x) type[$\Gamma$]}
\LeftLabel{I-$\Sigma$)}
\TrinaryInfC{$<$b,c$>$ $\in$ $\sum\limits_{x\in B}$ C(x) type[$\Gamma$]}
\end{prooftree}

\section{Regole di Eliminazione}
\label{subsec: eliminazione-indexed-sum-type}
\begin{adjustwidth}{-4em}{}
\begin{prooftree}
\AxiomC{M(z) type[$\Gamma$, z $\in$ $\sum\limits_{x \in B}$ C(x)]}
\AxiomC{\begin{tabular}[c]{cc}t $\in$ $\sum\limits_{x \in B}$ C(x)[$\Gamma$] \\ e(x,y) $\in$ M($<$x,y$>$)[$\Gamma$, x $\in$ B, y $\in$ C(x)]\end{tabular}}
\LeftLabel{E-$\Sigma$)}
\BinaryInfC{El$_\Sigma$(t,(x,y).e(x,y)) $\in$ M(t)[$\Gamma$]}
\end{prooftree}
\end{adjustwidth}

\section{Regole di Conservazione}
\label{subsec: conversione-indexed-sum-type}
\small
\begin{adjustwidth}{-4em}{}
\begin{prooftree}
\AxiomC{M(z) type[$\Gamma$, z $\in$ $\sum\limits_{x \in B}$ C(x)]}
\AxiomC{\begin{tabular}[c]{cc}b $\in$ B[$\Gamma$] \\ c $\in$ C(b)\end{tabular}}
\AxiomC{e(x,y) $\in$ M($<$x,y$>$)[$\Gamma$, x $\in$ B, y $\in$ C(x)]}
\LeftLabel{C-$\Sigma$)}
\TrinaryInfC{El$_\Sigma$($<$b,c$>$,e) $=$ e(b,c) $\in$ M($<$b,c$>$)[$\Gamma$]}
\end{prooftree}
\end{adjustwidth}

\section{Regole di Uguaglianza}
\label{subsec: uguaglianza-indexed-sum-type}
\small
\begin{prooftree}
\AxiomC{B$_1$ $=$ B$_2$ type[$\Gamma$]}
\AxiomC{C$_1$(x) $=$ C$_2$(x) type[$\Gamma$, x $\in$ B$_1$]}
\LeftLabel{eq-F-$\Sigma$)}
\BinaryInfC{$\sum\limits_{x \in B_1}$ C$_1$(x) $=$ $\sum\limits_{x \in B_2}$ C$_2$(x) type[$\Gamma$]}
\end{prooftree}
\begin{prooftree}
\AxiomC{$\sum\limits_{x \in B}$ C(x) type[$\Gamma$]}
\AxiomC{b$_1$ $=$ b$_2$ $\in$ B[$\Gamma$]}
\AxiomC{c$_1$ $=$ c$_2$ $\in$ C(b$_1$)[$\Gamma$]}
\LeftLabel{eq-I-$\Sigma$)}
\TrinaryInfC{$<$b$_1$,c$_1$$>$ $=$ $<$b$_2$,c$_2$$>$ $\in$ $\sum\limits_{x \in B}$ C(x) type[$\Gamma$]}
\end{prooftree}
\begin{adjustwidth}{-8em}{}
\begin{prooftree}
\AxiomC{M(z) type[$\Gamma$, z $\in$ $\sum\limits_{x \in B}$ C(x)]}
\AxiomC{\begin{tabular}[c]{cc}t$_1$ $=$ t$_2$ $\in$ $\sum\limits_{x \in B}$ C(x)[$\Gamma$] \\ e$_1$(x,y) $=$ e$_2$(x,y) $\in$ M($<$x,y$>$)[$\Gamma$, x $\in$ B, y $\in$ C(x)]\end{tabular}}
\LeftLabel{eq-E-$\Sigma$)}
\BinaryInfC{El$_\Sigma$(t$_1$,e$_1$) $=$ El$_\Sigma$(t$_2$,e$_2$) $\in$ M(t$_1$)[$\Gamma$]}
\end{prooftree}
\end{adjustwidth}
\normalsize

\section{Eliminatore dipendente}
\label{sec:eliminatore dipendente-indexed-sum-type}
L'eliminatore ha anche la forma dipendente.
\small
\begin{prooftree}
\AxiomC{M(z) type[$\Gamma$, z $\in$ $\sum\limits_{x \in B}$ C(x)]}
\AxiomC{e(x,y) $\in$ M($<$x,y$>$)[$\Gamma$, x $\in$ B, y $\in$ C(x)]}
\LeftLabel{E-$\Sigma_{dip}$)}
\BinaryInfC{El$_\Sigma$(z,(x,y).e(x,y)) $\in$ M(z)[$\Gamma$, z $\in$ $\sum\limits_{x \in B}$ C(x)]}
\end{prooftree}
\normalsize

\section{Semantica operazionale della somma indiciata forte}
\label{subsec: semantica-operazionale-indexed-sum-type}
La relazione $\rightarrow_1$ viene definita all'interno dei termini con l'uso delle seguenti regole di riduzione:
\begin{itemize}
\item $\beta_{\Sigma}$-red) El$_\Sigma$($<$b,c$>$, e) $\rightarrow_1$ e(b,c)
\item \AxiomC{t$_1$ $\rightarrow_1$ t$_2$}
\LeftLabel{$\Sigma$-red)}
\UnaryInfC{El$_\Sigma$ (t$_1$, e) $\rightarrow_1$ El$_\Sigma$ (t$_2$, e)}
\DisplayProof \qquad
\item novit\`a della somma indiciata forte rispetto al tipo singoletto\\
\AxiomC{b$_1$ $\rightarrow_1$ b$_2$}
\LeftLabel{$\Sigma$-red$_I$)}
\UnaryInfC{$<$b$_1$, c$>$ $\rightarrow_1$ $<$b$_2$, c$>$}
\DisplayProof
\item \AxiomC{c$_1$ $\rightarrow_1$ c$_2$}
\LeftLabel{$\Sigma$-red$_{II}$)}
\UnaryInfC{$<$b, c$_1>$ $\rightarrow_1$ $<$b, c$_2>$}
\DisplayProof
\end{itemize}

\section{Esercizi}
\label{sec: es-indixed-sum-type}
\paragraph{1)}
\textbf{Provare a scrivere le regole del prodotto cartesiano A $\times$ B di un tipo A con un tipo B.}\\\\
\noindent
\textbf{Svolgimento}\\\\
\noindent
A $\times$ B ${\overset{\mathit{def}}{=}}$ $\sum\limits_{x \in A}$ B\\\\
\noindent

\begin{itemize}
\item \textbf{Regole del tipo prodotto cartesiano:}
\begin{itemize}
\item \textit{Regole di Formazione}
\begin{prooftree}
\AxiomC{A type [$\Gamma$]}
\AxiomC{B type [$\Gamma$]}
\LeftLabel{F-x)} 
\BinaryInfC{A $\times$ B type[$\Gamma$]}
\end{prooftree}

\item \textit{Regole di Introduzione}
\begin{prooftree}
\AxiomC{a $\in$ A[$\Gamma$]}
\AxiomC{b $\in$ B[$\Gamma$]}
\LeftLabel{I-x)}
\BinaryInfC{$<$a,b$>$ $\in$ A $\times$ B[$\Gamma$]}
\end{prooftree}

\item \textit{Regole di Proiezione}
\begin{prooftree}
\AxiomC{d $\in$ A $\times$ B[$\Gamma$]}
\LeftLabel{PJ$_1$-x)}
\UnaryInfC{$\pi_1$d $\in$ A[$\Gamma$]}
\end{prooftree}
\begin{prooftree}
\AxiomC{d $\in$ A $\times$ B[$\Gamma$]}
\LeftLabel{PJ$_2$-x)}
\UnaryInfC{$\pi_2$d $\in$ B[$\Gamma$]}
\end{prooftree}
\item \textit{Regole di Uguaglianza delle proiezioni}
\begin{prooftree}
\AxiomC{a $\in$ A[$\Gamma$]}
\AxiomC{b $\in$ B[$\Gamma$]}
\LeftLabel{PJ$_1$-eq)}
\BinaryInfC{$\pi_1$($<$a,b$>$) $=$ a $\in$ A[$\Gamma$]}
\end{prooftree}
\begin{prooftree}
\AxiomC{a $\in$ A[$\Gamma$]}
\AxiomC{b $\in$ B[$\Gamma$]}
\LeftLabel{PJ$_2$-eq)}
\BinaryInfC{$\pi_2$($<$a,b$>$) $=$ b $\in$ B[$\Gamma$]}
\end{prooftree}
\end{itemize}
\item \textbf{Semantica operazionale del tipo prodotto cartesiano:}
\begin{itemize}
\item $\beta_x$-red$_1$) El$_\Sigma$($<$a,b$>$, e) $\rightarrow_1$ e(a,b)
\item $\beta_x$-red$_2$) $\pi_1$($<$a,b$>$) $\rightarrow_1$ a
\item $\beta_x$-red$_3$) $\pi_1$($<$a,b$>$) $\rightarrow_1$ b
\item \AxiomC{t$_1$ $\rightarrow_1$ t$_2$}
\LeftLabel{x-red)}
\UnaryInfC{El$_\Sigma$ (t$_1$, e) $\rightarrow_1$ El$_\Sigma$ (t$_2$, e)}
\DisplayProof \qquad
\item \AxiomC{a$_1$ $\rightarrow_1$ a$_2$}
\LeftLabel{x-red$_I$)}
\UnaryInfC{$<$a$_1$, b$>$ $\rightarrow_1$ $<$a$_2$, b$>$}
\DisplayProof
\item \AxiomC{b$_1$ $\rightarrow_1$ b$_2$}
\LeftLabel{x-red$_{II}$)}
\UnaryInfC{$<$a, b$_1>$ $\rightarrow_1$ $<$a, b$_2>$}
\DisplayProof
\end{itemize}
\end{itemize}

\paragraph{2)}
\textbf{Provare a scrivere le regole del tipo delle funzioni A $\rightarrow$ B[$\Gamma$] da un tipo A a un tipo B.}
\\\\
\noindent
\textbf{Svolgimento}\\\\

\begin{itemize}
\item \textbf{Regole del tipo delle funzioni:}
\begin{itemize}
\item \textit{Regole di Formazione}
\begin{prooftree}
\AxiomC{A type [$\Gamma$]}
\AxiomC{B type [$\Gamma$]}
\LeftLabel{F-$\rightarrow$)}
\BinaryInfC{A $\rightarrow$ B type[$\Gamma$]}
\end{prooftree}

\item \textit{Regole di Introduzione}
\begin{prooftree}
\AxiomC{b(x) $\in$ B[$\Gamma$,x$\in$A]}
\LeftLabel{I-$\rightarrow$)}
\UnaryInfC{$\lambda$x$^A$.b(x) $\in$ A $\rightarrow$ B[$\Gamma$]}
\end{prooftree}

\item \textit{Regole di Eliminazione}
\begin{prooftree}
\AxiomC{f $\in$ A$\rightarrow$B[$\Gamma$]}
\AxiomC{a $\in$ A[$\Gamma$]}
\LeftLabel{E-$\rightarrow$)}
\BinaryInfC{Ap(f,a) $\in$ B[$\Gamma$]}
\end{prooftree}

\item \textit{Regole di Conversione}
\begin{prooftree}
\AxiomC{b(x) $\in$ B[$\Gamma$, x $\in$ A]}
\AxiomC{a $\in$ A[$\Gamma$]}
\LeftLabel{C-$\rightarrow$)}
\BinaryInfC{Ap($\lambda$x$^A$.b(x),a) $=$ b(a) $\in$ B[$\Gamma$]}
\end{prooftree}

\item \textit{Regole di Uguaglianza}
\begin{prooftree}
\AxiomC{b$_1$(x) $=$ b$_2$(x) $\in$ B[$\Gamma$, x $\in$ A]}
\LeftLabel{eq-I-$\rightarrow$)}
\UnaryInfC{$\lambda$x$^A$.b$_1$(x) $=$ $\lambda$x$^A$.b$_2$(x) $\in$ A $\rightarrow$ B[$\Gamma$]}
\end{prooftree}
\begin{prooftree}
\AxiomC{f$_1$ $=$ f$_2$ $\in$ A$\rightarrow$B[$\Gamma$]}
\AxiomC{a$_1$ $=$ a$_2$ $\in$ A[$\Gamma$]}
\LeftLabel{eq-E-$\rightarrow$)}
\BinaryInfC{Ap(f$_1$,a$_1$) $=$ Ap(f$_2$,a$_2$) $\in$ B[$\Gamma$]}
\end{prooftree}
\end{itemize}

\item \textbf{Semantica operazionale del tipo prodotto cartesiano:}
\begin{itemize}
\item $\beta_x$-red) Ap($\lambda$x$^A$.b(x), a) $\rightarrow_1$ b(a)
\item \AxiomC{f$_1$ $\rightarrow_1$ f$_2$}
\LeftLabel{$\rightarrow$-red$_1$)}
\UnaryInfC{Ap(f$_1$, a) $\rightarrow_1$ Ap(f$_2$, a)}
\DisplayProof \qquad
\item \AxiomC{a$_1$ $\rightarrow_1$ a$_2$}
\LeftLabel{$\rightarrow$-red$_2$)}
\UnaryInfC{Ap(f, a$_1$) $\rightarrow_1$ Ap(f, a$_2$)}
\DisplayProof
\item \AxiomC{b$_1$ $\rightarrow_1$ b$_2$}
\LeftLabel{$\rightarrow$-red)}
\UnaryInfC{$\lambda$x$^A$.b$_1$ $\rightarrow$ $\lambda$x$^A$.b$_2$}
\DisplayProof
\end{itemize}
\end{itemize}

\section{Rappresentazione del tipo prodotto cartesiano}
\label{sec: rappresentazione-del-tipo-prodotto-cartesiano}
Il \textbf{tipo somma indiciata forte} \`e un potenziamento con tipi dipendenti del tipo prodotto cartesiano.\\
Allora ripartendo dal concetto di somma indiciata forte \\
\noindent
\begin{center}$\bigcup\limits_{x \epsilon B\hspace{0.1cm}set}^.$C(x) \qquad (C(x) set) x $\in$ B ${\overset{\mathit{def}}{=}}$ \{(b,c)$:$ b $\in$ B e c $\in$ C(b)\}\end{center}
Se
\noindent C(x) $\equiv$ C$^\backprime$ dove C$^\backprime$ non varia \quad $\Rightarrow$ \quad ${\overset{\mathit{def}}{=}}$ \{(b,c)$:$ b$\in$ B e c $\in$ C(b)\} $\simeq$ B $\times$
C$^\backprime$ \\\\
\noindent
\begin{center} B $\times$ C ${\overset{\mathit{def}}{=}}$ $\bigcup\limits_{x \epsilon B\hspace{0.1cm}set}^.$C\end{center}
\vspace{1cm}
\noindent
Supposto che 
\begin{prooftree}
\AxiomC{B type[$\Gamma$]}
\AxiomC{C type[$\Gamma$]}
\LeftLabel{ind-ty}
\UnaryInfC{C type[$\Gamma$, x $\in$ B]}
\LeftLabel{F-$\Sigma$}
\BinaryInfC{B $\times$ C ${\overset{\mathit{def}}{=}}$ $\sum\limits_{x \epsilon B}$C}
\end{prooftree}
\noindent
Il tipo prodotto cartesiano \`e definito dalle seguenti regole.

\subsection{Regole di Formazione}
\label{subsec: formazione-prod-cart}
\begin{prooftree}
\AxiomC{B type [$\Gamma$]}
\AxiomC{C type [$\Gamma$]}
\LeftLabel{F-x)}
\BinaryInfC{B $\times$ C type[$\Gamma$]}
\end{prooftree}

\subsection{Regole di Introduzione}
\label{subsec: introduzione-prod-cart}
\begin{prooftree}
\AxiomC{b $\in$ B[$\Gamma$]}
\AxiomC{c $\in$ C[$\Gamma$]}
\LeftLabel{I-x)}
\BinaryInfC{$<$b,c$>$ $\in$ B $\times$ C[$\Gamma$]}
\end{prooftree}

\subsection{Regole di Proiezione}
\label{subsec: proiezione-prod-cart}
\begin{prooftree}
\AxiomC{d $\in$ B $\times$ C[$\Gamma$]}
\LeftLabel{PJ$_1$-x)}
\UnaryInfC{$\pi_1$d $\in$ B[$\Gamma$]}
\end{prooftree}
\begin{prooftree}
\AxiomC{d $\in$ B $\times$ C[$\Gamma$]}
\LeftLabel{PJ$_2$-x)}
\UnaryInfC{$\pi_2$d $\in$ C[$\Gamma$]}
\end{prooftree}

\subsection{Regole di Uguaglianza delle proiezioni}
\label{subsec: ugualgianza-proiezioni-prod-cart}
\begin{prooftree}
\AxiomC{b $\in$ B[$\Gamma$]}
\AxiomC{c $\in$ C[$\Gamma$]}
\LeftLabel{PJ$_1$-eq)}
\BinaryInfC{$\pi_1$($<$b,c$>$) $=$ b $\in$ B[$\Gamma$]}
\end{prooftree}
\begin{prooftree}
\AxiomC{b $\in$ B[$\Gamma$]}
\AxiomC{c $\in$ C[$\Gamma$]}
\LeftLabel{PJ$_2$-eq)}
\BinaryInfC{$\pi_2$($<$b,c$>$) $=$ c $\in$ B[$\Gamma$]}
\end{prooftree}

\subsection{Semantica operazionale del prodotto cartesiano}
\label{subsec: semantica-operazionale-prod-cart}
\begin{itemize}
\item $\beta_x$-red$_1$) El$_\Sigma$($<$b,c$>$, e) $\rightarrow_1$ e(b,c)
\item $\beta_x$-red$_2$) $\pi_1$($<$b,c$>$) $\rightarrow_1$ b
\item $\beta_x$-red$_3$) $\pi_1$($<$b,c$>$) $\rightarrow_1$ c
\item \AxiomC{t$_1$ $\rightarrow_1$ t$_2$}
\LeftLabel{x-red)}
\UnaryInfC{El$_\Sigma$ (t$_1$, e) $\rightarrow_1$ El$_\Sigma$ (t$_2$, e)}
\DisplayProof \qquad
\item \AxiomC{b$_1$ $\rightarrow_1$ b$_2$}
\LeftLabel{x-red$_I$)}
\UnaryInfC{$<$b$_1$, c$>$ $\rightarrow_1$ $<$b$_2$, c$>$}
\DisplayProof
\item \AxiomC{c$_1$ $\rightarrow_1$ c$_2$}
\LeftLabel{x-red$_{II}$)}
\UnaryInfC{$<$b, c$_1>$ $\rightarrow_1$ $<$b, c$_2>$}
\DisplayProof
\end{itemize}
\noindent

\subsection{Correttezza del prodotto cartesiano}
\label{subsec:correttezza-del-prod-cart}
\textit{Dimostrazione}\\\\
\noindent
Dimostro che B $\times$ C ${\overset{\mathit{def}}{=}}$ $\bigcup\limits_{x \epsilon B\hspace{0.1cm}set}^.$C, trovando \textit{PJ$_1$-x} e \textit{PJ$_2$-x} usando E-$\Sigma_{dip}$.

\begin{center}
M(z) $\equiv$ B\\
\noindent
$\sum\limits_{x \epsilon B}$C $\equiv$ B $\times$ C
\end{center}
\noindent
\textit{Assumo derivabili B type[$\Gamma$] e C type[$\Gamma$]}
\begin{enumerate}
\item Sulla prima proiezione\\
\noindent

El$_\Sigma$(z,(x,y).x) $\equiv$ $\pi_1$z

\noindent
La prima proiezione sulla coppia canonica M($<$x,y$>$) \`e x\\\\
\noindent
\small
\begin{adjustwidth}{-14em}{}
\begin{prooftree}
\AxiomC{}
\UnaryInfC{B type[$\Gamma$]}

\AxiomC{}
\UnaryInfC{B type[$\Gamma$]}
\AxiomC{}
\UnaryInfC{C type[$\Gamma$]}
\LeftLabel{F-x}
\BinaryInfC{$\sum\limits_{x \epsilon B}$C type[$\Gamma$]}
\LeftLabel{F-c}\RightLabel{(z $\in$ $\sum\limits_{x \epsilon B}$C) $\notin$ $\Gamma$}
\UnaryInfC{$\Gamma$,z $\in$ $\sum\limits_{x \epsilon B}$C cont}
\LeftLabel{ind-ty}
\BinaryInfC{B type[$\Gamma$,z $\in$ $\sum\limits_{x \epsilon B}$C]}


\AxiomC{}
\UnaryInfC{C type[$\Gamma$]}
\AxiomC{}
\UnaryInfC{B type[$\Gamma$]}
\LeftLabel{F-c}\RightLabel{(x $\in$ B) $\notin$ $\Gamma$}
\UnaryInfC{$\Gamma$,x $\in$ B cont}
\LeftLabel{ind-ty}
\BinaryInfC{C type[$\Gamma$,x $\in$ B]}
\LeftLabel{F-c}\RightLabel{(y $\in$ C) $\notin$ ($\Gamma$, x $\in$ B)}
\UnaryInfC{$\Gamma$,x $\in$ B,y $\in$ C cont}
\LeftLabel{var}
\UnaryInfC{x $\in$ B[$\Gamma$,x $\in$ B,y $\in$ C]}
\LeftLabel{E-$\Sigma_{dip}$}
\BinaryInfC{$\pi_1$z $\in$ B[$\Gamma$,z $\in$ $\sum\limits_{x \epsilon B}$C]}
\end{prooftree}
\end{adjustwidth}
\noindent
\normalsize

\noindent
In conclusione
\begin{center}$\pi_1$z ${\overset{\mathit{def}}{=}}$ El$_\Sigma$(z,(x,y).x) $\in$ B[$\Gamma$,z $\in$ $\sum\limits_{x \epsilon B}$C]\end{center}

\item Sulla seconda proiezione\\
\noindent
El$_\Sigma$(z,(x,y).y) $\equiv$ $\pi_2$z

\noindent
La prima proiezione sulla coppia canonica M($<$x,y$>$) \`e y\\\\
\small
\begin{adjustwidth}{-3em}{}
\begin{prooftree}


\AxiomC{\textbf{($\ast$)}}
\UnaryInfC{B type[$\Gamma$, z $\in$ $\sum\limits_{x \epsilon B}$C]}

\AxiomC{}
\UnaryInfC{C type[$\Gamma$]}
\AxiomC{}
\UnaryInfC{B type[$\Gamma$]}
\LeftLabel{F-c}\RightLabel{(x $\in$ B) $\notin$ $\Gamma$}
\UnaryInfC{$\Gamma$,x $\in$ B cont}
\LeftLabel{ind-ty}
\BinaryInfC{C type[$\Gamma$,x $\in$ B]}
\LeftLabel{F-c}\RightLabel{(y $\in$ C) $\notin$ ($\Gamma$, x $\in$ B)}
\UnaryInfC{$\Gamma$,x $\in$ B,y $\in$ C cont}
\LeftLabel{var}
\UnaryInfC{y $\in$ C[$\Gamma$,x $\in$ B,y $\in$ C]}
\LeftLabel{E-$\Sigma_{dip}$}
\BinaryInfC{$\pi_2$z $\in$ C[$\Gamma$,z $\in$ $\sum\limits_{x \epsilon B}$C]}
\end{prooftree}
\end{adjustwidth}
\noindent
\normalsize

\noindent
In conclusione
\begin{center}$\pi_2$z ${\overset{\mathit{def}}{=}}$ El$_\Sigma$(z,(x,y).y) $\in$ C[$\Gamma$,z $\in$ $\sum\limits_{x \epsilon B}$C]\end{center}
\noindent
Ho concluso con ($\ast$) i sequenti gi\`a risolti, per evitare ripetizioni.\\
\end{enumerate}
Per verificare che effettivamente \textbf{$\pi_1$z} e \textbf{$\pi_2$z} siano \textit{PJ$_1$-x} e \textit{PJ$_2$-x}, utilizzo le riduzioni \textit{$\beta_x$-red$_1$} e \textit{$\beta_x$-red$_2$}.\\
La riduzione El$_\Sigma$($<$b,c$>$, e) $\rightarrow_1$ e(b,c) riscritta con sostituzioni diventa El$_\Sigma$($<$b,c$>$, e) $\rightarrow_1$ e[$\frac{x}{b}$,$\frac{y}{c}$]\\\\
\noindent
Allora applicata a una coppia $<$b,c$>$:
\begin{itemize}
\item  $\pi_1$($<$b,c$>$) ${\overset{\mathit{def}}{=}}$ El$_\Sigma$(z,(x,y).x) = El$_\Sigma$($<$b,c$>$,(x,y).x) $\rightarrow_1$ x[$\frac{x}{b}$, $\frac{y}{c}$] $\rightarrow$ b
\item $\pi_2$($<$b,c$>$) ${\overset{\mathit{def}}{=}}$ El$_\Sigma$(z,(x,y).y) = El$_\Sigma$($<$b,c$>$,(x,y).y) $\rightarrow_1$ y[$\frac{x}{b}$, $\frac{y}{c}$] $\rightarrow$ c
\end{itemize}
\noindent
Entrambe le \textit{$\beta_x$-red} sono verificate. Percui la definizione B $\times$ C ${\overset{\mathit{def}}{=}}$ $\sum\limits_{x \epsilon B}$C \`e giustificata.

\subsection{Esempi}
\label{subsec:esempi}
L'unione disgiunta non \`e l'unione insiemistica e questo lo si vede chiaramente dalla relazione sotto\\
\noindent
$\bigcup\limits_{x \epsilon Nat}^.$Nat $\equiv$ \{(x,y) $\bullet$ x $\in$ Nat,y $\in$ Nat\} $\equiv$ Nat $\times$ Nat $\neq$ $\bigcup\limits_{x \epsilon Nat}$Nat \{y: y $\in$ Nat, per un certo x $\in$ Nat\}\\
\noindent
La prima parte della relazione, se la famiglia \`e costante ingloba il prodotto cartesiano.\\\\
\noindent
Un esempio di unioni genuine, che si possono formare con tipi che non coincidono con il prodotto cartesiano, \`e $\sum\limits_{x \epsilon Nat}$ Mat(x $\bullet$ x), ove Mat sono le matrici quadratiche.\\
A livello insiemisitico potremmo dover proiettare, data una coppia $<$x,k$>$ (con x elementi e k una certa matrice), sull'indice o sull'elemento di cui stiamo parlando. Dunque dovremmo poter avere a disposizione le due proiezioni \textit{PJ$_1$-x} e \textit{PJ$_2$-x} anche su famiglie non costanti.\\\\
\noindent
Per individuare correttamente $\pi_1$(z) e $\pi_2$(z) assumo che $\sum\limits_{x \epsilon B}$C(x) type[$\Gamma$] sia derivabile. Tale ipotesi \`e vincolante per l'ottenimento della famiglia. \\\\
\noindent
Definisco $\omega$ = z $\in$ $\sum\limits_{x \epsilon B}$C(x)
\begin{enumerate}
\item Per la prima proiezione
\small
\begin{adjustwidth}{-8em}{}
\begin{prooftree}
\AxiomC{}
\UnaryInfC{B type[$\Gamma$]}
\AxiomC{}
%\UnaryInfC{C(x) type[$\Gamma$]}
%\AxiomC{($\ast$)}
%\UnaryInfC{$\Gamma$, x $\in$ B cont}
%\LeftLabel{ind-ty}
%\UnaryInfC{C(x) type[$\Gamma$, x $\in$ B]}
%\LeftLabel{F-$\Sigma$}
\UnaryInfC{$\sum\limits_{x \epsilon B}$C(x)type[$\Gamma$]}
\LeftLabel{F-c}\RightLabel{$\omega$ $\notin$ $\Gamma$}
\UnaryInfC{$\Gamma$, $\omega$ cont}
\LeftLabel{ind-ty}
\BinaryInfC{B type[$\Gamma$, $\omega$]}

\AxiomC{}
%\UnaryInfC{C(x) type[$\Gamma$]}
%\AxiomC{($\ast$)}
%\UnaryInfC{$\Gamma$,x $\in$ B cont}
%\LeftLabel{ind-ty}
\UnaryInfC{C(x) type[$\Gamma$,x $\in$ B]}
\LeftLabel{F-c}\RightLabel{(y $\in$ C(x)) $\notin$ ($\Gamma$,x $\in$ B)}
\UnaryInfC{$\Gamma$,x $\in$ B,y $\in$ C(x) cont}
\LeftLabel{var}
\UnaryInfC{x $\in$ B[$\Gamma$,x $\in$ B,y $\in$ C(x)]}
\LeftLabel{E-$\Sigma_{dip}$}
\BinaryInfC{El$_\Sigma$(z,(x,y).x) $\in$ B[$\Gamma$, $\omega$]}
\end{prooftree}
\end{adjustwidth}
\noindent
\normalsize
In conclusione
\begin{center}$\pi_1$z ${\overset{\mathit{def}}{=}}$ El$_\Sigma$(z,(x,y).x) $\in$ B[$\omega$]\end{center}

\item Per la seconda proiezione\\
\noindent
$\pi_2$z ${\overset{\mathit{def}}{=}}$ El$_\Sigma$(z,(x,y).y) $\in$ C($\pi_1(z)$)[$\omega$]\\\\
\noindent
Allora usando che M(z) $\equiv$ C($\pi_1$z) $\Rightarrow$ M($<$x,y$>$) $\equiv$ C($\pi_1<$x,y$>$)\\
\noindent
\begin{prooftree}
\AxiomC{\textbf{1}}
\UnaryInfC{C($\pi_1$z) type[$\Gamma$, $\omega$]}
\AxiomC{\textbf{2}}
\UnaryInfC{y $\in$ C($\pi_1<$x,y$>$)[$\Gamma$,x $\in$ B,y $\in$ C(x)]}
\LeftLabel{E-$\Sigma_{dip}$}
\BinaryInfC{El$_\Sigma$(z,(x,y).y) $\in$ C($\pi_1$z)[$\Gamma$, $\omega$]}
\end{prooftree}
\noindent
\normalsize
\vspace{0.5cm}
\textbf{\textbf{1}}
\small
\begin{adjustwidth}{-15em}{}
\begin{prooftree}
\AxiomC{($\ast$)}
\LeftLabel{E-$\Sigma_{dip}$}
\UnaryInfC{$\pi_1$z $\in$ B[$\Gamma$,$\omega$]}


\AxiomC{}
\UnaryInfC{C(x) type[$\Gamma$,x $\in$ B]}
\AxiomC{($\ast$)}
\UnaryInfC{$\Gamma$,x $\in$ B,$\omega$ cont}
\LeftLabel{ind-ty}
\BinaryInfC{C(x) type[$\Gamma$,x $\in$ B,$\omega$]}
\AxiomC{}
\UnaryInfC{B type[$\Gamma$]}
\AxiomC{}
\UnaryInfC{$\sum\limits_{x \epsilon B}$C(x) type[$\Gamma$]}
\LeftLabel{F-c}\RightLabel{$\omega$ $\notin$ $\Gamma$}
\UnaryInfC{$\Gamma$,$\omega$ cont}
\LeftLabel{ind-ty}
\BinaryInfC{B type[$\Gamma$,$\omega$]}
\LeftLabel{F-c}\RightLabel{(x $\in$ B) $\notin$ ($\Gamma$,$\omega$)}
\UnaryInfC{$\Gamma$,$\omega$, x $\in$ B cont}
\LeftLabel{ex-ty}
\BinaryInfC{C(x) type[$\Gamma$,$\omega$, x $\in$ B]}
\LeftLabel{sub-typ}
\BinaryInfC{C($\pi_1$z) type[$\Gamma$,$\omega$]}
\end{prooftree}
\end{adjustwidth}
\noindent
\normalsize
\vspace{0.5cm}
\textbf{\textbf{2}}
\small
\begin{adjustwidth}{-15em}{}
\begin{prooftree}
\AxiomC{($\ast$)}
\UnaryInfC{$\Gamma$,x $\in$ B,y $\in$ C(x) cont}
\LeftLabel{var}
\UnaryInfC{y $\in$ C(x)[$\Gamma$,x $\in$ B,y $\in$ C(x)]}

\AxiomC{\textbf{2$^A$}}
\UnaryInfC{x $=$ $\pi_1<$x,y$> \in$ B[$\Gamma$,x $\in$ B,y $\in$ C(x)]}

\AxiomC{\textbf{2$^B$}}
\UnaryInfC{C(w)type[$\Gamma$,x $\in$ B,y $\in$ C(x),w $\in$ B]}

\LeftLabel{sub-eq-typ}
\BinaryInfC{C(x) $=$ C($\pi_1<$x,y$>$)[$\Gamma$,x $\in$ B,y $\in$ C(x)]}
\LeftLabel{conv}
\BinaryInfC{y $\in$ C($\pi_1<$x,y$>$)[$\Gamma$,x $\in$ B,y $\in$ C(x)]}
\end{prooftree}
\end{adjustwidth}

\noindent
\normalsize
\vspace{0.5cm}
\textbf{\textbf{2$^A$}}
\small
\begin{adjustwidth}{-5em}{}
\begin{prooftree}
\AxiomC{($\ast$)}
\UnaryInfC{B type[$\Gamma$]}
\AxiomC{($\ast$)}
\UnaryInfC{x $\in$ B[$\Gamma$]}
\AxiomC{($\ast$)}
\UnaryInfC{y $\in$ C(x)}
\AxiomC{($\ast$)}
\UnaryInfC{x $\in$ B[$\Gamma$, x $\in$ B, y $\in$ C(x)]}
\LeftLabel{C-$\Sigma$}
\QuaternaryInfC{$\pi_1<$x,y$> $=$ x \in$ B[$\Gamma$]}
\AxiomC{($\ast$)}
\UnaryInfC{$\Gamma$,x $\in$ B,y $\in$ C(x) cont}
\LeftLabel{ind-ty}
\BinaryInfC{$\pi_1<$x,y$> $=$ x \in$ B[$\Gamma$,x $\in$ B,y $\in$ C(x)]}
\LeftLabel{sym}
\UnaryInfC{x $=$ $\pi_1<$x,y$> \in$ B[$\Gamma$,x $\in$ B,y $\in$ C(x)]}
\end{prooftree}
\end{adjustwidth}

\noindent
\normalsize
\vspace{0.5cm}
\textbf{\textbf{2$^B$}}
\small
\begin{adjustwidth}{-5em}{}
\begin{prooftree}
\AxiomC{}
\UnaryInfC{C(x)type[$\Gamma$,x $\in$ B]}
\AxiomC{($\ast$)}
\UnaryInfC{$\Gamma$,x $\in$ B cont}
\LeftLabel{ind-ty}
\BinaryInfC{C(x)type[$\Gamma$,x $\in$ B]}
\LeftLabel{$\alpha$-eq}
\UnaryInfC{C(w)type[$\Gamma$,w $\in$ B]}
\AxiomC{($\ast$)}
\UnaryInfC{$\Gamma$,w $\in$ B,x $\in$ B,y $\in$ C(x) cont}
\LeftLabel{ind-ty}
\BinaryInfC{C(w)type[$\Gamma$,w $\in$ B,x $\in$ B,y $\in$ C(x)]}
\AxiomC{($\ast$)}
\UnaryInfC{$\Gamma$,x $\in$ B,y $\in$ C(x),w $\in$ B cont}
\LeftLabel{ex-ty}
\BinaryInfC{C(w)type[$\Gamma$,x $\in$ B,y $\in$ C(x),w $\in$ B]}
\end{prooftree}
\end{adjustwidth}

\noindent
\normalsize
In conclusione vale $\pi_2$z\\
\noindent
Ho concluso con ($\ast$) una derivazione per semplificare o evitare ripetizioni nella derivazione stessa.

\end{enumerate}

\section{Usi del tipo somma indiciata per rappresentare l'assioma di separazione e di quantificazione universale}
\label{sec:usi-del-tipo-somma-indiciata}
Il tipo $\sum\limits_{x \epsilon B}$C(x) viene usato per rappresentare:
\begin{enumerate}
\item uso \textit{set} teoretico: l'unione indiciata disgiunta insiemistica (C(x) set[x $\in$ B])
\item uso \textit{set} teoretico: con assioma di separazione
\item uso logico: proposizionale
\end{enumerate}
\noindent
Il punto (1) l'ho gi\`a spiegato nella prima parte di questo capitolo, ora presento le implicazioni di (2) e (3).\\\\

\subsection{L'assioma di separazione}
\label{subsec:assioma-di-separazione}
L'assioma di separazione dice che dato B, insieme, esiste l'insieme ottenuto per separazione da B tramite $\varphi$(x) definito come  \{x $\in$ B $|$ $\varphi$(x)\} con $\varphi$(x) predicato.\\
Perci\`o per ogni y insieme, y \& \{x $\in$ B $|$ $\varphi$(x)\} $\Leftrightarrow$ $\varphi$(y) vale.\\
Si pu\`o, dunque, simulare l'insieme degli y per cui vale $\varphi$(y) nel momento in cui si ha una proposizione 
\begin{prooftree}
\AxiomC{$\varphi$(x) $\cancel{prop}_{\textbf{type}}$[$\Gamma$,x $\in$ B]}
\LeftLabel{F-$\Sigma$}
\UnaryInfC{$\sum\limits_{x \epsilon B}$ $\varphi$(x) $\cancel{type}_{\textbf{set}}$[$\Gamma$]}
\end{prooftree}
\noindent
dove, nella regola sopra, prop viene pensato come un type\\\\
\noindent
\textit{Dimostrazione}
\begin{itemize}
\item ($\Leftarrow$) Da $\sum\limits_{x \epsilon B}$ $\varphi$(x) significa avere b $\in$ B $+$ pf $\in$ $\varphi$(b), e quindi b soddisfa $\varphi$(x). Ecco che non \`e solo b ma $<$b,pf$>$ $\in$ \{x $\in$ B $|$ $\varphi$(x)\} con pf $\in$ $\varphi$(b) $_{I-\Sigma}\Leftarrow$ $\varphi$(b).\\
\item ($\Rightarrow$) z $\in$ \{x $\in$ B $|$ $\varphi$(x)\} con z che sarebbe una coppia e x $\in$ B $\equiv$ $\sum\limits_{x \epsilon B}$ $\varphi$(x) $\nRightarrow$ $\varphi$(z) vale, in quanto z \`e una coppia tipata.\\ 
Ma posso tuttavia dimostrare che $\varphi(\pi_1z)$ vale usando le proiezioni. Ecco che $\pi_2$(z) $\in$ $\varphi$($\pi_1$z). $\pi_2$(z) \`e cosi \textit{proof term} per cui $\varphi$($\pi_1$z) vale.
\end{itemize}
\noindent
Questo ha permesso a \textit{Martin-L$\ddot{o}$f} di inglobare la \textit{set theory} all'interno della teoria dei tipi.

\subsection{Proposizionale}
\label{subsec:proposizionale}
\begin{prooftree}
\AxiomC{C(x) type[$\Gamma$,x $\in$ B$_{set}$] $\equiv$ $\varphi$(x) prop}
\LeftLabel{$\Sigma$-F}
\UnaryInfC{$\sum\limits_{x \epsilon B}$ $\varphi$(x) $\cancel{type}_{\textbf{prop}}$[$\Gamma$] ${\overset{\mathit{def}}{=}} \exists_{x \in B} \varphi(x)$}
\end{prooftree}
\noindent
\`E la quantificazione esistenziale, in logica, di $\varphi$ che varia sull'insieme B.\\
\noindent
\paragraph{Giustificazione}\mbox{}\\\\
\noindent
$\phi$ prop[$\Gamma,x_1 \in \varphi_1,...,x_n \in \varphi_n]$\\
\noindent Sotto ipotesi $\varphi_i$ prop[$\Gamma$] \quad i $=$ 1..n\\
\noindent e $\varphi$ prop[$\Gamma$] con prop $\equiv$ type (ovvero tipi/\textit{set} delle loro dimostrazioni)\\\\
\noindent
\textit{Definizione} \textbf{\textit{proof-term}}\\ t $\in$ $\varphi[\Gamma]$ dove $\varphi$ \`e prop\\\\
\noindent 
Se devo dire che
pf $\in \phi[\Gamma,x_1 \in \varphi_1,...,x_n \in \varphi_n]$
$\equiv \phi$ \`e vero[$\Gamma, \varphi$ vero,...,$\varphi_n$ vero] (nessuno degli elementi del contesto dipende dal precedente)
$\equiv$ $\varphi_1,...,\varphi_n$ $\vdash_\Gamma$ $\phi$
\\\\\\
\noindent
Sappiamo che la logica lavora con variabili \textit{untyped}, dunque noi ora aggiungiamo in  $\exists_{ x\in B}$ $\varphi$(x) e $\forall$x$_{\in B}$ $\varphi$(x) l'informazione sul tipo.
\\\\
\noindent
Per quanto assunto sopra, la regola di Introduzione viene nel modo sequente riscritta\\\\
\noindent
Assumendo che $\exists_{x \in B}$ $\varphi$(x) type[$\Gamma$] sia ben formata
\begin{prooftree}
\AxiomC{b $\in$ B$_{set}$[$\Gamma$]}
\AxiomC{c $\in$ $\varphi$(b)[$\Gamma$]}
\LeftLabel{$\Sigma$-I}
\BinaryInfC{\cancel{$<b,c>$} $\in$ $\sum\limits_{x \epsilon B}$ C(x)[$\Gamma$]}
\end{prooftree}
\noindent
c $\in$ $\varphi$(b)[$\Gamma$] $\equiv$ $\varphi$(b) \`e vero [$\Gamma$,x$_1$ $\in$ $\gamma_1$,...,x$_n$ $\in$ $\gamma_n$] $\equiv$ $\gamma_1$,...,$\gamma_n$ $\vdash$ $\varphi$(b)\\
$\sum\limits_{x \epsilon B}$ C(x)[$\Gamma$] $\equiv$ $\exists$ x $\in$ B $\varphi$(x) vero[$\Gamma$]\\\\
\noindent
In conclusione, riscritta, la regola di Introduzione, con il calcolo dei sequenti diventa\\
\begin{prooftree}
\AxiomC{$\gamma_1$,...,$\gamma_n$ $\vdash$ $\varphi$(b)}
\LeftLabel{$\Sigma$-I}
\UnaryInfC{$\gamma_1$,...,$\gamma_n$ $\vdash_\Gamma$ $\exists$ x $\in$ B $\varphi$(x)}
\end{prooftree}
\noindent
Che non \`e altro che la regola a destra dell'esiste.
\\\\
\noindent
Ora definisco la regola dell'Eliminatore dipendente.\\
\noindent
Per farlo suppongo di avere un'altra proposizione definita come segue
\begin{prooftree}
\AxiomC{$\phi$ prop[$\Gamma$]}
\UnaryInfC{$\phi$ prop[$\Gamma$,z $\in$ $\sum\limits_{x \epsilon B}\varphi(x)$]}
\end{prooftree}
\noindent

\begin{prooftree}
\AxiomC{$\phi$ prop[$\Gamma$]}
\AxiomC{e(x,y) $\in$ $\phi$[$\Gamma$,x $\in$ B,y $\in$ $\varphi$(x)]}
\LeftLabel{E-$\Sigma_{dip}$}
\BinaryInfC{$\gamma$ $\in$ $\phi$[$\Gamma$,z $\in$ $\exists$ x $\in$ B $\varphi$(x)}
\end{prooftree}
\noindent
e(x,y) $\equiv$ $\varphi$ vero\\
\noindent
$\phi$[$\Gamma$,x $\in$ B,y $\in$ $\varphi$(x)] $\equiv$ $\phi$ \`e vero[$\Gamma$,x $\in$ B,$\varphi$(x) vero]\\
\noindent
$\gamma$ $\in$ $\phi$ $\equiv$ $\phi$ vera\\
\noindent
z $\in$ $\exists$ x $\in$ B $\varphi$(x) $\equiv$ $\exists$ x $\in$ B $\varphi$(x) vero\\\\
\noindent
In conclusione, riscritta, la regola dell'Eliminatore dipendente, con il calcolo dei sequenti diventa\\
\begin{prooftree}
\AxiomC{$\gamma_1$,...,$\gamma_n$ $\varphi$(x) vero $\vdash_{\Gamma,x \in B}$ $\phi$}
\LeftLabel{E-$\Sigma_{dip}$}
\UnaryInfC{$\gamma_1$,...,$\gamma_n$, $\exists$ x $\in$ B $\varphi$(x) $\vdash_\Gamma$ $\phi$}
\end{prooftree}
\noindent
Che non \`e altro che la regola a sinistra dell'esiste.

\subsection{Conclusioni sugli usi del tipo della somma indiciata disgiunta}
\label{subsec:conclusioni-usi-somma-indiciata-disgiunta}
Per
\begin{enumerate}
\item $\sum\limits_{x \epsilon B}C(x)$ \`e un \textit{set} (dove sia x $\in$ B che C(x) sono rispettivamente dei \textit{set}) $\Rightarrow$ B $\times$ C \`e il prodotto cartesiano
\item $\sum\limits_{x \epsilon B}\varphi(x)$ dove $\varphi$ \`e una proposizione $\Rightarrow$ rappresenta l'assioma di separazione \{x $\in$ B $|$ $\varphi$(x)\}\\\
\noindent
Esempio:
$\sum\limits_{z \epsilon list(Nat)}$ Id(Nat,lh(z),2) $\Rightarrow$ rappresenta il sottoinsieme delle liste di lunghezza 2
\item $\exists$ x $\in$ B $\varphi$(x) $\Rightarrow$ $\sum\limits_{x \epsilon B}$ $\varphi$(x), ove $\varphi$(x) e $\sum\limits_{x \epsilon B}$ $\varphi$(x) sono proposizioni
\end{enumerate}
\noindent
Questo serve per:
\begin{itemize}
\item Capire in che modo formalizzare
\item Comprendere il significato della teoria dei tipi che sto leggendo
\end{itemize}

\section{Congiunzione come tipo prodotto cartesiano}
\label{sec:congiunzione-come-tipo-prodotto-cartesiano}
Il tipo prodotto cartesiano permette di interpretare la congiunzione della logica.\\\\
\noindent
Date due preposizioni $\varphi$ prop[$\Gamma$] e $\psi$ prop[$\Gamma$]
\begin{center}
$\varphi$ \& $\psi$ ${\overset{\mathit{def}}{=}}$ $\varphi$ $\times$ $\psi$
\end{center}
L'uguaglianza vale in quanto $\varphi$ prop[$\Gamma$] $\equiv$ $\varphi$ type[$\Gamma$] che \`e il tipo delle sue dimostrazioni \\\\
\noindent 
La regola di \textit{Introduzione} del prodotto cartesiano \`e la seguente
\begin{prooftree}
\AxiomC{a $\in$ $\varphi$[$\Gamma$]}
\AxiomC{b $\in$ $\psi$[$\Gamma$]}
\LeftLabel{I-x}
\BinaryInfC{$<a,b>$ $\in$ $\varphi$ $\times$ $\psi$ [$\Gamma$]}
\end{prooftree}
\noindent
$\Rightarrow$ a $\in$ $\varphi$[$\Gamma$] $\equiv$ $\varphi$ true[$\Gamma$]\\
$\Rightarrow$ b $\in$ $\psi$[$\Gamma$] $\equiv$ $\psi$ true[$\Gamma$]\\
$\Rightarrow$ $\varphi$ $\times$ $\psi$ $\equiv$ $\varphi$ \& $\psi$\\
$\Rightarrow$ $\varphi$ \& $\psi$ vero[$\Gamma$]\\
\noindent
\textbf{$(1)$} Dunque $\varphi$ true[$\Gamma$] e $\psi$ true[$\Gamma$] ($\Rightarrow$) $\varphi$ \& $\psi$ true[$\Gamma$]\\\\
\noindent
Con la regole di \textit{Eliminazione} riesco a provare anche il verso opposto dell'implicazione. In quanto d $\in$ $\varphi$ $\times$ $\psi$[$\Gamma$] \`e come dire $\varphi$ \& $\psi$ true[$\Gamma$], per cui, grazie alle regole di Proiezione del prodotto cartesiano
\begin{center}
\AxiomC{d $\in$ $\varphi$ \& $\psi$[$\Gamma$]}
\LeftLabel{PJ$_1$-x}
\UnaryInfC{$\pi_1$d $\in$ $\varphi$ [$\Gamma$]}
\DisplayProof
\end{center}
d $\in$ $\varphi$ \& $\psi$[$\Gamma$] $\equiv$ $\varphi$ \& $\psi$ true[$\Gamma$]\\
$\pi_1$d $\in$ $\varphi$ [$\Gamma$] $\equiv$ $\varphi$ vero[$\Gamma$]\\
\begin{center}
\AxiomC{d $\in$ $\varphi$ \& $\psi$[$\Gamma$]}
\LeftLabel{PJ$_1$-x}
\UnaryInfC{$\pi_2$d $\in$ $\psi$ [$\Gamma$]}
\DisplayProof
\end{center}
d $\in$ $\varphi$ \& $\psi$[$\Gamma$] $\equiv$ $\varphi$ \& $\psi$ true[$\Gamma$]\\
$\pi_2$d $\in$ $\varphi$ [$\Gamma$] $\equiv$ $\psi$ vero[$\Gamma$]\\

\noindent
\textbf{$(2)$} Dunque con le regole di Eliminazione si \`e mostrato come se $\varphi$ \& $\psi$ true[$\Gamma$] $\Rightarrow$ $\varphi$ true[$\Gamma$] e $\psi$ true[$\Gamma$].
\\\\
\noindent
Quanto definito in (1) e (2) permettono di concludere che date $\varphi$ prop[$\Gamma$] e $\psi$ prop[$\Gamma$] (interpretate come tipi delle loro dimostrazioni), \`e corretto definire  $\varphi$ \& $\psi$ ${\overset{\mathit{def}}{=}}$ $\varphi$ $\times$ $\psi$.\\
Questo  perch\`e una formula e\`vera quando
$\varphi$ true[$\Gamma$] sse \textit{(def)} $\exists$ pf $\in$ $\varphi$[$\Gamma$]. Tale definizione di \textit{proof-term} rende vera entrambe le propriet\`a (1) e (2), e quindi si pu\`o ben interpretare la congiunzione come prodotto cartesiano.
\\\\
\noindent 
In conclusione $\varphi$ \& $\psi$ ${\overset{\mathit{def}}{=}}$  $\varphi$ $\times$ $\psi$ $\equiv$ $\sum\limits_{x \epsilon \varphi}$ $\psi$.\\\\
\noindent
Nella teoria dei tipi semplici $\varphi$ \& $\psi$ ${\overset{\mathit{def}}{=}}$ $\varphi$ $\times$ $\psi$ era gi\`a stato introdotto da \textit{Curry-Howard}.