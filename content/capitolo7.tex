\chapter{Tipo somma indiciata forte}
\label{cap: indexed-sum-type}
%17 inclusa
La somma indiciata forte \`e il potenzialmento indiciato della somma disgiunta binaria (\S\ref{cap: tipo-somma-disgiunta}). Tipo induttivo, ovvero generato con il principio d'induzione della regola di eliminazine, di tipi dipendente.\\\\
\noindent
\textit{Definizione set-teorica}\\
\noindent
\textbf{
\begin{center}$\bigcup\limits_{x \epsilon B\hspace{0.1cm}set}^.$C(x) \qquad (C(x) set) x $\in$ B \end{center}
$\neq$ \quad $\bigcup$ \{y : $\exists$ x $\in$ B \quad y $\in$ C(x)\}\\
\begin{center}${\overset{\mathit{def}}{=}}$ \{(b,c)$:$ b $\in$ B e c $\in$ C(b)\}\end{center}
}
\`E l'unione disgiunta di una famiglia di insiemi, definito dalle regole seguenti.

\section{Regole di Formazione}
\label{subsec: formazione-indexed-sum-type}
\begin{prooftree}
\AxiomC{C(x) type [$\Gamma$, x $\in$ B]}
\LeftLabel{F-$\Sigma$)}
\UnaryInfC{$\sum\limits_{x \in B}$ C(x) type[$\Gamma$]}
\end{prooftree}

\section{Regole di Introduzione}
\label{subsec: formazione-indexed-sum-type}
\begin{prooftree}
\AxiomC{k $\in$ B[$\Gamma$]}
\AxiomC{c $\in$ C(b)[$\Gamma$]}
\AxiomC{$\sum\limits_{x \in B}$ C(x) type[$\Gamma$]}
\LeftLabel{I-$\Sigma$)}
\TrinaryInfC{$<$b,c$>$ $\in$ $\sum\limits_{x\in B}$ C(x) type[$\Gamma$]}
\end{prooftree}

\section{Regole di Eliminazione}
\label{subsec: eliminazione-indexed-sum-type}
\begin{prooftree}
\AxiomC{M(z) type[$\Gamma$, z $\in$ $\sum\limits_{x \in B}$ C(x)]}
\AxiomC{t $\in$ $\sum\limits_{x \in B}$ C(x)[$\Gamma$]}
\AxiomC{e(x,y) $\in$ M($<$x,y$>$)[$\Gamma$, x $\in$ B, y $\in$ C(x)]}
\LeftLabel{E-$\Sigma$)}
\TrinaryInfC{El$_\Sigma$(t,(x,y).e(x,y)) $\in$ M(t)[$\Gamma$]}
\end{prooftree}

\section{Regole di Conservazione}
\label{subsec: conservazione-indexed-sum-type}
\small
\begin{adjustwidth}{-4em}{}
\begin{prooftree}
\AxiomC{M(z) type[$\Gamma$, z $\in$ $\sum\limits_{x \in B}$ C(x)]}
\AxiomC{b $\in$ B[$\Gamma$]}
\AxiomC{c $\in$ C(b)}
\AxiomC{e(x,y) $\in$ M($<$x,y$>$)[$\Gamma$, x $\in$ B, y $\in$ C(x)]}
\LeftLabel{C-$\Sigma$)}
\QuaternaryInfC{El$_\Sigma$($<$b,c$>$,e) $=$ e(b,c) $\in$ M($<$b,c$>$)[$\Gamma$]}
\end{prooftree}
\end{adjustwidth}

\section{Regole di Uguaglianza}
\label{subsec: uguaglianza-indexed-sum-type}
\small
\begin{prooftree}
\AxiomC{B$_1$ $=$ B$_2$ type[$\Gamma$]}
\AxiomC{C$_1$(x) $=$ C$_2$(x) type[$\Gamma$, x $\in$ B$_1$]}
\LeftLabel{eq-F-$\Sigma$)}
\BinaryInfC{$\sum\limits_{x \in B_1}$ C$_1$(x) $=$ $\sum\limits_{x \in B_2}$ C$_2$(x) type[$\Gamma$]}
\end{prooftree}
\begin{prooftree}
\AxiomC{$\sum\limits_{x \in B}$ C(x) type[$\Gamma$]}
\AxiomC{b$_1$ $=$ b$_2$ $\in$ B[$\Gamma$]}
\AxiomC{c$_1$ $=$ c$_2$ $\in$ C(b$_1$)[$\Gamma$]}
\LeftLabel{eq-I-$\Sigma$)}
\TrinaryInfC{$<$b$_1$,c$_1$$>$ $=$ $<$b$_2$,c$_2$$>$ $\in$ $\sum\limits_{x \in B}$ C(x) type[$\Gamma$]}
\end{prooftree}
\begin{adjustwidth}{-8em}{}
\begin{prooftree}
\AxiomC{M(z) type[$\Gamma$, z $\in$ $\sum\limits_{x \in B}$ C(x)]}
\AxiomC{t$_1$ $=$ t$_2$ $\in$ $\sum\limits_{x \in B}$ C(x)[$\Gamma$]}
\AxiomC{e$_1$(x,y) $=$ e$_2$(x,y) $\in$ M($<$x,y$>$)[$\Gamma$, x $\in$ B, y $\in$ C(x)]}
\LeftLabel{eq-E-$\Sigma$)}
\TrinaryInfC{El$_\Sigma$(t$_1$,e$_1$) $=$ El$_\Sigma$(t$_2$,e$_2$) $\in$ M(t$_1$)[$\Gamma$]}
\end{prooftree}
\end{adjustwidth}

\section{Eliminatore dipendente}
\label{sec:eliminatore dipendente-indexed-sum-type}
L'eliminatore ha anche la forma dipendente.
\small
\begin{prooftree}
\AxiomC{M(z) type[$\Gamma$, z $\in$ $\sum\limits_{x \in B}$ C(x)]}
\AxiomC{e(x,y) $\in$ M($<$x,y$>$)[$\Gamma$, x $\in$ B, y $\in$ C(x)]}
\LeftLabel{E-$\Sigma_{dip}$)}
\BinaryInfC{El$_\Sigma$(z,(x,y).e(x,y)) $\in$ M(z)[$\Gamma$, z $\in$ $\sum\limits_{x \in B}$ C(x)]}
\end{prooftree}
\normalsize

\section{Semantica operazionale della somma indiciata forte}
\label{subsec: semantica-operazionale-indexed-sum-type}
La relazione $\rightarrow_1$ viene definita all'interno dei termini con l'uso delle seguenti regole di riduzione:
\begin{itemize}
\item $\beta_{\Sigma}$-red) El$_\Sigma$($<$b,c$>$, e) $\rightarrow_1$ e(b,c)
\item \AxiomC{t$_1$ $\rightarrow_1$ t$_2$}
\LeftLabel{red$_I$)}
\UnaryInfC{El$_\Sigma$ (t$_1$, e) $\rightarrow_1$ El$_\Sigma$ (t$_2$, e)}
\DisplayProof \qquad
\item Novit\`a della somma indiciata forte rispetto al tipo singoletto
\AxiomC{b$_1$ $\rightarrow_1$ b$_2$}
\LeftLabel{$\Sigma$-red$_I$)}
\UnaryInfC{$<$b$_1$, c$>$ $\rightarrow_1$ $<$b$_2$, c$>$}
\DisplayProof
\item \AxiomC{c$_1$ $\rightarrow_1$ c$_2$}
\LeftLabel{$\Sigma$-red$_{II}$)}
\UnaryInfC{$<$b, c$_1>$ $\rightarrow_1$ $<$b, c$_2>$}
\DisplayProof
\end{itemize}

\section{Esercizi}
\label{sec: es-indixed-sum-type}
\paragraph{1}
\textbf{Provare a scrivere le regole del prodotto cartesiano A $\times$ B di un tipo A con un tipo B}\\\\
\noindent
\textbf{Svolgimento}
\noindent
A $\times$ B ${\overset{\mathit{def}}{=}}$ $\sum\limits_{x \in A}$ B\\\\
\noindent

\begin{itemize}
\item \textbf{Regole del tipo prodotto cartesiano:}
\begin{itemize}
\item \textit{Regole di Formazione}
\begin{prooftree}
\AxiomC{A type [$\Gamma$]}
\AxiomC{B type [$\Gamma$]}
\LeftLabel{F-x)}
\BinaryInfC{B $\times$ C type[$\Gamma$]}
\end{prooftree}

\item \textit{Regole di Introduzione}
\begin{prooftree}
\AxiomC{a $\in$ A[$\Gamma$]}
\AxiomC{b $\in$ B[$\Gamma$]}
\LeftLabel{I-x)}
\BinaryInfC{$<$a,b$>$ A $\times$ B[$\Gamma$]}
\end{prooftree}

\item \textit{Regole di Proiezione}
\begin{prooftree}
\AxiomC{d $\in$ A $\times$ B[$\Gamma$]}
\LeftLabel{PJ$_1$-x)}
\UnaryInfC{$\pi_1$d $\in$ A[$\Gamma$]}
\end{prooftree}
\begin{prooftree}
\AxiomC{d $\in$ A $\times$ B[$\Gamma$]}
\LeftLabel{PJ$_2$-x)}
\UnaryInfC{$\pi_2$d $\in$ B[$\Gamma$]}
\end{prooftree}

\item \textit{Regole di Uguaglianza delle proiezioni}
\begin{prooftree}
\AxiomC{a $\in$ A[$\Gamma$]}
\AxiomC{b $\in$ B[$\Gamma$]}
\LeftLabel{PJ$_1$-eq)}
\BinaryInfC{$\pi_1$($<$a,b$>$) $=$ a $\in$ A[$\Gamma$]}
\end{prooftree}
\begin{prooftree}
\AxiomC{a $\in$ A[$\Gamma$]}
\AxiomC{b $\in$ B[$\Gamma$]}
\LeftLabel{PJ$_2$-eq)}
\BinaryInfC{$\pi_2$($<$a,b$>$) $=$ b $\in$ B[$\Gamma$]}
\end{prooftree}
\item \textbf{Semantica operazionale del tipo prodotto cartesiano:}
\item $\beta_x$-red$_1$) El$_\Sigma$($<$a,b$>$, e) $\rightarrow_1$ e(a,b)
\item $\beta_x$-red$_2$) $\pi_1$($<$a,b$>$) $\rightarrow_1$ a
\item $\beta_x$-red$_3$) $\pi_1$($<$a,b$>$) $\rightarrow_1$ b
\item \AxiomC{t$_1$ $\rightarrow_1$ t$_2$}
\LeftLabel{red$_I$)}
\UnaryInfC{El$_\Sigma$ (t$_1$, e) $\rightarrow_1$ El$_\Sigma$ (t$_2$, e)}
\DisplayProof \qquad
\item \AxiomC{a$_1$ $\rightarrow_1$ a$_2$}
\LeftLabel{x-red$_I$)}
\UnaryInfC{$<$a$_1$, b$>$ $\rightarrow_1$ $<$a$_2$, b$>$}
\DisplayProof
\item \AxiomC{b$_1$ $\rightarrow_1$ b$_2$}
\LeftLabel{x-red$_{II}$)}
\UnaryInfC{$<$a, b$_1>$ $\rightarrow_1$ $<$a, b$_2>$}
\DisplayProof

\end{itemize}
\end{itemize}