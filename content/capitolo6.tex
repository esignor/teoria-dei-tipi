\chapter{Uguaglianza proposizionale}
\label{cap: uguaglianza-proposizionale}
%% lezione 16 e 17 inclusa
Pi\`u modi permettono di descrivere il tipo dell'uguaglianza, uno di questi \`e il tipo dell'uguaglianza proposizionale (o identit\`a proposizionale) di \textit{Martin-L$\ddot{o}$f}. Questi \`e dipendente in modo primitivo e definito dalle regole seguenti.

\section{Regole di Formazione}
\label{subsec: formazione-id}
\begin{prooftree}
\AxiomC{A type [$\Gamma$]}
\AxiomC{a $\in$ A[$\Gamma$]}
\AxiomC{b $\in$ A[$\Gamma$]}
\LeftLabel{F-Id)}
\TrinaryInfC{Id(A,a,b) type[$\Gamma$]}
\end{prooftree}

\section{Regole di Introduzione}
\label{subsec: introduzione-id}
\begin{prooftree}
\AxiomC{a $\in$ A[$\Gamma$]}
\LeftLabel{I-Id)}
\UnaryInfC{id(a) $\in$ Id(A,a,a)[$\Gamma$]}
\end{prooftree}

\section{Regole di Eliminazione}
\label{subsec: eliminazione-id}
\small
\begin{prooftree}
\AxiomC{\begin{tabular}[c]{cccc}M(z$_1$,z$_2$,z$_3$) type[$\Gamma$, z$_1$ $\in$ A, z$_2$ $\in$ A, z$_3$ $\in$ Id(A, z$_1$, z$_2$)]\\a $\in$ A[$\Gamma$] \\b $\in$ A[$\Gamma$] \\ t $\in$ Id(A,a,b)[$\Gamma$]\end{tabular}}
\AxiomC{e(x) $\in$ M(x,x,Id(x))[$\Gamma$, x $\in$ A]}
\LeftLabel{E-Id)}
\BinaryInfC{El$_{Id}$(t, (x).e(x)) $\in$ M(a,b,t)[$\Gamma$]}
\end{prooftree}

\section{Regole di Conservazione}
\label{subsec: conservazione-id}
\small
\begin{prooftree}
\AxiomC{\begin{tabular}[c]{cc}M(z$_1$,z$_2$,z$_3$) type[$\Gamma$, z$_1$ $\in$ A, z$_2$ $\in$ A, z$_3$ $\in$ Id(A, z$_1$, z$_2$)] \\ a $\in$ A[$\Gamma$]\end{tabular}}
\AxiomC{e(x) $\in$ M(x,x,Id(x))[$\Gamma$, x $\in$ A]}
\LeftLabel{E-Id)}
\BinaryInfC{El$_{Id}$(Id(a), (x).e(x)) = e(a) $\in$ M(a,a,Id(a))[$\Gamma$]}
\end{prooftree}


\section{Regole di Uguaglianza}
\label{subsec: uguaglianza-liste}
\normalsize
\AxiomC{A$_1$ = A$_2$ type[$\Gamma$]}
\AxiomC{a$_1$ = a$_2$ $\in$ A[$\Gamma$]}
\AxiomC{b$_1$ = b$_2$ $\in$ A[$\Gamma$]}
\LeftLabel{eq-F-Id)}
\TrinaryInfC{Id(A$_1$,a$_1$,b$_1$) = Id(A$_2$,a$_2$,b$_2$) type[$\Gamma$]}
\DisplayProof  \\
\vspace{0.5cm}\\
\AxiomC{a$_1$ = a$_2$ $\in$ A[$\Gamma$]}
\LeftLabel{eq-I-Id)}
\UnaryInfC{Id(a$_1$) = Id(a$_2$) $\in$ Id(A,a$_1$,a$_1$)[$\Gamma$]}
\DisplayProof

\section{Eliminatore dipendente}
\label{sec:eliminatore dipendente-id}
L'eliminatore ha anche la forma dipendente.
\small
\begin{prooftree}
\AxiomC{M(z$_1$,z$_2$,z$_3$) type[$\Gamma$, z$_1$ $\in$ A, z$_2$ $\in$ A, z$_3$ $\in$ Id(A, z$_1$, z$_2$)]}
\AxiomC{e(x) $\in$ M(x,x,Id(x)))[$\Gamma$, x $\in$ A]}
\LeftLabel{El-Id$_{dip}$)}
\BinaryInfC{El$_{Id}$(z$_3$, (x).e(x)) $\in$ M(z$_1$, z$_2$, z$_3$)[$\Gamma$,z$_1$ $\in$ A, z$_2$ $\in$ A, z$_3$ $\in$ Id(A,z$_1$, z$_2$)]}
\end{prooftree}
\normalsize

\section{Lemma dell'Id}
\label{sec: lemma-ind}
\`E derivabile la regola
\begin{prooftree}
\AxiomC{a = b $\in$ A[$\Gamma$]}
\UnaryInfC{Id(a) $\in$ Id(A,a,b)}
\end{prooftree}
\noindent
Con Id(a) = Id(b)\\\\
\noindent
Ovvero l'guaglianza proposizionale rende uguali i termini, definizionalmente, ma non vale il contrario. 
Percui se p $\in$ Id(A,a,b)[$\Gamma$] \`e derivabile $\nRightarrow$ a = b non \`e derivabile.\\
Formalmente si pu\`o trovare con il tipo singoletto, nel modo seguente: "? $\in$ Id(N$_1$,x,$\ast$)[x $\in$ N$_1$] $\nRightarrow$ x = $\ast$ $\in$ N$_1$[x $\in$ N$_1$] (non \`e vero che x \`e uguale a $\ast$)"\\\\
\textit{Dimostrazione} 
\begin{prooftree}
\AxiomC{a = b $\in$ A[$\Gamma$]}
\UnaryInfC{Id(a) $\in$ Id(A,a,b)[$\Gamma]}
\end{prooftree}



\section{Esercizi}
\label{sec: es-id}
\paragraph{1}
\textbf{Si dimostri che esiste una funzione h$_1$ che mi permetta di dimostrare la simmetria dell'uguaglianza. Si dia la prova che z$_2$ = z$_1$ non appena [$\Gamma$, z$_1$ $\in$ A, z$_2$ $\in$ A,  z$_3$ $\in$ Id(A,z$_1$,z$_2$)] con A type[$\Gamma$].
\begin{center}h$_1$(z$_1$,z$_2$,z$_3$) $\in$ Id(A,z$_1$,z$_2$)[$\Gamma$, z$_1$ $\in$ A, z$_2$ $\in$ A, z$_3$ $\in$ Id(A, z$_1$, z$_2$)] \end{center}}
\noindent
\textbf{Soluzione} (\textit{Si chiede di dimostrare che h$_1$ \`e \textit{proof term} nella forma indicata sopra})\\\\
\textit{Assunzione dell'esercizio: Id(A, z$_1$, z$_2$) type[$\Gamma$, z$_1$ $\in$ A, z$_2$ $\in$ A]}\\\\
M(z$_1$,z$_2$,z$_3$) $\equiv$ Id(A, z$_2$, z$_1$)\\
h$_1$(z$_1$,z$_2$,z$_3$) $\equiv$ El$_{Id}$(z$_3$,(x).id(x))

\small
\begin{prooftree}
\AxiomC{\textbf{1}}
\UnaryInfC{Id(A,z$_2$,z$_1$) type[$\Gamma$, z$_1$ $\in$ A, z$_2$ $\in$ A, z$_3$ $\in$ Id(A, z$_1$, z$_2$)]}
\AxiomC{\textbf{2}}
\UnaryInfC{id(x) $\in$ Id(A,x,x)[$\Gamma$, x $\in$ A]}
\LeftLabel{El-Id$_{dip}$)}
\BinaryInfC{h$_1$(z$_1$,z$2$,z$_3$) $\in$ Id(A,z$_2$,z$_1$)[$\Gamma$, z$_1$ $\in$ A, z$_2$ $\in$ A, z$_3$ $\in$ Id(A, z$_1$, z$_2$)]}
\end{prooftree}
\vspace{0.5cm}
\noindent
\textbf{1}
\scriptsize
\begin{adjustwidth}{-15em}{}
\begin{prooftree}
\AxiomC{\textbf{1$_A$}}
\UnaryInfC{A type [$\Gamma$, z$_1$ $\in$ A, z$_2$ $\in$ A, z$_3$ $\in$ Id(A,z$_1$,z$_2$)]}
\AxiomC{\textbf{1$_B$}}
\UnaryInfC{z$_2$ $\in$ A[$\Gamma$, z$_1$ $\in$ A, z$_2$ $\in$ A, z$_3$ $\in$ Id(A,z$_1$,z$_2$)]}
\AxiomC{\textbf{1$_C$}}
\UnaryInfC{z$_1$ $\in$ A[$\Gamma$, z$_1$ $\in$ A, z$_2$ $\in$ A, z$_3$ $\in$ Id(A,z$_1$,z$_2$)]}
\LeftLabel{F-id}
\TrinaryInfC{Id(A,z$_2$,z$_1$) type[$\Gamma$, z$_1$ $\in$ A, z$_2$ $\in$ A, z$_3$ $\in$ Id(A,z$_1$,z$_2$)]}
\end{prooftree}
\end{adjustwidth}

\vspace{0.5cm}
\noindent
\textbf{1$_A$}
\small
\begin{adjustwidth}{-8em}{}
\begin{prooftree}
\AxiomC{[ ] cont}
\LeftLabel{s-checks}
\UnaryInfC{A type [ ]}
\AxiomC{\textbf{1$_A^\backprime$}}
\UnaryInfC{A type[$\Gamma$, z$_1$ $\in$ A, z$_2$ $\in$ A]}
\AxiomC{\textbf{1$_B^\backprime$}}
\UnaryInfC{z$_1$ $\in$ [$\Gamma$, z$_1$ $\in$ A, z$_2$ $\in$ A]}
\AxiomC{\textbf{1$_C^\backprime$}}
\UnaryInfC{z$_2$ $\in$ [$\Gamma$, z$_1$ $\in$ A, z$_2$ $\in$ A]}
\LeftLabel{F-Id}
\TrinaryInfC{Id(A,z$_1$,z$_2$)[$\Gamma$, z$_1$ $\in$ A, z$_2$ $\in$ A]}
\LeftLabel{F-c}\RightLabel{\begin{tabular}[c]{cc}(z$_3$ $\in$ Id(A,z$_1$,z$_2$)) $\notin$ \\ ($\Gamma$, z$_1$ $\in$ A, z$_2$ $\in$ A)\end{tabular}}
\UnaryInfC{$\Gamma$, z$_1$ $\in$ A, z$_2$ $\in$ A, z$_3$ $\in$ Id(A,z$_1$,z$_2$) cont}
\LeftLabel{ind-te}
\BinaryInfC{A type [$\Gamma$, z$_1$ $\in$ A, z$_2$ $\in$ A, z$_3$ $\in$ Id(A,z$_1$,z$_2$)]}
\end{prooftree}
\end{adjustwidth}


\vspace{0.5cm}
\noindent
\textbf{1$_A^\backprime$}}
\small
\begin{prooftree}
\AxiomC{[ ] cont}
\LeftLabel{s-checks}
\UnaryInfC{A type [ ]}
\AxiomC{[ ] cont}
\LeftLabel{s-checks}
\UnaryInfC{A type [ ]}
\AxiomC{$\Gamma$ cont}
\LeftLabel{s-checks}
\UnaryInfC{A type[$\Gamma$]}
\LeftLabel{F-c}\RightLabel{(z$_1$ $\in$ A) $\notin$ $\Gamma$}
\UnaryInfC{$\Gamma$, z$_1$ $\in$ A cont}
\LeftLabel{ind-te}
\BinaryInfC{A type[$\Gamma$, z$_1$ $\in$ A]}
\LeftLabel{F-c}\RightLabel{(z$_2$ $\in$ A) $\notin$ ($\Gamma$, z$_1$ $\in$ A)}
\UnaryInfC{$\Gamma$, z$_1$ $\in$ A, z$_2$ $\in$ A cont}
\LeftLabel{ind-te}
\BinaryInfC{A type[$\Gamma$, z$_1$ $\in$ A, z$_2$ $\in$ A]}
\end{prooftree}

\noindent
\textbf{1$_B^\backprime$}}
\small
\begin{prooftree}
\AxiomC{[ ] cont}
\LeftLabel{s-checks}
\UnaryInfC{A type [ ]}
\AxiomC{$\Gamma$ cont}
\LeftLabel{s-checks}
\UnaryInfC{A type[$\Gamma$]}
\LeftLabel{F-c}\RightLabel{(z$_1$ $\in$ A) $\notin$ $\Gamma$}
\UnaryInfC{$\Gamma$, z$_1$ $\in$ A cont}
\LeftLabel{ind-te}
\BinaryInfC{A type[$\Gamma$, z$_1$ $\in$ A]}
\LeftLabel{F-c}\RightLabel{(z$_2$ $\in$ A) $\notin$ ($\Gamma$, z$_1$ $\in$ A)}
\UnaryInfC{$\Gamma$, z$_1$ $\in$ A, z$_2$ $\in$ A cont}
\LeftLabel{var}
\UnaryInfC{z$_1$ $\in$ A[$\Gamma$, z$_1$ $\in$ A, z$_2$ $\in$ A]}
\end{prooftree}

\noindent
\textbf{1$_C^\backprime$}
\small
\begin{prooftree}
\AxiomC{[ ] cont}
\LeftLabel{s-checks}
\UnaryInfC{A type [ ]}
\AxiomC{$\Gamma$ cont}
\LeftLabel{s-checks}
\UnaryInfC{A type[$\Gamma$]}
\LeftLabel{F-c}\RightLabel{(z$_1$ $\in$ A) $\notin$ $\Gamma$}
\UnaryInfC{$\Gamma$, z$_1$ $\in$ A cont}
\LeftLabel{ind-te}
\BinaryInfC{A type[$\Gamma$, z$_1$ $\in$ A]}
\LeftLabel{F-c}\RightLabel{(z$_2$ $\in$ A) $\notin$ ($\Gamma$, z$_1$ $\in$ A)}
\UnaryInfC{$\Gamma$, z$_1$ $\in$ A, z$_2$ $\in$ A cont}
\LeftLabel{var}
\UnaryInfC{z$_2$ $\in$ A[$\Gamma$, z$_1$ $\in$ A, z$_2$ $\in$ A]}
\end{prooftree}


\vspace{0.5cm}
\noindent
\textbf{2}
\small
\begin{prooftree}
\AxiomC{$\Gamma$}
\LeftLabel{s-checks}
\AxiomC{A type[$\Gamma$]}
\LeftLabel{F-c}\RightLabel{(x $\in$ A) $\notin$ $\Gamma$}
\UnaryInfC{$\Gamma$, x $\in$ A cont}
\LeftLabel{var}
\UnaryInfC{x $\in$ A[$\Gamma$, x $\in$ A]}
\LeftLabel{I-id}
\BinaryInfC{{id(x) $\in$ Id(A,x,x)[$\Gamma$, x $\in$ A]}}
\end{prooftree}
\noindent
\textit{Le derivazioni \textbf{2} e \textbf{3} sono simili a quelle in \textbf{1}. Le ometto per evitare una sovradimensionalit\`a del numero di derivazioni}\\\\
\vspace{1cm}
\noindent Vale che \textit{$\Gamma$  cont} \`e assioma ([ ] cont = $\Gamma$ cont con $\Gamma$ = $\varnothing$)\\
\noindent
\textbf{Verifico che Id(A, z$_1$ $\in$ A, z$_2$ $\in$ A)[$\Gamma$, z$_1$ $\in$ A, z$_2$ $\in$ A] \`e una premessa valida rispetto al giudizio di conclusione Id(A, z$_2$ $\in$ A, z$_1$ $\in$ A)[$\Gamma$, z$_1$ $\in$ A, z$_2$ $\in$ A, z$_3$ $\in$ Id(A,z$_1$,z$_2$)]}
\\\\
\noindent $\Delta$ = $\Gamma$, z$_1$ $\in$ A, z$_2$ $\in$ A, z$_3$ $\in$ Id(A,z$_1$,z$_2$)\\
\noindent $\theta$ = A, z$_1$ $\in$ A, z$_2$ $\in$ A\\
\scriptsize
\begin{adjustwidth}{-7em}{}
\begin{prooftree}
\AxiomC{($\ast$)}
\UnaryInfC{w $\in$ A[$\Delta$]}
\AxiomC{($\ast$)}
\UnaryInfC{z$_2$ $\in$ A[$\Delta$]}
\AxiomC{($\ast$)}
\UnaryInfC{z$_1$ $\in$ A[$\Delta$]}
\AxiomC{}
\UnaryInfC{\textbf{Id($\theta$)[$\Gamma$, z$_1$ $\in$ A, z$_2$ $\in$ A]}}
\AxiomC{\textbf{1}}
\UnaryInfC{z$_3$ $\in$ Id($\theta$), w $\in$ A cont}
\LeftLabel{ind-ty}
\BinaryInfC{Id($\theta$)[$\Delta$, w $\in$ A]}
\LeftLabel{sub-typ}
\BinaryInfC{Id(A, w $\in$ A, z$_2$ $\in$ A)[$\Delta$, w $\in$ A]}
\LeftLabel{sub-typ}
\BinaryInfC{Id(A, w $\in$ A, z$_1$ $\in$ A)[$\Delta$, w $\in$ A]}
\LeftLabel{sub-typ}
\BinaryInfC{Id(A, z$_2$ $\in$ A, z$_1$ $\in$ A)[$\Delta$]}
\end{prooftree}
\end{adjustwidth}
\vspace{0.5cm}
\noindent
\normalsize \textbf{1}
\small
\begin{prooftree}
\AxiomC{$\Gamma$ cont}
\LeftLabel{s-checks}
\UnaryInfC{A type[ ]}
\AxiomC{($\ast$)}
\UnaryInfC{A type[ ]}
\AxiomC{($\ast$)}
\UnaryInfC{$z_1$ $\in$ A type[ ]}
\AxiomC{($\ast$)}
\UnaryInfC{$z_2$ $\in$ A type[ ]}
\LeftLabel{F-Id}
\TrinaryInfC{Id(A,z$_1$,z$_2$) type[ ]}
\LeftLabel{F-c}\RightLabel{(z$_3$ $\in$ Id($\theta$)) $\notin$ [ ]}
\UnaryInfC{z$_3$ $\in$ Id($\theta$) cont}
\LeftLabel{ind-ty}
\BinaryInfC{A type[z$_3$ $\in$ Id($\theta$)]}
\LeftLabel{F-c}\RightLabel{(w $\in$ A) $\notin$ z$_3$ $\in$ Id($\theta$)}
\UnaryInfC{z$_3$ $\in$ Id($\theta$), w $\in$ A cont}
\end{prooftree}
\noindent
Con il simbolo ($\ast)$ indico quelle derivazioni che terminano su un assioma, che per evitare ripetizioni nella derivazione ho omesso.