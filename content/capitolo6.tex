\chapter{Tipo dell'uguaglianza proposizionale}
\label{cap: uguaglianza-proposizionale}
%% lezione 16
Pi\`u modi permettono di descrivere il tipo dell'uguaglianza, uno di questi \`e il tipo dell'uguaglianza proposizionale (o identit\`a proposizionale) di \textit{Martin-L$\ddot{o}$f}. \`E un primo esempio di tipo dipendente, e se opportunamente abbinato al costruttore pu\`o costruire altri tipi dipendenti.\\
\noindent
Questi \`e dipendente, in modo primitivo ed \`e definito dalle regole seguenti.

\section{Regole di Formazione}
\label{subsec: formazione-id}
\begin{prooftree}
\AxiomC{A type [$\Gamma$]}
\AxiomC{a $\in$ A[$\Gamma$]}
\AxiomC{b $\in$ A[$\Gamma$]}
\LeftLabel{F-Id)}
\TrinaryInfC{Id(A,a,b) type[$\Gamma$]}
\end{prooftree}

\section{Regole di Introduzione}
\label{subsec: introduzione-id}
\begin{prooftree}
\AxiomC{a $\in$ A[$\Gamma$]}
\LeftLabel{I-Id)}
\UnaryInfC{id(a) $\in$ Id(A,a,a)[$\Gamma$]}
\end{prooftree}

\section{Regole di Eliminazione}
\label{subsec: eliminazione-id}
\small
\begin{prooftree}
\AxiomC{\begin{tabular}[c]{cccc}M(z$_1$,z$_2$,z$_3$) type[$\Gamma$, z$_1$ $\in$ A, z$_2$ $\in$ A, z$_3$ $\in$ Id(A, z$_1$, z$_2$)]\\a $\in$ A[$\Gamma$] \\b $\in$ A[$\Gamma$] \\ t $\in$ Id(A,a,b)[$\Gamma$]\end{tabular}}
\AxiomC{e(x) $\in$ M(x,x,Id(x))[$\Gamma$, x $\in$ A]}
\LeftLabel{E-Id)}
\BinaryInfC{El$_{Id}$(t, (x).e(x)) $\in$ M(a,b,t)[$\Gamma$]}
\end{prooftree}

\section{Regole di Conservazione}
\label{subsec: conservazione-id}
\small
\begin{prooftree}
\AxiomC{\begin{tabular}[c]{cc}M(z$_1$,z$_2$,z$_3$) type[$\Gamma$, z$_1$ $\in$ A, z$_2$ $\in$ A, z$_3$ $\in$ Id(A, z$_1$, z$_2$)] \\ a $\in$ A[$\Gamma$]\end{tabular}}
\AxiomC{e(x) $\in$ M(x,x,Id(x))[$\Gamma$, x $\in$ A]}
\LeftLabel{E-Id)}
\BinaryInfC{El$_{Id}$(Id(a), (x).e(x)) $=$ e(a) $\in$ M(a,a,Id(a))[$\Gamma$]}
\end{prooftree}


\section{Regole di Uguaglianza}
\label{subsec: uguaglianza-id}
\normalsize
\AxiomC{A$_1$ = A$_2$ type[$\Gamma$]}
\AxiomC{a$_1$ = a$_2$ $\in$ A[$\Gamma$]}
\AxiomC{b$_1$ = b$_2$ $\in$ A[$\Gamma$]}
\LeftLabel{eq-F-Id)}
\TrinaryInfC{Id(A$_1$,a$_1$,b$_1$) = Id(A$_2$,a$_2$,b$_2$) type[$\Gamma$]}
\DisplayProof  \\
\vspace{0.5cm}\\
\AxiomC{a$_1$ = a$_2$ $\in$ A[$\Gamma$]}
\LeftLabel{eq-I-Id)}
\UnaryInfC{Id(a$_1$) = Id(a$_2$) $\in$ Id(A,a$_1$,a$_1$)[$\Gamma$]}
\DisplayProof

\section{Eliminatore dipendente}
\label{sec:eliminatore dipendente-id}
L'eliminatore ha anche la forma dipendente.
\small
\begin{prooftree}
\AxiomC{M(z$_1$,z$_2$,z$_3$) type[$\Gamma$, z$_1$ $\in$ A, z$_2$ $\in$ A, z$_3$ $\in$ Id(A, z$_1$, z$_2$)]}
\AxiomC{e(x) $\in$ M(x,x,Id(x)))[$\Gamma$, x $\in$ A]}
\LeftLabel{E-Id$_{dip}$)}
\BinaryInfC{El$_{Id}$(z$_3$, (x).e(x)) $\in$ M(z$_1$, z$_2$, z$_3$)[$\Gamma$,z$_1$ $\in$ A, z$_2$ $\in$ A, z$_3$ $\in$ Id(A,z$_1$, z$_2$)]}
\end{prooftree}
\normalsize

\section{Lemma dell'Id}
\label{sec: lemma-ind}
\`E derivabile la regola
\begin{prooftree}
\AxiomC{a $=$ b $\in$ A[$\Gamma$]}
\UnaryInfC{Id(a) $\in$ Id(A,a,b)}
\end{prooftree}
\noindent
Con Id(a) $=$ Id(b)\\\\
\noindent
Ovvero l'guaglianza proposizionale rende uguali i termini, definizionalmente, ma non vale il contrario. 
Percui se p $\in$ Id(A,a,b)[$\Gamma$] \`e derivabile $\nRightarrow$ a $=$ b non \`e derivabile.\\
Formalmente si pu\`o trovare con il tipo singoletto, nel modo seguente: "? $\in$ Id(N$_1$,x,$\ast$)[x $\in$ N$_1$] $\nRightarrow$ x $=$ $\ast$ $\in$ N$_1$[x $\in$ N$_1$] (non \`e vero che x \`e uguale a $\ast$)"\\\\
\textit{Dimostrazione} 
\begin{prooftree}
\AxiomC{}
\UnaryInfC{a $=$ b $\in$ A[$\Gamma$]}
\LeftLabel{s-checks}
\UnaryInfC{a $\in$ A[$\Gamma$]}
\LeftLabel{I-Id}
\UnaryInfC{Id(a) $\in$ Id(A,a,a)[$\Gamma$]}
\AxiomC{\textbf{1}}
\UnaryInfC{Id(A,a,a) $=$ Id(A,a,b) type[$\Gamma$]}
\LeftLabel{conv}
\BinaryInfC{Id(a) $\in$ Id(A,a,b)[$\Gamma$]}
\end{prooftree}
\vspace{0.5cm}
\textbf{1}\\
\begin{prooftree}
\AxiomC{}
\UnaryInfC{a $=$ b $\in$ [$\Gamma$]}
\LeftLabel{s-checks}
\UnaryInfC{a $\in$ A[$\Gamma$]}
\LeftLabel{s-checks}
\UnaryInfC{A type[$\Gamma$]}
\LeftLabel{ref}
\UnaryInfC{A $=$ A type[$\Gamma$]}
\AxiomC{}
\UnaryInfC{a $=$ b $\in$ [$\Gamma$]}
\LeftLabel{s-checks}
\UnaryInfC{a $\in$ [$\Gamma$]}
\LeftLabel{ref}
\UnaryInfC{a $=$ a $\in$ [$\Gamma$]}
\AxiomC{}
\UnaryInfC{a $=$ b $\in$ [$\Gamma$]}
\LeftLabel{Eq-F-Id}
\TrinaryInfC{Id(A,a,a)[$\Gamma$] $=$ Id(A,a,b)[$\Gamma$]}
\end{prooftree}

\section{Esercizio di dimostrazione per induzione su addizione con zero}
\label{sec:esercizio-di-dimostrazione-per-induzione-su-addizione-con-zero}
\textit{Nel dettaglio l'esercizio presentato in questa sezione, e svolto a lezione, viene risolto in \S \ref{sec: es-id}, esercizio 5}.\\
$\varphi$(0) $\&$ $\forall_{x \in Nat}$($\varphi$(x) $\rightarrow$ $\varphi$(succ(x))) $\rightarrow$ $\forall_{z \in Nat}$ $\varphi$(z) (\S \ref{cap:principi-di-induzione-per-tipi-induttivi}).
\\\\
\noindent
\textit{Definizione} \textbf{Somma dei numeri Naturali}
w $+$ z $\in$ Nat[w $\in$ Nat,z $\in$ Nat]\\
w $+$ z ${\overset{\mathit{def}}{\equiv}}$ El$_{Nat}$(z,w,(x,y).succ(y))\\\\ 
\noindent
w $+$ 0 $=$ w $\in$ Nat[w $\in$ Nat] per \textit{$\beta_{1Nat}$-red}\\
ma definizionalmente 0 $+$ z $\neq$ z\\
El$_{Nat}$(z,0,(x,y).succ(y)) \`e in forma normale $\neq$ da z, anch'essa in forma normale\\
$\rightarrow$ 0 $+$ z $=$ z $\in$ Nat[z $\in$ Nat] non \`e derivabile in teoria dei tipi sui Naturali.\\\\
\noindent 
In realt\`a, si pu\`o risolvere 0 $+$ z $=$ z $\in$ Nat[z $\in$ Nat], non con l'uso dell'uguaglianza definizionale, ma con l'identit\`a proposizionale, nel modo seguente.\\
\begin{center}\textbf{Obiettivo:} trovare pf $\in$ $\forall_{z \in Nat}$ Id(Nat,0$+$z,z)\end{center}
Sappiamo che\\
$\forall_{w \in Nat}$ Id(Nat,w+0,w) $\equiv$ $\lambda$w.id(w) [ ]\\
Invece pf $\in$ $\forall_{z \in Nat}$ Id(Nat,0$+$z,z) va dimostrato per induzione.
\noindent
\\\\
Uso il \textit{proof-term} cos\`i definito in una precedente lezione:
\begin{center} h$_2$(z$_1$,z$_2$,z$_3$) $\in$ Id(Nat,succ(z$_1$),succ(z$_2$))[z$_1$ $\in$ Nat,z$_2$ $\in$ Nat,z$_3$ $\in$ Id(Nat,z$_2$,z$_3$)]\end{center}

\small
\begin{adjustwidth}{-9.5em}{}
\begin{prooftree}

\AxiomC{}
\UnaryInfC{Id(Nat,0+z,z)type[z $\in$ Nat]}
\AxiomC{}
\UnaryInfC{id(0) $\in$ Id(Nat,0+0,0)[ ]}
\AxiomC{}
\UnaryInfC{h$_2$(0+x,x,y) $\in$ Id(Nat,0+succ(x),succ(x))[x $\in$ Nat,y $\in$ Id(Nat,0+x,x)]}
\LeftLabel{E-Nat$_{dip}$}
\TrinaryInfC{El$_{Nat}$(z,id(0),(x,y).h$_2$(0+x,x,y))Id(Nat,0+z,z)[z $\in$ Nat]}
\LeftLabel{I-{\scriptsize $\prod$}}
\UnaryInfC{? $\in$ $\forall_{z \in Nat}$ Id(Nat,0+z,z)[ ]}
\end{prooftree}
\end{adjustwidth}
\noindent
\vspace{0.3cm}
\normalsize \begin{center}\textbf{In conclusione:} $\forall_{z \in Nat}$ Id(Nat,0+z,z)\end{center}





\section{Esercizi}
\label{sec: es-id}
\paragraph{1)}
\textbf{Si dimostri che esiste una funzione h$_1$ che mi permetta di dimostrare la simmetria dell'uguaglianza. Si dia la prova che z$_2$ = z$_1$ non appena [$\Gamma$, z$_1$ $\in$ A, z$_2$ $\in$ A,  z$_3$ $\in$ Id(A,z$_1$,z$_2$)] con A type[$\Gamma$].
\begin{center}h$_1$(z$_1$,z$_2$,z$_3$) $\in$ Id(A,z$_2$,z$_1$)[$\Gamma$, z$_1$ $\in$ A, z$_2$ $\in$ A, z$_3$ $\in$ Id(A, z$_1$, z$_2$)] \end{center}}
\noindent
\textbf{Soluzione} (\textit{Si chiede di dimostrare che h$_1$ \`e \textit{proof term} nella forma indicata sopra})\\\\
\textit{Assunzione dell'esercizio: Id(A, z$_1$, z$_2$) type[$\Gamma$, z$_1$ $\in$ A, z$_2$ $\in$ A]}\\\\
M(z$_1$,z$_2$,z$_3$) $\equiv$ Id(A, z$_2$, z$_1$)\\
h$_1$(z$_1$,z$_2$,z$_3$) $\equiv$ El$_{Id}$(z$_3$,(x).id(x))

\small
\begin{prooftree}
\AxiomC{\textbf{1}}
\UnaryInfC{Id(A,z$_2$,z$_1$) type[$\Gamma$, z$_1$ $\in$ A, z$_2$ $\in$ A, z$_3$ $\in$ Id(A, z$_1$, z$_2$)]}
\AxiomC{\textbf{2}}
\UnaryInfC{id(x) $\in$ Id(A,x,x)[$\Gamma$, x $\in$ A]}
\LeftLabel{E-Id$_{dip}$)}
\BinaryInfC{h$_1$(z$_1$,z$_2$,z$_3$) $\in$ Id(A,z$_2$,z$_1$)[$\Gamma$, z$_1$ $\in$ A, z$_2$ $\in$ A, z$_3$ $\in$ Id(A, z$_1$, z$_2$)]}
\end{prooftree}
\vspace{0.5cm}
\noindent
\normalsize
\textbf{1}
\scriptsize
\begin{adjustwidth}{-14em}{}
\begin{prooftree}
\AxiomC{\textbf{1$_A$}}
\UnaryInfC{A type [$\Gamma$, z$_1$ $\in$ A, z$_2$ $\in$ A, z$_3$ $\in$ Id(A,z$_1$,z$_2$)]}
\AxiomC{\textbf{1$_B$}}
\UnaryInfC{z$_2$ $\in$ A[$\Gamma$, z$_1$ $\in$ A, z$_2$ $\in$ A, z$_3$ $\in$ Id(A,z$_1$,z$_2$)]}
\AxiomC{\textbf{1$_C$}}
\UnaryInfC{z$_1$ $\in$ A[$\Gamma$, z$_1$ $\in$ A, z$_2$ $\in$ A, z$_3$ $\in$ Id(A,z$_1$,z$_2$)]}
\LeftLabel{F-id}
\TrinaryInfC{Id(A,z$_2$,z$_1$) type[$\Gamma$, z$_1$ $\in$ A, z$_2$ $\in$ A, z$_3$ $\in$ Id(A,z$_1$,z$_2$)]}
\end{prooftree}
\end{adjustwidth}

\vspace{0.5cm}
\noindent
\normalsize
\textbf{1$_A$}
\small
\begin{adjustwidth}{-8em}{}
\begin{prooftree}
\AxiomC{}
\UnaryInfC{A type [$\Gamma$]}
\AxiomC{\textbf{1$_A^\backprime$}}
\UnaryInfC{A type[$\Gamma$, z$_1$ $\in$ A, z$_2$ $\in$ A]}
\AxiomC{\textbf{1$_B^\backprime$}}
\UnaryInfC{z$_1$ $\in$ A [$\Gamma$, z$_1$ $\in$ A, z$_2$ $\in$ A]}
\AxiomC{\textbf{1$_C^\backprime$}}
\UnaryInfC{z$_2$ $\in$ A [$\Gamma$, z$_1$ $\in$ A, z$_2$ $\in$ A]}
\LeftLabel{F-Id}
\TrinaryInfC{Id(A,z$_1$,z$_2$)type[$\Gamma$, z$_1$ $\in$ A, z$_2$ $\in$ A]}
\LeftLabel{F-c}\RightLabel{\begin{tabular}[c]{cc}(z$_3$ $\in$ Id(A,z$_1$,z$_2$)) $\notin$ \\ ($\Gamma$, z$_1$ $\in$ A, z$_2$ $\in$ A)\end{tabular}}
\UnaryInfC{$\Gamma$, z$_1$ $\in$ A, z$_2$ $\in$ A, z$_3$ $\in$ Id(A,z$_1$,z$_2$) cont}
\LeftLabel{ind-ty}
\BinaryInfC{A type [$\Gamma$, z$_1$ $\in$ A, z$_2$ $\in$ A, z$_3$ $\in$ Id(A,z$_1$,z$_2$)]}
\end{prooftree}
\end{adjustwidth}


\vspace{0.5cm}
\noindent
\normalsize
\textbf{1$_A^\backprime$}}
\small
\begin{adjustwidth}{4em}{}
\begin{prooftree}
\AxiomC{}
\UnaryInfC{A type [$\Gamma$]}
\AxiomC{}
\UnaryInfC{A type [$\Gamma$]}
\AxiomC{}
\UnaryInfC{A type[$\Gamma$]}
\LeftLabel{F-c}\RightLabel{(z$_1$ $\in$ A) $\notin$ $\Gamma$}
\UnaryInfC{$\Gamma$, z$_1$ $\in$ A cont}
\LeftLabel{ind-ty}
\BinaryInfC{A type[$\Gamma$, z$_1$ $\in$ A]}
\LeftLabel{F-c}\RightLabel{(z$_2$ $\in$ A) $\notin$ ($\Gamma$, z$_1$ $\in$ A)}
\UnaryInfC{$\Gamma$, z$_1$ $\in$ A, z$_2$ $\in$ A cont}
\LeftLabel{ind-ty}
\BinaryInfC{A type[$\Gamma$, z$_1$ $\in$ A, z$_2$ $\in$ A]}
\end{prooftree}
\end{adjustwidth}

\noindent
\normalsize
\textbf{1$_B^\backprime$}}
\small
\begin{adjustwidth}{4em}{}
\begin{prooftree}
\AxiomC{}
\UnaryInfC{A type [$\Gamma$]}
\AxiomC{}
\UnaryInfC{A type[$\Gamma$]}
\LeftLabel{F-c}\RightLabel{(z$_1$ $\in$ A) $\notin$ $\Gamma$}
\UnaryInfC{$\Gamma$, z$_1$ $\in$ A cont}
\LeftLabel{ind-ty}
\BinaryInfC{A type[$\Gamma$, z$_1$ $\in$ A]}
\LeftLabel{F-c}\RightLabel{(z$_2$ $\in$ A) $\notin$ ($\Gamma$, z$_1$ $\in$ A)}
\UnaryInfC{$\Gamma$, z$_1$ $\in$ A, z$_2$ $\in$ A cont}
\LeftLabel{var}
\UnaryInfC{z$_1$ $\in$ A[$\Gamma$, z$_1$ $\in$ A, z$_2$ $\in$ A]}
\end{prooftree}
\end{adjustwidth}

\noindent
\normalsize
\textbf{1$_C^\backprime$}
\small
\begin{prooftree}
\AxiomC{}
\UnaryInfC{A type [$\Gamma$]}
\AxiomC{}
\UnaryInfC{A type[$\Gamma$]}
\LeftLabel{F-c}\RightLabel{(z$_1$ $\in$ A) $\notin$ $\Gamma$}
\UnaryInfC{$\Gamma$, z$_1$ $\in$ A cont}
\LeftLabel{ind-ty}
\BinaryInfC{A type[$\Gamma$, z$_1$ $\in$ A]}
\LeftLabel{F-c}\RightLabel{(z$_2$ $\in$ A) $\notin$ ($\Gamma$, z$_1$ $\in$ A)}
\UnaryInfC{$\Gamma$, z$_1$ $\in$ A, z$_2$ $\in$ A cont}
\LeftLabel{var}
\UnaryInfC{z$_2$ $\in$ A[$\Gamma$, z$_1$ $\in$ A, z$_2$ $\in$ A]}
\end{prooftree}


\vspace{0.5cm}
\noindent
\normalsize
\textbf{2}
\small
\begin{prooftree}
\AxiomC{}
\UnaryInfC{A type[$\Gamma$]}
\LeftLabel{F-c}\RightLabel{(x $\in$ A) $\notin$ $\Gamma$}
\UnaryInfC{$\Gamma$, x $\in$ A cont}
\LeftLabel{var}
\UnaryInfC{x $\in$ A[$\Gamma$, x $\in$ A]}
\LeftLabel{I-id}
\UnaryInfC{{id(x) $\in$ Id(A,x,x)[$\Gamma$, x $\in$ A]}}
\end{prooftree}
\noindent
\normalsize
\textit{Le derivazioni \textbf{1$_B$} e \textbf{1$_C$} sono simili a quelle in \textbf{1$_A$}. Le ometto per evitare una sovradimensionalit\`a del numero di derivazioni.}\\\\
\noindent
\textbf{Verifico che Id(A, z$_1$ $\in$ A, z$_2$ $\in$ A)type[$\Gamma$, z$_1$ $\in$ A, z$_2$ $\in$ A] \`e una premessa valida rispetto al giudizio di conclusione Id(A, z$_2$ $\in$ A, z$_1$ $\in$ A)type[$\Gamma$, z$_1$ $\in$ A, z$_2$ $\in$ A, z$_3$ $\in$ Id(A,z$_1$,z$_2$)]}
\\\\
\noindent $\Delta$ $=$ $\Gamma$, z$_1$ $\in$ A, z$_2$ $\in$ A, z$_3$ $\in$ Id(A,z$_1$,z$_2$)\\
\noindent $\theta$ $=$ A, z$_1$ $\in$ A, z$_2$ $\in$ A\\
\scriptsize
\begin{adjustwidth}{-10em}{}
\begin{prooftree}
\AxiomC{($\ast$)}
\UnaryInfC{z$_2$ $\in$ A[$\Delta$]}
\AxiomC{($\ast$)}
\UnaryInfC{z$_1$ $\in$ A[$\Delta$, w $\in$ A]}
\AxiomC{($\ast$)}
\UnaryInfC{w $\in$ A[$\Delta$, w $\in$ A]}
\AxiomC{}
\UnaryInfC{\textbf{Id($\theta$)type[$\Gamma$, z$_1$ $\in$ A, z$_2$ $\in$ A]}}
\AxiomC{($\ast$)}
\UnaryInfC{$\Delta$, w $\in$ A cont}
\LeftLabel{ind-ty}
\BinaryInfC{Id($\theta$)type[$\Delta$, w $\in$ A]}
\LeftLabel{sub-typ}
\BinaryInfC{Id(A, w $\in$ A, z$_2$ $\in$ A)type[$\Delta$, w $\in$ A]}
\LeftLabel{sub-typ}
\BinaryInfC{Id(A, w $\in$ A, z$_1$ $\in$ A)type[$\Delta$, w $\in$ A]}
\LeftLabel{sub-typ}
\BinaryInfC{Id(A, z$_2$ $\in$ A, z$_1$ $\in$ A)type[$\Delta$]}
\end{prooftree}
\end{adjustwidth}
\vspace{0.5cm}
\noindent
\normalsize Con il simbolo ($\ast$) indico quelle derivazioni che terminano su un assioma, che per evitare ripetizioni nella derivazione ho omesso.

\paragraph{2)}
\textbf{Si dimostri che esiste una funzione h$_2$ che mi permetta di dimostrare che il successore preserva l'uguaglianza proposizionale. Si dia la prova che succ(z$_1$) $=$ succ(z$_2$) non appena [z$_1$ $\in$ Nat, z$_2$ $\in$ Nat,  z$_3$ $\in$ Id(Nat,z$_1$,z$_2$)] con Nat type[$\Gamma$] non dipendente.
\begin{center}h$_2$(z$_1$, z$_2$, z$_3$) $\in$ Id(Nat, succ(z$_1$), succ(z$_2$))[z$_1$ $\in$ Nat, z$_2$ $\in$ Nat, z$_3$ $\in$ Id(Nat,z$_1$,z$_2$)]\end{center}}
\noindent
\\\\
\textbf{Soluzione}\\\\
\noindent
M(z$_1$,z$_2$,z$_3$) $\equiv$ Id(Nat,succ(z$_1$),succ(z$_2$))\\
\noindent
h$_2$(z$_1$,z$_2$,z$_3$) $\equiv$ El$_{Id}$(z$_3$,(x).id(succ(x)))\\\\
\noindent
$\Sigma$ $\equiv$ z$_1$ $\in$ Nat, z$_2$ $\in$ Nat, z$_3$ $\in$ Id(Nat,z$_1$,z$_2$)
\noindent

\small
\begin{prooftree}
\AxiomC{\textbf{1}}
\UnaryInfC{Id(Nat,succ(z$_1$),succ(z$_2$)) $\in$ [$\Sigma$]}
\AxiomC{\textbf{2}}
\UnaryInfC{Id(succ(x)) $\in$ Id(Nat,succ(x),succ(x))[x $\in$ Nat]}
\LeftLabel{E-Id$_{dip}$}
\BinaryInfC{h$_2$(z$_1$, z$_2$, z$_3$) $\in$ Id(Nat, succ(z$_1$), succ(z$_2$))[$\Sigma$]}
\end{prooftree}
\vspace{0.5cm}
\noindent
\normalsize
\textbf{1}
\small
\begin{adjustwidth}{4em}{}
\begin{prooftree}
\AxiomC{\textbf{1$_A$}}
\UnaryInfC{Nat type [$\Sigma$]}
\AxiomC{\textbf{1$_B$}}
\UnaryInfC{succ(z$_1$) $\in$ Nat[$\Sigma$]}
\AxiomC{\textbf{1$_C$}}
\UnaryInfC{succ(z$_2$) $\in$ Nat[$\Sigma$]}
\LeftLabel{F-id}
\TrinaryInfC{Id(Nat,succ(z$_1$),succ(z$_2$)) $\in$ [$\Sigma$]}
\end{prooftree}
\end{adjustwidth}

\vspace{0.5cm}
\noindent
\normalsize
\textbf{1$_A$}
\small
\begin{adjustwidth}{-11em}{}
\begin{prooftree}
\AxiomC{}
\UnaryInfC{Nat type [ ]}
\AxiomC{\textbf{1$_A^\backprime$}}
\UnaryInfC{Nat type[z$_1$ $\in$ Nat, z$_2$ $\in$ Nat]}
\AxiomC{\textbf{1$_B^\backprime$}}
\UnaryInfC{z$_1$ $\in$ Nat[z$_1$ $\in$ Nat, z$_2$ $\in$ Nat]}
\AxiomC{\textbf{1$_C^\backprime$}}
\UnaryInfC{z$_2$ $\in$ Nat[z$_1$ $\in$ Nat, z$_2$ $\in$ Nat]}
\LeftLabel{F-Id}
\TrinaryInfC{Id(Nat,z$_1$,z$_2$)type[z$_1$ $\in$ Nat, z$_2$ $\in$ Nat]}
\LeftLabel{F-c}\RightLabel{\begin{tabular}[c]{cc}(z$_3$ $\in$ Id(Nat,z$_1$,z$_2$)) $\notin$ \\ (z$_1$ $\in$ Nat, z$_2$ $\in$ Nat)\end{tabular}}
\UnaryInfC{$\Sigma$ cont}
\LeftLabel{ind-ty}
\BinaryInfC{Nat type [$\Sigma$]}
\end{prooftree}
\end{adjustwidth}


\vspace{0.5cm}
\noindent
\normalsize
\textbf{1$_A^\backprime$}
\small
\begin{adjustwidth}{4em}{}
\begin{prooftree}
\AxiomC{[ ] cont}
\LeftLabel{F-Nat}
\UnaryInfC{Nat type [ ]}
\AxiomC{[ ] cont}
\LeftLabel{F-Nat}
\UnaryInfC{Nat type [ ]}
\AxiomC{[ ] cont}
\LeftLabel{F-Nat}
\UnaryInfC{Nat type[ ]}
\LeftLabel{F-c}\RightLabel{(z$_1$ $\in$ Nat) $\notin$ [ ]}
\UnaryInfC{z$_1$ $\in$ Nat cont}
\LeftLabel{ind-ty}
\BinaryInfC{Nat type[z$_1$ $\in$ Nat]}
\LeftLabel{F-c}\RightLabel{(z$_2$ $\in$ Nat) $\notin$ z$_1$ $\in$ Nat}
\UnaryInfC{z$_1$ $\in$ Nat, z$_2$ $\in$ Nat cont}
\LeftLabel{ind-ty}
\BinaryInfC{Nat type[z$_1$ $\in$ Nat, z$_2$ $\in$ Nat]}
\end{prooftree}
\end{adjustwidth}

\noindent
\normalsize
\textbf{1$_B^\backprime$}
\small
\begin{adjustwidth}{-6em}{}
\begin{prooftree}
\AxiomC{[ ] cont}
\LeftLabel{F-Nat}
\UnaryInfC{Nat type [ ]}
\AxiomC{[ ] cont}
\LeftLabel{F-Nat}
\UnaryInfC{Nat type[ ]}
\LeftLabel{F-c}\RightLabel{(z$_2$ $\in$ Nat) $\notin$ [ ]}
\UnaryInfC{z$_2$ $\in$ Nat cont}
\LeftLabel{ind-ty}
\BinaryInfC{Nat type[z$_2$ $\in$ Nat]}
\LeftLabel{F-c}\RightLabel{(z$_1$ $\in$ Nat) $\notin$ z$_2$ $\in$ Nat}
\UnaryInfC{z$_2$ $\in$ Nat, z$_1$ $\in$ Nat cont}
\LeftLabel{var}
\UnaryInfC{z$_1$ $\in$ Nat [z$_2$ $\in$ Nat, z$_1$ $\in$ Nat]}
\AxiomC{[ ] cont}
\LeftLabel{F-Nat}
\UnaryInfC{Nat type[ ]}
\LeftLabel{F-c}\RightLabel{(z$_1$ $\in$ Nat) $\notin$ [ ]}
\UnaryInfC{z$_1$ $\in$ Nat cont}
\LeftLabel{F-Nat}
\UnaryInfC{Nat type[z$_1$ $\in$ Nat]}
\LeftLabel{F-c}\RightLabel{(z$_2$ $\in$ Nat) $\notin$ z$_1$ $\in$ Nat}
\UnaryInfC{z$_1$ $\in$ Nat, z$_2$ $\in$ Nat cont}
\LeftLabel{ex-te}
\BinaryInfC{z$_1$ $\in$ Nat [z$_1$ $\in$ Nat, z$_2$ $\in$ Nat]}
\end{prooftree}
\end{adjustwidth}

\noindent
\normalsize
\textbf{1$_C^\backprime$}
\small
\begin{adjustwidth}{6em}{}
\begin{prooftree}
\AxiomC{[ ] cont}
\LeftLabel{F-Nat}
\UnaryInfC{Nat type [ ]}
\AxiomC{[ ] cont}
\LeftLabel{F-Nat}
\UnaryInfC{Nat type[ ]}
\LeftLabel{F-c}\RightLabel{(z$_1$ $\in$ Nat) $\notin$ [ ]}
\UnaryInfC{z$_1$ $\in$ Nat cont}
\LeftLabel{ind-ty}
\BinaryInfC{Nat type[z$_1$ $\in$ Nat]}
\LeftLabel{F-c}\RightLabel{(z$_2$ $\in$ Nat) $\notin$ z$_1$ $\in$ Nat}
\UnaryInfC{z$_1$ $\in$ Nat, z$_2$ $\in$ Nat cont}
\LeftLabel{var}
\UnaryInfC{z$_2$ $\in$ Nat[z$_1$ $\in$ Nat, z$_2$ $\in$ Nat]}
\end{prooftree}
\end{adjustwidth}



\vspace{0.5cm}
\noindent
\normalsize
\textbf{2}
\small
\begin{prooftree}
\AxiomC{[ ] cont}
\LeftLabel{F-Nat}
\UnaryInfC{Nat type[ ]}
\LeftLabel{F-c}\RightLabel{(x $\in$ Nat) $\notin$ [ ]}
\UnaryInfC{x $\in$ Nat cont}
\LeftLabel{var}
\UnaryInfC{x $\in$ Nat[x $\in$ Nat]}
\LeftLabel{I$_2$-Nat}
\UnaryInfC{succ(x) $\in$ Nat[x $\in$ Nat]}
\LeftLabel{I-id}
\UnaryInfC{Id(succ(x)) $\in$ Id(Nat,succ(x),succ(x))[x $\in$ Nat]}
\end{prooftree}
\noindent
\normalsize
\textit{Le derivazioni \textbf{1$_B$} e \textbf{1$_C$} sono simili a quelle in \textbf{1$_A$}. Le ometto per evitare una sovradimensionalit\`a del numero di derivazioni.}

\paragraph{3)}
\textbf{Si dimostri che se a $=$ b $\in$ A[$\Gamma$] \`e derivabile nella teoria dei tipi, con le regole finora introdotte, allora esiste un \textit{proof-term} tale che \begin{center}pf $\in$ Id(A,a,b)\end{center}\noindent \`e derivabile.}\\\\
\noindent
\textbf{Soluzione} (\textit{Ricalca la soluzione del lemma in \S\ref{sec: lemma-ind}})\\\\
\noindent
Un \textbf{pf} \`e un qualsiasi elemento di un qualsiasi tipo di uguaglianza proposizionale. Per essere conforme con le regole fornite in \S\ref{cap: uguaglianza-proposizionale} definisco pf $\equiv$ id(a).

\small
\begin{adjustwidth}{-4em}{}
\begin{prooftree}

\AxiomC{}
\UnaryInfC{a $=$ b $\in$ A[$\Gamma$]}
\LeftLabel{s-checks}
\UnaryInfC{a $\in$ A[$\Gamma$]}
\LeftLabel{I-Id}
\UnaryInfC{id(a) $\in$ Id(A,a,a)[$\Gamma$]}

\AxiomC{}
\UnaryInfC{a $=$ b $\in$ A[$\Gamma$]}
\LeftLabel{s-checks}
\UnaryInfC{a $\in$ A[$\Gamma$]}
\LeftLabel{s-checks}
\UnaryInfC{A type[$\Gamma$]}
\LeftLabel{ref}
\UnaryInfC{A $=$ A type[$\Gamma$]}

\AxiomC{}
\UnaryInfC{a $=$ b $\in$ A[$\Gamma$]}
\LeftLabel{s-checks}
\UnaryInfC{a $\in$ A[$\Gamma$]}
\LeftLabel{ref}
\UnaryInfC{a $=$ a $\in$ A[$\Gamma$]}

\AxiomC{}
\UnaryInfC{a $=$ b $\in$ A[$\Gamma$]}

\LeftLabel{eq-F-Id}
\TrinaryInfC{Id(A,a,a) $=$ Id(A,a,b)type[$\Gamma$]}
\LeftLabel{conv}
\BinaryInfC{id(a) $\in$ Id(A,a,b)[$\Gamma$]}
\end{prooftree}
\end{adjustwidth}

\paragraph{4)}
\textbf{Si dimostri che esiste un \textit{proof-term} pf del tipo
\begin{center} pf $\in$ Id(N$_1$,x,$\ast$)[x $\in$ N$_1$] \end{center}}
\noindent
\\\\
\textbf{Soluzione}\\\\
\noindent
Uso l'eliminatore dipendente del tipo N$_1$\\\\\
z $\equiv$ x\\
El$_{N1}$(z,c) $\equiv$ pf $\equiv$ El$_{N1}$(x,x.id($\ast$))\\
M(z) $\equiv$ Id(N$_1$,x,$\ast$)\\
c $\in$ M($\ast$)[$\Gamma$] $\equiv$ id($\ast$) $\in$ Id(N$_1$,$\ast$,$\ast$)[x $\in$ N$_1$]\\

\small
\begin{adjustwidth}{-4em}{}
\begin{prooftree}

\AxiomC{}
\UnaryInfC{N$_1$ type[x $\in$ N$_1$]}
\AxiomC{}
\UnaryInfC{x $\in$ N$_1$[x $\in$ N$_1$]}
\AxiomC{}
\UnaryInfC{$\ast$ $\in$ N$_1$[x $\in$ N$_1$]}
\LeftLabel{F-Id}
\TrinaryInfC{Id(N$_1$,x,$\ast$)type[x $\in$ N$_1$]}

\AxiomC{[ ] cont}
\LeftLabel{I-S}
\UnaryInfC{$\ast$ $\in$ N$_1$[ ]}
\LeftLabel{I-Id}
\UnaryInfC{id($\ast$) $\in$ Id(N$_1$,$\ast$,$\ast$)[ ]}
\LeftLabel{E-S$_{dip}$}
\BinaryInfC{El$_{N1}$(x,x.id($\ast$)) $\in$ Id(N$_1$,x,$\ast$)[x $\in$ N$_1$]}
\end{prooftree}
\end{adjustwidth}

\paragraph{5)}
\textbf{Si dimostri che esiste un \textit{proof-term} pf tale che sia possibile definire l'addizione tra numeri naturali
\begin{center} x $+$ y$ \in$ Nat[x $\in$ Nat,y $\in$ Nat] \end{center} 
in modo tale che esistano dei \textit{proof-term} pf$_1$ e pf$_2$ tali che
\begin{center}
pf$_1$ $\in$ Id(Nat,x+0,x)[x $\in$ Nat]
\quad
pf$_2$ $\in$ Id(Nat,0+x,x)[x $\in$ Nat]
\end{center}
}
\noindent
\\\\
\textbf{Soluzione}\\\\
\begin{enumerate}
\item pf$_1$ $\in$ Id(Nat,x+0,x)[x $\in$ Nat]\\\\
$\forall_x \in Nat$ Id(Nat,x+0,x)[x $\in$ Nat] $\equiv$ $\lambda$x.id(x)\\
\noindent
Id(Nat,x+0,x)[x $\in$ Nat] ${\overset{\mathit{def}}{\equiv}}$ {\scriptsize $\prod$}$_{x \in Nat}$Id(Nat,x+0,x)[x $\in$ Nat]\\

\scriptsize
\begin{adjustwidth}{-17.5em}{}
\begin{prooftree}

\AxiomC{[ ] cont}
\LeftLabel{F-Nat}
\UnaryInfC{Nat type [ ]}
\LeftLabel{F-c}\RightLabel{(x $\in$ Nat) $\notin$ [ ]}
\UnaryInfC{x $\in$ Nat cont}
\LeftLabel{var}
\UnaryInfC{x $\in$ Nat[x $\in$ Nat]}
\LeftLabel{I-Id}
\UnaryInfC{id(x) $\in$ Id(Nat,x,x)[x $\in$ Nat]}

\AxiomC{\textbf{($\ast$)}}
\UnaryInfC{Nat type[ ]}
\AxiomC{\textbf{($\ast$)}}
\UnaryInfC{x $\in$ Nat cont}
\LeftLabel{ind-ty}
\BinaryInfC{Nat type[x $\in$ Nat]}
\LeftLabel{ref}
\UnaryInfC{Nat $=$ Nat type[x $\in$ Nat]}

\AxiomC{\textbf{1}}
\UnaryInfC{x $=$ x+0 $\in$ Nat [x $\in$ Nat]}


\AxiomC{\textbf{($\ast$)}}
\UnaryInfC{x $\in$ Nat [x $\in$ Nat]}
\LeftLabel{ref}
\UnaryInfC{x $=$ x $\in$ Nat [x $\in$ Nat]}
\LeftLabel{eq-F-Id}
\TrinaryInfC{Id(Nat,x,x) $=$ Id(Nat,x+0,x) type[x $\in$ Nat]}
\LeftLabel{conv}
\BinaryInfC{id(x) $\in$ Id(Nat,x+0,x)[x $\in$ Nat]}
\LeftLabel{I-\scriptsize $\prod$}
\UnaryInfC{$\lambda$x.id(x) $\in$ {\scriptsize $\prod$}$_{x \in Nat}$Id(Nat,x+0,x)[x $\in$ Nat]}
\end{prooftree}
\end{adjustwidth}
\vspace{0.5cm}
\noindent
\normalsize
\textbf{1}\\\\
\noindent
El$_{Nat}$(0,c,e) $\equiv$ El$_{Nat}$(0,x,(w,z).succ(z))\\
c $\in$ M(0)[$\Gamma$] $\equiv$ x $\in$ Nat[x $\in$ Nat]\\\\
\noindent
$\Delta$ $\equiv$ x $\in$ Nat, w $\in$ Nat, z $\in$ Nat\\
$\Delta^\backprime$ $\equiv$ z $\in$ Nat, x $\in$ Nat, w $\in$ Nat
\small
\begin{adjustwidth}{-12em}{}
\begin{prooftree}
\AxiomC{\textbf{($\ast$)}}
\UnaryInfC{Nat type [x $\in$ Nat]}
\AxiomC{\textbf{($\ast$)}}
\UnaryInfC{x $\in$ Nat [x $\in$ Nat]}

\AxiomC{[ ] cont}
\LeftLabel{I$_2$-Nat$_{prog}$}
\UnaryInfC{succ(z) $\in$ Nat [z $\in$ Nat]}
\AxiomC{\textbf{$\bigstar$}}
\UnaryInfC{$\Delta^\backprime$ cont}
\LeftLabel{ind-te}
\BinaryInfC{succ(z) $\in$ Nat [$\Delta^\backprime$]}

\AxiomC{\textbf{$\bigstar$}}
\UnaryInfC{$\Delta$}
\LeftLabel{ex-te}
\BinaryInfC{succ(z) $\in$ Nat [$\Delta$]}
\LeftLabel{C$_1$-Nat}
\TrinaryInfC{x+0 $=$ x $\in$ Nat [x $\in$ Nat]}
\LeftLabel{sym}
\UnaryInfC{x $=$ x+0 $\in$ Nat [x $\in$ Nat]}
\end{prooftree}
\end{adjustwidth}
\noindent
\normalsize
\\\\
\noindent Ho usato \textbf{($\ast$)} per concludere le derivazioni gi\`a svolte all'interno dell'albero.\\
\textbf{$\heartsuit$} derivazione gi\`a risolta in \S\ref{sec: es-naturali}, esercizio 3 (necessaria applicazione di \textit{$\alpha$-eq}).\\
\textbf{$\bigstar$} derivazione gi\`a risolte negli esercizi precedenti, Prevede una combinazioni di istruzioni di indebolimento/assunzione di variabili/formazione di contesto per verificare l'assioma [ ] cont.\\

\noindent
\item pf$_2$ $\in$ Id(Nat,0+x,x)[x $\in$ Nat]\\\\
\noindent
Uso l'eliminatore dipendente del tipo Nat.\\
Svolgo la ricorsione su x (altrimenti su y cambierei il contesto e "uscirei" dalle richieste dell'esercizio), dunque
\begin{itemize}
\item Premesse:
\begin{itemize}
\item M(z) type[$\Gamma$,z $\in$ Nat] $\equiv$ Id(Nat,0+x,x)[x $\in$ Nat]
\item c $\in$ M(0)[$\Gamma$] $\equiv$ id(0) $\in$ Id(Nat,0+0,0)[ ]
\item M(z) $\equiv$ M(x) per $\alpha$-eq (dimostrazione in \S\ref{sec: es-naturali} esercizio 5)
\item e(x,y) $\in$ M(succ(x))[$\Gamma$,x $\in$ Nat,y $\in$ M(x)] $\equiv$ h(0+x,x,y) $\in$ Id(Nat,0+succ(x),succ(x))[$\Gamma$,x $\in$ Nat,y $\in$ Id(Nat,0+x,x)]
\end{itemize}
\item Conclusioni:
\begin{itemize}
\item El$_{Nat}$(z,c,e) $\equiv$ El$_{Nat}$(x,id(0),(z,v).h(0+z,z,v))
\item M(z)[$\Gamma$,z $\in$ Nat] $\equiv$ Id(Nat,0+x,x)[x $\in$ Nat]
\end{itemize}
\end{itemize}


\small
\begin{adjustwidth}{-11.5em}{}
\begin{prooftree}

\AxiomC{\textbf{1}}
\UnaryInfC{Id(Nat,0+x,x)type[x $\in$ Nat]}
\AxiomC{\textbf{2}}
\UnaryInfC{id(0) $\in$ Id(Nat,0+0,0)[ ]}
\AxiomC{\textbf{3}}
\UnaryInfC{h(z+0,z,v) $\in$ Id(Nat,0+succ(z),succ(z))[z $\in$ Nat,v $\in$ Id(Nat,0+z,z)]}
\LeftLabel{E-Nat$_{dip}$}
\TrinaryInfC{El$_{Nat}$(x,id(0),(z,v).h(0+z,z,v))Id(Nat,0+x,x)[x $\in$ Nat]}
\end{prooftree}
\end{adjustwidth}
\vspace{0.5cm}
\normalsize \textbf{1}
\small
\begin{prooftree}

\AxiomC{[ ] cont}
\LeftLabel{F-Nat}
\UnaryInfC{Nat type [ ]}
\LeftLabel{F-c}\RightLabel{(x $\in$ Nat) $\notin$ [ ]}
\UnaryInfC{x $\in$ Nat cont}
\LeftLabel{F-Nat}
\UnaryInfC{Nat type[x $\in$ Nat]}

\AxiomC{\textbf{$\heartsuit$}}
\UnaryInfC{0+x $\in$ Nat[x $\in$ Nat]}

\AxiomC{\textbf{($\ast$)}}
\UnaryInfC{x $\in$ Nat cont}
\LeftLabel{var}
\UnaryInfC{x $\in$ Nat[x $\in$ Nat]}
\LeftLabel{F-Id}
\TrinaryInfC{Id(Nat,0+x,x)type[x $\in$ Nat]}
\end{prooftree}


\vspace{0.5cm}
\normalsize \textbf{2}
\small
\begin{adjustwidth}{-5em}{}
\begin{prooftree}

\AxiomC{[ ] cont}
\LeftLabel{I-Nat}
\UnaryInfC{0 $\in$ Nat[ ]}
\LeftLabel{I-Id}
\UnaryInfC{id(0) $\in$ Id(Nat,0,0)[ ]}


\AxiomC{[ ] cont}
\LeftLabel{F-Nat}
\UnaryInfC{Nat type[ ]}
\LeftLabel{ref}
\UnaryInfC{Nat $=$ Nat type[ ]}

\AxiomC{\textbf{2$^\backprime$}}
\UnaryInfC{0 $=$ 0+0 $\in$ Nat[ ]}

\AxiomC{\textbf{($\ast$)}}
\UnaryInfC{0 $\in$ Nat[ ]}
\LeftLabel{ref}
\UnaryInfC{0 $=$ 0 $\in$ Nat[ ]}
\LeftLabel{eq-F-Id}
\TrinaryInfC{Id(Nat,0,0) $=$ Id(Nat,0+0,0)[ ]}
\LeftLabel{conv}
\BinaryInfC{id(0) $\in$ Id(Nat,0+0,0)[ ]}
\end{prooftree}
\end{adjustwidth}
\normalsize

\vspace{0.5cm}
\normalsize \textbf{2$^\backprime$}\\\\
El$_{Nat}$(0,c,e) $\equiv$ El$_{Nat}$(0,0,(w,x).succ(x))\\
c $\in$ M(0)[$\Gamma$] $\equiv$ 0 $\in$ Nat[ ]
\small
\begin{adjustwidth}{-4em}{}
\begin{prooftree}
\AxiomC{[ ] cont}
\LeftLabel{F-Nat}
\UnaryInfC{Nat type[ ]}

\AxiomC{[ ] cont}
\LeftLabel{I-Nat}
\UnaryInfC{0 $\in$ Nat[ ]}

\AxiomC{\textbf{($\ast$)}}
\UnaryInfC{Nat type[ ]}
\LeftLabel{F-c}\RightLabel{(w $\in$ Nat) $\notin$ [ ]}
\UnaryInfC{w $\in$ Nat cont}
\LeftLabel{F-Nat}
\UnaryInfC{Nat type[w $\in$ Nat]}
\LeftLabel{F-c}\RightLabel{(x $\in$ Nat) $\notin$ w $\in$ Nat}
\UnaryInfC{w $\in$ Nat,x $\in$ Nat cont}
\LeftLabel{var}
\UnaryInfC{x $\in$ Nat[w $\in$ Nat,x $\in$ Nat]}
\LeftLabel{I$_2$-Nat}
\UnaryInfC{succ(x) $\in$ Nat[w $\in$ Nat,x $\in$ Nat]}
\LeftLabel{C$_1$-Nat}
\TrinaryInfC{0+0 $=$ 0 $\in$ Nat[ ]}
\LeftLabel{sym}
\UnaryInfC{0 $=$ 0+0 $\in$ Nat[ ]}
\end{prooftree}
\end{adjustwidth}


\normalsize \textbf{3}\\\\
$\Delta$ $\equiv$ z $\in$ Nat,v $\in$ Id(Nat,0+z,z)
\small
\begin{adjustwidth}{-8em}{}
\begin{prooftree}
\AxiomC{\textbf{$\diamondsuit$}}
\UnaryInfC{h(0+z,z,v) $\in$ Id(Nat,succ(0+z),succ(z))[$\Delta$]}
\AxiomC{\textbf{3$^\backprime$}}
\UnaryInfC{Id(Nat,succ(0+z),succ(z)) $=$ Id(Nat,0+succ(z),succ(z))[$\Delta$]}
\LeftLabel{conv}
\BinaryInfC{h(0+z,z,v) $\in$ Id(Nat,0+succ(z),succ(z))[$\Delta$]}
\end{prooftree}
\end{adjustwidth}

\vspace{0.5cm}
\normalsize \textbf{3$^\backprime$}\\\\
\small
\begin{adjustwidth}{-14em}{}
\begin{prooftree}
\AxiomC{\textbf{3$^\backprime_A$}}
\UnaryInfC{Id(Nat,0+z,z)type[z $\in$ Nat]}
\LeftLabel{F-c}\RightLabel{\begin{tabular}[c]{cc}(y $\in$ Id(Nat,0+z,z))\\ $\notin$ z $\in$ Nat\end{tabular}}
\UnaryInfC{$\Delta$ cont}
\LeftLabel{F-Nat}
\UnaryInfC{Nat type[$\Delta$]}
\LeftLabel{ref}
\UnaryInfC{Nat $=$ Nat type[$\Delta$]}

\AxiomC{\textbf{3$^\backprime_B$}}
\UnaryInfC{succ(0+z) $=$ 0+succ(z) $\in$ Nat[$\Delta$]}


\AxiomC{\textbf{$\bigstar$}}
\UnaryInfC{z $\in$ Nat[$\Delta$]}
\LeftLabel{I$_2$-Nat}
\UnaryInfC{succ(z) $\in$ Nat[$\Delta$]}
\LeftLabel{ref}
\UnaryInfC{succ(z) $=$ succ(z) $\in$ Nat[$\Delta$]}
\LeftLabel{eq-F-Id}
\TrinaryInfC{Id(Nat,succ(0+z),succ(z)) $=$ Id(Nat,0+succ(z),succ(z))[$\Delta$]}
\end{prooftree}
\end{adjustwidth}

\vspace{0.5cm}
\normalsize \textbf{3$^\backprime_A$}\\\\
\small
\begin{adjustwidth}{-5em}{}
\begin{prooftree}
\AxiomC{[ ] cont}
\LeftLabel{F-Nat}
\UnaryInfC{Nat type[ ]}
\LeftLabel{F-c}\RightLabel{(z $\in$ Nat) $\notin$ [ ]}
\UnaryInfC{z $\in$ Nat cont}
\LeftLabel{F-Nat}
\UnaryInfC{Nat type[z $\in$ Nat]}

\AxiomC{\textbf{$\heartsuit$}}
\UnaryInfC{0+z $\in$ Nat[x $\in$ Nat]}

\AxiomC{\textbf{($\ast$)}}
\UnaryInfC{z $\in$ Nat cont}
\LeftLabel{var}
\UnaryInfC{z $\in$ Nat[z $\in$ Nat]}
\LeftLabel{F-Id}
\TrinaryInfC{Id(Nat,0+z,z)type[z $\in$ Nat]}

\end{prooftree}
\end{adjustwidth}


\vspace{0.5cm}
\normalsize \textbf{3$^\backprime_B$}\\\\
El$_{Nat}$(0,c,e) $\equiv$ El$_{Nat}$(0,succ(0+z),(w,u).succ(u))\\
c $\in$ M(0)[$\Gamma$] $\equiv$ succ(0+z) $\in$ Nat[$\Delta$]
\small
\begin{adjustwidth}{-10em}{}
\begin{prooftree}

\AxiomC{\textbf{$\bigstar$}}
\UnaryInfC{Nat type[$\Delta$]}

\AxiomC{\textbf{$\heartsuit$}}
\UnaryInfC{0+z $\in$ Nat[z $\in$ Nat]}
\AxiomC{\textbf{$\bigstar$}}
\UnaryInfC{$\Delta$ cont}
\LeftLabel{ind-te}
\BinaryInfC{0+z $\in$ Nat[$\Delta$]}
\LeftLabel{I$_2$-Nat}
\UnaryInfC{succ(0+z) $\in$ Nat[$\Delta$]}

\AxiomC{\textbf{$\bigstar$}}
\UnaryInfC{$\Delta$, w $\in$ Nat, u $\in$ Nat cont}
\LeftLabel{var}
\UnaryInfC{u $\in$ Nat[$\Delta$, w $\in$ Nat, u $\in$ Nat]}
\LeftLabel{I$_2$-Nat}
\UnaryInfC{succ(u) $\in$ Nat[$\Delta$, w $\in$ Nat, u $\in$ Nat]}

\LeftLabel{C$_1$-Nat}
\TrinaryInfC{0+succ(z)  $=$ succ(0+z) $\in$ Nat[$\Delta$]}
\LeftLabel{sym}
\UnaryInfC{succ(0+z) $=$ 0+succ(z) $\in$ Nat[$\Delta$]}

\end{prooftree}
\end{adjustwidth}
\vspace{0.5cm}
\normalsize
\noindent Ho usato \textbf{($\ast$)} per concludere le derivazioni gi\`a svolte all'interno dell'albero.\\
\textbf{$\heartsuit$} derivazione gi\`a risolta in \S\ref{sec: es-naturali}, esercizio 3 (necessaria applicazione di \textit{$\alpha$-eq}).\\
\textbf{$\diamondsuit$} derivazione gi\`a risolta nell'esercizio 2, di questa sezione (necessaria applicazione di \textit{$\alpha$-eq}).\\
\textbf{$\bigstar$} derivazione gi\`a risolte negli esercizi precedenti, Prevede una combinazioni di istruzioni di indebolimento/assunzione di variabili/formazione di contesto per verificare l'assioma [ ] cont.\\

\end{enumerate}

%\paragraph{6)}
%\textbf{Si dimostri che esiste un \textit{proof-term} del tipo
%\begin{center} pf $\in$ Id(A,x,z)[x $\in$ A,y $\in$ A,z $\in$ A,w$_1$ $\in$ Id(A,x,y),w$_2$ $\in$ Id(A,y,z)] \end{center}}
