\chapter{Uguaglianza proposizionale}
\label{cap: uguaglianza-proposizionale}
%% lezione 16 e 17 inclusa
Pi\`u modi permettono di descrivere il tipo dell'uguaglianza, uno di questi \`e il tipo dell'uguaglianza proposizionale (o identit\`a proposizionale) di \textit{Martin-L$\ddot{o}$f}. Questi \`e dipendente in modo primitivo e definito dalle regole seguenti.

\section{Regole di Formazione}
\label{subsec: formazione-id}
\begin{prooftree}
\AxiomC{A type [$\Gamma$]}
\AxiomC{a $\in$ A[$\Gamma$]}
\AxiomC{b $\in$ A[$\Gamma$]}
\LeftLabel{F-Id)}
\TrinaryInfC{Id(A,a,b) type[$\Gamma$]}
\end{prooftree}

\section{Regole di Introduzione}
\label{subsec: introduzione-id}
\begin{prooftree}
\AxiomC{a $\in$ A[$\Gamma$]}
\LeftLabel{I-Id)}
\UnaryInfC{id(a) $\in$ Id(A,a,a)[$\Gamma$]}
\end{prooftree}

\section{Regole di Eliminazione}
\label{subsec: eliminazione-id}
\small
\begin{prooftree}
\AxiomC{\begin{tabular}[c]{cccc}M(z$_1$,z$_2$,z$_3$) type[$\Gamma$, z$_1$ $\in$ A, z$_2$ $\in$ A, z$_3$ $\in$ Id(A, z$_1$, z$_2$)]\\a $\in$ A[$\Gamma$] \\b $\in$ A[$\Gamma$] \\ t $\in$ Id(A,a,b)[$\Gamma$]\end{tabular}}
\AxiomC{e(x) $\in$ M(x,x,Id(x))[$\Gamma$, x $\in$ A]}
\LeftLabel{E-Id)}
\BinaryInfC{El$_{Id}$(t, (x).e(x)) $\in$ M(a,b,t)[$\Gamma$]}
\end{prooftree}

\section{Regole di Conservazione}
\label{subsec: conservazione-id}
\small
\begin{prooftree}
\AxiomC{\begin{tabular}[c]{cc}M(z$_1$,z$_2$,z$_3$) type[$\Gamma$, z$_1$ $\in$ A, z$_2$ $\in$ A, z$_3$ $\in$ Id(A, z$_1$, z$_2$)] \\ a $\in$ A[$\Gamma$]\end{tabular}}
\AxiomC{e(x) $\in$ M(x,x,Id(x))[$\Gamma$, x $\in$ A]}
\LeftLabel{E-Id)}
\BinaryInfC{El$_{Id}$(Id(a), (x).e(x)) = e(a) $\in$ M(a,a,Id(a))[$\Gamma$]}
\end{prooftree}


\section{Regole di Uguaglianza}
\label{subsec: uguaglianza-liste}
\normalsize
\AxiomC{A$_1$ = A$_2$ type[$\Gamma$]}
\AxiomC{a$_1$ = a$_2$ $\in$ A[$\Gamma$]}
\AxiomC{b$_1$ = b$_2$ $\in$ A[$\Gamma$]}
\LeftLabel{eq-F-Id)}
\TrinaryInfC{Id(A$_1$,a$_1$,b$_1$) = Id(A$_2$,a$_2$,b$_2$) type[$\Gamma$]}
\DisplayProof  \\
\vspace{0.5cm}\\
\AxiomC{a$_1$ = a$_2$ $\in$ A[$\Gamma$]}
\LeftLabel{eq-I-Id)}
\UnaryInfC{Id(a$_1$) = Id(a$_2$) $\in$ Id(A,a$_1$,a$_1$)[$\Gamma$]}
\DisplayProof

\section{Eliminatore dipendente}
\label{subsec:eliminatore dipendente-id}
L'eliminatore ha anche la forma dipendente.
\small
\begin{prooftree}
\AxiomC{M(z$_1$,z$_2$,z$_3$) type[$\Gamma$, z$_1$ $\in$ A, z$_2$ $\in$ A, z$_3$ $\in$ Id(A, z$_1$, z$_2$)]}
\AxiomC{e(x) $\in$ M(x,x,Id(x)))[$\Gamma$, x $\in$ A]}
\LeftLabel{El-Id$_{dip}$)}
\BinaryInfC{El$_{Id}$(z$_3$, (x).e(x)) $\in$ M(z$_1$, z$_2$, z$_3$)[$\Gamma$,z$_1$ $\in$ A, z$_2$ $\in$ A, z$_3$ $\in$ Id(A,z$_1$, z$_2$)]}
\end{prooftree}