\chapter{Tipi Universo \`a la Tarski}
\label{cap:tipo-universo}
Il tipo Universo non \`e induttivo ed \`e rivolto alla teoria dei tipi di  \textit{Martin-L$\ddot{o}$f} intensionale.
\begin{center} U$_0$, U$_1$,..., U$_n$ \qquad con n $\in$ Nat\end{center}.
\begin{center} U$_0$ $\notin$ U$_0$ \qquad U$_0$ $\in$ U$_0$ \`e una contraddizione\end{center}
U$_i$, nel senso di  \textit{Martin-L$\ddot{o}$f},  \`e un contenitore predicativo,  tramite codifica, dei tipi fino costruiti, ed \`e chiuso sulla loro formazione.\\\\
\noindent
Il tipo Universo alla Tarski \`e definito dalle regole seguenti.\\\\
\noindent
\textbf{T} indica la decodifica.
\section{Regole di Formazione}
\label{sec: formazione-U0}
\begin{prooftree}
\AxiomC{$\Gamma$ cont}
\LeftLabel{F-Uno)}
\UnaryInfC{U$_0$ type[$\Gamma$]}
\end{prooftree}

\section{Regole di Introduzione}
\label{sec: introduzione-U0}
\begin{center}
\AxiomC{$\Gamma$ cont}
\LeftLabel{I$_1$-Uno)}
\UnaryInfC{$\hat{N_0}$ $\in$ U$_0$[$\Gamma$]}
\DisplayProof \qquad
\AxiomC{$\Gamma$ cont}
\LeftLabel{I$_2$-Uno)}
\UnaryInfC{$\hat{N_1}$ $\in$ U$_0$[$\Gamma$]}
\DisplayProof
\vspace{0.5cm}
\AxiomC{$\Gamma$ cont}
\LeftLabel{I$_3$-Uno)}
\UnaryInfC{Nat $\in$ U$_0$[$\Gamma$]}
\DisplayProof
\vspace{0.5cm}
\AxiomC{c(x) $\in$ U$_0$[$\Gamma$,x $\in$ \textbf{T}(b)]}
\AxiomC{b $\in$ U$_0$[$\Gamma$]}
\LeftLabel{I$_4$-Uno)}
\BinaryInfC{$\hat{\sum}_{x \in b}$ c(x) $\in$ U$_0$[$\Gamma$]}
\DisplayProof \quad
\AxiomC{c(x) $\in$ U$_0$[$\Gamma$,x $\in$ \textbf{T}(b)]}
\AxiomC{b $\in$ U$_0$[$\Gamma$]}
\LeftLabel{I$_5$-Uno)}
\BinaryInfC{$\hat{\prod}_{x \in b}$ c(x) $\in$ U$_0$[$\Gamma$]}
\DisplayProof \\
\vspace{0.5cm}
\AxiomC{b $\in$ U$_0$[$\Gamma$]}
\AxiomC{c $\in$ U$_0$[$\Gamma$]}
\LeftLabel{I$_6$-Uno)}
\BinaryInfC{b $\hat{+}$ c $\in$ U$_0$[$\Gamma$]}
\DisplayProof \quad
\AxiomC{b $\in$ U$_0$[$\Gamma$]}
\LeftLabel{I$_7$-Uno)}
\UnaryInfC{List(b) $\in$ U$_0$[$\Gamma$]}
\DisplayProof\\
\vspace{0.5cm}
\begin{prooftree}
\AxiomC{b $\in$ U$_0$[$\Gamma$]}
\AxiomC{c $\in$ \textbf{T}(b)[$\Gamma$]}
\AxiomC{d $\in$ \textbf{T}(b)[$\Gamma$]}
\LeftLabel{I$_8$-Uno)}
\TrinaryInfC{Id(b,c,d) $\in$ U$_0$[$\Gamma$]}
\end{prooftree}
\end{center}

\section{Regole di Eliminazione}
\label{sec:eliminzione-U0}
\begin{prooftree}
\AxiomC{a $\in$ U$_0$[$\Gamma$]}
\LeftLabel{E-Uno)}
\UnaryInfC{\textbf{T}(a) type[$\Gamma$]}
\end{prooftree}

\section{Regole di Conversione}
\label{sec:conversione-U0}
\begin{center}
\AxiomC{$\Gamma$ cont}
\LeftLabel{C$_1$-Uno)}
\UnaryInfC{\textbf{T}($\hat{N_0}$) $=$ N$_0$ type[$\Gamma$]}
\DisplayProof \quad
\AxiomC{$\Gamma$ cont}
\LeftLabel{C$_2$-Uno)}
\UnaryInfC{\textbf{T}($\hat{N_1}$) $=$ N$_1$ type[$\Gamma$]}
\DisplayProof\\
\vspace{0.5cm}
\AxiomC{$\Gamma$ cont}
\LeftLabel{C$_3$-Uno)}
\UnaryInfC{\textbf{T}(Nat) $=$ Nat type[$\Gamma$]}
\DisplayProof\\
\vspace{0.5cm}
\AxiomC{c(x) $\in$ U$_0$[$\Gamma$,x $\in$ \textbf{T}(b)]}
\AxiomC{b $\in$ U$_0$[$\Gamma$]}
\LeftLabel{C$_4$-Uno)}
\BinaryInfC{\textbf{T}($\hat{\sum}_{x \in b} c(x)$ $=$ ($\sum$ x $\in$ \textbf{T}(b))T(c(x)) type[$\Gamma$]}
\DisplayProof\\
\vspace{0.5cm}
\AxiomC{c(x) $\in$ U$_0$[$\Gamma$,x $\in$ \textbf{T}(b)]}
\AxiomC{b $\in$ U$_0$[$\Gamma$]}
\LeftLabel{C$_5$-Uno)}
\BinaryInfC{\textbf{T}($\hat{\prod}_{x \in b} c(x)$ $=$ ($\prod$ x $\in$ \textbf{T}(b))T(c(x)) type[$\Gamma$]}
\DisplayProof\\
\end{center}
\vspace{0.5cm}
\AxiomC{b $\in$ U$_0$[$\Gamma$]}
\AxiomC{c $\in$ U$_0$[$\Gamma$]}
\LeftLabel{C$_6$-Uno)}
\BinaryInfC{\textbf{T}(b $\hat{+}$ c) $=$ \textbf{T}(b) $+$ \textbf{T}(c) type[$\Gamma$]}
\DisplayProof \quad
\AxiomC{b $\in$ U$_0$[$\Gamma$]}
\LeftLabel{C$_7$-Uno)}
\UnaryInfC{\textbf{T}(List(b)) $=$ List(\textbf{T}(b)) type[$\Gamma$]}
\DisplayProof
\begin{center}
\vspace{0.5cm}
\AxiomC{b $\in$ U$_0$[$\Gamma$]}
\AxiomC{c $\in$ \textbf{T}(b)[$\Gamma$]}
\AxiomC{d $\in$ \textbf{T}(b)[$\Gamma$]}
\LeftLabel{C$_8$-Uno)}
\TrinaryInfC{\textbf{T}(Id(b,c,d)) $=$ Id(\textbf{T}(b),c,d) type[$\Gamma$]}
\DisplayProof
\end{center}

\section{Regole di Uguaglianza}
\label{subsec: uguaglianza-U0}
\begin{center}
\vspace{0.5cm}
\AxiomC{c$_1$(x) $=$ c$_2$(x) $\in$ U$_0$[$\Gamma$,x $\in$ \textbf{T}(b)]}
\AxiomC{b$_1$ $=$ b$_2$ $\in$ U$_0$[$\Gamma$]}
\LeftLabel{eq-I$_4$-Uno)}
\BinaryInfC{$\hat{\sum}_{x \in b_1}$ c$_1$(x) $=$ $\hat{\sum}_{x \in b_2}$ c$_2$(x) $\in$ U$_0$[$\Gamma$]}
\DisplayProof \\ \vspace{0.5cm}
\AxiomC{c$_1$(x) $=$ c$_2$(x) $\in$ U$_0$[$\Gamma$,x $\in$ \textbf{T}(b)]}
\AxiomC{b$_1$ $=$ b$_2$ $\in$ U$_0$[$\Gamma$]}
\LeftLabel{eq-I$_5$-Uno)}
\BinaryInfC{$\hat{\prod}_{x \in b_1}$ c$_1$(x) $=$ $\hat{\prod}_{x \in b_2}$ c$_2$(x) $\in$ U$_0$[$\Gamma$]}
\DisplayProof \\
\vspace{0.5cm}
\AxiomC{b$_1$ $=$ b$_2$ $\in$ U$_0$[$\Gamma$]}
\AxiomC{c$_1$ $=$ c$_2$ $\in$ U$_0$[$\Gamma$]}
\LeftLabel{eq-I$_6$-Uno)}
\BinaryInfC{b$_1$ $\hat{+}$ c$_1$ $=$ b$_2$ $\hat{+}$ c$_2$ $\in$ U$_0$[$\Gamma$]}
\DisplayProof \\ \vspace{0.5cm}
\AxiomC{b$_1$ $=$ b$_2$ $\in$ U$_0$[$\Gamma$]}
\LeftLabel{eq-I$_7$-Uno)}
\UnaryInfC{List(b$_1$) $=$ List(b$_2$) $\in$ U$_0$[$\Gamma$]}
\DisplayProof\\
\vspace{0.5cm}
\begin{prooftree}
\AxiomC{b$_1$ $=$ b$_2$ $\in$ U$_0$[$\Gamma$]}
\AxiomC{c$_1$ $=$ c$_2$ $\in$ \textbf{T}(b)[$\Gamma$]}
\AxiomC{d$_1$ $=$ d$_2$ $\in$ \textbf{T}(b)[$\Gamma$]}
\LeftLabel{eq-I$_8$-Uno)}
\TrinaryInfC{Id(b$_1$,c$_1$,d$_1$) $=$ Id(b$_2$,c$_2$,d$_2$) $\in$ U$_0$[$\Gamma$]}
\end{prooftree}
\end{center}

\section{Semantica operazionale tipo Universo}
\label{subsec: semantica-operazionale-U0}
Negli universi normalizzare i termini \`e pi\`u difficile e richiede una relazione di riduzione che coinvolge anche i tipi, vale dunque il \textbf{teorema di forte normalizzazione} (\S \ref{subsec: applicazione-uguaglianza-definizionale-tra-termini}).\\
La relazione $\rightarrow_1$ viene definita all'interno dei termini con l'uso delle seguenti regole di riduzione:
\begin{itemize}
\item $\beta_{Uno}$-red) \textbf{T}($\hat{Nat}$) $\rightarrow_1$ Nat
\item \AxiomC{a$_1$ $\rightarrow$ a$_2$}
\LeftLabel{Uno-red)}
\UnaryInfC{\textbf{T}(a$_1$) $\rightarrow$ \textbf{T}(a$_2$) type[$\Gamma$]}
\DisplayProof \qquad
\item Novit\`a del tipo Universo rispetto al tipo singoletto
\AxiomC{b$_1$ $\rightarrow_1$ b$_2$}
\LeftLabel{$Uno$-I$_6$-red$_I$)}
\UnaryInfC{b$_1$ $\hat{+}$ c $\rightarrow_1$ b$_2$ $\hat{+}$ c}
\DisplayProof
\item \AxiomC{c$_1$ $\rightarrow_1$ c$_2$}
\LeftLabel{$Uno$-I$_6$-red$_{II}$)}
\UnaryInfC{b$\hat{+}$ c$_1$  $\rightarrow_1$ b $\hat{+}$ c$_2$}
\DisplayProof
\item \AxiomC{d$_1$ $\rightarrow_1$ d$_2$}
\LeftLabel{$Uno$-I$_7$-red$_I$)}
\UnaryInfC{List(d$_1$) $\rightarrow_1$ List(d$_2$)}
\DisplayProof
\item \AxiomC{t$_1$ $\rightarrow_1$ t$_2$}
\LeftLabel{$Uno$-I$_8$-red$_I$)}
\UnaryInfC{Id(t$_1$,w,z) $\rightarrow_1$ Id(t$_2$,w,z)}
\DisplayProof
\item \AxiomC{w$_1$ $\rightarrow_1$ w$_2$}
\LeftLabel{$Uno$-I$_8$-red$_{II}$)}
\UnaryInfC{Id(t,w$_1$,z) $\rightarrow_1$ Id(t,w$_2$,z)}
\DisplayProof
\item \AxiomC{z$_1$ $\rightarrow_1$ z$_2$}
\LeftLabel{$Uno$-I$_8$-red$_{III}$)}
\UnaryInfC{Id(t,w,z$_1$) $\rightarrow_1$ Id(t,w,z$_2$)}
\DisplayProof
\end{itemize}




