\chapter{Tipi Universo \`a la Tarski}
\label{cap:tipo-universo}
Il tipo Universo, \textbf{schema di tipi}, non \`e induttivo ed \`e rivolto, in questo caso, alla teoria dei tipi di  \textit{Martin-L$\ddot{o}$f} intensionale.\\
Esiste il tipo Universo \`a la \textit{Tarski} e alla \textit{Russell}. Il primo, che espongo nel capitolo, pi\`u preciso ma pesante nella sintassi, pi\`u facile da capire e da modellare (soprattutto per fare delle semantiche nella teoria dei tipi), per cui migliore quando si programma con un \textit{proof assistant}; e il secondo pi\`u semplice da utilizzare nelle dimostrazioni, ma meno preciso.\\\\
Caratteristica del tipo Universo \`e che permette la ripetizione delle costruzioni per generare una gerarchia di universi, legati fra di loro
\begin{center} U$_0$, U$_1$,..., U$_n$ \qquad con n $\in$ Nat\end{center}
Nella prima versione \textit{Martin-L$\ddot{o}$f} permetteva che U$_0$ appartenesse a U$_0$, generando per\`o una contraddizione.
\textbf{\begin{center} U$_0$ $\notin$ U$_0$ \quad $\rightarrow$ \quad U$_0$ $\in$ U$_0$ \`e una contraddizione\end{center}}
\noindent
Universo \`e un contenitore, di tipi, predicativo, ovvero U$_0$ $\notin$ U$_0$.\\\\
\noindent
\textit{Fondamentale}\\
Essenziale per avere commutativit\`a  \`e non permettere che $\hat{U_0}$ $\in$ U$_0$ (un codice U$_0$ dentro a U$_0$) altrimenti, come gi\`a detto, si ricade nella contraddizione. Per cui si dichiara  come "\textit{small}" o piccoli tutti i tipi con codice in U$_0$; e invece come "\textit{large}" i tipi costruiti con U$_0$, ad esempio U$_0$ $\rightarrow$ U$_0$ type[ ].\\
Ecco che per la natura della regola di Eliminazione e dal fatto che contiene esclusivamente \textit{small} e non se stesso, posso affermare che l'Universo \`e \textbf{predicativo e non induttivo}.\\\\
\noindent
Il tipo del \textbf{Primo Universo (U$_0$) alla Tarski} \`e definito dalle regole seguenti, dove \textbf{T} indica la decodifica.
\section{Regole di Formazione}
\label{sec: formazione-U0}
\begin{prooftree}
\AxiomC{$\Gamma$ cont}
\LeftLabel{F-Uno)}
\UnaryInfC{U$_0$ type[$\Gamma$]}
\end{prooftree}
\textit{U$_0$ (come N$_0$, N$_1$, ecc..) \`e una costante di tipo.}

\section{Regole di Introduzione}
\label{sec: introduzione-U0}
\textit{Per le regole di Introduzione U$_0$ funge da contenitore di tipi, tramite codifica, ed \`e chiuso rispetto ai tipi finora introdotti, nei capitolo precedenti. Inoltre le regole di Introduzione usano quelle di Eliminazione.}
\begin{center}
\AxiomC{$\Gamma$ cont}
\LeftLabel{I$_1$-Uno)}
\UnaryInfC{$\hat{N_0}$ $\in$ U$_0$[$\Gamma$]}
\DisplayProof \quad
\AxiomC{$\Gamma$ cont}
\LeftLabel{I$_2$-Uno)}
\UnaryInfC{$\hat{N_1}$ $\in$ U$_0$[$\Gamma$]}
\DisplayProof \quad
\vspace{0.5cm}
\AxiomC{$\Gamma$ cont}
\LeftLabel{I$_3$-Uno)}
\UnaryInfC{$\hat{Nat}$ $\in$ U$_0$[$\Gamma$]}
\DisplayProof
\vspace{0.5cm}
\AxiomC{c(x) $\in$ U$_0$[$\Gamma$,x $\in$ \textbf{T}(b)]}
\AxiomC{b $\in$ U$_0$[$\Gamma$]}
\LeftLabel{I$_4$-Uno)}
\BinaryInfC{$\hat{\sum}_{x \in b}$ c(x) $\in$ U$_0$[$\Gamma$]}
\DisplayProof \quad
\AxiomC{c(x) $\in$ U$_0$[$\Gamma$,x $\in$ \textbf{T}(b)]}
\AxiomC{b $\in$ U$_0$[$\Gamma$]}
\LeftLabel{I$_5$-Uno)}
\BinaryInfC{$\hat{\prod}_{x \in b}$ c(x) $\in$ U$_0$[$\Gamma$]}
\DisplayProof \\
\vspace{0.5cm}
\AxiomC{b $\in$ U$_0$[$\Gamma$]}
\AxiomC{c $\in$ U$_0$[$\Gamma$]}
\LeftLabel{I$_6$-Uno)}
\BinaryInfC{b $\hat{+}$ c $\in$ U$_0$[$\Gamma$]}
\DisplayProof \quad
\AxiomC{b $\in$ U$_0$[$\Gamma$]}
\LeftLabel{I$_7$-Uno)}
\UnaryInfC{$\hat{List}(b)$ $\in$ U$_0$[$\Gamma$]}
\DisplayProof
\end{center}
\noindent
\textit{In (I$_7$-Uno) b rappresenta la codifica di un tipo in un contenitore e $\hat{List}(b)$ il codice delle liste associate a b.}
\vspace{0.5cm}
\begin{center}
\begin{prooftree}
\AxiomC{b $\in$ U$_0$[$\Gamma$]}
\AxiomC{c $\in$ \textbf{T}(b)[$\Gamma$]}
\AxiomC{d $\in$ \textbf{T}(b)[$\Gamma$]}
\LeftLabel{I$_8$-Uno)}
\TrinaryInfC{$\hat{Id}$(b,c,d) $\in$ U$_0$[$\Gamma$]}
\end{prooftree}
\end{center}


\section{Regole di Eliminazione}
\label{sec:eliminzione-U0}
\begin{prooftree}
\AxiomC{a $\in$ U$_0$[$\Gamma$]}
\LeftLabel{E-Uno)}
\UnaryInfC{\textbf{T}(a) type[$\Gamma$]}
\end{prooftree}
\noindent 
\textit{\textbf{T} indica la decodifica del codice a, e trasforma il codice in un tipo.}

\section{Regole di Conversione}
\label{sec:conversione-U0}
\begin{center}
\AxiomC{$\Gamma$ cont}
\LeftLabel{C$_1$-Uno)}
\UnaryInfC{\textbf{T}($\hat{N_0}$) $=$ N$_0$ type[$\Gamma$]}
\DisplayProof \quad
\AxiomC{$\Gamma$ cont}
\LeftLabel{C$_2$-Uno)}
\UnaryInfC{\textbf{T}($\hat{N_1}$) $=$ N$_1$ type[$\Gamma$]}
\DisplayProof\\
\vspace{0.5cm}
\AxiomC{$\Gamma$ cont}
\LeftLabel{C$_3$-Uno)}
\UnaryInfC{\textbf{T}(Nat) $=$ Nat type[$\Gamma$]}
\DisplayProof\\
\vspace{0.5cm}
\AxiomC{c(x) $\in$ U$_0$[$\Gamma$,x $\in$ \textbf{T}(b)]}
\AxiomC{b $\in$ U$_0$[$\Gamma$]}
\LeftLabel{C$_4$-Uno)}
\BinaryInfC{\textbf{T}($\hat{\sum}_{x \in b} c(x)$) $=$ $\sum_{x \in \textbf{T}(b)}$T(c(x)) type[$\Gamma$]}
\DisplayProof\\
\vspace{0.5cm}
\AxiomC{c(x) $\in$ U$_0$[$\Gamma$,x $\in$ \textbf{T}(b)]}
\AxiomC{b $\in$ U$_0$[$\Gamma$]}
\LeftLabel{C$_5$-Uno)}
\BinaryInfC{\textbf{T}($\hat{\prod}_{x \in b} c(x)$) $=$ $\prod_{x \in \textbf{T}(b)}$T(c(x)) type[$\Gamma$]}
\DisplayProof\\
\end{center}
\vspace{0.5cm}
\AxiomC{b $\in$ U$_0$[$\Gamma$]}
\AxiomC{c $\in$ U$_0$[$\Gamma$]}
\LeftLabel{C$_6$-Uno)}
\BinaryInfC{\textbf{T}(b $\hat{+}$ c) $=$ \textbf{T}(b) $+$ \textbf{T}(c) type[$\Gamma$]}
\DisplayProof \quad
\AxiomC{b $\in$ U$_0$[$\Gamma$]}
\LeftLabel{C$_7$-Uno)}
\UnaryInfC{\textbf{T}($\hat{List}$(b)) $=$ List(\textbf{T}(b)) type[$\Gamma$]}
\DisplayProof
\begin{center}
\vspace{0.5cm}
\AxiomC{b $\in$ U$_0$[$\Gamma$]}
\AxiomC{c $\in$ \textbf{T}(b)[$\Gamma$]}
\AxiomC{d $\in$ \textbf{T}(b)[$\Gamma$]}
\LeftLabel{C$_8$-Uno)}
\TrinaryInfC{\textbf{T}($\hat{Id}$(b,c,d)) $=$ Id(\textbf{T}(b),c,d) type[$\Gamma$]}
\DisplayProof
\end{center}

\section{Regole di Uguaglianza}
\label{subsec: uguaglianza-U0}
\begin{center}
\vspace{0.5cm}
\AxiomC{c$_1$(x) $=$ c$_2$(x) $\in$ U$_0$[$\Gamma$,x $\in$ \textbf{T}(b)]}
\AxiomC{b$_1$ $=$ b$_2$ $\in$ U$_0$[$\Gamma$]}
\LeftLabel{eq-I$_4$-Uno)}
\BinaryInfC{$\hat{\sum}_{x \in b_1}$ c$_1$(x) $=$ $\hat{\sum}_{x \in b_2}$ c$_2$(x) $\in$ U$_0$[$\Gamma$]}
\DisplayProof \\ \vspace{0.5cm}
\AxiomC{c$_1$(x) $=$ c$_2$(x) $\in$ U$_0$[$\Gamma$,x $\in$ \textbf{T}(b)]}
\AxiomC{b$_1$ $=$ b$_2$ $\in$ U$_0$[$\Gamma$]}
\LeftLabel{eq-I$_5$-Uno)}
\BinaryInfC{$\hat{\prod}_{x \in b_1}$ c$_1$(x) $=$ $\hat{\prod}_{x \in b_2}$ c$_2$(x) $\in$ U$_0$[$\Gamma$]}
\DisplayProof \\
\vspace{0.5cm}
\AxiomC{b$_1$ $=$ b$_2$ $\in$ U$_0$[$\Gamma$]}
\AxiomC{c$_1$ $=$ c$_2$ $\in$ U$_0$[$\Gamma$]}
\LeftLabel{eq-I$_6$-Uno)}
\BinaryInfC{b$_1$ $\hat{+}$ c$_1$ $=$ b$_2$ $\hat{+}$ c$_2$ $\in$ U$_0$[$\Gamma$]}
\DisplayProof \\ \vspace{0.5cm}
\AxiomC{b$_1$ $=$ b$_2$ $\in$ U$_0$[$\Gamma$]}
\LeftLabel{eq-I$_7$-Uno)}
\UnaryInfC{$\hat{List}$(b$_1$) $=$ $\hat{List}$(b$_2$) $\in$ U$_0$[$\Gamma$]}
\DisplayProof\\
\vspace{0.5cm}
\begin{prooftree}
\AxiomC{b$_1$ $=$ b$_2$ $\in$ U$_0$[$\Gamma$]}
\AxiomC{c$_1$ $=$ c$_2$ $\in$ \textbf{T}(b)[$\Gamma$]}
\AxiomC{d$_1$ $=$ d$_2$ $\in$ \textbf{T}(b)[$\Gamma$]}
\LeftLabel{eq-I$_8$-Uno)}
\TrinaryInfC{$\hat{Id}$(b$_1$,c$_1$,d$_1$) $=$ $\hat{Id}$(b$_2$,c$_2$,d$_2$) $\in$ U$_0$[$\Gamma$]}
\end{prooftree}
\end{center}

\section{Semantica operazionale tipo Primo Universo}
\label{subsec: semantica-operazionale-U0}
Negli universi normalizzare i termini \`e pi\`u difficile e richiede una relazione di riduzione che coinvolge anche i tipi, vale dunque il \textbf{teorema di forte normalizzazione} (\S \ref{subsec: applicazione-uguaglianza-definizionale-tra-termini}).\\
La relazione $\rightarrow_1$ viene definita all'interno dei termini con l'uso delle seguenti regole di riduzione:
\begin{itemize}
\item $\beta_{Uno}$-C$_1$-red) \textbf{T}($\hat{N_0}$) $\rightarrow_1$ N$_0$
\item $\beta_{Uno}$-C$_2$-red) \textbf{T}($\hat{N_1}$) $\rightarrow_1$ N$_1$
\item $\beta_{Uno}$-C$_3$-red) \textbf{T}($\hat{Nat}$) $\rightarrow_1$ Nat
\item $\beta_{Uno}$-C$_4$-red) \textbf{T}($\hat{\sum}_{x \in b} c(x)$) $\rightarrow_1$ $\sum_{x \in \textbf{T}(b)} \textbf{T}(c(x))$
\item $\beta_{Uno}$-C$_5$-red) \textbf{T}($\hat{\prod}_{x \in b} c(x)$) $\rightarrow_1$ $\prod_{x \in \textbf{T}(b)} \textbf{T}(c(x))$
\item $\beta_{Uno}$-C$_6$-red) \textbf{T}(b $\hat{+}$ c) $\rightarrow_1$ \textbf{T}(b) $+$ \textbf{T}(c)
\item $\beta_{Uno}$-C$_7$-red) \textbf{T}($\hat{List}$(b)) $\rightarrow_1$ List(\textbf{T}(b))
\item $\beta_{Uno}$-C$_8$-red) \textbf{T}($\hat{Id}(b,c,d)$) $\rightarrow_1$ Id(\textbf{T}(b),c,d)
\item \AxiomC{a$_1$ $\rightarrow$ a$_2$}
\LeftLabel{Uno-red)}
\UnaryInfC{\textbf{T}(a$_1$) $\rightarrow_1$ \textbf{T}(a$_2$) type[$\Gamma$]}
\DisplayProof \qquad
\item Novit\`a del tipo Universo rispetto al tipo singoletto
\AxiomC{c$_1$ $\rightarrow_1$ c$_2$}
\AxiomC{b$_1$ $\rightarrow_1$ b$_2$}
\LeftLabel{$Uno$-I$_4$-red$_I$)}
\BinaryInfC{$\hat{\sum}_{x \in b_1}$ c$_1$(x) $\rightarrow_1$ $\hat{\sum}_{x \in b_2}$ c$_2$(x)}
\DisplayProof
\item \AxiomC{c$_1$ $\rightarrow_1$ c$_2$}
\AxiomC{b$_1$ $\rightarrow_1$ b$_2$}
\LeftLabel{$Uno$-I$_5$-red$_I$)}
\BinaryInfC{$\hat{\prod}_{x \in b_1}$ c$_1$(x) $\rightarrow_1$ $\hat{\prod}_{x \in b_2}$ c$_2$(x)}
\DisplayProof\\
\item \AxiomC{b$_1$ $\rightarrow_1$ b$_2$}
\LeftLabel{$Uno$-I$_6$-red$_I$)}
\UnaryInfC{b$_1$ $\hat{+}$ c $\rightarrow_1$ b$_2$ $\hat{+}$ c}
\DisplayProof \quad
\AxiomC{c$_1$ $\rightarrow_1$ c$_2$}
\LeftLabel{$Uno$-I$_6$-red$_{II}$)}
\UnaryInfC{b$\hat{+}$ c$_1$  $\rightarrow_1$ b $\hat{+}$ c$_2$}
\DisplayProof
\item \AxiomC{d$_1$ $\rightarrow_1$ d$_2$}
\LeftLabel{$Uno$-I$_7$-red$_I$)}
\UnaryInfC{$\hat{List}$(d$_1$) $\rightarrow_1$ $\hat{List}$(d$_2$)}
\DisplayProof
\item \AxiomC{t$_1$ $\rightarrow_1$ t$_2$}
\LeftLabel{$Uno$-I$_8$-red$_I$)}
\UnaryInfC{$\hat{Id}$(t$_1$,w,z) $\rightarrow_1$ $\hat{Id}$(t$_2$,w,z)}
\DisplayProof \quad
\AxiomC{w$_1$ $\rightarrow_1$ w$_2$}
\LeftLabel{$Uno$-I$_8$-red$_{II}$)}
\UnaryInfC{$\hat{Id}$(t,w$_1$,z) $\rightarrow_1$ $\hat{Id}$(t,w$_2$,z)}
\DisplayProof \quad
\AxiomC{z$_1$ $\rightarrow_1$ z$_2$}
\LeftLabel{$Uno$-I$_8$-red$_{III}$)}
\UnaryInfC{$\hat{Id}$(t,w,z$_1$) $\rightarrow_1$ $\hat{Id}$(t,w,z$_2$)}
\DisplayProof
\end{itemize}

\section{Esercizio su Universo: $0$ diverso da $1$ proposizionalmente}
\label{sec:esercizio-su-universo}
\textit{Il sequente esercizio \`e utile a comprendere come la logica si interpreta all'interno della teoria dei tipi.\\
Inoltre mette in luce, come l'uso degli Universi, permette di dimostrare alcuni fatti banali dell'aritmetica, e in questo caso, l'assioma di Peano.}\\\\
\noindent
\textbf{Voglio dimonstrare che $\neg$Id(Nat,0,1) \`e derivabile, ovvero che esiste pf tale che} 
\begin{center}\textbf{pf $\in$ Id(Nat,0,1) $\rightarrow$ N$_0$[ ]}\end{center}
Questo perch\`e N$_0$ $\equiv$ $\bot$ e Id(Nat,0,1) $\equiv$ $0$ $=_{Nat}$ 1\\
Che \`e equivalente  a trovare 
\begin{center}\textbf{pf(z) $\in$ N$_0$[z $\in$ Id(Nat,0,1)]}\end{center}
\noindent \textit{Ricordo che l'uguaglianza \`e Id e non 0 $=$ 1 $\in$ Nat (0 \`e una NF $\neq$ succ(0) $=$ 1 che \`e una NF) che difatti non \`e derivabile}.\\\\
\noindent \textbf{L'unico modo per trovare pf \`e solo utilizzando un Universo (U$_0$)}.
\begin{center}
U$_0$\\
$\hat{U_0}$ $\in$ U$_1$\\
$\hat{U_1}$ $\in$ U$_2$\\
...
\end{center}
\noindent
Nella teoria dei tipi di \textit{Martin-L$\ddot{o}$f} \`e possibile sempre definire U$_0$ e poi mettere il codice di U$_0$ dento a U$_1$, il codice di U$_1$ dentro a U$_2$, e cosi via, senza mai far collassare gli universi.\\
Al netto di ci\`o, in questo esercizio, basterebbe usare un Universo U$_k$ e non solo U$_0$.

\subsection{Lemma 1}
Definisco la funzione \textit{isZero}, indispensabile per capire se z \`e 0 oppure 1. Dunque\\
\begin{center}\textbf{isZero(z)[z $\in$ Nat]}\end{center}
Idea: pensare a U$_0$ come se fosse Bool (in realt\`a U$_0$ ne contiene molti di pi\`u).
\begin{itemize}
\item vero: $\hat{N}_1$ $\rightarrow$ T(N$_1$) $=$ N$_1$ = \textit{tt}$^I$
\item falso: $\hat{N}_0$ $\rightarrow$ T(N$_0$) $=$ N$_0$ = \textit{$\bot$}$^I$
\end{itemize}
\noindent
\textit{\`E vero che z \`e zero, per cui nel caso base va messo N$_1$\\
Il successore di qualcosa non \`e zero, e dunque va messo N$_0$}
\scriptsize
\begin{adjustwidth}{-7em}{}
\begin{prooftree}
\AxiomC{[ ] cont}
\LeftLabel{F-Nat}
\UnaryInfC{Nat type [ ]}
\LeftLabel{F-c}\RightLabel{(z $\in$ Nat) $\notin$ [ ]}
\UnaryInfC{z $\in$ Nat cont}
\LeftLabel{F-Uno}
\UnaryInfC{U$_0$ type[z $\in$ Nat]}

\AxiomC{[ ] cont}
\LeftLabel{I$_2$-Uno}
\UnaryInfC{$\hat{N}_1$ $\in$ U$_0$[ ]}

\AxiomC{[ ] cont}
\LeftLabel{I$_1$-Uno}
\UnaryInfC{$\hat{N}_0$ $\in$ U$_0$[ ]}

\AxiomC{[ ] cont}
\LeftLabel{F-Nat}
\UnaryInfC{Nat type [ ]}
\LeftLabel{F-c}\RightLabel{(x $\in$ Nat) $\notin$ [ ]}
\UnaryInfC{x $\in$ Nat cont}
\LeftLabel{F-Uno}
\UnaryInfC{U$_0$ type[x $\in$ Nat]}
\LeftLabel{F-c}\RightLabel{\begin{tabular}[c]{cc}(y $\in$ U$_0$) \\ $\notin$ x $\in$ Nat\end{tabular}}
\UnaryInfC{x $\in$ Nat, y $\in$ U$_0$ cont}
\LeftLabel{ind-te}
\BinaryInfC{$\hat{N}_0$ $\in$ U$_0$[x $\in$ Nat, y $\in$ U$_0$]}

\LeftLabel{E-Nat$_{dip}$}
\TrinaryInfC{El$_{Nat}$(z,$\hat{N_1}$,(x,y).$\hat{N_0}$) $\in$ U$_0$[z $\in$ Nat]}
\end{prooftree}
\end{adjustwidth}
\noindent
\normalsize
\begin{enumerate}
\item Verifico che \textit{isZero}(0) $=$ $\hat{N}_1$ $\in$ U$_0$[ ] (\textit{per farlo uso la regola di Conversione dei Naturali})

\begin{prooftree}
\AxiomC{[ ] cont}
\LeftLabel{F-Uno}
\UnaryInfC{U$_0$ type[ ]}

\AxiomC{[ ] cont}
\LeftLabel{I$_2$-Uno}
\UnaryInfC{$\hat{N}_1$ $\in$ U$_0$[ ]}


\AxiomC{[ ] cont}
\LeftLabel{I$_1$-Uno}
\UnaryInfC{$\hat{N}_0$ $\in$ U$_0$[ ]}
\LeftLabel{C$_1$-Nat}
\TrinaryInfC{El$_{Nat}$(0,$\hat{N_1}$,(x,y).$\hat{N_0}$) $=$ $\hat{N_1}$ $\in$ U$_0$ [ ]}
\end{prooftree}
\noindent

\item Verifico che \textit{isZero}(1) $=$ $\hat{N}_0$ $\in$ U$_0$[ ]

\begin{prooftree}
\AxiomC{[ ] cont}
\LeftLabel{I$_1$-Nat}
\UnaryInfC{0 $\in$ Nat [ ]}

\AxiomC{[ ] cont}
\LeftLabel{F-Uno}
\UnaryInfC{U$_0$ type[ ]}

\AxiomC{[ ] cont}
\LeftLabel{I$_2$-Uno}
\UnaryInfC{$\hat{N}_1$ $\in$ U$_0$[ ]}


\AxiomC{[ ] cont}
\LeftLabel{I$_1$-Uno}
\UnaryInfC{$\hat{N}_0$ $\in$ U$_0$[ ]}
\LeftLabel{C$_2$-Nat}
\QuaternaryInfC{El$_{Nat}$(succ(0),$\hat{N_1}$,(x,y).$\hat{N_0}$) $=$ $\hat{N_0}$ $\in$ U$_0$ [ ]}
\end{prooftree}
\noindent
\end{enumerate}
\noindent
Dunque \begin{center}\textbf{\textit{isZero}(z) equivale a El$_{Nat}$(z,$\hat{N}_1$,(x,y).$\hat{N}_0$)\\
con \textit{isZero}(0) $=$ $\hat{N}_1$ $\in$ U$_0$[ ] e \textit{isZero}(1) $=$ $\hat{N}_0$ $\in$ U$_0$[ ]}\end{center}

\subsection{Lemma 2}
h$_{isZero}$(z$_1$,z$_2$,z$_3$) $\in$ Id(U$_0$,isZero(z$_1$),isZero(z$_2$))[z$_1$ $\in$ Nat,z$_2$ $\in$ Nat,z$_3$ $\in$ Id(Nat,z$_1$,z$_2$)] \`e derivabile.\\
h$_{isZero}$(z$_1$,z$_2$,z$_3$) \`e equivalente a El$_{Id}$(z$_3$,(x).id(isZero(x)) (derivazione presente in \S\ref{sec: es-id} esercizio 2) \`e anch'essa derivabile.

\subsubsection{Corollario}
h$_{isZero}$(0,1,z) $\in$ Id(U$_0$,isZero(0),isZero(1))[z $\in$ Id(Nat,0,1)] \`e derivabile.\\
Inoltre visto che \textit{isZero}(0) $=$ N$_1$ e \textit{isZero}(1) $=$ N$_0$ anche
\begin{center} \textbf{h$_{isZero}$(0,1,z) Id(U$_0$,N$_1$,N$_0$)[z $\in$ Id(Nat,0,1)]} \end{center}\`e derivabile.
\small
\begin{adjustwidth}{-11em}{}
\begin{prooftree}
\AxiomC{U$_0$ $=$ U$_0$ type[z $\in$ Id(Nat,0,1)}
\AxiomC{isZero(0) $=$ N$_1$ $\in$ U$_0$ [z $\in$ Id(Nat,0,1)]}
\AxiomC{isZero(1) $=$ N$_0$ $\in$ U$_0$ [z $\in$ Id(Nat,0,1)]}
\LeftLabel{eq-F-Id}
\TrinaryInfC{Id(U$_0$,isZero(0),isZero(1) $=$ Id(U$_0$,N$_1$,N$_0$)[z $\in$ Id(Nat,0,1)}
\end{prooftree}
\end{adjustwidth}
\vspace{0.5cm}
\noindent
\normalsize
Ecco che se 0 $=$ 1 siamo vicino alla contraddizione, in quanto usando che \textit{isZero}(0) $=$ N$_1$ e \textit{isZero}(1) $=$ N$_0$, allora anche N$_0$ $=$ N$_1$ nell'Universo. Ma N$_0$ \`e \textit{true} e  N$_1$ \`e \textit{false} e se \textit{true} fosse uguale a \textit{false} avremmo una logica inconsistente.
\subsection{Lemma 3}
\textit{Serve per arrivare ad avere una contraddizione.}\\\\
\noindent
q$_{U_0}$(x,y,z) $\in$ (T(x) $\rightarrow$ T(y)) $\times$ (T(y) $\rightarrow$ T(x))[x $\in$ U$_0$,y $\in$ U$_0$,z $\in$ Id(U$_0$,x,y)] \`e derivabile, equivalente a El$_{Id}$(z,(x).$<\lambda$w.w,$\lambda$w.w$>$) $\in$ (T(x) $\rightarrow$ T(y)) $\times$ (T(x) $\rightarrow$ T(y)).\\\\
\noindent
$\Delta$ $\equiv$ x $\in$ U$_0$,y $\in$ U$_0$,z $\in$ Id(U$_0$,x,y)
\scriptsize
\begin{adjustwidth}{-13em}{}
\begin{prooftree}
\AxiomC{}
\UnaryInfC{(T(x) $\rightarrow$ T(y))[$\Delta$]}
\AxiomC{}
\UnaryInfC{(T(x) $\rightarrow$ T(y))[$\Delta$]}
\LeftLabel{F-x}
\BinaryInfC{(T(x) $\rightarrow$ T(y)) $\times$ (T(x) $\rightarrow$ T(y))[$\Delta$]}

\AxiomC{}
\UnaryInfC{T(x) type[x $\in$ U$_0$]}
\LeftLabel{F-c}\RightLabel{(w $\in$ T(x)) $\notin$ x $\in$ U$_0$}
\UnaryInfC{x $\in$ U$_0$,w $\in$ T(x) cont}
\LeftLabel{var}
\UnaryInfC{w $\in$ T(x)[x $\in$ U$_0$,w $\in$ T(x)]}
\LeftLabel{I-$\rightarrow$}
\UnaryInfC{$\lambda$w.w $\in$ T(x) $\rightarrow$ T(x)[x $\in$ U$_0$]}
\AxiomC{$\ast$}
\UnaryInfC{$\lambda$w.w $\in$ T(x) $\rightarrow$ T(x)[x $\in$ U$_0$]}
\LeftLabel{I-x}
\BinaryInfC{$<\lambda$w.w,$\lambda$w.w$>$ $\in$ (T(x) $\rightarrow$ T(x)) $\times$ (T(x) $\rightarrow$ T(x))[x $\in$ U$_0$]}
\LeftLabel{E-Id$_{dip}$}
\BinaryInfC{El$_{Id}$(z,(x).$<\lambda$w.w,$\lambda$w.w$>$) $\in$ (T(x) $\rightarrow$ T(y)) $\times$ (T(x) $\rightarrow$ T(y))[]}
\end{prooftree}
\end{adjustwidth}
\normalsize
\vspace{0.3cm}
\noindent
Ho usato \textbf{($\ast$)} per concludere le derivazioni gi\`a svolte all'interno dell'albero.
\subsubsection{Corollario}
Da q$_{U_0}$(x,y,z) $\in$ (T(x) $\rightarrow$ T(y)) $\times$ (T(y) $\rightarrow$ T(x))[$\Delta$] sostituendo x con N$_1$ e y con N$_0$ ottengo \begin{center} q$_{U_0}$($\hat{N_1}$,$\hat{N_0}$,z) $\in$ (T(N$_1$) $\rightarrow$ T(N$_0$)) $\times$ (T(N$_0$) $\rightarrow$ T(N$_1$))[z $\in$ Id(Nat,0,1)] \end{center} che \`e derivabile. Inoltre mediante sostituzione trovo che
\scriptsize
\begin{adjustwidth}{-6em}{}
\begin{prooftree}
\AxiomC{$\Gamma$ cont}
\UnaryInfC{T(N$_1$) $=$ N$_1$ type[$\Gamma$]}
\AxiomC{$\Gamma$ cont}
\UnaryInfC{T(N$_0$) $=$ N$_0$ type[$\Gamma$]}
\LeftLabel{eq-F-{\scriptsize$\prod$}}
\BinaryInfC{T(N$_1$) $\rightarrow$ T(N$_0$) $=$ N$_1$ $\rightarrow$ N$_0$}
\AxiomC{$\Gamma$ cont}
\UnaryInfC{T(N$_0$) $=$ N$_0$ type[$\Gamma$]}
\AxiomC{$\Gamma$ cont}
\UnaryInfC{T(N$_1$) $=$ N$_1$ type[$\Gamma$]}
\LeftLabel{eq-F-{\scriptsize$\prod$}-F}
\BinaryInfC{T(N$_0$) $\rightarrow$ T(N$_1$) $=$ N$_0$ $\rightarrow$ N$_1$}
\LeftLabel{eq-F-x}
\BinaryInfC{(T(N$_1$) $\rightarrow$ T(N$_0$)) $\times$ (T(N$_0$) $\rightarrow$ T(N$_1$)) $=$ (N$_1$ $\rightarrow$ N$_0$) $\times$ (N$_0$ $\rightarrow$ N$_1$)}
\end{prooftree}
\end{adjustwidth}

\normalsize
\begin{center}\textbf{q$_{U_0}$($\hat{N_1}$,$\hat{N_0}$,z) $\in$ (N$_1$ $\rightarrow$ N$_0$) $\times$ (N$_0$ $\rightarrow$ N$_1$)}\end{center} \`e derivabile.
\\\\
\noindent
\textbf{Corollario finale}\\\\
\noindent
Ap($\pi_1 q_{U_0}(\hat{N}_1,\hat{N}_0,z^\backprime))$, $\ast$)$\in$ N$_0$[z$\backprime$ $\in$ Id(U$_0$,$\hat{N}_1$,$\hat{N}_0$)]\\
Dunque $\bot$ true [Id(U$_0$,$\hat{N}_1$,$\hat{N}_0$) true] \`e derivabile.\\\\
\noindent
Inoltre h$_{isZero}$(0,1,z) $\in$ Id(U$_0$,N$_1$,N$_0$)[z $\in$ Id(Nat,0,1)] \`e equivalente a Id(U$_0$,$\hat{N}_1$,$\hat{N}_0$) true[z $\in$ Id(Nat,0,1)] che  \`e derivabile.\\
Per cui mettendo assieme, usando la logica, che Id(U$_0$,$\hat{N}_1$,$\hat{N}_0$) true[z $\in$ Id(Nat,0,1)]  e $\bot$ true [Id(U$_0$,$\hat{N}_1$,$\hat{N}_0$) true] si ottiene $\bot$ true [Id(Nat,0,1)] che \`e ancora derivabile.\\\\
\noindent 
Il \textit{proof-term} non \`e altro che prendere Ap($\pi_1 q_{U_0}(\hat{N}_1,\hat{N}_0,z^\backprime))$,$\ast$)$\in$ N$_0$[z$\backprime$ $\in$ Id(U$_0$,$\hat{N}_1$,$\hat{N}_0$)] e comporlo con h$_{isZero}$(0,1,z) $\in$ Id(U$_0$,N$_1$,N$_0$)[z $\in$ Id(Nat,0,1)]
\begin{center}\textbf{Ap($\pi_1 q_{U_0}(\hat{N}_1,\hat{N}_0,h_{isZero}$(0,1,z)),$\ast$) $\in$ N$_0$[z $\in$ Id(Nat,0,1)]} \\ \textbf{$\lambda$z.Ap($\pi_1 q_{U_0}(\hat{N}_1,\hat{N}_0,h_{isZero}$(0,1,z)),$\ast$) $\in$ $\neg$Id(Nat,0,1) }\end{center}
\noindent
Tale \textit{proof-term}, non \`e altro che la codifica del \textit{proof-assistant}, della dimostrazione che $\neg$Id(Nat,0,1) true[ ].

\section{Peculiarit\`a della teoria dei tipi di \textit{Martin-L$\ddot{o}$f}}
\label{sec:peculiarit\`a-della-teoria-dei-tipi}
Per teoria dei tipi di \textit{Martin-L$\ddot{o}$f} si intende
\begin{center}
N$_0$,N$_1$, Nat, List, B $+$ C, Id(A,a,b), $\sum\limits_{x \in A}$ B(x), $\prod\limits_{x \in A}$ B(x), U$_0$ $+$ altri universi
\end{center}
\noindent
Questa \`e la teoria \textbf{MLTT}.\\
La teoria \textbf{MLTT} \`e aperta, ovvero si intende anche l'estensione della teoria specifica, con i tipi elencati, con altri tipi induttivi, secondo lo schema delle regole date in \S\ref{sec: schema-generale}, in modo predicativo. In tale teoria deve sempre valere il teorema della forma normale \S \ref{subsec: applicazione-uguaglianza-definizionale-tra-termini} (la confluenza), e in generale la correttezza dei 4 giudizi deve essere decidibile (\textbf{decidable type checking}).\\\\
\noindent
\begin{center}\textbf{MLTT$_0$} ${\overset{\mathit{def}}{=}}$ \textbf{MLLT} $-$ Universo e con solo i tipi elencati sopra\\
\textbf{MLLT$_1$} ${\overset{\mathit{def}}{=}}$ \textbf{MLTT$_0$} $+$ U$_0$ universo\end{center}
\noindent
La teoria \textbf{MLLT$_1$} \`e quella spiegata nella teoria del corso.

\subsection{Necessit\`a di un Universo per formalizzare l'aritmetica}
\label{subsec:Necessit\`a-di-un-Universo-per-formalizzare-l'aritmetica}
Se prendiamo la teoria \textbf{MLTT$_0$ senza l'Universo} \begin{center}$\equiv$ N$_0$, N$_1$, List(A), Nat, B $+$ C, Id(A,a,b), $\sum\limits_{x \in A}$ B(x), $\prod\limits_{x \in A}$ B(x)\end{center}
\noindent
\`e \textbf{inadeguata} per formalizzare la matematica, perch\`e non dimostra che i Naturali sono formati da infiniti numeri. In particolare non \`e in grado di dimostrare che 0 $\neq$ 1 proposizionalmente. Formalmente significa
\begin{center}\textbf{$\neg$ Id(Nat,0,1) true[ ] non \`e derivabile in MLTT$_0$ (ma lo \`e con l'Universo)}\end{center}
\noindent
Al massimo in \textbf{MLTT} si pu\`o formalizzare in \textit{Costructive Math} $=$ \textit{Math}(matematica usuale che si pu\`o dimostrare con la Logica intuizionisitica) $+$ Logica Intuizionistica.\\
Invece \textit{Classic Math} $=$ \textit{Math} $+$ Logica Intuizionistica $+$ $\varphi \vee \neq\varphi$ true[$\Gamma$] per ogni $\varphi$\\
$\Rightarrow$ Logica Intuizionistica $+$ $\varphi \vee \neq\varphi$ true[$\Gamma$] per ogni $\varphi$ $=$ Logica Classica\\
$\Rightarrow$ $\varphi \vee \neq\varphi$ true[$\Gamma$] per ogni $\varphi$ non vale in \textbf{MLLT}.\\
In \textbf{MLTT} non si pu\`o formalizzare la \textit{Classic Math}, al massimo solo quella \textit{Costructive}.
\subsubsection{Dimostrazione del perch\`e MLLT$_0$, senza Universo, \`e indeguata}
\label{subsec:dimostro-perch\`e-MLLT$_0$-senza-Universo-\`e-indeguata}
Per dimostrare l'aritmetica intuizionistica bisogna almeno essere in MLTT$_1$, ci vuole almeno U$_0$ Universo.\\\\
\textit{Dimostro che MLTT$_0$ non \`e sufficiente, con il seguente esempio}
\begin{center}
\textbf{0 $\neq$ 1 prop $\equiv$ $\neg$Id(Nat,0,1)}
\end{center}
\noindent
\begin{enumerate}
\item \textbf{Costruire un contromodello interno di MLLT$_0$}, che renda valido tutte le regole di \textbf{MLLT$_0$} con l'assunzione che \textbf{MLLT$_0$} sia consistente.\\
Ma ($\neg$Id(Nat,0,1))$^J$ $=$ N$_0$ type [ ] $=$ $\bot$ type [ ].\\\\
\noindent
Idea: ( )$^J$ tipi di MLLT$_0$ $\rightarrow$ \{N$_0$,N$_1$\} $\equiv$ Bool (ovvero ogni tipo viene interpretato come proposizione \textit{true} o \textit{false})\\
Ci\`o significa che interpreto (B type[$\Gamma$])$^J$, all'interno di \textbf{MLLT$_0$} stesso, come B$^J$ type[z $\in$ N$^J$]\\
Inoltre interpreto un \textit{proof-term} (pf $\in$ B type[$\Gamma$])$^J$ ${\overset{\mathit{def}}{=}}$ $\ast$ $\in$ B$^J$[$\Gamma$] sse B$^J$ $=$ N$_1$ (\textit{interpretazione parziale}).\\\\
\noindent
Con tale interpretazione si ottiene il seguente \textbf{Teorema}\\
pf $\in$ B[$\Gamma$] \`e derivabile $\rightarrow$ $\ast$ $\in$ B$^J$[$\Gamma^J$] \`e derivabile in \textbf{MTT$_0$}. Questo vale anche per i tipi, dunque se B type[$\Gamma$] \`e derivabile $\rightarrow$ B$^J$ type[$\Gamma^J$]
\item \textbf{Interpretazione}
\begin{center}
N$_0^J$ ${\overset{\mathit{def}}{=}}$ N$_0$\\
N$_1^J$ ${\overset{\mathit{def}}{=}}$ N$_1$\\
List(A)$^J$ ${\overset{\mathit{def}}{=}}$ List(A)\\
Nat$^J$ ${\overset{\mathit{def}}{=}}$ N$_1$\\
Id(A,a,b)$^J$ ${\overset{\mathit{def}}{=}}$ A$^J$
\end{center}
\noindent
Con questo modello allora se scrivo Id(A,a,b)$^J$ viene interpretato come Nat$^J$ $=$ N$_1$\\
Inoltre possiamo derivare $\ast$ $\in$ Id(A,a,b)$^J$ in \textbf{MTT$_0$}.
\textit{Se Id(A,a,b)$^J$ \`e abitata non pu\`o per\`o essere abitata la negazione.} Formalmente deve essere interpretata la negazione come
\begin{center}\textbf{$\neg$ Id(Nat,0,1) ${\overset{\mathit{def}}{=}}$ (Id(Nat,0,1)$\rightarrow$ N$_0$)$^J$ }\end{center}
Qui va data l'interpretazione dello spazio di funzioni, caso particolare di $\scriptsize$prod
\begin{center}
($\prod\limits_{x \in A}$ B)$^J$ $=$
$
\begin{cases}
\textbf{N}_0\text{ se } A^J =  N_1  \text{ e } B^J = N_0\\
\textbf{N}_1 \text{ altrimenti}
\end{cases}
$
\end{center}
\noindent
Pensate come preposizioni \\
($\prod\limits_{x \in A}$ B)$^J$ $\equiv$ A$^J$ $\rightarrow$ B$^J$ $\equiv$\\
\textit{tt} $\rightarrow$ \textit{tt} $\equiv$ \textit{tt}\\
$\bot$ $\rightarrow$ \textit{tt} $\equiv$ \textit{tt}\\
$\bot$ $\rightarrow$ $\bot$ $\equiv$ \textit{tt}\\
\textit{tt} $\rightarrow$ $\bot$ $\equiv$ $\bot$\\\\
\noindent
Dunque $\neg$ Id(Nat,0,1) ${\overset{\mathit{def}}{=}}$ (Id(Nat,0,1)$\rightarrow$ N$_0$)$^J$ $=$ $\prod\limits_{x \in Id(Nat,0,1)^J}$ N$_0$ $=$ $\prod\limits_{x \in N_1}$ N$_0$ $=$ N$_0$ $\rightarrow$ N$_0$($\bot$)\\
\textbf{In questo modo ho dimostrato che non \`e vero che 0 $\neq$ 1 proposizionalemente.}\\\\
\noindent
Per completezza do anche le interpretazioni di  {\scriptsize}$\sum$ e  $+$

\begin{center}
($\sum\limits_{x \in A}$ B)$^J$ $=$
$
\begin{cases}
\textbf{N}_1\text{ se } A^J =  N_1  \text{ e } B^J = N_1\\
\textbf{N}_0 \text{ altrimenti}
\end{cases}
$
\end{center}
\noindent
Pensate come preposizioni \\
($\sum\limits_{x \in A}$ B)$^J$ $\equiv$ A$^J$ $\times$ B$^J$ $\equiv$\\
\textit{tt} $\times$ \textit{tt} $\equiv$ \textit{tt}\\
$\bot$ $\times$ \textit{tt} $\equiv$ $\bot$  \\
$\bot$ $\times$ $\bot$ $\equiv$ $\bot$ \\
\textit{tt} $\rightarrow$ $\bot$ $\equiv$ $\bot$\\\\

\begin{center}
(A $+$ B)$^J$ $=$
$
\begin{cases}
\textbf{N}_0\text{ se } A^J = B^J = N_0\\
\textbf{N}_1 \text{ altrimenti}
\end{cases}
$
\end{center}
\noindent
Pensate come preposizioni \\
(A $+$ B)$^J$ $\equiv$ A$^J$ $\vee$ B$^J$ $\equiv$\\
$\bot$  $\vee$ $\bot$  $\equiv$ $\bot$ \\
$\bot$ $\vee$ \textit{tt} $\equiv$ \textit{tt}  \\
\textit{tt} $\vee$ \textit{tt} $\equiv$ \textit{tt} \\
\textit{tt} $\vee$ $\bot$ $\equiv$ \textit{tt}\\\\

\end{enumerate}
\noindent
In questo modo abbiamo costruito (tipi $+$ termini di \textbf{MLTT$_0$})$^J$ $\rightarrow$ \textbf{MLTT$_0$}. Che rappresenta, difatti, uno schiacciamento dei tipi \textit{tt}(N$_1$) o $\bot$(N$_0$), pensando ai tipi come proposizioni, riuscendo a provare che 0 $=$1 \`e vero, in quanto tutti i tipi dei Naturali vengono interpretati unicamente dal singoletto.\\\\
\noindent
\textbf{Questa interpretazione non funziona pi\`u con l'Universo}, perch\`e
\begin{center}\textbf{U$_0^J$  $=$ N$_1$ $\Leftrightarrow$ MLLT$_1$ \`e inconsistente, poich\`e ($\bot$)$^J$ $=$ (\textit{tt})$^J$} \end{center}
in quanto U$_0$, essendo \textit{Bool}, contiene sia il codice di $\hat{N_1}$ che il codice di $\hat{N_0}$



