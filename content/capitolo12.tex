\chapter{Tipi Universo \`a la Tarski}
\label{cap:tipo-universo}
Il tipo Universo, \textbf{schema di tipi}, non \`e induttivo ed \`e rivolto, in questo caso, alla teoria dei tipi di  \textit{Martin-L$\ddot{o}$f} intensionale.\\
Esiste il tipo Universo \`a la \textit{Tarski} e alla \textit{Russell}. Il primo, che espongo nel capitolo, pi\`u preciso ma pesante nella sintassi, pi\`u facile da capire e da modellare (soprattutto per fare delle semantiche nella teoria dei tipi), per cui migliore quando si programma con un \textit{proof assistant}; e il secondo pi\`u semplice da utilizzare nelle dimostrazioni, ma meno preciso.\\\\
Caratteristica del tipo Universo \`e che permette la ripetizione delle costruzioni per generare una gerarchia di universi, legati fra di loro
\begin{center} U$_0$, U$_1$,..., U$_n$ \qquad con n $\in$ Nat\end{center}
Nella prima versione \textit{Martin-L$\ddot{o}$f} permetteva che U$_0$ appartenesse a U$_0$, generando per\`o una contraddizione.
\textbf{\begin{center} U$_0$ $\notin$ U$_0$ \quad $\rightarrow$ \quad U$_0$ $\in$ U$_0$ \`e una contraddizione\end{center}}
\noindent
Universo \`e un contenitore, di tipi, predicativo, ovvero U$_0$ $\notin$ U$_0$.\\\\
\noindent
\textit{Fondamentale}\\
Essenziale per avere commutativit\`a  \`e non permettere che $\hat{U_0}$ $\in$ U$_0$ (un codice U$_0$ dentro a U$_0$) altrimenti, come gi\`a detto, si ricade nella contraddizione. Per cui si dichiara  come "\textit{small}" o piccoli tutti i tipi con codice in U$_0$; e invece come "\textit{large}" i tipi costruiti con U$_0$, ad esempio U$_0$ $\rightarrow$ U$_0$ type[ ].\\
Ecco che per la natura della regola di Eliminazione e dal fatto che contiene esclusivamente \textit{small} e non se stesso, posso affermare che l'Universo \`e \textbf{predicativo e non induttivo}.\\\\
\noindent
Il tipo del \textbf{Primo Universo (U$_0$) alla Tarski} \`e definito dalle regole seguenti, dove \textbf{T} indica la decodifica.
\section{Regole di Formazione}
\label{sec: formazione-U0}
\begin{prooftree}
\AxiomC{$\Gamma$ cont}
\LeftLabel{F-Uno)}
\UnaryInfC{U$_0$ type[$\Gamma$]}
\end{prooftree}
\textit{U$_0$ (come N$_0$, N$_1$, ecc..) \`e una costante di tipo.}

\section{Regole di Introduzione}
\label{sec: introduzione-U0}
\textit{Per le regole di Introduzione U$_0$ funge da contenitore di tipi, tramite codifica, ed \`e chiuso rispetto ai tipi finora introdotti, nei capitolo precedenti. Inoltre le regole di Introduzione usano quelle di Eliminazione.}
\begin{center}
\AxiomC{$\Gamma$ cont}
\LeftLabel{I$_1$-Uno)}
\UnaryInfC{$\hat{N_0}$ $\in$ U$_0$[$\Gamma$]}
\DisplayProof \quad
\AxiomC{$\Gamma$ cont}
\LeftLabel{I$_2$-Uno)}
\UnaryInfC{$\hat{N_1}$ $\in$ U$_0$[$\Gamma$]}
\DisplayProof \quad
\vspace{0.5cm}
\AxiomC{$\Gamma$ cont}
\LeftLabel{I$_3$-Uno)}
\UnaryInfC{$\hat{Nat}$ $\in$ U$_0$[$\Gamma$]}
\DisplayProof
\vspace{0.5cm}
\AxiomC{c(x) $\in$ U$_0$[$\Gamma$,x $\in$ \textbf{T}(b)]}
\AxiomC{b $\in$ U$_0$[$\Gamma$]}
\LeftLabel{I$_4$-Uno)}
\BinaryInfC{$\hat{\sum}_{x \in b}$ c(x) $\in$ U$_0$[$\Gamma$]}
\DisplayProof \quad
\AxiomC{c(x) $\in$ U$_0$[$\Gamma$,x $\in$ \textbf{T}(b)]}
\AxiomC{b $\in$ U$_0$[$\Gamma$]}
\LeftLabel{I$_5$-Uno)}
\BinaryInfC{$\hat{\prod}_{x \in b}$ c(x) $\in$ U$_0$[$\Gamma$]}
\DisplayProof \\
\vspace{0.5cm}
\AxiomC{b $\in$ U$_0$[$\Gamma$]}
\AxiomC{c $\in$ U$_0$[$\Gamma$]}
\LeftLabel{I$_6$-Uno)}
\BinaryInfC{b $\hat{+}$ c $\in$ U$_0$[$\Gamma$]}
\DisplayProof \quad
\AxiomC{b $\in$ U$_0$[$\Gamma$]}
\LeftLabel{I$_7$-Uno)}
\UnaryInfC{$\hat{List}(b)$ $\in$ U$_0$[$\Gamma$]}
\DisplayProof
\end{center}
\noindent
\textit{In (I$_7$-Uno) b rappresenta la codifica di un tipo in un contenitore e $\hat{List}(b)$ il codice delle liste associate a b.}
\vspace{0.5cm}
\begin{center}
\begin{prooftree}
\AxiomC{b $\in$ U$_0$[$\Gamma$]}
\AxiomC{c $\in$ \textbf{T}(b)[$\Gamma$]}
\AxiomC{d $\in$ \textbf{T}(b)[$\Gamma$]}
\LeftLabel{I$_8$-Uno)}
\TrinaryInfC{$\hat{Id}$(b,c,d) $\in$ U$_0$[$\Gamma$]}
\end{prooftree}
\end{center}


\section{Regole di Eliminazione}
\label{sec:eliminzione-U0}
\begin{prooftree}
\AxiomC{a $\in$ U$_0$[$\Gamma$]}
\LeftLabel{E-Uno)}
\UnaryInfC{\textbf{T}(a) type[$\Gamma$]}
\end{prooftree}
\noindent 
\textit{\textbf{T} indica la decodifica del codice a, e trasforma il codice in un tipo.}

\section{Regole di Conversione}
\label{sec:conversione-U0}
\begin{center}
\AxiomC{$\Gamma$ cont}
\LeftLabel{C$_1$-Uno)}
\UnaryInfC{\textbf{T}($\hat{N_0}$) $=$ N$_0$ type[$\Gamma$]}
\DisplayProof \quad
\AxiomC{$\Gamma$ cont}
\LeftLabel{C$_2$-Uno)}
\UnaryInfC{\textbf{T}($\hat{N_1}$) $=$ N$_1$ type[$\Gamma$]}
\DisplayProof\\
\vspace{0.5cm}
\AxiomC{$\Gamma$ cont}
\LeftLabel{C$_3$-Uno)}
\UnaryInfC{\textbf{T}(Nat) $=$ Nat type[$\Gamma$]}
\DisplayProof\\
\vspace{0.5cm}
\AxiomC{c(x) $\in$ U$_0$[$\Gamma$,x $\in$ \textbf{T}(b)]}
\AxiomC{b $\in$ U$_0$[$\Gamma$]}
\LeftLabel{C$_4$-Uno)}
\BinaryInfC{\textbf{T}($\hat{\sum}_{x \in b} c(x)$) $=$ $\sum_{x \in \textbf{T}(b)}$T(c(x)) type[$\Gamma$]}
\DisplayProof\\
\vspace{0.5cm}
\AxiomC{c(x) $\in$ U$_0$[$\Gamma$,x $\in$ \textbf{T}(b)]}
\AxiomC{b $\in$ U$_0$[$\Gamma$]}
\LeftLabel{C$_5$-Uno)}
\BinaryInfC{\textbf{T}($\hat{\prod}_{x \in b} c(x)$) $=$ $\prod_{x \in \textbf{T}(b)}$T(c(x)) type[$\Gamma$]}
\DisplayProof\\
\end{center}
\vspace{0.5cm}
\AxiomC{b $\in$ U$_0$[$\Gamma$]}
\AxiomC{c $\in$ U$_0$[$\Gamma$]}
\LeftLabel{C$_6$-Uno)}
\BinaryInfC{\textbf{T}(b $\hat{+}$ c) $=$ \textbf{T}(b) $+$ \textbf{T}(c) type[$\Gamma$]}
\DisplayProof \quad
\AxiomC{b $\in$ U$_0$[$\Gamma$]}
\LeftLabel{C$_7$-Uno)}
\UnaryInfC{\textbf{T}($\hat{List}$(b)) $=$ List(\textbf{T}(b)) type[$\Gamma$]}
\DisplayProof
\begin{center}
\vspace{0.5cm}
\AxiomC{b $\in$ U$_0$[$\Gamma$]}
\AxiomC{c $\in$ \textbf{T}(b)[$\Gamma$]}
\AxiomC{d $\in$ \textbf{T}(b)[$\Gamma$]}
\LeftLabel{C$_8$-Uno)}
\TrinaryInfC{\textbf{T}($\hat{Id}$(b,c,d)) $=$ Id(\textbf{T}(b),c,d) type[$\Gamma$]}
\DisplayProof
\end{center}

\section{Regole di Uguaglianza}
\label{subsec: uguaglianza-U0}
\begin{center}
\vspace{0.5cm}
\AxiomC{c$_1$(x) $=$ c$_2$(x) $\in$ U$_0$[$\Gamma$,x $\in$ \textbf{T}(b)]}
\AxiomC{b$_1$ $=$ b$_2$ $\in$ U$_0$[$\Gamma$]}
\LeftLabel{eq-I$_4$-Uno)}
\BinaryInfC{$\hat{\sum}_{x \in b_1}$ c$_1$(x) $=$ $\hat{\sum}_{x \in b_2}$ c$_2$(x) $\in$ U$_0$[$\Gamma$]}
\DisplayProof \\ \vspace{0.5cm}
\AxiomC{c$_1$(x) $=$ c$_2$(x) $\in$ U$_0$[$\Gamma$,x $\in$ \textbf{T}(b)]}
\AxiomC{b$_1$ $=$ b$_2$ $\in$ U$_0$[$\Gamma$]}
\LeftLabel{eq-I$_5$-Uno)}
\BinaryInfC{$\hat{\prod}_{x \in b_1}$ c$_1$(x) $=$ $\hat{\prod}_{x \in b_2}$ c$_2$(x) $\in$ U$_0$[$\Gamma$]}
\DisplayProof \\
\vspace{0.5cm}
\AxiomC{b$_1$ $=$ b$_2$ $\in$ U$_0$[$\Gamma$]}
\AxiomC{c$_1$ $=$ c$_2$ $\in$ U$_0$[$\Gamma$]}
\LeftLabel{eq-I$_6$-Uno)}
\BinaryInfC{b$_1$ $\hat{+}$ c$_1$ $=$ b$_2$ $\hat{+}$ c$_2$ $\in$ U$_0$[$\Gamma$]}
\DisplayProof \\ \vspace{0.5cm}
\AxiomC{b$_1$ $=$ b$_2$ $\in$ U$_0$[$\Gamma$]}
\LeftLabel{eq-I$_7$-Uno)}
\UnaryInfC{$\hat{List}$(b$_1$) $=$ $\hat{List}$(b$_2$) $\in$ U$_0$[$\Gamma$]}
\DisplayProof\\
\vspace{0.5cm}
\begin{prooftree}
\AxiomC{b$_1$ $=$ b$_2$ $\in$ U$_0$[$\Gamma$]}
\AxiomC{c$_1$ $=$ c$_2$ $\in$ \textbf{T}(b)[$\Gamma$]}
\AxiomC{d$_1$ $=$ d$_2$ $\in$ \textbf{T}(b)[$\Gamma$]}
\LeftLabel{eq-I$_8$-Uno)}
\TrinaryInfC{$\hat{Id}$(b$_1$,c$_1$,d$_1$) $=$ $\hat{Id}$(b$_2$,c$_2$,d$_2$) $\in$ U$_0$[$\Gamma$]}
\end{prooftree}
\end{center}

\section{Semantica operazionale tipo Primo Universo}
\label{subsec: semantica-operazionale-U0}
Negli universi normalizzare i termini \`e pi\`u difficile e richiede una relazione di riduzione che coinvolge anche i tipi, vale dunque il \textbf{teorema di forte normalizzazione} (\S \ref{subsec: applicazione-uguaglianza-definizionale-tra-termini}).\\
La relazione $\rightarrow_1$ viene definita all'interno dei termini con l'uso delle seguenti regole di riduzione:
\begin{itemize}
\item $\beta_{Uno}$-C$_1$-red) \textbf{T}($\hat{N_0}$) $\rightarrow_1$ N$_0$
\item $\beta_{Uno}$-C$_2$-red) \textbf{T}($\hat{N_1}$) $\rightarrow_1$ N$_1$
\item $\beta_{Uno}$-C$_3$-red) \textbf{T}($\hat{Nat}$) $\rightarrow_1$ Nat
\item $\beta_{Uno}$-C$_4$-red) \textbf{T}($\hat{\sum}_{x \in b} c(x)$) $\rightarrow_1$ $\sum_{x \in \textbf{T}(b)} \textbf{T}(c(x))$
\item $\beta_{Uno}$-C$_5$-red) \textbf{T}($\hat{\prod}_{x \in b} c(x)$) $\rightarrow_1$ $\prod_{x \in \textbf{T}(b)} \textbf{T}(c(x))$
\item $\beta_{Uno}$-C$_6$-red) \textbf{T}(b $\hat{+}$ c) $\rightarrow_1$ \textbf{T}(b) $+$ \textbf{T}(c)
\item $\beta_{Uno}$-C$_7$-red) \textbf{T}($\hat{List}$(b)) $\rightarrow_1$ List(\textbf{T}(b))
\item $\beta_{Uno}$-C$_8$-red) \textbf{T}($\hat{Id}(b,c,d)$) $\rightarrow_1$ Id(\textbf{T}(b),c,d)
\item \AxiomC{a$_1$ $\rightarrow$ a$_2$}
\LeftLabel{Uno-red)}
\UnaryInfC{\textbf{T}(a$_1$) $\rightarrow_1$ \textbf{T}(a$_2$) type[$\Gamma$]}
\DisplayProof \qquad
\item Novit\`a del tipo Universo rispetto al tipo singoletto
\AxiomC{c$_1$ $\rightarrow_1$ c$_2$}
\AxiomC{b$_1$ $\rightarrow_1$ b$_2$}
\LeftLabel{$Uno$-I$_4$-red$_I$)}
\BinaryInfC{$\hat{\sum}_{x \in b_1}$ c$_1$(x) $\rightarrow_1$ $\hat{\sum}_{x \in b_2}$ c$_2$(x)}
\DisplayProof
\item \AxiomC{c$_1$ $\rightarrow_1$ c$_2$}
\AxiomC{b$_1$ $\rightarrow_1$ b$_2$}
\LeftLabel{$Uno$-I$_5$-red$_I$)}
\BinaryInfC{$\hat{\prod}_{x \in b_1}$ c$_1$(x) $\rightarrow_1$ $\hat{\prod}_{x \in b_2}$ c$_2$(x)}
\DisplayProof\\
\item \AxiomC{b$_1$ $\rightarrow_1$ b$_2$}
\LeftLabel{$Uno$-I$_6$-red$_I$)}
\UnaryInfC{b$_1$ $\hat{+}$ c $\rightarrow_1$ b$_2$ $\hat{+}$ c}
\DisplayProof \quad
\AxiomC{c$_1$ $\rightarrow_1$ c$_2$}
\LeftLabel{$Uno$-I$_6$-red$_{II}$)}
\UnaryInfC{b$\hat{+}$ c$_1$  $\rightarrow_1$ b $\hat{+}$ c$_2$}
\DisplayProof
\item \AxiomC{d$_1$ $\rightarrow_1$ d$_2$}
\LeftLabel{$Uno$-I$_7$-red$_I$)}
\UnaryInfC{$\hat{List}$(d$_1$) $\rightarrow_1$ $\hat{List}$(d$_2$)}
\DisplayProof
\item \AxiomC{t$_1$ $\rightarrow_1$ t$_2$}
\LeftLabel{$Uno$-I$_8$-red$_I$)}
\UnaryInfC{$\hat{Id}$(t$_1$,w,z) $\rightarrow_1$ $\hat{Id}$(t$_2$,w,z)}
\DisplayProof \quad
\AxiomC{w$_1$ $\rightarrow_1$ w$_2$}
\LeftLabel{$Uno$-I$_8$-red$_{II}$)}
\UnaryInfC{$\hat{Id}$(t,w$_1$,z) $\rightarrow_1$ $\hat{Id}$(t,w$_2$,z)}
\DisplayProof \quad
\AxiomC{z$_1$ $\rightarrow_1$ z$_2$}
\LeftLabel{$Uno$-I$_8$-red$_{III}$)}
\UnaryInfC{$\hat{Id}$(t,w,z$_1$) $\rightarrow_1$ $\hat{Id}$(t,w,z$_2$)}
\DisplayProof
\end{itemize}




