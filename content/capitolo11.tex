\chapter{Principi di induzione per tipi induttivi}
\label{cap:principi-di-induzione-per-tipi-induttivi}
Di seguito tratto i principi di induzione, associati alle regole di \textit{Eliminazione} dei cosiddetti tipi induttivi \textit{T}, per renderne la ragione del nome; quando il tipo M(z) type[$\Gamma$,z $\in$ \textit{T}], verso cui si elimina, \`e in realt\`a una $\varphi$(z) prop[$\Gamma$,z $\in$ \textit{T}].\\\\
\noindent
\textbf{Tipi induttivi:} N$_0$, N$_1$, List(A), Nat, B $+$ C, {\scriptsize $\sum$} $_{x \in B}$ C(x), Id(A,a,b)

\section{Propriet\`a per i tipi induttivi}
\label{sec:prorieta-per-i-tipi-induttivi}
Per poter enunciare il principio di induzione, ci servono delle \textbf{propriet\`a di manipolazione dei contesti}.
\begin{itemize}
\item {\scriptsize $\sum$} serve per \textbf{raggruppare le assunzioni in un contesto}:\\
c(x,y) $\in$ C(x,y)[x $\in$ A,y $\in$ B(x)] derivabile $\Rightarrow$ c($\pi_1$z,$\pi_2$z) $\in$ C($\pi_1$z,$\pi_2$z)[z $\in$ {\scriptsize $\sum\limits$}$_{x \in A}$B(x)] \`e derivabile\\
Si pu\`o anche andare nella direzione opposta, dunque\\
d(z) $\in$ D(z)[z $\in$ {\scriptsize $\sum\limits$}$_{x \in A}$B(x)] derivabile $\Rightarrow$ d($<$x,y$>$) $\in$  D($<$x,y$>$)[x $\in$ A,y $\in$ B(x)] derivabile
\item {\scriptsize $\prod$} serve per \textbf{togliere le assunzioni in un contesto}:\\
c(x,y) $\in$ C(x,y)[x $\in$ A,y $\in$ B(x)] derivabile $\Rightarrow$ $\lambda$y.c(x) $\in$ {\scriptsize $\prod$}$_{y \in B(x)}$C(x,y)[x $\in$ A]\\
Per cui posso sempre togliere l'ultima assunzione.
\end{itemize}
\noindent
\textbf{Notazione:} quando ho $\varphi$(z) $\rightarrow$ $\varphi^I$(z) \\\\

\noindent
Di seguito enuncio i vari tipi di induzione, con spiegazione per il principio dei numeri Naturali.
\section{Principio di induzione numeri Naturali}
\label{sec:principio-di-induzione-numeri-Naturali}
\begin{prooftree}
\AxiomC{M(z)type[$\Gamma$, z $\in$ Nat]}
\AxiomC{\begin{tabular}[c]{cc}c $\in$ M(0)[$\Gamma$] \\ e(x,y) $\in$ M(succ(x))[$\Gamma$,x $\in$ Nat,y $\in$ M(x)]\end{tabular}}
\LeftLabel{E-Nat$_{dip}$)}
\BinaryInfC{El$_{Nat}$(z,c,e) $\in$ M(z)[$\Gamma$,z $\in$ Nat]}
\end{prooftree}
\noindent
\`E da tale regola che si ricavano i principi di induzione.\\\\
\noindent
\textit{Esempio}
\begin{prooftree}
\AxiomC{$\varphi$(z)prop[$\Gamma$, z $\in$ Nat]}
\AxiomC{\begin{tabular}[c]{cc}c $\in$ $\varphi$(0)[$\Gamma$] \\ e(x,y) $\in$ $\varphi$(succ(x))[$\Gamma$,z $\in$ Nat,y $\in$ M(x)]\end{tabular}}
\LeftLabel{E-Nat$_{dip}$)}
\BinaryInfC{El$_{Nat}$(z,c,e) $\in$ $\varphi$(z)[$\Gamma$,z $\in$ Nat]}
\end{prooftree}
\noindent
$\Leftrightarrow$ $\lambda$z.El$_{Nat}$(z,c,e) $\in$ $\forall_{z \in Nat} \hspace{0.1cm}\varphi$(z) (usando {\scriptsize $\prod$} per togliere le assunzioni di contesto)\\
A questo punto posso raggrupparlo in un unico termine, non solo con la manipolazione del contesto ma anche trasformando le assunzioni della regola in variabili\\
$\lambda$z.El$_{Nat}$(z,w,(x,y)e(x,y)) $\in$ $\forall_{z \in Nat} \varphi(z)$[$\Gamma$,w $\in$ $\varphi$(0);z$_f$ $\in$ $\forall_{x \in Nat}$ ($\varphi$(x) $\rightarrow$ $\varphi$(succ(x)))] $\Rightarrow$
$\lambda$z.El$_{Nat}$(z,w,(x,y).Ap(Ap(z$_f$,x),y) $\in$ $\forall_{z \in Nat} \varphi(z)$[$\Gamma$,w $\in$ $\varphi$(0) $\times$/$\&$ z$_f$ $\in$ $\forall_{x \in Nat}$ ($\varphi$(x) $\rightarrow$ $\varphi$(succ(x)))]\\\\
\noindent
Assumendo che, quando astraggo, $\varphi$ true[$\Gamma$,$\Psi$ true] derivabile $\Leftrightarrow$ $\Psi$ $\rightarrow$ $\varphi$ true[$\Gamma$]\\\\
\noindent
In questo modo arrivo ad avere il principio di induzione che rende valido l'eliminatore dei numeri Naturali\\
\begin{center}Nat) $\varphi$(0) $\&$ $\forall_{x \in Nat}$($\varphi$(x) $\rightarrow$ $\varphi$(succ(x)) $\rightarrow$ $\forall_{z \in Nat}$ $\varphi$(z)\end{center}
\`e valido e segue dalla \textit{(E-Nat$_{dip}$)}

\section{Principio di induzione su N$_0$}
\label{sec:principio-di-induzione-N0}
\begin{center}N$0$) $\bot$ true $\rightarrow$ $\varphi$[$\Gamma$]\end{center} \`e valido per ogni $\varphi$ prop[$\Gamma$]
\section{Principio di induzione su N$_1$}
\label{sec:principio-di-induzione-N1}
\begin{center}N$_1$) $\varphi(\ast)$ $\rightarrow$ $\forall_{x \in N_1}$ $\varphi$(x) true[$\Gamma$] \end{center}\`e valido per ogni $\varphi(z)$ prop[$\Gamma$,z $\in$ N$_1$]

\section{Principio di induzione su List(A)}
\label{sec:principio-di-induzione-List(A)}
Supponiamo A type[$\Gamma$] derivabile \\
\begin{center}List(A)) $\varnothing(nil)$ $\&$ $\forall_{x \in List(A)}$ $\forall_{w \in A}$($\varnothing(x)$ $\rightarrow$ $\varnothing(cons(x,w))$) $\rightarrow$ $\forall_{z \in List(A)}$ $\varphi$(z) true[$\Gamma$]\end{center} \`e derivabile\\

\section{Principio di induzione su B+C}
\label{sec:principio-di-induzione-B+C}
$\varphi$(z) prop[$\Gamma$, z $\in$ B $+$ C] derivabile
\noindent
\begin{center}B$+$C) $\forall_{x_1 \in B}$  $\varphi(nl(x_1))$ $\&$ $\forall_{x_2 \in C}$  $\varphi(nr(x_2))$ $\rightarrow$ $\forall_{z\in B+C}$  $\varphi(z)$ true[$\Gamma$]\end{center} \`e derivabile
\section{Principio di induzione su $\sum\limits_{x \in B}$ C(x)}
\label{sec:principio-di-induzione-sum}
Supponiamo $\varphi$(z) prop[$\Gamma$,z $\in$ B $+$ C]
\begin{center}{\scriptsize $\sum$}) $\forall_{x \in B}$ $\forall_{y \in C(x)}$ $\varnothing(<x,y>)$ $\rightarrow$ $\forall_{z \in}$ {\scriptsize$_{\sum}$}$_{x \in B \hspace{0.1cm}C(x)}$ $\varphi$(z) true[$\Gamma$]\end{center} \`e derivabile\\
con $\forall_{z \in}$ {\scriptsize$_{\sum}$}$_{x \in B \hspace{0.1cm}C(x)}$ famiglia di insiemi $=$ unione indiciata disgiunta\\

\section{Principio di induzione su Id(A,a,b)}
\label{sec:principio-di-induzione-Id}
\`E un tipo induttivo, grande novit\`a, introdotta da \textit{Martin-L$\ddot{o}$f}, all'interno della teoria dei tipi.\\
\begin{center}Dato $\varphi(z_1,z_2,z_3)$ prop[$\Gamma$,z$_1$ $\in$ A,z$_2$ $\in$ A,z$_3$ $\in$ Id(A,z$_1$,z$_2$)] derivabile\\
Ind) $\forall_{x \in A}$ $\varphi(x,x,id(x))$ $\rightarrow$ $\forall_{z_1 \in A}$ $\forall_{z_2 \in A}$ $\forall_{z_3 \in Id(A,z_1,z_2)}$ $\varphi(z_1,z_2,z_3)$ true[$\Gamma$]\end{center} \`e derivabile\\\\
\noindent
Tale principio non \`e formulato su $\varphi$(z) type[$\Gamma$,$z \in Id(A,a,b)$] con A e a,b fissati, ma dipende da tre parametri z$_1$, z$_2$ e z$_3$, come accade nella regola del \textit{(E-Id)}.\\\\
\noindent In conclusione,  ho mostrato, come i \textbf{principi di induzione} servono per dimostrare le proposizioni; e le \textbf{regola di eliminazione} non determinano solo ricorsori, ma anche regole di induzione.

\section{Esercizi}
\label{sec:esercizi-principi-induzione}
\paragraph{1)}
\textbf{Dimostrare che i principi di induzione sono validi in teoria dei tipi.}\\\\
\textbf{Soluzione}\\\\
\begin{itemize}
\item \begin{center}N$_0$) $\bot$ true $\rightarrow$ $\varphi$[$\Gamma$]\end{center}
\noindent\\
Devo verificare che pf $\in$ $\bot$ $\rightarrow$ $\varphi[\Gamma]$ $\equiv$ pf $\in$ {\scriptsize $\prod$}$_{z \in \bot}$ $\varphi$ type[$\Gamma$] $\equiv$ pf $\in$ {\scriptsize $\prod$}$_{z \in N_0}$ $\varphi$ type[$\Gamma$] sia derivabile in teoria dei tipi.\\\\
\noindent
pf $\in$ {\scriptsize $\prod$}$_{z \in N_0}$ $\varphi$ type[$\Gamma$] applicando \textit{(I-{\scriptsize $\prod$})} diventa El$_{N_0}$(z) $\in$ $\varphi$ type[$\Gamma$,z $\in$ N$_0$]

\begin{prooftree}
\AxiomC{}
\UnaryInfC{$\varphi$ type[$\Gamma$]}

\AxiomC{$\Gamma$ cont}
\LeftLabel{F-N$_0$}
\UnaryInfC{N$_0$ type[$\Gamma$]}
\LeftLabel{F-c}\RightLabel{(z $\in$ N$_0$) $\notin$ $\Gamma$}
\UnaryInfC{$\Gamma$,z $\in$ N$_0$ cont}
\LeftLabel{ind-ty}
\BinaryInfC{$\varphi$ type[$\Gamma$,z $\in$ N$_0$]}
\LeftLabel{E-N$_{0dip}$}
\UnaryInfC{El$_{N_0}$(z) $\in$ $\varphi$ type[$\Gamma$,z $\in$ N$_0$]}
\end{prooftree}


\item \begin{center}Nat) $\varphi$(0) $\&$ $\forall_{x \in Nat}$($\varphi$(x) $\rightarrow$ $\varphi$(succ(x)) $\rightarrow$ $\forall_{z \in Nat}$ $\varphi$(z) true[$\Gamma$]\end{center}
Devo verificare che pf $\in$ ${\scriptsize \prod}_{y \in (\varphi(0) \times \prod_{x \in Nat}(\prod_{s \in \varphi(x)}\varphi(succ(x)))}$ ${\scriptsize \prod}_{z \in Nat} \varphi(z)$ type[$\Gamma$] che equivale a\\
$\lambda z$.El$_{Nat}$(z,w,Ap(Ap(k,x),y)) $\in$ ${\scriptsize \prod}_{z \in Nat} \varphi(z)$ type[$\Gamma$, z $\in$ $(w \in \varphi(0) \times k \in \prod_{x \in Nat}(\prod_{s \in \varphi(x)}\varphi(succ(x))))$]\\
Applico \textit{(I-{\scriptsize $\prod$})} e ottengo \\
El$_{Nat}$(z,w,Ap(Ap(k,x),y)) $\in$ $\varphi(z)$ type[$\Gamma$, z $\in$ $(w \in \varphi(0) \times k \in \prod_{x \in Nat}(\prod_{s \in \varphi(x)}\varphi(succ(x))))$, z $\in$ Nat]\\
\\\\
\noindent
$\Delta$ $\equiv$ $\Gamma$, z $\in$ $(w \in \varphi(0) \times k \in \prod_{x \in Nat}(\prod_{s \in \varphi(x)}\varphi(succ(x))))$
\small
\begin{adjustwidth}{-7em}{}
\begin{prooftree}
\AxiomC{}
\UnaryInfC{$\varphi(z)$ type[$\Delta$,z $\in$ Nat]}
\AxiomC{\textbf{1}}
\UnaryInfC{w $\in$ $\varphi$(0)[$\Delta$]}
\AxiomC{\textbf{2}}
\UnaryInfC{Ap(Ap(k,x),y) $\in$ $\varphi(succ(x))$[$\Delta$, x $\in$ Nat, y $\in$ $\varphi(x)$]}
\LeftLabel{E-Nat$_{dip}$}
\TrinaryInfC{El$_{Nat}$(z,w,Ap(Ap(k,x),y)) $\in$ $\varphi(z)$ type[$\Delta$,z $\in$ Nat]}
\end{prooftree}
\end{adjustwidth}
\normalsize
\textbf{1}\\\\
\noindent
\small
\begin{adjustwidth}{-7em}{}
\begin{prooftree}
\AxiomC{}
\UnaryInfC{$w \in \varphi(0)$ type[$\Gamma$]}
\AxiomC{}
\UnaryInfC{$k \in \prod_{x \in Nat}(\prod_{s \in \varphi(x)}\varphi(succ(x))))$ type[$\Gamma$]}
\LeftLabel{F-{\scriptsize $\sum$}}
\BinaryInfC{$w \in \varphi(0) \times k \in \prod_{x \in Nat}(\prod_{s \in \varphi(x)}\varphi(succ(x))))$ type[$\Gamma$]}
\LeftLabel{F-c}\RightLabel{\begin{tabular}[cc]{c}(z $\in$ $(w \in \varphi(0) \times k \in \prod_{x \in Nat}$ \\$(\prod_{s \in \varphi(x)}\varphi(succ(x))))$) $\notin$ $\Gamma$\end{tabular}}
\UnaryInfC{$\Delta$ cont}
\LeftLabel{var}
\UnaryInfC{w $\in$ $\varphi$(0)[$\Delta$]}
\end{prooftree}
\end{adjustwidth}
\normalsize










\item \begin{center}N$_1$) $\varphi(\ast)$ $\rightarrow$ $\forall_{x \in N_1}$ $\varphi$(x) true[$\Gamma$] \end{center}
$\equiv$ $\prod_{z \in \varphi(\ast)}$ $\forall_{x \in N_1}$ $\varphi$(x) true[$\Gamma$]\\
$\equiv$ $\prod_{z \in \varphi(\ast)}$ $\prod_{x \in N_1}$ $\varphi$(x) true[$\Gamma$]\\

$\Rightarrow$ Devo verificare che pf $\in$ $\prod_{z \in \varphi(\ast)}$ $\prod_{x \in N_1}$ $\varphi$(x)[$\Gamma$]\\\\
\noindent






\item \begin{center}List(A)) $\varnothing(nil)$ $\&$ $\forall_{x \in List(A)}$ $\forall_{w \in A}$($\varnothing(x)$ $\rightarrow$ $\varnothing(cons(x,w))$) $\rightarrow$ $\forall_{z \in List(A)}$ $\varphi$(z) true[$\Gamma$]\end{center}
\item \begin{center}B$+$C) $\forall_{x_1 \in B}$  $\varphi(nl(x_1))$ $\&$ $\forall_{x_2 \in C}$  $\varphi(nr(x_2))$ $\rightarrow$ $\forall_{z\in B+C}$  $\varphi(z)$ true[$\Gamma$]\end{center}
\item \begin{center}{\scriptsize $\sum$}) $\forall_{x \in B}$ $\forall_{y \in C(x)}$ $\varnothing(<x,y>)$ $\rightarrow$ $\forall_{z \in}$ {\scriptsize$_{\sum}$} $_{x \in B \hspace{0.1cm}C(x)}$ $\varphi$(z) true[$\Gamma$]\end{center}
\item \begin{center}Ind)$\forall_{x \in A}$ $\varphi(x,x,id(x))$ $\rightarrow$ $\forall_{z_1 \in A}$ $\forall_{z_2 \in A}$ $\forall_{z_3 \in Id(z_1,z_2,z_3)}$ $\varphi(z_1,z_2,z_3)$ true[$\Gamma$]\end{center}
\end{itemize}

\paragraph{2)}
\textbf{Mostrare che per ogni tipo B $+$ C esiste un \textit{proof-term} del tipo\\
\begin{center}Id(B $+$ C,z,z$^\backprime$) $\leftrightarrow$ \\ $\exists_{x \in B}$ $\exists_{x\backprime \in B}$ Id(B $+$ C,z,inl(x)) $\wedge$ Id(B $+$ C,z$\backprime$,inl(x$^\backprime$)) $\wedge$ Id(B,x,x$^\backprime$)\\
$\vee$\\
$\exists_{y \in C}$ $\exists_{y^\backprime \in C}$ Id(B $+$ C,z,inr(y)) $\wedge$ Id(B $+$ C,z$^\backprime$,inr(y$^\backprime$)) $\wedge$ Id(B,y,y$^\backprime$)) type[z $\in$ B $+$ C,z$^\backprime$ $\in$ B $+$ C]\end{center}
}
\noindent
\textbf{Soluzione}\\\\
\noindent
Definisco:
\begin{enumerate}
\item $\phi$ $\equiv$ Id(B $+$ C,z,z$^\backprime$)
\item $\psi$ $\equiv$ $\exists_{x \in B}$ $\exists_{x\backprime \in B}$ Id(B $+$ C,z,inl(x)) $\wedge$ Id(B $+$ C,z$\backprime$,inl(x$^\backprime$)) $\wedge$ Id(B,x,x$^\backprime$)\\
$\vee$\\
$\exists_{y \in C}$ $\exists_{y^\backprime \in C}$ Id(B $+$ C,z,inr(y)) $\wedge$ Id(B $+$ C,z$^\backprime$,inr(y$^\backprime$)) $\wedge$ Id(B,y,y$^\backprime$)) type[z $\in$ B $+$ C,z$^\backprime$ $\in$ B $+$ C] $\equiv$ $\sum_{x \in B}$ $\sum_{x\backprime \in B}$ (Id(B $+$ C,z,inl(x)) $\times$ Id(B $+$ C,z$\backprime$,inl(x$^\backprime$)) $\times$ Id(B,x,x$^\backprime$))\\
$+$\\
$\sum_{y \in C}$ $\sum_{y^\backprime \in C}$ (Id(B $+$ C,z,inr(y)) $\times$ Id(B $+$ C,z$^\backprime$,inr(y$^\backprime$)) $\times$ Id(B,y,y$^\backprime$))\\
type[z $\in$ B $+$ C,z$^\backprime$ $\in$ B $+$ C]
\item Dunque in
\begin{itemize}
\item teoria dei tipi
\begin{center}pf $\in$ ($\phi \rightarrow \psi$) $\&$ ($\psi \rightarrow \phi$) type[z $\in$ B $+$ C,z$^\backprime$ $\in$ B $+$ C]\end{center}
\item logica predicativa con l'uguaglianza
\begin{center}pf $\in$ ($\prod_{w \in \phi} \psi$) $\times$ ($\prod_{w \in \psi} \phi$) type[z $\in$ B $+$ C,z$^\backprime$ $\in$ B $+$ C]\end{center}
\end{itemize}
\end{enumerate}
\small

\begin{adjustwidth}{-7em}{}
\begin{prooftree}
\AxiomC{}
\UnaryInfC{$<pf_1,pf_2>$ $\in$ ($\prod_{w \in \phi} \psi$) $\times$ ($\prod_{w \in \psi} \phi$) type[z $\in$ B $+$ C,z$^\backprime$ $\in$ B $+$ C]}
\end{prooftree}
\end{adjustwidth}

\normalsize
\noindent
\begin{itemize}
\item Devo individuare $pf_1$ tale che $\prod_{w \in Id(B + C,z,z^\backprime)}$ $\psi$ type[z $\in$ B $+$ C,z$^\backprime$ $\in$ B $+$ C]
\item Devo individuare $pf_2$ tale che $\prod_{w \in \psi}$ $\phi$ type[z $\in$ B $+$ C,z$^\backprime$ $\in$ B $+$ C]] 
\end{itemize}

\noindent
Devo individuare un pf che appartenga all'Id, dunque uso la regola dell'eliminatore dipendente:\\
pf$_1$ $\equiv$ $\lambda$w.pf$_1$(w) $\equiv$  $\lambda$w.El$_{Id}$(w,(t).El$_+$(t,e$_B$,e$_C$))\\\\\\
\noindent
e$_B$???\\
e$_C$???\\\\\\
\noindent
pf$_2$ $\equiv$ $\lambda$w.pf$_2$(w) \\
?\\


\paragraph{3)}
\textbf{Si dimostri che esiste un \textit{proof-term} del tipo
\begin{center} pf $\in$ Id(A,x,z)[x $\in$ A,y $\in$ A,z $\in$ A,w$_1$ $\in$ Id(A,x,y),w$_2$ $\in$ Id(A,y,z)]\end{center}}
\noindent
\textbf{Soluzione}\\\\
\noindent
Utilizzo {\scriptsize $\prod$} per togliere le assunzioni in un contesto (\S \ref{sec:prorieta-per-i-tipi-induttivi}):\\
$\Rightarrow$ $\lambda w_2.id(w_2)$ $\in$ $\prod_{w_2 \in Id(A,y,z)}$ Id(A,x,z)[x $\in$ A,y $\in$ A,z $\in$ A,w$_1$ $\in$ Id(A,x,y)]\\
$\Rightarrow$ $\lambda z.\lambda w_2.id(w_2)$ $\in$ $\prod_{z \in A} \prod_{w_2 \in Id(A,y,z)}$ Id(A,x,z)[x $\in$ A,y $\in$ A,w$_1$ $\in$ Id(A,x,y)]\\
$\Rightarrow$ $\lambda w_1. \lambda z.\lambda w_2.id(w_2)$ $\in$ $\prod_{w_1 \in Id(A,x,y)}$ $\prod_{z \in A}$ $\prod_{w_2 \in Id(A,y,z)}$ $\in$ Id(A,x,z)[x $\in$ A,y $\in$ A]\\
$\Rightarrow$ $\lambda w_1.(El_{Id}(w_1, (k).\lambda z.\lambda w_2.w_2))$ $\in$ $\prod_{w_1 \in Id(A,x,y)}$ $\prod_{z \in A}$ $\prod_{w_2 \in Id(A,y,z)}$ $\in$ Id(A,x,z)[x $\in$ A,y $\in$ A]\\
Applico \textit{(I-{\scriptsize $\prod$})} per poter poi derivare con \textit(E-Id$_{dip}$).

\begin{itemize}
\item Premesse:
\begin{itemize}
\item M(z$_1$,z$_2$,z$_3$) $\equiv$ M(x,y,w$_1$)
\item M(z$_1$,z$_2$,z$_3$) type[$\Gamma,z_1 \in A,z_2 \in A,z_3 \in Id(A,z_1,z_2)]$ $\equiv$ {\scriptsize$\prod$}$_{z \in A}$ {\scriptsize$\prod$}$_{w_2 \in Id(A,y,z)}$ Id(A,x,z)[x $\in$ A,y $\in$ A,w$_1$ $\in$ Id(A,x,y)]
\item e(k) $\in$ M(k,k,id(k))[$\Gamma,x \in A$] $\equiv$ $\lambda z.\lambda w_2.w_2$ $\in$ ${\scriptsize\prod}_{z \in A}$ ${\scriptsize\prod}_{w_2 \in Id(A,k,z)}$ Id(A,k,z)[x $\in$ A,y $\in$ A,w$_1$ $\in$ Id(A,x,y),k $\in$ A]
\end{itemize}
\item Conclusione:
\begin{itemize}
\item El$_{Id}$(z$_3$, (k).e(k)) $\equiv$ El$_{Id}$(w$_1$, (k).$\lambda z.\lambda w_2.w_2$)
\end{itemize}
\end{itemize}

\scriptsize
\begin{adjustwidth}{-11em}{}
\begin{prooftree}
\AxiomC{\textbf{1}}
\UnaryInfC{{\scriptsize$\prod$}$_{z \in A}$ {\scriptsize$\prod$}$_{w_2 \in Id(A,y,z)}$ Id(A,x,z)[x $\in$ A,y $\in$ A,w$_1$ $\in$ Id(A,x,y)]}
\AxiomC{\textbf{2}}
\UnaryInfC{$\lambda z.\lambda w_2.w_2$ $\in$ ${\scriptsize\prod}_{z \in A}$ ${\scriptsize\prod}_{w_2 \in Id(A,k,z)}$ Id(A,k,z)[x $\in$ A,y $\in$ A,w$_1$ $\in$ Id(A,x,y),k $\in$ A]}
\LeftLabel{E-Id$_{dip}$}
\BinaryInfC{El$_{Id}$(w$_1$, (k).$\lambda z.\lambda w_2.w_2$) $\in$ {\scriptsize$\prod$}$_{z \in A}$ {\scriptsize$\prod$}$_{w_2 \in Id(A,y,z)}$ Id(A,x,z)[x $\in$ A,y $\in$ A,w$_1$ $\in$ Id(A,x,y)]}
\end{prooftree}
\end{adjustwidth}
\normalsize
\textbf{1}\\\\
\noindent 
$\Delta$ $\equiv$ x $\in$ A,y $\in$ A,w$_1$ $\in$ Id(A,x,y), z $\in$ A
\scriptsize
\begin{adjustwidth}{-5em}{}
\begin{prooftree}
\AxiomC{}
\UnaryInfC{A type[ ]}

\AxiomC{}
\UnaryInfC{A type[x $\in$ A,y $\in$ A]}
\AxiomC{}
\UnaryInfC{x $\in$ A[x $\in$ A,y $\in$ A]}
\AxiomC{}
\UnaryInfC{y $\in$ A[x $\in$ A,y $\in$ A]}
\LeftLabel{F-Id}
\TrinaryInfC{Id(A,x,y) type[x $\in$ A,y $\in$ A]}
\LeftLabel{F-c}\RightLabel{\begin{tabular}[c]{cc}(w$_1$ $\in$ Id(A,x,y)) \\ $\notin$ (x $\in$ A,y $\in$ A)\end{tabular}}
\UnaryInfC{x $\in$ A,y $\in$ A,w$_1$ $\in$ Id(A,x,y) cont}
\LeftLabel{ind-ty}
\BinaryInfC{A type[x $\in$ A,y $\in$ A,w$_1$ $\in$ Id(A,x,y)]}

\AxiomC{\textbf{1$^\backprime$}}
\UnaryInfC{{\scriptsize$\prod$}$_{w_2 \in Id(A,y,z)}$ Id(A,x,z)[$\Delta$]}
\LeftLabel{F-${\scriptsize\prod}$}
\BinaryInfC{{\scriptsize$\prod$}$_{z \in A}$ {\scriptsize$\prod$}$_{w_2 \in Id(A,y,z)}$ Id(A,x,z)[x $\in$ A,y $\in$ A,w$_1$ $\in$ Id(A,x,y)]}
\end{prooftree}
\end{adjustwidth}

\normalsize
\textbf{1$^\backprime$}\\\\
\scriptsize
\begin{adjustwidth}{-8em}{}
\begin{prooftree}
\AxiomC{}
\UnaryInfC{A type[$\Delta$]}
\AxiomC{}
\UnaryInfC{y $\in$ A[$\Delta$]}
\AxiomC{}
\UnaryInfC{z $\in$ A[$\Delta$]}
\LeftLabel{F-Id}
\TrinaryInfC{Id(A,y,z) type[$\Delta$]}


\AxiomC{}
\UnaryInfC{A type[$\Delta$, w$_2$ $\in$ Id(A,y,z)]}
\AxiomC{}
\UnaryInfC{x $\in$ A[$\Delta$, w$_2$ $\in$ Id(A,y,z)]}
\AxiomC{}
\UnaryInfC{z $\in$ A[$\Delta$, w$_2$ $\in$ Id(A,y,z)]}
\LeftLabel{F-Id}
\TrinaryInfC{Id(A,x,z)[$\Delta$, w$_2$ $\in$ Id(A,y,z)]}
\LeftLabel{F-${\scriptsize\prod}$}
\BinaryInfC{{\scriptsize$\prod$}$_{w_2 \in Id(A,y,z)}$ Id(A,x,z)[$\Delta$]}
\end{prooftree}
\end{adjustwidth}


\normalsize
\textbf{2}\\\\
\scriptsize
\begin{adjustwidth}{2em}{}
\begin{prooftree}
\AxiomC{}
\UnaryInfC{A type[$\Delta$, k $\in$ A, z $\in$ A]}
\AxiomC{}
\UnaryInfC{k $\in$ A [$\Delta$, k $\in$ A, z $\in$ A]}
\AxiomC{}
\UnaryInfC{z $\in$ A [$\Delta$, k $\in$ A, z $\in$ A]}
\LeftLabel{F-Id}
\TrinaryInfC{Id(A,k,z)[$\Delta$, k $\in$ A, z $\in$ A]}
\LeftLabel{F-c}\RightLabel{\begin{tabular}[c]{cc}(w$_2$ $\in$ Id(A,k,z)) $\notin$ \\ ($\Delta$, k $\in$ A, z $\in$ A)\end{tabular}}
\UnaryInfC{$\Delta$, k $\in$ A, z $\in$ A, w$_2$ $\in$ Id(A,k,z) cont}
\LeftLabel{var}
\UnaryInfC{w$_2$ $\in$ Id(A,k,z)[$\Delta$, k $\in$ A, z $\in$ A, w$_2$ $\in$ Id(A,k,z)]}
\LeftLabel{I-${\scriptsize\prod}$}
\UnaryInfC{$\lambda w_2.w_2$ $\in$ ${\scriptsize\prod}_{w_2 \in Id(A,k,z)}$ Id(A,k,z)[$\Delta$, k $\in$ A, z $\in$ A]}
\LeftLabel{I-${\scriptsize\prod}$}
\UnaryInfC{$\lambda z.\lambda w_2.w_2$ $\in$ ${\scriptsize\prod}_{z \in A}$ ${\scriptsize\prod}_{w_2 \in Id(A,k,z)}$ Id(A,k,z)[$\Delta$,k $\in$ A]}
\end{prooftree}
\end{adjustwidth}
\normalsize
