
\chapter{Tipo della somma disgiunta}
\label{cap: tipo-somma-disgiunta}
Il tipo somma disgiunta \`e un costruttore di tipo. Questi non \`e dipendente se da solo, lo diventa solo quando agisce su un tipo dipendente.\\
Anche con il tipo somma disgiunta si parla di tipo induttivo (accade gi\`a per N$_1$, Nat e List(A)).\\
Le regole del tipo della somma disgiunta sono le seguenti.

\section{Regole di Formazione}
\label{sec: formazione-disjointsum}
\begin{prooftree}
\AxiomC{B type [$\Gamma$]}
\AxiomC{C type [$\Gamma$]}
\LeftLabel{F-$+$)}
\BinaryInfC{B + C type [$\Gamma$]}
\end{prooftree}

\section{Regole di Introduzione}
\label{sec: introduzione-disjointsum}
\begin{center}
\AxiomC{b $\in$ B[$\Gamma$]}
\AxiomC{B + C type[$\Gamma$]}
\LeftLabel{I$_1$-$+$)}
\BinaryInfC{inl(b) $\in$ B + C[$\Gamma$]}
\DisplayProof \qquad
\AxiomC{c $\in$ C[$\Gamma$]}
\AxiomC{B + C type[$\Gamma$]}
\LeftLabel{I$_2$-$+$)}
\BinaryInfC{inr(c) $\in$ B + C[$\Gamma$]}
\DisplayProof
\end{center}
\noindent
\textit{Nelle regole di Introduzione inl e inr sevono, rispettivamente, per introdurre un termine a sinistra e a destra.}

\section{Regole di Eliminazione}
\label{sec: eliminazione-disjointsum}
\small
\begin{adjustwidth}{-5em}{}
\begin{prooftree}
\AxiomC{\begin{tabular}[c]{cc}M(z) type[$\Gamma$, z $\in$ B + C]\\t $\in$ B + C[$\Gamma$]\end{tabular}}
\AxiomC{e$_B$(x$_1$) $\in$ M(inl(x$_1$))[$\Gamma$, x$_1$ $\in$ B]}
\AxiomC{e$_C$(x$_2$) $\in$ M(inr(x$_2$))[$\Gamma$, x$_2$ $\in$ C]}
\LeftLabel{E-$+$)}
\TrinaryInfC{El$_+$(t,e$_B$,e$_C$) $\in$ M(t)[$\Gamma$]}
\end{prooftree}
\end{adjustwidth}

\section{Regole di Conservazione}
\label{sec: conservazione-disjointsum}
\small
\begin{adjustwidth}{-5em}{}
\begin{prooftree}
\AxiomC{\begin{tabular}[c]{cc}M(z) type[$\Gamma$, z $\in$ B + C]\\b $\in$ B[$\Gamma$]\end{tabular}}
\AxiomC{e$_B$(x$_1$) $\in$ M(inl(x$_1$))[$\Gamma$, x$_1$ $\in$ B]}
\AxiomC{e$_C$(x$_2$) $\in$ M(inr(x$_2$))[$\Gamma$, x$_2$ $\in$ C]}
\LeftLabel{C$_1$-$+$)}
\TrinaryInfC{El$_+$(inl(b),e$_B$,e$_C$) $=$ e$_B$(b) $\in$ M(inl(b))[$\Gamma$]}
\end{prooftree}
\end{adjustwidth}
\small
\begin{adjustwidth}{-5em}{}
\begin{prooftree}
\AxiomC{\begin{tabular}[c]{cc}M(z) type[$\Gamma$, z $\in$ B + C]\\c $\in$ C[$\Gamma$]\end{tabular}}
\AxiomC{e$_B$(x$_1$) $\in$ M(inl(x$_1$))[$\Gamma$, x$_1$ $\in$ B]}
\AxiomC{e$_C$(x$_2$) $\in$ M(inr(x$_2$))[$\Gamma$, x$_2$ $\in$ C]}
\LeftLabel{C$_2$-$+$)}
\TrinaryInfC{El$_+$(inr(c),e$_B$,e$_C$) $=$ e$_C$(c) $\in$ M(inr(c))[$\Gamma$]}
\end{prooftree}
\end{adjustwidth}


\section{Regole di Uguaglianza}
\label{sec: uguaglianza-disjointsum}
\normalsize
\begin{prooftree}
\AxiomC{B$_1$ $=$ B$_2$ type[$\Gamma$]}
\AxiomC{C$_1$ $=$ C$_2$ type[$\Gamma$]}
\LeftLabel{eq-F-$+$)}
\BinaryInfC{B$_1$ + C$_1$ $=$ B$_2$ + C$_2$ type($\Gamma$)}
\end{prooftree}

\section{Eliminatore dipendente}
\label{sec:eliminatore dipendente-disjointsum}
L'eliminatore ha, anche nel caso della somma disgiunta, la forma dipendente.
\small
\begin{adjustwidth}{-5em}{}
\begin{prooftree}
\AxiomC{M(z) type[$\Gamma$, z $\in$ B + C]}
\AxiomC{e$_B$(x$_1$) $\in$ M(inl(x$_1$))[$\Gamma$, x$_1$ $\in$ B]}
\AxiomC{e$_C$(x$_2$) $\in$ M(inr(x$_2$))[$\Gamma$, x$_2$ $\in$ C]}
\LeftLabel{E$_{dip}$-$+$)}
\TrinaryInfC{El$_+$(z,e$_B$,e$_C$) $\in$ M(z)[$\Gamma$,z $\in$ B + C]}
\end{prooftree}
\end{adjustwidth}
\noindent
\normalsize
Da sottolineare come con questo eliminatore non definisce solo che si ricorre su B $+$ C, ma da, nascosto, anche un principio di induzione. Inoltre si possono vedere B $+$ C come due proposizioni $\beta$ e $\gamma$ (come illustro in \ref{sec:osservazioni-dal-punto-di-vista-logico-disjointsum}).

\section{Semantica operazionale della somma disgiunta}
\label{sec: semantica-operazionale-somma-disgiunta}
La relazione $\rightarrow_1$ viene definita all'interno dei termini con l'uso delle seguenti regole di riduzione:
\begin{itemize}
\item $\beta_{1+}$-red) El$_+$(inl(b), e$_B$, e$_C$) $\rightarrow_1$ e$_B$(b)
\item $\beta_{2+}$-red) El$_+$(inr(c), e$_B$, e$_C$) $\rightarrow_1$ e$_C$(c)
\item \AxiomC{t$_1$ $\rightarrow_1$ t$_2$}
\LeftLabel{$+$-red)}
\UnaryInfC{El$_+$(t$_1$, e$_B$, e$_C$) $\rightarrow_1$ El$_+$(t$_2$, e$_B$, e$_C$)}
\DisplayProof \qquad
\item novit\`a della somma disgiunta rispetto al tipo singoletto
\AxiomC{b$_1$ $\rightarrow_1$ b$_2$}
\LeftLabel{$+$-red$_I$)}
\UnaryInfC{inl(b$_1$) $\rightarrow_1$ inl(b$_2$)}
\DisplayProof
\item \AxiomC{c$_1$ $\rightarrow_1$ c$_2$}
\LeftLabel{$+$-red$_{II}$)}
\UnaryInfC{inr(c$_1$) $\rightarrow_1$ inr(c$_2$)}
\DisplayProof
\end{itemize}

\section{Osservazioni dal punto di vista logico}
\label{sec:osservazioni-dal-punto-di-vista-logico-disjointsum} 
Le regole di \textit{Eliminazione/Introduzione} partono dalle regole di congiunzione nella logica (nel caso in cui sia B che C siano proposizioni), inoltre nel caso dell'eliminazione si va a eliminare verso altre proposizioni.\\Tale concetto \`e stato voluto da \textit{Martin-L$\ddot{o}$f}, per riuscire a interpretare la logica con la teoria dei tipi.
\subsection{La regola di Formazione}
\label{subsec:la-regola-di-formazione-osservazioni-disjointsum}
Se si hanno le proposizioni $\beta$ prop[$\Gamma$] e $\gamma$ prop[$\Gamma$] e si interpreta la somma come disgiunzione, allora la regola di \textit{Formazione} diventa
\begin{prooftree}
\AxiomC{$\beta$ prop [$\Gamma$]}
\AxiomC{$\gamma$ prop [$\Gamma$]}
\LeftLabel{F-$+$)}
\BinaryInfC{B $\vee$ C prop [$\Gamma$]}
\end{prooftree}
Questi ha influenza sia sulla regola \textit{(E-$_{dip}+$)} che  \textit{(I-$+$)}. 

\subsubsection{La regola di Eliminazione}
\begin{prooftree}
\AxiomC{$\xi$ prop[$\Gamma$]}
\AxiomC{$\xi$ \`e vero[$\Gamma$,supponiamo $\beta$ vero]}
\AxiomC{$\xi$ \`e vero[$\Gamma$,$\gamma$ \`e vero]}
\LeftLabel{E$_{dip}$-$+$)}
\TrinaryInfC{\cancel{El$_+$(z,e$_B$,e$_C$)} $\in$ $\xi$ vero[$\Gamma$,z $\in$ $\beta$ $\vee$ $\gamma$ vero]}
\end{prooftree}
che, nel calcolo dei sequenti, equivale alla or a sinistra
\begin{prooftree}
\AxiomC{$\beta$ $\vdash_\Gamma$ $\xi$}
\AxiomC{$\gamma$ $\vdash_\Gamma$ $\xi$}
\LeftLabel{$\vee$-S)}
\BinaryInfC{$\beta$ $\vee$ $\gamma$ $\vdash_\Gamma$ $\xi$}
\end{prooftree}
\noindent
Allo stesso modo le regole di \textit{Introduzione} si possono vedere in modo molto semplice.

\subsubsection{Le regole di Introduzione}
\begin{prooftree}
\AxiomC{b $\in$ $\beta$[$\Gamma$]}
\AxiomC{$\beta$ + $\gamma$ prop[$\Gamma$]}
\LeftLabel{I$_1$-$+$)}
\BinaryInfC{inl(b) $\in$ $\beta$ $\vee$ $\gamma$}
\end{prooftree}
b $\in$ $\beta$ $\equiv$ $\beta$ vero\\
$\beta$ $\vee$ $\gamma$ $\equiv$ $\beta$ $\vee$ $\gamma$ vero\\
Questa, nel calcolo dei sequenti, equivale alla or a destra\\
Con $\Delta$ contesto, allora
\begin{prooftree}
\AxiomC{$\Delta$ $\vdash_\Gamma$ $\beta$}
\LeftLabel{$\vee$-D)}
\UnaryInfC{$\Delta$ $\vdash_\Gamma$ $\beta$ $\vee$ $\gamma$}
\end{prooftree}
\noindent
Lo stesso si pu\`o fare sulla seconda regola di \textit{Introduzione}.
\begin{prooftree}
\AxiomC{c $\in$ $\gamma$[$\Gamma$]}
\AxiomC{$\beta$ + $\gamma$ prop[$\Gamma$]}
\LeftLabel{I$_2$-$+$)}
\BinaryInfC{inr(c) $\in$ $\beta$ $\vee$ $\gamma$}
\end{prooftree}
c $\in$ $\gamma$ $\equiv$ $\gamma$ vero\\
$\beta$ $\vee$ $\gamma$ $\equiv$ $\beta$ $\vee$ $\gamma$ vero\\
Questa, nel calcolo dei sequenti, equivale alla regola di riduzione naturale nella logica\\
Con $\Delta$ contesto, allora
\begin{prooftree}
\AxiomC{$\Delta$ $\vdash_\Gamma$ $\gamma$}
\LeftLabel{I$_2$-$+$)}
\UnaryInfC{$\Delta$ $\vdash_\Gamma$ $\beta$ $\vee$ $\gamma$}
\end{prooftree}
\noindent
\vspace{0.5cm}


\section{Esercizi}
\label{sec: es-somma-disgiunta}
\paragraph{1)}
\textbf{Si scrivano le regole del tipo booleano come tipo semplice e si provi che \`e rappresentabile come N$_1$ + N$_2$.}\\\\
\textbf{Soluzione}\\\\
Definisco il tipo Bool nel modo seguente
Bool $=$ {true || false}\\
inl($\ast$) $\equiv$ true\\
inr($\ast$) $\equiv$ false\\
\noindent Per definire regole e semantica operazionale uso che N$_1$ + N$_2$ $=$ Bool.
\begin{itemize}
\item \textbf{Regole del tipo Bool:}

\begin{itemize}
\item\textit{Regole di Formazione}
\begin{prooftree}
\AxiomC{$\Gamma$ cont}
\LeftLabel{F-Bool)}
\UnaryInfC{Bool type[$\Gamma$]}
\end{prooftree}

\item\textit{Regole di Introduzione}
\begin{center}
\AxiomC{$\Gamma$ cont}
\LeftLabel{I$_1$-Bool)}
\UnaryInfC{true $\in$ Bool[$\Gamma$]}
\DisplayProof \qquad
\AxiomC{$\Gamma$ cont}
\LeftLabel{I$_2$-Bool)}
\UnaryInfC{false $\in$ Bool[$\Gamma$]}
\DisplayProof
\end{center}

\item\textit{Regole di Eliminazione}
\small
\begin{prooftree}
\AxiomC{M(z) type[$\Gamma$, z $\in$ Bool]}
\AxiomC{t $\in$ Bool[$\Gamma$]}
\AxiomC{\begin{tabular}[c]{cc}e$_B$ $\in$ M(true)[$\Gamma$] \\ e$_C$ $\in$ M(false)[$\Gamma$]\end{tabular}}
\LeftLabel{E-Bool)}
\TrinaryInfC{El$_{Bool}$(t,e$_B$,e$_C$) $\in$ M(t)[$\Gamma$]}
\end{prooftree}
\small
\begin{prooftree}
\AxiomC{M(z) type[$\Gamma$, z $\in$ Bool]}
\AxiomC{\begin{tabular}[c]{cc}e$_B$ $\in$ M(true)[$\Gamma$] \\ e$_C$ $\in$ M(false)[$\Gamma$]\end{tabular}}
\LeftLabel{E-Bool$_{dip}$)}
\BinaryInfC{El$_{Bool}$(z,e$_B$,e$_C$) $\in$ M(z)[$\Gamma$,z $\in$ Bool]}
\end{prooftree}

\item\normalsize\textit{Regole di Conversione}
\small
\begin{prooftree}
\AxiomC{M(z) type[$\Gamma$, z $\in$ Bool]}
\AxiomC{e$_B$ $\in$ M(true)[$\Gamma$]}
\AxiomC{e$_C$ $\in$ M(false)[$\Gamma$]}
\LeftLabel{C$_1$-Bool)}
\TrinaryInfC{El$_{Bool}$(true,e$_B$,e$_C$) $=$ e$_B$ $\in$ M(true)[$\Gamma$]}
\end{prooftree}
\begin{prooftree}
\AxiomC{M(z) type[$\Gamma$, z $\in$ Bool]}
\AxiomC{e$_B$ $\in$ M(true)[$\Gamma$]}
\AxiomC{e$_C$ $\in$ M(false)[$\Gamma$]}
\LeftLabel{C$_2$-Bool)}
\TrinaryInfC{El$_{Bool}$(false,e$_B$,e$_C$) $=$ e$_C$ $\in$ M(false)[$\Gamma$]}
\end{prooftree}

\item\normalsize\textit{Regole di Uguaglianza}
\small
\begin{center}
\AxiomC{M(z) type [$\Gamma$, z $\in$ Bool]}
\AxiomC{t $=$ t$^\backprime$ $\in$ Bool[$\Gamma$]}
\AxiomC{\begin{tabular}[c]{cc}e$_B$ $=$ s $\in$ M(true)[$\Gamma$]\\e$_C$ $=$ u $\in$ M(false)[$\Gamma$]\end{tabular}}
\LeftLabel{eq-E-Bool)}
\TrinaryInfC{El$_{Bool}$(t,e$_B$,e$_C$) $=$ El$_{Bool}$(t$^\backprime$,s,u) $\in$ M(t$^\backprime$)[$\Gamma$]}
\DisplayProof
\end{center}
\end{itemize}

\item \textbf{Semantica operazionale del tipo Bool:}
\begin{itemize}
\item $\beta_{1Bool}$-red) El$_{Bool}$(true, e$_B$, e$_C$) $\rightarrow_1$ e$_B$
\item $\beta_{2Bool}$-red) El$_{Bool}$(false, e$_B$, e$_C$) $\rightarrow_1$ e$_C$
\item IF-true) if true then M else N $\rightarrow_1$ M
\item IF-false) if false then M else N $\rightarrow_1$ N
\item \AxiomC{t$_1$ $\rightarrow_1$ t$_2$}
\LeftLabel{Bool-red)}
\UnaryInfC{El$_{Bool}$(t$_1$, e$_B$, e$_C$) $\rightarrow_1$ El$_{Bool}$(t$_2$, e$_B$, e$_C$)}
\DisplayProof
\item \AxiomC{M$_1$ $\rightarrow_1$ M$_1^\backprime$}
\LeftLabel{IF)}
\UnaryInfC{if M$_1$ then M$_2$ else M$_3$ $\rightarrow_1$ if M$_1^\backprime$ then M$_2$ else M$_3$}
\DisplayProof
\end{itemize}

\end{itemize}




