\chapter{Tipo dei numeri Naturali}
\label{sec: tipo-naturali}

\section{Regole di Formazione}
\label{subsec: formazione-nat}
\begin{prooftree}
\AxiomC{$\Gamma$ cont}
\LeftLabel{F-Nat)}
\UnaryInfC{Nat type[$\Gamma$]}
\end{prooftree}

\section{Regole di Introduzione}
\label{subsec: eliminazione-nat}
\begin{center}
\AxiomC{$\Gamma$ cont}
\LeftLabel{I$_1$-Nat)}
\UnaryInfC{0 $\in$ Nat[$\Gamma$]}
\DisplayProof \qquad
\AxiomC{m $\in$ Nat[$\Gamma$]}
\LeftLabel{I$_2$-Nat)}
\UnaryInfC{succ(m) $\in$ Nat[$\Gamma$]}
\DisplayProof
\end{center}

\section{Regole di Eliminazione}
\label{subsec: eliminazione-nat}
\small
\begin{prooftree}
\AxiomC{t $\in$ Nat[$\Gamma$]}
\AxiomC{M(z) type[$\Gamma$, z $\in$ Nat]}
\AxiomC{c $\in$ M(0)[$\Gamma$]}
\AxiomC{e(x,y) $\in$ M(succ(x))[$\Gamma$, x $\in$ Nat, y $\in$ M(x)]}
\LeftLabel{E-Nat)}
\QuaternaryInfC{El$_{Nat}$(t,c,e) $\in$ M(t)[$\Gamma$]}
\end{prooftree}

\section{Regole di Conversione}
\label{subsec: conversione-nat}
\small
\begin{prooftree}
\AxiomC{M(z) type[$\Gamma$, z $\in$ Nat]}
\AxiomC{c $\in$ M(0)[$\Gamma$]}
\AxiomC{e(x,y) $\in$ M(succ(x))[$\Gamma$, x $\in$ Nat, y $\in$ M(x)]}
\LeftLabel{C$_1$-Nat)}
\TrinaryInfC{El$_{Nat}$(0,c,e) = c $\in$ M(0)[$\Gamma$]}
\end{prooftree}

\begin{prooftree}
\AxiomC{m  $\in$ Nat[$\Gamma$]}
\AxiomC{M(z) type[$\Gamma$, z $\in$ Nat]}
\AxiomC{c $\in$ M(0)[$\Gamma$]}
\AxiomC{e(x,y) $\in$ M(succ(x))[$\Gamma$, x $\in$ Nat, y $\in$ M(x)]}
\LeftLabel{C$_2$-Nat)}
\QuaternaryInfC{El$_{Nat}$(succ(m),c,e) = e(m, El$_{Nat}$(m,c,e)) $\in$ M(succ(m))[$\Gamma$]}
\end{prooftree}

\section{Regole di Uguaglianza}
\label{subsec: uguaglianza-nat}
\normalsize
\begin{center}
\begin{prooftree}
\AxiomC{t$_1$ = t$_2$ $\in$ Nat[$\Gamma$]}
\LeftLabel{eq-Nat)}
\UnaryInfC{succ(t$_1$) = succ(t$_2$) $\in$ Nat[$\Gamma$]}
\end{prooftree}
\end{center}

\section{Introduzione ed Eliminatore dipendente}
\label{subsec:introduzione-eliminatore dipendente-naturale}
Le regole di formazione dei tipi e dei loro termini sono formulate in modo da rendere la regole si sostituzione per tipi e termini ammissibili.\\
Ad esempio la regola di introduzione del successore di un numero naturale si pu\`o formulare come un esplicito programma funzionale visto come termine dipendente.
\begin{prooftree}
\AxiomC{$\Gamma$ cont}
\LeftLabel{I$_2$-Nat$_{prog}$)}
\UnaryInfC{succ(x) $\in$ Nat[$\Gamma$, x $\in$ Nat]}
\end{prooftree}
Il medesimo discorso vale per la regola di eliminazione
\begin{prooftree}
\AxiomC{M(z) type[$\Gamma$, z $\in$ Nat]}
\AxiomC{c $\in$ M(0)[$\Gamma$]}
\AxiomC{e(x,y) $\in$ M(succ(x))[$\Gamma$, x $\in$ Nat, y $\in$ M(x)]}
\LeftLabel{E-Nat$_{dip}$)}
\TrinaryInfC{El$_{Nat}$(z,c,e) $\in$ M(z)[$\Gamma$, z $\in$ Nat]}
\end{prooftree}
\noindent
\textit{E-Nat} \`e equivalente a \textit{E-Nat$_{dip}$}. Difatti la teoria \textit{T$_{N1Nat}$}, in cui c'\`e \textit{E-Nat}, \`e equivalente a \textit{T$\backprime$} senza \textit{E-Nat}, ma con \textit{E-Nat$_{dip}$}, le regole di sosituzione e di \textit{sanitary check}.

\section{Primitiva ricorsiva}
\label{subsec: primitiva-ricorsiva}
\textit{Definizione}\\\\
Nat$^n$ $\times$ Nat $\rightarrow$ Nat\\
Dati g$_0$: Nat$^m$ $\rightarrow$ Nat e g$_1$: Nat$^m$ $\times$ Nat $\times$ Nat $\rightarrow$ Nat\\
n$_1$...n$_m$ $\in$ Nat allora\\\\
rec(n$_1$...n$_m$, 0) $\equiv$ g$_0$(n$_1$...n$_m$)\\
rec(n$_1$...n$_m$, k+1) $\equiv$ g$_0$(n$_1$...n$_m$, k, rec(n$_1$...n$_m$, k))

\section{Semantica operazionale dei numeri naturali}
\label{subsec: semantica-operazionale-naturali}
La relazione $\rightarrow_1$ viene definita all'interno dei termini con l'uso delle seguenti regole di riduzione:
\begin{itemize}
\item $\beta_{1Nat}$-red) El$_{Nat}$(0, c, e) $\rightarrow_1$ c
\item $\beta_{2Nat}$-red) El$_{Nat}$(succ(m), c, e) $\rightarrow_1$ e(m,El$_{Nat}$(m, c, e))
\item \AxiomC{t$_1$ $\rightarrow_1$ t$_2$}
\LeftLabel{red$_I$)}
\UnaryInfC{El$_{N1}$(t$_1$, c, e) $\rightarrow_1$ El$_{N1}$(t$_2$, c, e)}
\DisplayProof \qquad
\AxiomC{c$_1$ $\rightarrow_1$ c$_2$}
\LeftLabel{red$_{II}$)}
\UnaryInfC{El$_{N1}$(t, c$_1$, e) $\rightarrow_1$ El$_{N1}$(t, c$_2$, e)}
\DisplayProof 
\item Novit\`a dei numeri naturali rispetto al tipo singoletto
\AxiomC{t$_1$ $\rightarrow_1$ t$_2$}
\LeftLabel{N-red)}
\UnaryInfC{succ(t$_1$) $\rightarrow_1$ succ(t$_2$)}
\DisplayProof
\item + riduzione $\rightarrow_1$ rispetto a N$_1$
\end{itemize}

\section{Addizione per ricorsione}
\label{subsec: addizione-per-ricorsione}
\textit{Di seguito riporto un esercizio, svolto in aula, con lo scopo di comprendere come l'uguaglianza definizionale, tra numeri naturali, non coincide con l'uguaglianza aritmetica in matematica.}
\\\\
La somma, tra numeri naturali, viene definita usando l'eliminatore dipendente \textit{E-Nat$_{dip}$}(\S \ref{subsec:introduzione-eliminatore dipendente-naturale}). In questo modo si riesce a definire per la nostra teoria \textit{T$_{1Nat}$}
\begin{center} w + z $\in$ Nat [w $\in$ Nat, z $\in$ Nat] \\ come \\ El$_{Nat}$(z, w, (x,y).succ(y))[w $\in$ Nat, z $\in$ Nat]\end{center}
Usando la nozione di primitava ricorsiva e decidendo di ricorrere su z \begin{center} w + 0 $\equiv$ w \\ w + succ(z) $\equiv$ succ(w+z) $\equiv$ succ(y)\end{center}
Ecco che l'albero di derivazione assume la seguente forma:
\scriptsize
\begin{adjustwidth}{-11em}{}
\begin{prooftree}
\AxiomC{[ ] cont}
\LeftLabel{F-Nat}
\UnaryInfC{Nat type [ ]}
\LeftLabel{F-c}\RightLabel{(w $\in$ Nat) $\notin$ [ ]}
\UnaryInfC{w $\in$ Nat cont}
\LeftLabel{F-Nat}
\UnaryInfC{Nat type[w $\in$ Nat]}
\LeftLabel{F-c}\RightLabel{(z $\in$ Nat) $\notin$ w $\in$ Nat}
\UnaryInfC{w $\in$ Nat, z $\in$ Nat cont}
\LeftLabel{F-Nat}
\UnaryInfC{Nat type[w $\in$ Nat, z $\in$ Nat]}
\AxiomC{[ ] cont}
\LeftLabel{F-Nat}
\UnaryInfC{Nat type [ ]}
\LeftLabel{F-c}\RightLabel{(w $\in$ Nat) $\notin$ [ ]}
\UnaryInfC{w $\in$ Nat cont}
\LeftLabel{var}
\UnaryInfC{w $\in$ Nat[w $\in$ Nat]}
\AxiomC{[ ] cont}
\LeftLabel{F-Nat}
\UnaryInfC{Nat type [ ]}
\LeftLabel{F-c}\RightLabel{(w $\in$ Nat) $\notin$ [ ]}
\UnaryInfC{w $\in$ Nat cont}
\LeftLabel{F-Nat}
\UnaryInfC{Nat type[w $\in$ Nat]}
\LeftLabel{F-c}\RightLabel{\begin{tabular}[c]{cc}(z $\in$ Nat) $\notin$ \\ w $\in$ Nat \end{tabular}}
\UnaryInfC{w $\in$ Nat, z $\in$ Nat cont}
\LeftLabel{I$_2$-Nat$_{prog}$}
\UnaryInfC{succ(y) $\in$ [w $\in$ Nat, z $\in$ Nat, y $\in$ Nat]}
\LeftLabel{E-Nat$_{dip}$}
\TrinaryInfC{El$_{Nat}$(z, w, (x,y).succ(y))[w $\in$ Nat, z $\in$ Nat]}
\end{prooftree}
\end{adjustwidth}
\noindent
\normalsize
\begin{itemize}
\item Prova induttiva (minimo dei controlli da fare per verificare l'esattezza della derivazione)
\begin{itemize}
\item Caso base: cosa accade se poniamo al posto di z lo 0?\\
El$_{Nat}$(0, w, (x,y).succ(y)) $\rightarrow_1$ w per \textit{$\beta_{1Nat}$-red} ($\equiv$ w + 0 = w)
\item Passo induttivo: ricorro su z (esempio succ(0) = 1)\\
El$_{Nat}$(succ(0), w, (x,y).succ(y)) per \textit{$\beta_{1Nat}$-red} $\rightarrow_1$ succ(El$_{Nat}$(0, w, (x,y).succ(y))) $\rightarrow_1$ succ(w) \\
Dunque w + 1 = succ(w) $\in$ Nat
\end{itemize}
\noindent
$\Rightarrow$ Il programma fa effettivamente quello che dovrebbe.
\end{itemize}

\subsection{Osservazioni sull'addizione}
\label{subsubsec: osservazioni-addizione}
w $+_1$ z $\equiv$ El$_{Nat}$(z, w, (x,y), succ(y)) $\neq$ come NF da z.\\
Se sostituisco w con 0, allora 0 $+_1$ z $\equiv$ w $+_1$ z[$\frac{w}{0}$] $\equiv$ El$_{Nat}$(z, 0, (x,y), succ(y)) \`e in NF (dunque non riesco pi\`u a ridurla ulteriormente). 
\\Ecco che 0 $+_1$ z \`e un valore in NF $\neq$ da z, da cui \`e impossibile dimostrare che 0 $+_1$ z = z $\in$ Nat[z $\in$ Nat].\\
Questo non accade per w $+_1$ 0 = w (\S \ref{subsec: addizione-per-ricorsione}).\\ 
Perci\`o, se noi scriviamo la somma riccorrendo sul secondo membro, riusciamo a dire che \textit{primo-membro $+_1$ 0 = primo-membro}, ma non che \textit{0 $+_1$ secondo-membro = secondo - membro}. In quanto non esiste alcuna sottostrategia deterministica, per il secondo caso, per cui il programma si ferma.
\\\\
In conclusione, l'uguaglianza definizionale di termini con variabili \`e diversa dall'uguaglianza aritmentica. Eccezione fatta per le espressioni chiuse, senza variabili, perch\`e il termine chiuso si riduce a un'unica NF, che si dimostra essere un numero arabico.

\section{Esercizi}
\label{sec: es-naturali}

\paragraph{1}
\textbf{Dimostrare che le regole enunciate in \S\ref{sec: tipo-naturali}, \textit{I$_2$-Nat$_{prog}$} ed \textit{E-Nat$_{dip}$}, sono ammissibili nel sistema di teoria dei tipi dei numeri naturali.}
\\\\
\textbf{Soluzione}\\\\
\noindent
In I$_2$-Nat$_{prog}$ il contesto \`e $\Sigma$ = $\Gamma$, x $\in$ Nat che permette l'applicazione di I$_2$ Nat.
\begin{prooftree}
\AxiomC{$\Gamma$ cont}
\LeftLabel{F-Nat}
\UnaryInfC{Nat type[$\Gamma$]}
\LeftLabel{F-c}\RightLabel{(x $\in$ Nat) $\notin$ $\Gamma$}
\UnaryInfC{$\Gamma$, x $\in$ Nat cont}
\LeftLabel{var}
\UnaryInfC{x $\in$ Nat[$\Gamma$, x $\in$ Nat]}
\LeftLabel{I$_2$ Nat}
\UnaryInfC{succ(x) $\in$ Nat[$\Gamma$, x $\in$ Nat]}
\end{prooftree}
\vspace{0.5cm}
\noindent
\normalsize{Assumo che le premesse di \textit{I$_2$-Nat$_{prog}$} (\textit{$\Gamma$ count}) sia valida, perci\`o \`e valido, dalla prova sopra, anche il giudizio di conclusione \textit{succ(x) $\in$ Nat[$\Gamma$, x $\in$ Nat]}, di conseguenza derivabile in \textit{T$^\backprime$}.}
\\\\
\noindent
In E-Nat$_{dip}$ il contesto \`e $\Sigma$ = $\Gamma$, z $\in$ Nat, che permette l'applicazione di E-Nat.
\scriptsize
\begin{adjustwidth}{-15em}{}
\begin{prooftree}
\AxiomC{$\Gamma$ cont}
\LeftLabel{F-Nat}
\UnaryInfC{Nat type[$\Gamma$]}
\LeftLabel{F-c}\RightLabel{(z $\in$ Nat) $\notin$ $\Gamma$}
\UnaryInfC{$\Gamma$, z $\in$ Nat cont}
\LeftLabel{var}
\UnaryInfC{z $\in$ Nat[$\Gamma$, z $\in$ Nat]}
\AxiomC{}
\UnaryInfC{M(z) type[$\Gamma$,z $\in$ Nat]}
\AxiomC{\textbf{1}}
\UnaryInfC{c $\in$ M(0)[$\Gamma$,z $\in$ Nat]}
\AxiomC{\textbf{2}}
\UnaryInfC{e(x,y) $\in$ M(succ(x))[$\Gamma$, z $\in$ Nat, x $\in$ Nat, y $\in$ Nat]}
\LeftLabel{E-Nat}
\QuaternaryInfC{El$_{Nat}$(z,c,e) $\in$ M(z)[$\Gamma$, z $\in$ Nat]}
\end{prooftree}
\end{adjustwidth}
\vspace{0.5cm}
\small
\textbf{1}
\begin{prooftree}
\AxiomC{}
\UnaryInfC{c $\in$ M(0)[$\Gamma$]}
\AxiomC{$\Gamma$ cont}
\LeftLabel{F-Nat}
\UnaryInfC{Nat type[$\Gamma$]}
\LeftLabel{F-c}\RightLabel{(z $\in$ Nat) $\notin$ $\Gamma$}
\UnaryInfC{$\Gamma$, z $\in$ Nat cont}
\LeftLabel{ind-te}
\BinaryInfC{c $\in$ M(0)[$\Gamma$,z $\in$ Nat]}
\end{prooftree}
\noindent
\textbf{2}
\scriptsize
\begin{adjustwidth}{-15em}{}
\begin{prooftree}
\AxiomC{}
\UnaryInfC{e(x,y) $\in$ M(succ(x))[$\Gamma$, x $\in$ Nat, y $\in$ Nat]}
\AxiomC{$\Gamma$ cont}
\UnaryInfC{Nat type[$\Gamma$]}
\LeftLabel{F-c}\RightLabel{(z $\in$ Nat) $\notin$ $\Gamma$}
\UnaryInfC{z $\in$ Nat cont}
\LeftLabel{ind-te}
\BinaryInfC{e(x,y) $\in$ M(succ(x))[$\Gamma$, x $\in$ Nat, y $\in$ Nat, z $\in$ Nat]}
\AxiomC{$\Gamma$ cont}
\UnaryInfC{Nat type[$\Gamma$]}
\LeftLabel{F-c}\RightLabel{(z $\in$ Nat) $\notin$ $\Gamma$}
\UnaryInfC{$\Gamma$, z $\in$ Nat cont}
\LeftLabel{F-Nat}
\UnaryInfC{Nat type[$\Gamma$, z $\in$ Nat]}
\LeftLabel{F-c}\RightLabel{(x $\in$ Nat) $\notin$ ($\Gamma$,z $\in$ Nat)}
\UnaryInfC{$\Gamma$, z $\in$ Nat, x $\in$ Nat cont}
\LeftLabel{F-Nat}
\UnaryInfC{Nat type[$\Gamma$, z $\in$ Nat, x $\in$ Nat]}
\LeftLabel{F-c}\RightLabel{\begin{tabular}[c]{cc}(y $\in$ Nat) $\notin$ ($\Gamma$, \\ z $\in$ Nat, x $\in$ Nat)\end{tabular}}
\UnaryInfC{$\Gamma$, z $\in$ Nat, x $\in$ Nat, y $\in$ Nat cont}
\LeftLabel{ex-te}
\BinaryInfC{e(x,y) $\in$ M(succ(x))[$\Gamma$, z $\in$ Nat, x $\in$ Nat, y $\in$ Nat]}
\end{prooftree}
\end{adjustwidth}
\vspace{0.5cm}
\noindent
\normalsize{Assumo che le premesse di \textit{E-Nat$_{dip}$} (\textit{M(z) type[$\Gamma$, z $\in$ Nat], c $\in$ M(0)[$\Gamma$], e(x,y) $\in$ M(succ(x))[$\Gamma$, x $\in$ Nat, y $\in$ M(x)]}) siano valide e che $\Gamma$ cont sia assioma ([ ] cont = [$\Gamma$] cont con $\Gamma$ = $\varnothing$), perci\`o \`e valido, dalla prova sopra, anche il giudizio di conclusione \textit{El$_{Nat}$(z,c,e) $\in$ M(z)[$\Gamma$, w $\in$ Nat]}, di conseguenza derivabile in \textit{T$^\backprime$}.}

\paragraph{2}
\textbf{Definire w + 2 $\in$ Nat[w $\in$ Nat], ove 2 \`e l'abbreviazione del termine ottenuto applicando 2 $\equiv$ succ(succ(0)).}
\\\\
\textbf{Soluzione}\\\\
La ricorsione la faccio w, usando lo schema di ricorsione primitva, vale che w + 2 $\equiv$ El$_{Nat}$(w, 2, (x,y).succ(y)) $\in$ Nat[w $\in$ Nat]

\small
\begin{adjustwidth}{-5em}{}
\begin{prooftree}
\AxiomC{[ ] cont}
\LeftLabel{F-Nat}
\UnaryInfC{Nat type[ ]}
\LeftLabel{F-c}\RightLabel{(w $\in$ Nat) $\notin$ [ ]}
\UnaryInfC{w $\in$ Nat cont}
\LeftLabel{F-Nat}
\UnaryInfC{Nat type[w $\in$ Nat]}
\AxiomC{[ ] cont}
\LeftLabel{I$1$-Nat}
\UnaryInfC{0 $\in$ Nat[ ]}
\LeftLabel{I$2$-Nat}
\UnaryInfC{1 $\in$ Nat[ ]}
\LeftLabel{I$2$-Nat}
\UnaryInfC{2 $\in$ Nat[ ]}
\AxiomC{[ ] cont}
\LeftLabel{F-Nat}
\UnaryInfC{Nat type [ ]}
\LeftLabel{F-c}\RightLabel{(x $\in$ Nat) $\notin$ [ ]}
\UnaryInfC{x $\in$ Nat cont}
\LeftLabel{I$2$-Nat$_{prog}$}
\UnaryInfC{succ(y) $\in$ Nat[x $\in$ Nat, y $\in$ Nat]}
\LeftLabel{E-Nat$_{dip}$}
\TrinaryInfC{El$_{Nat}$(w, 2, (x,y).succ(y)) $\in$ Nat[w $\in$ Nat]}
\end{prooftree}
\end{adjustwidth}
\noindent
\normalsize \textit{Dimostrazione di correttezza di El$_{Nat}$(w, 2, (x,y).succ(y)) $\in$ Nat[w $\in$ Nat]}
\begin{itemize}
\item El$_{Nat}$(0, 2, (x,y).succ(y)) $\rightarrow_1$ 2 per \textit{$\beta_{1Nat}$-red}
\item El$_{Nat}$(succ(m), 2, (x,y).succ(y)) $\rightarrow_1$ succ(El$_{Nat}$(m, 2, (x,y).succ(y))) per \textit{$\beta_{2Nat}$-red} $\Rightarrow$ per m = 0 $\equiv$ succ(El$_{Nat}$(0, 2, (x,y).succ(y))) $\rightarrow_1$ succ(2) $\in$ Nat = 3 (dal punto precedente).
\end{itemize}

\paragraph{3}
\textbf{Definire l'operazione di addizione usando le regole del tipo dei numeri naturali.}
\begin{center} x + y $\in$ Nat[x $\in$ Nat, y $\in$ Nat]\end{center}
in modo tale che valga x + 0 = x $\in$  Nat[x $\in$ Nat]
\\\\
\textbf{Soluzione}\\\\
La ricorsione la faccio y, usando lo schema di ricorsione primitva, vale che x + y $\equiv$ El$_{Nat}$(y, x, (w,z).succ(z)) $\in$ Nat[x $\in$ Nat, y $\in$ Nat]


\scriptsize
\begin{adjustwidth}{-10em}{}
\begin{prooftree}
\AxiomC{[ ] cont}
\LeftLabel{F-Nat}
\UnaryInfC{Nat type [ ]}
\LeftLabel{F-c}\RightLabel{(x $\in$ Nat) $\notin$ [ ]}
\UnaryInfC{x $\in$ Nat cont}
\LeftLabel{F-Nat}
\UnaryInfC{Nat type[x $\in$ Nat]}
\LeftLabel{F-c}\RightLabel{\begin{tabular}[c]{cc}(y $\in$ Nat) $\notin$ \\ x $\in$ Nat \end{tabular}}
\UnaryInfC{x $\in$ Nat, y $\in$ Nat cont}
\LeftLabel{F-Nat}
\UnaryInfC{Nat type[x $\in$ Nat, y $\in$ Nat]}
\AxiomC{[ ] cont}
\LeftLabel{F-Nat}
\UnaryInfC{Nat type [ ]}
\LeftLabel{F-c}\RightLabel{(x $\in$ Nat) $\notin$ [ ]}
\UnaryInfC{x $\in$ Nat cont}
\LeftLabel{var}
\UnaryInfC{x $\in$ Nat[x $\in$ Nat]}
\AxiomC{[ ] cont}
\LeftLabel{F-Nat}
\UnaryInfC{Nat type[ ]}
\LeftLabel{F-c}\RightLabel{(x $\in$ Nat) $\notin$ [ ]}
\UnaryInfC{x $\in$ Nat cont}
\LeftLabel{F-Nat}
\UnaryInfC{Nat type[x $\in$ Nat]}
\LeftLabel{F-c}\RightLabel{\begin{tabular}[c]{ccc}(w $\in$ Nat) \\ $\notin$ x $\in$ Nat \end{tabular}}
\UnaryInfC{x $\in$ Nat, w $\in$ Nat cont}
\LeftLabel{I$_{2}$-Nat$_{prog}$}
\UnaryInfC{succ(z) $\in$ Nat[x $\in$ Nat, w $\in$ Nat, z $\in$ Nat]}
\LeftLabel{E-Nat$_{dip}$}
\TrinaryInfC{El$_{Nat}$(y, x, (w,z).succ(z)) $\in$ Nat[x $\in$ Nat, y $\in$ Nat]}
\end{prooftree}
\end{adjustwidth}
\noindent
\normalsize \textit{Dimostrazione di correttezza di El$_{Nat}$(y, x, (w,z).succ(z)) $\in$ Nat[x $\in$ Nat, y $\in$ Nat]}
\begin{itemize}
\item El$_{Nat}$(0, x, (w,z).succ(z))  $\rightarrow_1$ x per \textit{$\beta_{1Nat}$-red}
\item El$_{Nat}$(succ(y), x, (w,z).succ(z)) $\rightarrow_1$ succ(El$_{Nat}$(y, x, (w,z).succ(z))) per \textit{$\beta_{2Nat}$-red} $\Rightarrow$ per y=0 $\equiv$ succ(El$_{Nat}$(0, x, (w,z).succ(z))) $\rightarrow_1$ succ(x) $\in$ Nat = x + 1  (dal punto precedente).
\end{itemize}

\paragraph{4}
\textbf{Definire l'operazione di addizione usando le regole del tipo dei numeri naturali.}
\begin{center} x + y $\in$ Nat[x $\in$ Nat, y $\in$ Nat\end{center}
in modo tale che valga 0 + y = x $\in$  Nat[x $\in$ Nat]
\\\\
\textbf{Soluzione}\\\\
La ricorsione la faccio x, percui, usando lo schema di ricorsione primitva, vale che x + y $\equiv$ El$_{Nat}$(x, y, (w,z).succ(z)) $\in$ Nat[x $\in$ Nat, y $\in$ Nat]

\small
\begin{adjustwidth}{-6em}{}
\begin{prooftree}
\AxiomC{\textbf{1}}
\UnaryInfC{El$_{Nat}$(x, y, (w,z).succ(z)) $\in$ Nat[y $\in$ Nat, x $\in$ Nat]}
\AxiomC{[ ] cont}
\LeftLabel{F-Nat}
\UnaryInfC{Nat type [ ]}
\LeftLabel{F-c}\RightLabel{(x $\in$ Nat) $\notin$ [ ]}
\UnaryInfC{x $\in$ Nat cont}
\LeftLabel{F-Nat}
\UnaryInfC{Nat type[x $\in$ Nat]}
\LeftLabel{F-c}\RightLabel{(y $\in$ Nat) $\notin$ x $\in$ Nat}
\UnaryInfC{x $\in$ Nat, y $\in$ Nat cont}
\LeftLabel{ex-te}
\BinaryInfC{El$_{Nat}$(x, y, (w,z).succ(z)) $\in$ Nat[x $\in$ Nat, y $\in$ Nat]}
\end{prooftree}
\end{adjustwidth}
\vspace{0.5cm}
\textbf{1}
\scriptsize
\begin{adjustwidth}{-17em}{}
\begin{prooftree}
\AxiomC{[ ] cont}
\LeftLabel{F-Nat}
\UnaryInfC{Nat type[ ]}
\LeftLabel{F-c}\RightLabel{(y $\in$ Nat) $\notin$ [ ]}
\UnaryInfC{y $\in$ Nat cont}
\LeftLabel{F-Nat}
\UnaryInfC{Nat type [y $\in$ Nat]}
\LeftLabel{F-c}\RightLabel{(x $\in$ Nat) $\notin$ y $\in$ Nat}
\UnaryInfC{y $\in$ Nat, x $\in$ Nat cont}
\LeftLabel{F-Nat}
\UnaryInfC{Nat type[y $\in$ Nat, x $\in$ Nat]}
\AxiomC{[ ] cont}
\LeftLabel{F-Nat}
\UnaryInfC{Nat type [ ]}
\LeftLabel{F-c}\RightLabel{(y $\in$ Nat) $\notin$ [ ]}
\UnaryInfC{y $\in$ Nat cont}
\LeftLabel{var}
\UnaryInfC{y $\in$ Nat[y $\in$ Nat]}
\AxiomC{[ ] cont}
\LeftLabel{F-Nat}
\UnaryInfC{Nat type [ ]}
\LeftLabel{F-c}\RightLabel{(y $\in$ Nat) $\notin$ [ ]}
\UnaryInfC{y $\in$ Nat cont}
\LeftLabel{F-Nat}
\UnaryInfC{Nat type[y $\in$ Nat]}
\LeftLabel{F-c}\RightLabel{\begin{tabular}[c]{cc}(w $\in$ Nat) $\notin$ \\ y $\in$ Nat \end{tabular}}
\UnaryInfC{y $\in$ Nat, w $\in$ Nat cont}
\LeftLabel{I$_2$-Nat$_{prog}$}
\UnaryInfC{succ(z) $\in$[y $\in$ Nat, w $\in$ Nat, z $\in$ Nat]}
\LeftLabel{E-Nat$_{dip}$}
\TrinaryInfC{El$_{Nat}$(x, y, (w,z).succ(z)) $\in$ Nat[y $\in$ Nat, x $\in$ Nat]}
\end{prooftree}
\end{adjustwidth}
\noindent\\
\normalsize \textit{Dimostrazione di correttezza di El$_{Nat}$(x, y, (w,z).succ(z)) $\in$ Nat[y $\in$ Nat, x $\in$ Nat]}
\begin{itemize}
\item El$_{Nat}$(0, y, (w,z).succ(z))  $\rightarrow_1$ y per \textit{$\beta_{1Nat}$-red}
\item El$_{Nat}$(succ(x), y, (w,z).succ(z)) $\rightarrow_1$ succ(El$_{Nat}$(x, y, (w,z).succ(z))) per \textit{$\beta_{2Nat}$-red} $\Rightarrow$ per x=0 $\equiv$ succ(El$_{Nat}$(0, y, (w,z).succ(z))) $\rightarrow_1$ succ(y) $\in$ Nat = y + 1  (dal punto precedente).
\end{itemize}

