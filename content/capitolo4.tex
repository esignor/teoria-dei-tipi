\chapter{Tipo delle liste di un tipo}
\label{cap: tipo-delle-liste-di-un-tipo}
Il tipo delle liste di elementi costruisce un costruttore delle liste ed \`e definito dalle regole seguenti.

\section{Regole di Formazione}
\label{subsec: formazione-liste}
\begin{prooftree}
\AxiomC{A type [$\Gamma$]}
\LeftLabel{F-cost)}
\UnaryInfC{List(A) type[$\Gamma$]}
\end{prooftree}

\section{Regole di Introduzione}
\label{subsec: introduzione-liste}
\begin{center}
\AxiomC{List(A) type [$\Gamma$]}
\LeftLabel{I$_1$-List)}
\UnaryInfC{nil $\in$ List(A)[$\Gamma$]}
\DisplayProof \qquad
\AxiomC{s $\in$ List(A)[$\Gamma$]}
\AxiomC{a $\in$ A[$\Gamma$]}
\LeftLabel{I$_2$-List)}
\BinaryInfC{cons(s,a) $\in$ List(A)[$\Gamma$]}
\DisplayProof
\end{center}

\section{Regole di Eliminazione}
\label{subsec: eliminazione-liste}
\small
\begin{adjustwidth}{-5em}{}
\begin{prooftree}
\AxiomC{\begin{tabular}[c]{ccc}M(z) type[$\Gamma$, z $\in$ List(A)]\\t $\in$ List(A)[$\Gamma$] \\c $\in$ M(nil)[$\Gamma$]\end{tabular}}
\AxiomC{e(x,w,y) $\in$ M(cons(x,w))[$\Gamma$, x $\in$ List(A),  w $\in$ A, y $\in$ M(x)]}
\LeftLabel{E-List)}
\BinaryInfC{El$_{List}$(t,c,e) $\in$ M(t)[$\Gamma$]}
\end{prooftree}
\end{adjustwidth}

\section{Regole di Conservazione}
\label{subsec: conservazione-liste}
\small
\begin{adjustwidth}{-5em}{}
\begin{prooftree}
\AxiomC{\begin{tabular}[c]{cc}M(z) type[$\Gamma$, z $\in$ List(A)] \\ c $\in$ M(nil)[$\Gamma$]\end{tabular}}
\AxiomC{e(x,w,y) $\in$ M(cons(x,w))[$\Gamma$, x $\in$ List(A),  w $\in$ A, y $\in$ M(x)]}
\LeftLabel{C$_1$-List)}
\BinaryInfC{El$_{List}$(nil,c,e) $=$ c $\in$ M(nil)[$\Gamma$]}
\end{prooftree}
\end{adjustwidth}
\small
\begin{adjustwidth}{-6em}{}
\begin{prooftree}
\AxiomC{M(z) type[$\Gamma$, z $\in$ List(A)]}
\AxiomC{\begin{tabular}[c]{ccc} s $\in$ List(A)[$\Gamma$] \\ a $\in$ A[$\Gamma$] \\ c $\in$ M(nil)[$\Gamma$]\end{tabular}}
\AxiomC{e(x,w,y) $\in$ M(cons(x,w))[$\Gamma$, x $\in$ List(A), w $\in$ A, y $\in$ M(x)]}
\LeftLabel{C$_2$-List)}
\TrinaryInfC{El$_{List}$(cons(s,a),c,e) $=$ e(s,a, E$_{List}$(s,c,e)) $\in$ M(cons(s,a))[$\Gamma$]}
\end{prooftree}
\end{adjustwidth}


\section{Regole di Uguaglianza}
\label{subsec: uguaglianza-liste}
\normalsize
\begin{prooftree}
\AxiomC{A$_1$ $=$ A$_2$ type[$\Gamma$]}
\LeftLabel{eq-I$_1$-List)}
\UnaryInfC{List(A$_1$) $=$ List(A$_2$) type($\Gamma$)}
\end{prooftree}
\begin{prooftree}
\AxiomC{s$_1$ $=$ s$_2$ $\in$ List(A)[$\Gamma$]}
\AxiomC{a$_1$ $=$ a$_2$ $\in$ A[$\Gamma$]}
\LeftLabel{eq-I$_2$-List)}
\BinaryInfC{cons(s$_1$,a$_1$) $=$ cons(s$_2$,a$_2$) $\in$ List(A)($\Gamma$)}
\end{prooftree}
\small
\begin{adjustwidth}{-8em}{}
\begin{prooftree}
\AxiomC{\begin{tabular}[c]{ccc} M(z) type[$\Gamma$, z $\in$ List(A)] \\ t$_1$ $=$ t$_2$ $\in$ List(A)[$\Gamma$] \\ c$_1$ $=$ c$_2$ $\in$ M(nil)[$\Gamma$]\end{tabular}}
\AxiomC{e$_1$(x,w,y) $=$ e$_2$(x,w,y) $\in$ M(cons(x,w))[$\Gamma$, x $\in$ List(A), w $\in$ A, y $\in$ M(x)]}
\LeftLabel{eq-E-List)}
\BinaryInfC{El$_{List}$(t$_1$, c$_1$, e$_1$) $=$ El$_{List}$(t$_2$, c$_2$, e$_2$) $\in$ M(t$_1$)[$\Gamma$]}
\end{prooftree}
\end{adjustwidth}
\normalsize
\section{Eliminatore dipendente}
\label{subsec:eliminatore dipendente-lista}
L'eliminatore ha anche la forma dipendente, conveniente da usare soprattutto per motivi pratici.
\small
\begin{adjustwidth}{-6em}{}
\begin{prooftree}
\AxiomC{\begin{tabular}[c]{cc}M(z) type[$\Gamma$, z $\in$ List(A)] \\ c $\in$ M(nil)[$\Gamma$]\end{tabular}}
\AxiomC{e(x,w,y) $\in$ M(cons(x,w))[$\Gamma$, x $\in$ List(A),  w $\in$ A, y $\in$ M(x)]}
\LeftLabel{E-List$_{dip}$)}
\BinaryInfC{El$_{List}$(z,c,e) $\in$ M(z)[$\Gamma$, z $\in$ List(A)]}
\end{prooftree}
\end{adjustwidth}
\normalsize
Quando si scrive un programma \`e bene scrivere la sostituzione espressamente, come riporto sotto.
\small
\begin{adjustwidth}{-7em}{}
\begin{prooftree}
\AxiomC{\begin{tabular}[c]{cc}D type[$\Gamma$, z $\in$ List(A)] \\ c $\in$ D($\frac{z}{nil}$)[$\Gamma$]\end{tabular}}
\AxiomC{e(x,w,y) $\in$ D($\frac{z}{cons(x,w)}$)[$\Gamma$, x $\in$ List(A),  w $\in$ A, y $\in$ D($\frac{z}{x}$)]}
\LeftLabel{E-List$_{dip}$)}
\BinaryInfC{El$_{List}$(z,c,e) $\in$ D(z)[$\Gamma$, z $\in$ List(A)]}
\end{prooftree}
\end{adjustwidth}

\normalsize
\section{Semantica operazionale del tipo lista}
\label{subsec: semantica-operazionale-lista}
La relazione $\rightarrow_1$ viene definita all'interno dei termini con l'uso delle seguenti regole di riduzione:
\begin{itemize}
\item $\beta_{1List}$-red) El$_{List}$(nil, c, e) $\rightarrow_1$ c
\item $\beta_{2List}$-red) El$_{List}$(cons(s,a), c, e) $\rightarrow_1$ e(s,a,El$_{List}$(s, c, e))
\item \AxiomC{t$_1$ $\rightarrow_1$ t$_2$}
\LeftLabel{List-red$_1$)}
\UnaryInfC{El$_{List}$(t$_1$, c, e) $\rightarrow_1$ El$_{List}$(t$_2$, c, e)}
\DisplayProof \vspace{0.3cm} \\
\AxiomC{c$_1$ $\rightarrow_1$ c$_2$}
\LeftLabel{List-red$_2$)}
\UnaryInfC{El$_{List}$(t, c$_1$, e) $\rightarrow_1$ El$_{List}$(t, c$_2$, e)}
\DisplayProof 
\item novit\`a delle liste rispetto al tipo singoletto
\AxiomC{s$_1$ $\rightarrow_1$ s$_2$}
\LeftLabel{List-red$_I$)}
\UnaryInfC{cons(s$_1$,a) $\rightarrow_1$ cons(s$_2$,a)}
\DisplayProof
\item \AxiomC{a$_1$ $\rightarrow_1$ a$_2$}
\LeftLabel{List-red$_{II}$)}
\UnaryInfC{cons(s,a$_1$) $\rightarrow_1$ cons(s,a$_2$)}
\DisplayProof
\end{itemize}

\section{Esercizi}
\label{sec: es-liste}
\paragraph{1)}
\textbf{Dati i tipi singoletto e delle liste \`e possibile definire un tipo dei numeri Naturali \textit{Nat}?}
\\\\
\textbf{Soluzione}\\\\
Posso vedere una lista come una collezione di n singoletti. Ecco che posso definire il tipo dei numeri Naturali su questa lista, per ricorsione primitiva, nel modo sottostante:
\begin{center}
0 $=$ nil \\
1 $=$ cons(nil, $\ast$) \\
n $=$ cons(cons(n-1), $\ast$)
\end{center}

\paragraph{2)}
\textbf{Dato un tipo A semplice, ovvero non dipendente, per esempio A $=$ N{$_1$}, se si vuole definire un'operazione \begin{center} append$_1$(x$^\backprime$,y$^\backprime$) $\in$ List(A)[x$^\backprime$ $\in$ List(A), y$^\backprime$ $\in$ A]\end{center}
\noindent tale che l'elemento y$^\backprime$ venga posto alla fine della lista x$^\backprime$, allora basta definire append$_1$ nel modo seguente \begin{center}append$_1$(x$^\backprime$,y$^\backprime$) $=$ cons(x$^\backprime$,y$^\backprime$) $\in$ List(A)[x$^\backprime$ $\in$ List(A), y$^\backprime$ $\in$ A]\end{center}
\noindent Ma se si vuole definire un'operazione \begin{center} append$_2$(x$^\backprime$,y$^\backprime$) $\in$ List(A)[x$^\backprime$ $\in$ List(A), y$^\backprime$ $\in$ A]\end{center} 
\noindent tale che y$^\backprime$ venga posto all'inizio della lista x$^\backprime$, allora come occorre procedere?}\\\\
\textbf{Soluzione}
\\\\
Per riuscire a porre y$^\backprime$ all'inizio della lista x$^\backprime$, devo usare la regola di eliminazione sulla lista x$^\backprime$. Per farlo, devo prima di tutto dare equazionalmente la definizione ricorsiva di append$_2$
\begin{center}
append$_2$(nil, y$^\backprime$) $=$  cons(nil, y$^\backprime$) $=$ y$^\backprime$\\
append$_2$(cons(s, x$^\backprime$), y$^\backprime$) $=$ cons(append$_2$(s, y$^\backprime$), x$^\backprime$)\\
\end{center}
Ora posso applicare la regola dell'eliminazione nel modo seguente:
\begin{itemize}
\item Premesse:
\begin{itemize}
\item M(z) type[$\Gamma$, z $\in$ List(A)] $\equiv$ List(A) type[x$^\backprime$ $\in$ List(A), y$^\backprime$ $\in$ A] (\textit{z $\in$ List(A) risulta superfluo})
\item t $\in$ List(A)[$\Gamma$] $\equiv$ x$^\backprime$ $\in$ List(A)[x$^\backprime$ $\in$ List(A), y$^\backprime$ $\in$ A]
\item c $\in$ M(nil)[$\Gamma$] $\equiv$ y$^\backprime$ $\in$ A[x$^\backprime$ $\in$ List(A), y$^\backprime$ $\in$ A]
\item e(x,w,y) $\in$ M(cons(x,w))[$\Gamma$, x $\in$ List(A),  w $\in$ A, y $\in$ M(x)] $\equiv$
cons(z,y) $\in$ List(A)[x$^\backprime$ $\in$ List(A), y$^\backprime$ $\in$ A, y$\in$ A, z $\in$ List(A)]
\end{itemize}
\noindent
\item Giudizio di conclusione:
\begin{itemize}
\item El$_{List}$(t,c,e) $\in$ M(t)[$\Gamma$] $\equiv$ El$_{List}$(x$^\backprime$, y$^\backprime$,(x,y,z).cons(z,y)) $\in$ List(A)[x$^\backprime$ $\in$ List(A), y$^\backprime$ $\in$ A]
\end{itemize}
\end{itemize}

\vspace{0.5cm}
\begin{prooftree}
\AxiomC{\textbf{1}}
\AxiomC{\textbf{2}}
\AxiomC{\textbf{3}}
\AxiomC{\textbf{4}}
\LeftLabel{E-List}
\QuaternaryInfC{El$_{List}$(x$^\backprime$, y$^\backprime$,(x,y,z).cons(z,y)) $\in$ List(A)[x$^\backprime$ $\in$ List(A), y$^\backprime$ $\in$ A]}
\end{prooftree}
\vspace{1cm}
\normalsize \textbf{1}
\small
\begin{prooftree}
\AxiomC{}
\UnaryInfC{A type[ ]}
\AxiomC{}
\LeftLabel{}
\UnaryInfC{A type[ ]}
\AxiomC{}
\UnaryInfC{A type[ ]}
\LeftLabel{F-cost}
\UnaryInfC{List(A) type[ ]}
\LeftLabel{F-c}\RightLabel{(x$^\backprime$ $\in$ List(A)) $\notin$ [ ]}
\UnaryInfC{x$^\backprime$ $\in$ List(A) cont}
\LeftLabel{ind-ty}
\BinaryInfC{A type[x$^\backprime$ $\in$ List(A)]}
\LeftLabel{F-c}\RightLabel{\begin{tabular}[c]{cc}(y$^\backprime$ $\in$ A) $\notin$ \\ x$^\backprime$ $\in$ List(A)\end{tabular}}
\UnaryInfC{x$^\backprime$ $\in$ List(A), y$^\backprime$ $\in$ A cont}
\LeftLabel{ind-ty}
\BinaryInfC{A type[x$^\backprime$ $\in$ List(A), y$^\backprime$ $\in$ A]}
\LeftLabel{F-cost}
\UnaryInfC{List(A) type[x$^\backprime$ $\in$ List(A), y$^\backprime$ $\in$ A]}
\end{prooftree}
\vspace{1cm}
\normalsize \textbf{2}\\\\
\noindent
\textit{La regola di scambio di contesto (ex-ty) pu\`o venire anche omessa.}
\small
\begin{adjustwidth}{-11em}{}
\begin{prooftree}
\AxiomC{}
\UnaryInfC{A type[ ]}
\AxiomC{}
\UnaryInfC{A type[ ]}
\LeftLabel{F-c}\RightLabel{(y$^\backprime$ $\in$ A) $\notin$ [ ]}
\UnaryInfC{y$^\backprime$ $\in$ A cont}
\LeftLabel{ind-ty}
\BinaryInfC{A type[y$^\backprime$ $\in$ A]}
\LeftLabel{F-cost}
\UnaryInfC{List(A) type[y$^\backprime$ $\in$ A]}
\LeftLabel{F-c}\RightLabel{\begin{tabular}[c]{cc}(x$^\backprime$ $\in$ List(A)) $\notin$ \\ y$^\backprime$ $\in$ A\end{tabular}}
\UnaryInfC{y$^\backprime$ $\in$ A, x$^\backprime$ $\in$ List(A) cont}
\LeftLabel{var}
\UnaryInfC{x$^\backprime$ $\in$ List(A)[y$^\backprime$ $\in$ A, x$^\backprime$ $\in$ List(A)]}
\AxiomC{}
\UnaryInfC{A type[ ]}
\AxiomC{}
\UnaryInfC{A type[ ]}
\LeftLabel{F-cost}
\UnaryInfC{List(A) type[ ]}
\LeftLabel{F-c}\RightLabel{(x$^\backprime$ $\in$ List(A)) $\notin$ [ ]}
\UnaryInfC{x$^\backprime$ $\in$ List(A) cont}
\LeftLabel{ind-ty}
\BinaryInfC{A type[x$^\backprime$ $\in$ List(A)]}
\LeftLabel{F-c}\RightLabel{(y$^\backprime$ $\in$ A) $\notin$ x$^\backprime$ $\in$ List(A)}
\UnaryInfC{x$^\backprime$ $\in$ List(A), y$^\backprime$ $\in$ A cont}
\LeftLabel{ex-ty}
\BinaryInfC{x$^\backprime$ $\in$ List(A)[x$^\backprime$ $\in$ List(A), y$^\backprime$ $\in$ A]}
\end{prooftree}
\end{adjustwidth}
\vspace{1cm}
\normalsize \textbf{3}
\small
\begin{prooftree}
\AxiomC{}
\UnaryInfC{A type[ ]}
\AxiomC{}
\UnaryInfC{A type[ ]}
\LeftLabel{F-cost}
\UnaryInfC{List(A) type [ ]}
\LeftLabel{F-c}\RightLabel{(x$^\backprime$ $\in$ List(A)) $\notin$ [ ]}
\UnaryInfC{x$^\backprime$ $\in$ List(A) cont}
\LeftLabel{ind-ty}
\BinaryInfC{A type[x$^\backprime$ $\in$ List(A)]}
\LeftLabel{F-c}\RightLabel{(y$^\backprime$ $\in$ A) $\notin$ x$^\backprime$ $\in$ List(A)}
\UnaryInfC{x$^\backprime$ $\in$ List(A), y$^\backprime$ $\in$ A cont}
\LeftLabel{var}
\UnaryInfC{y$^\backprime$ $\in$ A[x$^\backprime$ $\in$ List(A), y$^\backprime$ $\in$ A]}
\end{prooftree}
\vspace{1cm}
\normalsize \textbf{4}
\small
\begin{adjustwidth}{-5em}{}
\begin{prooftree}
\AxiomC{\textbf{4$^\backprime$}}
\UnaryInfC{x$^\backprime$ $\in$ List(A), y$^\backprime$ $\in$ A, z $\in$ List(A), y$\in$ A, z $\in$ List(A) cont}
\LeftLabel{var}
\UnaryInfC{z $\in$ List(A)[x$^\backprime$ $\in$ List(A), y$^\backprime$ $\in$ A, y$\in$ A, z $\in$ List(A)]}
\AxiomC{\textbf{4$^\backprime$}}
\UnaryInfC{x$^\backprime$ $\in$ List(A), y$^\backprime$ $\in$ A, y$\in$ A, z $\in$ List(A) cont}
\LeftLabel{var}
\UnaryInfC{y $\in$ A[x$^\backprime$ $\in$ List(A), y$^\backprime$ $\in$ A, y$\in$ A, z $\in$ List(A)]}
\LeftLabel{I$_2$-List}
\BinaryInfC{cons(z,y) $\in$ List(A)[x$^\backprime$ $\in$ List(A), y$^\backprime$ $\in$ A, y$\in$ A, z $\in$ List(A)]}
\end{prooftree}
\end{adjustwidth}
\vspace{1cm}
\normalsize \textbf{4$^\backprime$}
\small
\begin{adjustwidth}{-3em}{}
\begin{prooftree}
\AxiomC{}
\UnaryInfC{A type[ ]}
\AxiomC{}
\UnaryInfC{A type[ ]}
\AxiomC{}
\UnaryInfC{A type[ ]}
\AxiomC{}
\UnaryInfC{A type[ ]}
\LeftLabel{F-cost}
\UnaryInfC{List(A) type[ ]}
\LeftLabel{F-c}\RightLabel{(x$^\backprime$ $\in$ List(A)) $\notin$ [ ]}
\UnaryInfC{x$^\backprime$ $\in$ List(A) cont}
\LeftLabel{ind-ty}
\BinaryInfC{A type[x$^\backprime$ $\in$ List(A)]}
\LeftLabel{F-c}\RightLabel{(y$^\backprime$ $\in$ A) $\notin$ x$^\backprime$ $\in$ List(A)}
\UnaryInfC{x$^\backprime$ $\in$ List(A), y$^\backprime$ $\in$ A cont}
\LeftLabel{ind-ty}
\BinaryInfC{A type[x$^\backprime$ $\in$ List(A), y$^\backprime$ $\in$ A]}
\LeftLabel{F-c}\RightLabel{\begin{tabular}[c]{cc}(y $\in$ A) $\notin$ (x$^\backprime$ $\in$ List(A), \\ y$^\backprime$ $\in$ A)\end{tabular}}
\UnaryInfC{x$^\backprime$ $\in$ List(A), y$^\backprime$ $\in$ A, y $\in$ A cont}
\LeftLabel{ind-ty}
\BinaryInfC{List(A) type[x$^\backprime$ $\in$ List(A), y$^\backprime$ $\in$ A, y $\in$ A]}
\LeftLabel{F-c}\RightLabel{(z $\in$ List(A)) $\notin$  (x$^\backprime$ $\in$ List(A), y$^\backprime$ $\in$ A, y $\in$ A)}
\UnaryInfC{x$^\backprime$ $\in$ List(A), y$^\backprime$ $\in$ A, y $\in$ A, z $\in$ List(A) cont}
\end{prooftree}
\end{adjustwidth}
\vspace{0.5cm}
\noindent
\normalsize \textit{Dimostrazione di correttezza di El$_{List}$(x$^\backprime$, y$^\backprime$,(x,y,z).cons(z,y)) $\in$ List(A)[x$^\backprime$ $\in$ List(A), y$^\backprime$ $\in$ A]}
\begin{itemize}
\item El$_{List}$(nil, y$^\backprime$,(x,y,z).cons(z,y)) $\rightarrow_1$ y$^\backprime$ per \textit{$\beta_{1List}$-red}
\item El$_{List}$(cons(s,x$^\backprime$), y$^\backprime$,(x,y,z).cons(z,y)) $\rightarrow_1$ cons(El$_{List}$(x$^\backprime$, y$^\backprime$, (x,y,z).cons(z,y))) per \textit{$\beta_{2List}$-red} $\Rightarrow$ per x$^\backprime$ $=$ nil $\equiv$ cons(El$_{List}$(nil, y$^\backprime$, (x,y,z).cons(z,y))) $\rightarrow_1$ cons(nil, y$^\backprime$) $\in$ List(A) $=$ y$^\backprime$ (dal punto precedente).
\end{itemize}



\paragraph{3)}
\textbf{Definire l'operazione \textit{append} di accostamento di una lista a un'altra di tipo A type[ ] usando le regole del tipo delle liste
\begin{center}append(x,y) $\in$ List(A)[x $\in$ List(A), y $\in$ List(A)]\end{center}
in modo tale che valga append(x,nil) = x $\in$ List(A)[x $\in$ List(A)]
}
\\\\
\noindent
\textbf{Soluzione}
\\\\
Per riuscire a poter concatenare la lista x alla lista y, devo usare la regola di eliminazione sulla lista y. Per farlo, devo prima di tutto definire equazionalmente la definizione ricorsiva di append
\begin{center}
append(x, nil) $=$  cons(x, nil) $=$ x\\
append(x, cons(y, a)) $=$ cons(x, append(y, a))\\
\end{center}
Ora posso applicare la regola dell'eliminazione nel modo seguente:
\begin{itemize}
\item Premesse:
\begin{itemize}
\item M(z) type[$\Gamma$, z $\in$ List(A)] $\equiv$ List(A) type[x $\in$ List(A), y $\in$ List(A)] (\textit{z $\in$ List(A) risulta superfluo})
\item t $\in$ List(A)[$\Gamma$] $\equiv$ y $\in$ List(A)[x $\in$ List(A), y $\in$ List(A)]
\item c $\in$ M(nil)[$\Gamma$] $\equiv$ x $\in$ List(A)[x $\in$ List(A), y $\in$ List(A)]
\item e(x,w,y) $\in$ M(cons(x,w))[$\Gamma$, x $\in$ List(A),  w $\in$ A, y $\in$ M(x)] $\equiv$
cons(z,v) $\in$ List(A)[x $\in$ List(A) y $\in$ List(A), v $\in$ A, z $\in$ List(A)]
\end{itemize}
\noindent
\item Giudizio di conclusione:
\begin{itemize}
\item El$_{List}$(t,c,e) $\in$ M(t)[$\Gamma$] $\equiv$ El$_{List}$(y, x,(u,v,z).cons(z,v)) $\in$ List(A)[x $\in$ List(A), y $\in$ List(A)]
\end{itemize}
\end{itemize}
\vspace{0.5cm}
\begin{prooftree}
%\AxiomC{[ ] cont}
%\LeftLabel{s-checks}
%\UnaryInfC{A type[ ]}
%\LeftLabel{F-cost}
%\UnaryInfC{List(A) type [ ]}
%\LeftLabel{F-c}\RightLabel{(x $\in$ List(A)) $\notin$ [ ]}
%\UnaryInfC{x $\in$ List(A) cont}
%\AxiomC{[ ] cont}
%\LeftLabel{s-checks}
%\UnaryInfC{A type[ ]}
%\LeftLabel{ind-ty}
%\BinaryInfC{A type[x $\in$ List(A)]}
%\AxiomC{[ ] cont}
%\LeftLabel{s-checks}
%\UnaryInfC{A type[ ]}
%\LeftLabel{F-cost}
%\UnaryInfC{List(A) type [ ]}
%\LeftLabel{F-c}\RightLabel{(x $\in$ List(A)) $\notin$ [ ]}
%\UnaryInfC{x $\in$ List(A) cont}
%\AxiomC{[ ] cont}
%\LeftLabel{s-checks}
%\UnaryInfC{A type[ ]}
%\LeftLabel{ind-ty}
%\BinaryInfC{A type[x $\in$ List(A)]}
%\LeftLabel{F-cost}
%\UnaryInfC{List(A) type[x $\in$ List(A)]}
%\LeftLabel{F-c}\RightLabel{(y $\in$ List(A)) $\notin$ x $\in$ List(A)}
%\UnaryInfC{x $\in$ List(A), y $\in$ List(A) cont}
%\LeftLabel{ind-ty}
%\BinaryInfC{A type[x $\in$ List(A), y $\in$ List(A)]}
%\LeftLabel{F-cost}
\AxiomC{\textbf{1}}%\UnaryInfC{List(A) type[x $\in$ List(A), y $\in$ List(A)]}
%\AxiomC{[ ] cont}
%\LeftLabel{s-checks}
%\UnaryInfC{A type[ ]}
%\LeftLabel{F-cost}
%\UnaryInfC{List(A) type [ ]}
%\LeftLabel{F-c}\RightLabel{(x $\in$ List(A)) $\notin$ [ ]}
%\UnaryInfC{x $\in$ List(A) cont}
%\AxiomC{[ ] cont}
%\LeftLabel{s-checks}
%\UnaryInfC{A type[ ]}
%\LeftLabel{ind-ty}
%\BinaryInfC{A type[x $\in$ List(A)]}
%\LeftLabel{F-cost}
%\UnaryInfC{List(A) type[x $\in$ List(A)]}
%\LeftLabel{F-c}\RightLabel{(y $\in$ List(A)) $\notin$ x $\in$ List(A)}
%\UnaryInfC{x $\in$ List(A), y $\in$ List(A) cont}
%\LeftLabel{var}
\AxiomC{\textbf{2}}%\UnaryInfC{y $\in$ List(A)[x $\in$ List(A), y $\in$ List(A)]}
%\AxiomC{[ ] cont}
%\LeftLabel{s-checks}
%\UnaryInfC{A type[ ]}
%\LeftLabel{F-cost}
%\UnaryInfC{List(A) type [ ]}
%\LeftLabel{F-c}\RightLabel{(x $\in$ List(A)) $\notin$ [ ]}
%\UnaryInfC{x $\in$ List(A) cont}
%\AxiomC{[ ] cont}
%\LeftLabel{s-checks}
%\UnaryInfC{A type[ ]}
%\LeftLabel{ind-ty}
%\BinaryInfC{A type[x $\in$ List(A)]}
%\LeftLabel{F-cost}
%\UnaryInfC{List(A) type[x $\in$ List(A)]}
%\LeftLabel{F-c}\RightLabel{(y $\in$ List(A)) $\notin$ x $\in$ List(A)}
%\UnaryInfC{x $\in$ List(A), y $\in$ List(A) cont}
%\AxiomC{[ ] cont}
%\LeftLabel{s-checks}
%\UnaryInfC{A type[ ]}
%\LeftLabel{F-cost}
%\UnaryInfC{List(A) type [ ]}
%\LeftLabel{F-c}\RightLabel{(y $\in$ List(A)) $\notin$ [ ]}
%\UnaryInfC{y $\in$ List(A) cont}
%\AxiomC{[ ] cont}
%\LeftLabel{s-checks}
%\UnaryInfC{A type[ ]}
%\LeftLabel{ind-ty}
%\UnaryInfC{A type[y $\in$ List(A)]}
%\LeftLabel{F-cost}
%\UnaryInfC{List(A) type[y $\in$ List(A)]}
%\LeftLabel{F-c}\RightLabel{(x $\in$ List(A)) $\notin$ y $\in$ List(A)}
%\UnaryInfC{y $\in$ List(A), x $\in$ List(A) cont}
%\LeftLabel{var}
%\UnaryInfC{x $\in$ List(A)[y $\in$ List(A), x $\in$ List(A)]}
%\LeftLabel{ex-te}
\AxiomC{\textbf{3}}%\BinaryInfC{x $\in$ List(A)[x $\in$ List(A), y $\in$ List(A)]}
\AxiomC{\textbf{4}}%\UnaryInfC{cons(z,w) $\in$ List(A)[x $\in$ List(A), y $\in$ List(A), z $\in$ List(A), w$\in$ A]}
\LeftLabel{E-List}
\QuaternaryInfC{El$_{List}$(y, x,(u,v,z).cons(z,v)) $\in$ List(A)[x $\in$ List(A), y $\in$ List(A)]}
\end{prooftree}
\vspace{1cm}
\normalsize \textbf{1}
\small
\begin{prooftree}
\AxiomC{}
\UnaryInfC{A type[ ]}
\AxiomC{}
\UnaryInfC{A type[ ]}
\AxiomC{}
\UnaryInfC{A type[ ]}
\LeftLabel{F-cost}
\UnaryInfC{List(A) type[ ]}
\LeftLabel{F-c}\RightLabel{(x $\in$ List(A)) $\notin$ [ ]}
\UnaryInfC{x $\in$ List(A) cont}
\LeftLabel{ind-ty}
\BinaryInfC{A type[x $\in$ List(A)]}
\LeftLabel{F-cost}
\UnaryInfC{List(A) type[x $\in$ List(A)]}
\LeftLabel{F-c}\RightLabel{\begin{tabular}[c]{cc}(y $\in$ List(A)) $\notin$ \\ x $\in$ List(A)\end{tabular}}
\UnaryInfC{x $\in$ List(A), y $\in$ List(A) cont}
\LeftLabel{ind-ty}
\BinaryInfC{A type[x $\in$ List(A), y $\in$ List(A)]}
\LeftLabel{F-cost}
\UnaryInfC{List(A) type[x $\in$ List(A), y $\in$ List(A)]}
\end{prooftree}
\vspace{1.5cm}
\normalsize \textbf{2}
\small
\begin{prooftree}
\AxiomC{}
\UnaryInfC{A type[ ]}
\AxiomC{}
\UnaryInfC{A type[ ]}
\LeftLabel{F-cost}
\UnaryInfC{List(A) type[ ]}
\LeftLabel{F-c}\RightLabel{(x $\in$ List(A)) $\notin$ [ ]}
\UnaryInfC{x $\in$ List(A) cont}
\LeftLabel{ind-ty}
\BinaryInfC{A type[x $\in$ List(A)]}
\LeftLabel{F-cost}
\UnaryInfC{List(A) type[x $\in$ List(A)]}
\LeftLabel{F-c}\RightLabel{(y $\in$ List(A)) $\notin$ x $\in$ List(A)}
\UnaryInfC{x $\in$ List(A), y $\in$ List(A) cont}
\LeftLabel{var}
\UnaryInfC{y $\in$ List(A)[x $\in$ List(A), y $\in$ List(A)]}
\end{prooftree}
\vspace{1cm}
\normalsize \textbf{3}\\\\
\noindent
\textit{La regola di scambio di contesto (ex-ty) pu\`o venire anche omessa.}
\scriptsize
\begin{adjustwidth}{-4.5em}{}
\begin{prooftree}
\AxiomC{}
\UnaryInfC{A type[ ]}
\AxiomC{}
\UnaryInfC{A type[ ]}
\LeftLabel{F-cost}
\UnaryInfC{List(A) type[ ]}
\LeftLabel{F-c}\RightLabel{(y $\in$ List(A)) $\notin$ [ ]}
\UnaryInfC{y $\in$ List(A) cont}
\LeftLabel{ind-ty}
\BinaryInfC{A type[y $\in$ List(A)]}
\LeftLabel{F-cost}
\UnaryInfC{List(A) type[y $\in$ List(A)]}
\LeftLabel{F-c}\RightLabel{\begin{tabular}[c]{cc}(x $\in$ List(A)) $\notin$ \\ y $\in$ List(A)\end{tabular}}
\UnaryInfC{y $\in$ List(A), x $\in$ List(A) cont}
\LeftLabel{var}
\UnaryInfC{x $\in$ List(A)[y $\in$ List(A), x $\in$ List(A)]}

\AxiomC{}
\UnaryInfC{A type[ ]}
\AxiomC{}
\UnaryInfC{A type[ ]}
\LeftLabel{F-cost}
\UnaryInfC{List(A) type[ ]}
\LeftLabel{F-c}\RightLabel{(x $\in$ List(A)) $\notin$ [ ]}
\UnaryInfC{x $\in$ List(A) cont}
\LeftLabel{ind-ty}
\BinaryInfC{A type[x $\in$ List(A)]}
\LeftLabel{F-cost}
\UnaryInfC{List(A)type[x $\in$ List(A)]}
\LeftLabel{F-c}\RightLabel{\begin{tabular}[c]{cc}(y $\in$ List(A)) $\notin$ \\ x $\in$ List(A)\end{tabular}}
\UnaryInfC{x $\in$ List(A), y $\in$ List(A) cont}
\LeftLabel{ex-ty}
\BinaryInfC{x $\in$ List(A)[x $\in$ List(A), y $\in$ List(A)]}
\end{prooftree}
\end{adjustwidth}
\vspace{1cm}
\normalsize \textbf{4}
\small
\begin{adjustwidth}{-7em}{}
\begin{prooftree}
\AxiomC{\textbf{4$^\backprime$}}
\UnaryInfC{x $\in$ List(A), y $\in$ List(A), v $\in$ A, z $\in$ List(A) cont}
\LeftLabel{var}
\UnaryInfC{z $\in$ List(A)[x $\in$ List(A), y $\in$ List(A), v $\in$ A, z $\in$ List(A)]}
\AxiomC{\textbf{4$^\backprime$}}
\UnaryInfC{x $\in$ List(A) y $\in$ List(A), v$\in$ A, z $\in$ List(A) cont}
\LeftLabel{var}
\UnaryInfC{v $\in$ A[x $\in$ List(A) y $\in$ List(A), v $\in$ A, z $\in$ List(A)]}
\LeftLabel{I$_2$-List}
\BinaryInfC{cons(z,v) $\in$ List(A)[x $\in$ List(A), y $\in$ List(A), v $\in$ A, z $\in$ List(A)]}
\end{prooftree}
\end{adjustwidth}
\vspace{1cm}
\normalsize \textbf{4$^\backprime$}
\small
\begin{adjustwidth}{-4em}{}
\begin{prooftree}
\AxiomC{A type[ ]}
\LeftLabel{F-cost}
\UnaryInfC{List(A) type[ ]}
\AxiomC{}
\UnaryInfC{A type[ ]}
\AxiomC{}
\UnaryInfC{A type[ ]}
\AxiomC{}
\UnaryInfC{A type[ ]}
\LeftLabel{F-cost}
\UnaryInfC{List(A) type[ ]}
\LeftLabel{F-c}\RightLabel{(x $\in$ List(A)) $\notin$ [ ]}
\UnaryInfC{x $\in$ List(A) cont}
\LeftLabel{ind-ty}
\BinaryInfC{A type[x $\in$ List(A)]}
\LeftLabel{F-cost}
\UnaryInfC{List(A) type[x $\in$ List(A)]}
\LeftLabel{F-c}\RightLabel{(y $\in$ List(A)) $\notin$ x $\in$ List(A)}
\UnaryInfC{x $\in$ List(A), y $\in$ List(A) cont}
\LeftLabel{ind-ty}
\BinaryInfC{A type[x $\in$ List(A), y $\in$ List(A)]}
\LeftLabel{F-c}\RightLabel{\begin{tabular}[c]{cc}(v $\in$ A) $\notin$ (x $\in$ List(A), \\ y $\in$ List(A)))\end{tabular}}
\UnaryInfC{x $\in$ List(A), y $\in$ List(A), v $\in$ A cont}
\LeftLabel{ind-ty}
\BinaryInfC{List(A) type[x $\in$ List(A), y $\in$ List(A), v $\in$ A]}
\LeftLabel{F-c}\RightLabel{(z $\in$ List(A)) $\notin$  (x $\in$ List(A), y $\in$ List(A), v $\in$ A)}
\UnaryInfC{x $\in$ List(A) y $\in$ List(A), v $\in$ A,  z $\in$ List(A) cont}
\end{prooftree}
\end{adjustwidth}
\vspace{0.5cm}
\noindent
\normalsize \textit{Dimostrazione di correttezza di El$_{List}$(y, x,(u,v,z).cons(z,v)) $\in$ List(A)[x $\in$ List(A), y $\in$ List(A)]}
\begin{itemize}
\item El$_{List}$(nil, x,(u,v,z).cons(z,v)) $\rightarrow_1$ x per \textit{$\beta_{1List}$-red}
\item  El$_{List}$(cons(y,a), x,(u,v,z).cons(z,v)) $\rightarrow_1$ cons(El$_{List}$(y, x, (u,v,z).cons(z,v))) per \textit{$\beta_{2List}$-red} $\Rightarrow$ per y $=$ nil $\equiv$ cons(El$_{List}$(nil, x, (u,v,z).cons(z,v))) $\rightarrow_1$ cons(x, nil) $\in$ List(A) $=$ x (dal punto precedente).
\end{itemize}