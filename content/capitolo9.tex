\chapter{Tipo vuoto}
\label{cap: N0}
%lezione 20 addendum
Il tipo vuoto (N$_0$, \textit{empty-set}) \`e un tipo induttivo generato dalle sue regole di introduzione (che non ci sono). \\ Il tipo N$_0$ \`e definito dalle seguenti regole.

\section{Regole di Formazione}
\label{sec: formazione-N0}
\begin{prooftree}
\AxiomC{$\Gamma$ cont}
\LeftLabel{F-N$_0$)}
\UnaryInfC{N$_0$ type [$\Gamma$]}
\end{prooftree}

\section{Regole di Introduzione}
\label{sec: introduzione-N0}
N$_0$[$\Gamma$] $=$ $\varnothing$ $\Rightarrow$ non ci sono valori canonici\\\\
\noindent
\textit{Non ci sono le regole di Introduzione, in quanto N$_0$ si vuole che sia un tipo vuoto, e di conseguenza non serve introdurre elementi}

\section{Regole di Eliminazione}
\label{sec: eliminazione-N0}
\begin{prooftree}
\AxiomC{t $\in$ N$_0$[$\Gamma$]}
\AxiomC{M(z) type[$\Gamma$,z $\in$ N$_0$]}
\LeftLabel{E-N$_0$)}
\BinaryInfC{El$_{N0}$(t) $\in$ M(t)[$\Gamma$]}
\end{prooftree}

\section{Regole di Conservazione}
\label{sec: conservazione-funzione}
\textit{Non esiste alcuna regola di Conversione, a causa della mancanza di ipotesi dell'eliminatore sui termini M(0).}

\section{Regole di Uguaglianza}
\label{sec: uguaglianza-funzione}
\begin{prooftree}
\AxiomC{t$_1$ $=$ t$_2$ $\in$ N$_0$[$\Gamma$]}
\AxiomC{M(z) type[$\Gamma$,z $\in$ N$_0$]}
\LeftLabel{eq-E-N$_0$)}
\BinaryInfC{El$_{N0}$(t$_1$) $=$ El$_{N0}$(t$_2$) $\in$ M(t$_1$)[$\Gamma$]}
\end{prooftree}
\noindent
\textit{Non serve nessun'altra ipotesi non essendoci alcun elemento canonico.\\ Inoltre questo comporta la non necessit\`a delle altre regole di uguaglianza}

\section{Eliminatore dipendente}
\label{sec:eliminatore dipendente-N0}
\begin{prooftree}
\AxiomC{M(z) type[$\Gamma$,z $\in$ N$_0$]}
\LeftLabel{E-N$_{0dip}$)}
\UnaryInfC{El$_{N0}$(z) $\in$ M(z)[$\Gamma$,z $\in$ N$_0$]}
\end{prooftree}


\section{Semantica operazionale del tipo funzione}
\label{sec: semantica-operazionale-funzione}
La relazione $\rightarrow_1$ viene definita all'interno dei termini con l'uso delle seguenti regole di riduzione:
\begin{itemize}
\item $\beta_{\rightarrow}$-red) NON ESISTE 
\item Per far interagire N$_0$ con altri tipi (altrimenti N$_0$ da solo non fa nulla)
\AxiomC{t$_1$ $\rightarrow_1$ t$_2$}
\LeftLabel{N$_0$-red)}
\UnaryInfC{El$_{N0}$(t$_1$) $\rightarrow_1$ El$_{N0}$(t$_2$)}
\DisplayProof
\end{itemize}
\noindent
La regola \textit{(N$_0$-red)} \`e associata all'uguaglianza dell'eliminazione.

\section{Osservazioni dal punto di vista logico}
\label{sec:osservazioni-dal-punto-di-vista-logico-N0}
Il tipo vuoto serve in logica per interpretare il falso
\begin{center}$\bot$ ${\overset{\mathit{def}}{\equiv}}$ N$_0$\end{center}
\noindent
$\bot$ \`e vero sse $\exists$ $\in$ N$_0$
\begin{prooftree}
\AxiomC{t $\in$ N$_0$[$\Gamma$]}
\AxiomC{$\bot$ \`e vero[$\Gamma$]}
\AxiomC{$\varphi$ prop[$\Gamma$]}
\LeftLabel{ind-ty}
\BinaryInfC{$\varphi$ prop[$\Gamma$,z $\in$ N$_0$]}
\LeftLabel{E$_{N0}$}
\BinaryInfC{El$_{N0}$(t) $\in$ $\varphi$[$\Gamma$]}
\end{prooftree}
Quella appena definita rappresenta la regola
\AxiomC{$\bot$ vero[$\Gamma$]}
\UnaryInfC{$\varphi$ vero[$\Gamma$]}
\DisplayProof
\\\\
\noindent Inoltre, con la regola di \textit{Eliminazione dipendente}, definita in \S \ref{sec:eliminatore dipendente-N0}, posso dire che, per ogni $\varphi$ prop[$\Gamma$]\\
$\Rightarrow$M(z) type[$\Gamma$,z $\in$ N$_0$] $\equiv$ $\varphi$(z) \`e prop\\
$\Rightarrow$El$_{N0}$(z) $\in$ M(z)[$\Gamma$,z $\in$ N$_0$] $\equiv$ $\varphi$ \`e vero[$\Gamma$,$\Delta$,N$_0$ vero] $\equiv$ $\bot$ $\vdash$ $\varphi$ che \`e l'\textbf{assioma del falso} in logica \textit{($\bot$-ax)}\\\\
\noindent
In conclusione, dunque, la regola di eliminazione di N$_0$ rappresenta \textit{$\bot$-ax} in calcolo dei sequenti/regola del falso \textit{quodlibet} in deduzione naturale.


