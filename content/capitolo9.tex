\chapter{Tipo vuoto}
\label{cap: N0}
%lezione 20 addendum
Il tipo vuoto (N$_0$) \`e un tipo induttivo generato dalle sue regole di introduzione (che non ci sono). Queste regole sono le seguenti.

\section{Regole di Formazione}
\label{sec: formazione-N0}
\begin{prooftree}
\AxiomC{$\Gamma$ cont}
\LeftLabel{N$_0$-F)}
\UnaryInfC{N$_0$ type [$\Gamma$]}
\end{prooftree}

\section{Regole di Introduzione}
\label{sec: introduzione-N0}
N$_0$ $=$ $\varnothing$[$\Gamma$] $\Rightarrow$ non ci sono valori canonici
\textit{Non ci sono le regole di Introduzione, in quanto N$_0$ si vuole che sia un tipo vuoto}

\section{Regole di Eliminazione}
\label{sec: eliminazione-N0}
\begin{prooftree}
\AxiomC{t $\in$ N$_0$[$\Gamma$]}
\AxiomC{M(z) type[$\Gamma$,z $\in$ N$_0$]}
\LeftLabel{E-N$_0$)}
\BinaryInfC{El$_{N0}$(t) $\in$ M(t)[$\Gamma$]}
\end{prooftree}

\section{Regole di Conservazione}
\label{sec: conservazione-funzione}
\textit{Non esiste alcuna regola di Conversione. In quanto non vi \`e alcuna ipotesi dell'eliminatore sui temini M(0), essendo il tipo vuoto privo di elementi canonici.}

\section{Regole di Uguaglianza}
\label{sec: uguaglianza-funzione}
\begin{prooftree}
\AxiomC{t$_1$ $=$ t$_2$ $\in$ N$_0$[$\Gamma$]}
\AxiomC{M(z) type[$\Gamma$,z $\in$ N$_0$]}
\LeftLabel{eq-E-N$_0$)}
\BinaryInfC{El$_{N0}$(t$_1$) $=$ El$_{N0}$(t$_2$) $\in$ M(t$_1$)[$\Gamma$]}
\end{prooftree}
\noindent
\textit{Non servono nessun'altra ipotesi non essendoci alcun elemento canonico.\\ Inoltre questo comporta la non necessit\`a delle altre regole di uguaglianza}

\section{Eliminatore dipendente}
\label{sec:eliminatore dipendente-N0}
\begin{prooftree}
\AxiomC{M(z) type[$\Gamma$,z $\in$ N$_0$]}
\LeftLabel{E-N$_{0dip}$)}
\UnaryInfC{El$_{N0}$(z) $\in$ M(z)[$\Gamma$,z $\in$ N$_0$]}
\end{prooftree}


\section{Semantica operazionale del tipo funzione}
\label{sec: semantica-operazionale-funzione}
La relazione $\rightarrow_1$ viene definita all'interno dei termini con l'uso delle seguenti regole di riduzione:
\begin{itemize}
\item $\beta_{\rightarrow}$-red) NON ESISTE 
\item Per far interagire N$_0$ con altri tipi
\AxiomC{t$_1$ $\rightarrow_1$ t$_2$}
\LeftLabel{N$_0$-red)}
\UnaryInfC{El$_{N0}$(t$_1$) $\rightarrow_1$ El$_{N0}$(t$_2$)}
\DisplayProof
\end{itemize}

\section{Osservazioni dal punto di vista logico}
\label{sec:osservazioni-dal-punto-di-vista-logico-N0}
L'applicazione in logica di N$_0$ viene definita 
\begin{center}$\bot$ ${\overset{\mathit{def}}{\equiv}}$ N$_0$\end{center}
\noindent
$\exists$t $\in$ N$_0$ $\Leftrightarrow$
\AxiomC{$\bot$ \`e vero[$\Gamma$]}
\AxiomC{$\varphi$(z) prop[$\Gamma$]}
\LeftLabel{)}
\BinaryInfC{El$_{N0}$(t$_1$) $\rightarrow_1$ El$_{N0}$(t$_2$)}
\DisplayProof
