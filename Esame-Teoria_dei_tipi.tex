\documentclass[10pt,a4paper, italian]{book}
\usepackage[italian]{babel}
\usepackage[T1]{fontenc} % riconosce gli acenti
\usepackage[utf8]{inputenc} % riconosce gli acenti
\usepackage{amssymb}
\usepackage[table]{xcolor}
\usepackage{tabularx}
\usepackage{float}
\usepackage{hyperref}
\hypersetup{
	colorlinks=true,
	linkcolor=blue,
	urlcolor=linkcolor
}	
\usepackage{amsmath}
\usepackage{amsfonts}
\usepackage{bussproofs} % per alberi di deduzione naturale
\usepackage{graphicx} % per le immagini
\usepackage[strict]{changepage}


\begin{document}
\setcounter{section}{0}
\setcounter{page}{1}

%-----------------------------------------------------------------------------------%
% FIRST PAGE
\begin{figure}[H]
\centering
  \includegraphics[width=0.6\linewidth]{./img/logo_dip.png}
   \label{fig: logo-unip-dipartimento-matematica}
\end{figure}
\begin{center}\Large{\textbf{Universit\`a degli Studi di Padova\\Dipartimento di Matematica "Tullio Levi-Civita\\Corso di Laurea Magistrale in Informatica}}\end{center}
\begin{center}\Large{Esame di Teoria dei Tipi}\end{center}
\begin{flushright} 
Teoria dei Tipi\\
\textit{Elaborato scritto - Settembre 2020}\\
Eleonora Signor, 1237581
\end{flushright}
\pagebreak

%-----------------------------------------------------------------------------------%
% INDEX OF PAGES
\tableofcontents \pagebreak
%-----------------------------------------------------------------------------------%

\chapter{Introduzione}
\label{cap:introduzione}
\section{La triplice faccia della teoria dei tipi}
\label{sec:la-triplice-faccia}
La teoria dei tipi offre una base teorica a fondamento dello sviluppo di:
\begin{itemize}
\item \textbf{Matematica}: nella teoria degli insiemi;
\item \textbf{Logica}: come fondamento dei connettivi logici e dei quantificatori, con trattazione mediate tecniche di \textit{proof-theory} per dimostrarne la non falsit\`a o non contradditoriet\`a;
\item \textbf{Informatica}: per la correttezza dei programmi, da una semantica operazionale a un certo tipo di operazioni.\\ 
Con riferimento alla teoria degli insiemi, visto come linguaggio di programmazione funzionale, \`e possibile specificare con formule l'obiettivo di un programma e dimostrarne la correttezza attraverso la specifica.
\end{itemize}
\noindent
La teoria dei tipi nasce per garantire la \textit{Certified Proof Correctness}. Ovvero la correttezza dei programmi, volta a costruire gli assistenti automatici per le formalizzazioni.

\section{Come nasce la teoria dei tipi}
\label{sec:come-nasce}
Gli errori di programmazione sono stati preponderanti alla nascita di metodi automatici, che assicurassero la correttezza del \textit{software}. Alcuni di questi, degni di nota, sono stati:
\begin{itemize}
\item incidente nel lancio dell'Apollo 11;
\item tragedie sanitarie: incidenti avvenuti tra il 1985-1987, in cui dei pazienti ricevettero una massiccia \textit{overdose} di radiazioni e per la quali alcuni morirono;
\item errori di vita civile: riserva di solo due cifre per il campo et\`a all'interno dei \textit{database}. Ecco che una signora danese ricevette per il suo 107-esimo compleanno, una mail, dalle autorit\`a della scuola locale, per iscriversi alla prima elementare.
\end{itemize}
\noindent 
Per la matematica la correttezza delle dimostrazioni \`e irrilevante solo quando la soluzione \`e certa (come accade con il cubo di \textit{Rubik}, dove so che la soluzione \`e corretta quando ognuno dei lati \`e uniformemente colorato); e in generale questo \`e difficile che accada.\\
Un'esempio di problema, dove la soluzione non \`e certa, \`e il Teorema dei Quattro Colori, risolto da un \textit{computer} e la cui prova di correttezza della dimostrazione fu data dal \textit{proof- assistant} Coq. Quest'ultimo basato sulla teoria dei tipi e intellegibile dall'essere umano.\\\\
Una citazione importante va al matematico Russo V.V. \textit{Voevodsky}, vincitore della medaglia \textit{ Fields}. Esso si batt\`e per la creazione di un \textit{proof assistant}, per rendere le dimostrazioni da informali, per problemi complessi, a completamente formalizzate, con l'impiego della teoria dei tipi. I suoi studi trovano principale applicazione in campo algebrico e geometrico; ma i concetti emersi assunsero delle connotazioni pi\`u ampie. \textit{Voevodsky}, difatti, si rese conto che formalizzare equivale a programmare. Ci\`o significa che la teoria dei tipi permette di vedere una dimostrazione come un programma.
\\\\
Esiste la certezza assoluta per una certa teoria, esclusivamente, quando ha un numero di assiomi, accettati per fede, molto limitato. In quanto assiomatizzabile da un calcolatore.
\\\\
In conclusione formalizzare in una teoria dei tipi (come quella degli insiemi) equivale a programmare un programma.

\section{Il Paradosso di Russell}
\label{sec:paradosso-di-russell}
La base della teoria dei tipi, compresa quella di \textit{Martin-L$\ddot{o}$f}, si deve a B. \textit{Russell}. \\
Siamo nel 1907 quando nasce la teoria dei tipi, sviluppata nei \textit{Principia Mathematica} da B. \textit{Russel} assieme ad A.N. \textit{Whitehead}. Tale teoria, intesa come logica e non informatica, nasce come soluzione alternativa alla teoria degli insiemi, di allora, con lo scopo di fondare la matematica su un sistema formale accettabile e non contraddittorio.\\
Di seguito espongo un sistema contraddittorio della teoria degli insiemi.\\
\\
\textbf{Linguaggio \textit{L} di una teoria degli insiemi \textit{F}}\mbox{}
\begin{itemize}
\item \textit{L} linguaggio del primo ordine ($=$, $\&$, $\rightarrow$, $\lor$, $\forall{x}$, $\exists{x}$), con l'aggiunta del predicato $\in$ "appartiene"
\item variabili VAR $\ni$ \{x, y, z, w,\dots \}
\end{itemize}
\noindent
dove x, y, z sono da intendersi come insiemi e x $\in$ y $=$ "x appartiene a y".\\\\
All'interno di \textit{L} c'\`e una teoria degli insiemi. Tra cui prende posto l'\textbf{assioma di comprensione di \textit{Frege}}, definito nel modo seguente:\\
Per ogni formula $\phi$(x) vale che $\exists$z $\forall$y (y$\in$z $\Leftrightarrow$ $\phi$(y)) [$\equiv$ $\exists$z z $=$ \{x $\big|$ $\phi$(x)\}]\\\\
\textbf{Teorema (o Paradosso) di \textit{Russell}}: la teoria \textit{F} \`e contraddittoria.\\

\noindent \textbf{Dimostrazione}:\\
$\phi$(x) $=$ x$\notin$x ($\equiv$ $\neg$ (x $\in$ x))\\
Per l'assioma di comprensione	$\exists$z z $=$ \{x $|$ x$\notin$x\} ($\exists$z $\forall$y (y$\in$z $\Leftrightarrow$ y$\notin$y)).\\
Ponendo y$=$z ottengo che z$\in$z $\Leftrightarrow$ z$\notin$z, che risulta una \textbf{contraddizione}.\\\\
L'assioma di comprensione \`e contraddittorio perch\`e permette di formare insiemi che non appartengono a se stessi.\\\\
\textit{Come correggere la contraddizione?}\\
La soluzione accettabile \`e porre agli insiemi una \textbf{gerarchia di tipi}. In questo modo l'assioma di comprensione diventa\\
\begin{center}$\exists$z $\forall$y (y$\in$a $\to$ (y$\in$z $\Leftrightarrow$ $\phi$(y)) $\equiv$ z $=$ \{x$\in$a $\big|$ $\phi$(x)\}\end{center}
\noindent
In questo modo non posso pi\`u creare il Paradosso di \textit{Russell}.\\\\
Al momento questa teoria dei tipi non \`e utilizzata. Una sua evoluzione diretta \`e 
la teoria dei tipi di \textit{Martin-L$\ddot{o}$f}.\\\\
L'idea di \textit{Russell} fu dunque quella di costruire insiemi partendo da una gerarchia.

\section{Idee principali nelle teorie di tipo moderne}
\label{sec:idee-teorie-moderne}
Le teorie di tipo moderne (chiamate $\lambda$-calcolo tipato) nascono, nel corso degli anni '30, dalla combinazione della teoria di tipo di \textit{Russell} con il $\lambda$-calcolo di \textit{Church}.

\subsection{Richiamo della teoria del $\lambda$-calcolo di \textit{Church}}
\label{subsec:lambda-calcolo}
Ha origine dalla logica, \`e un linguaggio in grado di trattare le funzioni e rivolto alla loro formalizzazione. Consiste in un linguaggio formale, le cui componenti principali sono programmi chiamati termini (pensati come funzioni).\\ La grammatica \`e la seguente:
\begin{center} t $:=$ x $|$ b$_1$(b$_2$) $|$ $\lambda$x.t \end{center}
Esempio di applicazione: tg(x) $\equiv$ $\lambda$x.tg(x)

\paragraph{Regole di computazione di base}\mbox{}\\\\ 
\[ (\lambda x.t)(b) \rightarrow t[\frac{x}{b}] \qquad
\frac{b_1 \rightarrow b_2	\qquad a_1 \rightarrow a_2 }{b_1(a_1) \rightarrow b_2(a_2)} \qquad 
\frac{b_1 \rightarrow b_2}{\lambda x.b_1 \rightarrow \lambda x.b_2} \]
\noindent
Si dice che un programma si riduca a un altro, cio\`e converge, solo se c'\`e una sequenza di riduzioni (applicazione di regole e/o assiomi), che connettono il primo programma con l'ultimo. Si parla, in questo modo, di \textbf{chiusura transitiva e simmetrica}, che si conclude quando il programma non \`e pi\`u riducibile. Quanto appena descritto pu\`o venire definito in\\\\ $t \rightarrow t'$ sse esiste un numero finito di passi per cui $t$ si riduce a $t'$, ovvero esiste $b_1 \dots b_m$ t.c. $t \rightarrow b_1 \rightarrow b_2 \dots \rightarrow b_m \rightarrow t'$.\\\\
Il $\lambda$ calcolo permette di codificare qualsiasi programma scritto in qualunque linguaggio (imperativo, dichiarativo, Java, C++, BASIC, \dots). Tuttavia tale linguaggio non codifica solo programmi che terminano, ma anche programmi che non lo fanno.
Un esempio di applicazione, per quest'ultima categoria, \`e un programma con computazione infinita: $\lambda x.x(x)$.\\
$\lambda x.x(x)$ lo applichiamo a se stesso. Perci\'o diventa $\Lambda \equiv (\lambda x.x(x))(\lambda x.x(x))$
che seguendo la computazione si riduce a \[x(x)[\frac{x}{\lambda x.x(x)}] \equiv (\lambda x.x(x))(\lambda x.x(x))\]\\
Dunque esiste una catena di (t$_i$)$_{i\in\mathbb{N}}$ di termini t$_i \rightarrow$ t$_{i+1}$. Ci\`o significa che
$\Lambda$ non termina in qualunque linguaggio sia interpretato.\\\\
$\Lambda$ risulta un buon metodo per rappresentare le funzioni, ma non \`e completo, rispetto all'intuizione matematica di funzione. \`E necessario, per questo, tipare le variabili; ovvero $\lambda$x.x$\in$A$\rightarrow$B(x$\in$A).\\\\
Il $\lambda$-calcolo tipato, nato dal $\lambda$-calcolo "puro", \`e anch'esso un linguaggio di programmazione. Essendo tipato pu\`o essere trattato come una teoria degli insiemi.

\section{Che cosa \`e un tipo?}
\label{sec:cosa e un tipo}
Per rispondere a questa domanda \`e necessario fornire la semantica intuitiva di tipo. Per farlo \`e utile pensare alla teoria dei tipi come paradigma di fondazione sia logico che matematico che informatico.

\begin{table}[h]
\centering
\begin{tabularx}{\textwidth}{XXXX}
\hline 
\rowcolor{amethyst}
{\color[HTML]{FFFFFF}\textbf{Sintassi in teoria dei tipi moderna}} & {\color[HTML]{FFFFFF}\textbf{Sintassi in teoria degli insiemi}} & {\color[HTML]{FFFFFF} \textbf{Sintassi in un linguaggio logico/per una logica (anche predicativo)}} & {\color[HTML]{FFFFFF}\textbf{Sintassi in un linguaggio di programmazione}} \\
\hline\hline 
A \textit{type} & A \textit{set} & A \textit{prop} & A \textit{data type} \\ 
\hline 
a$\in$A & a$\in$A & a$\in$A & a$\in$A \\ 
\hline 
\end{tabularx}
\caption{\label{tab:sintassi-paradigmi-funzionali}Sintassi per i diversi paradigmi funzionali.} 
\end{table}
\noindent
Per la sintassi:
\begin{itemize} 
\item nella \textbf{teoria dei tipi moderna} \textit{a} rappresenta un termine e \textit{A} un tipo;
\item nella sintassi in una \textbf{teoria degli insiemi} \textit{a} \`e un elemento e \textit{A} un insieme. Coincidendo con la corrispondenza originale in mente da \textit{Russell}.
\item nella sintassi in un \textbf{linguaggio logico} \textit{a} rappresenta una dimostrazione di \textit{A}, e \textit{A} viene inteso come insieme o tipo delle sue dimostrazioni. Perci\`o \textit{a} rappresenta un \textit{proof-term} affermante come la proposizione di A sia vera.
\item nella teoria in una sintassi di un \textbf{linguaggio di programmazione} \textit{a} rappresenta un programma e \textit{A} una specifica.
\end{itemize}
\noindent
Dunque quando parliamo di tipo ci riferiamo a un insieme, una proposizione o \textit{data type}, a seconda dell'applicazione di tipo che si ha in mente.\\\\
Dal punto di vista logico non si hanno solo proposizioni, ma anche predicati. Parlare solo di tipo non risulta quindi sufficiente. Per questo se si vuole rappresentare non una proposizione, ma un predicato A(x) si usa la sintassi \textbf{A(x) prop[x$\in$D]}.\\
Dalla logica si sa che i predicati $\phi$(x) hanno x senza un dominio specifico, perch\`e la sintassi non determina che cosa \`e in x. Al seguito di tutto questo i predicati hanno una variabile che deve essere tipata come \textbf{$\phi$(x) prop[x$\in$D]}.\\
Dunque (definizione di predicato)
\begin{center}\textbf{$\exists$ z $\quad$ z=\{x$\in$ a$|$ $\phi$(x)\} $\qquad$ $\equiv$ $\qquad$ $\phi$(x)prop[x$\in$ a]}\end{center}
\noindent
Quanto appena definito da origine al concetto di \textbf{tipo dipendente}, nel quale vengono tipate tutte le variabili che appartengono a una \textbf{famiglia di tipo}.\\\\
Le famiglie di tipo sono indispensabili per rappresentare il concetto di predicato. Di seguito ho riassunto in forma tabellare le diverse famiglie.\\

\begin{table}[h]
\centering
\begin{tabularx}{\textwidth}{XXXp{3.1cm}}
\hline 
\rowcolor{amethyst}
{\color[HTML]{FFFFFF}\textbf{di tipo}} & {\color[HTML]{FFFFFF}\textbf{negli insiemi}} & {\color[HTML]{FFFFFF} \textbf{in logica}} & {\color[HTML]{FFFFFF}\textbf{dati dipendenti}} \\
\hline\hline 
A(x) prop[x$\in$D] & A(x) set[x$\in$D] & A(x) prop[x$\in$D] & A(x) datatype[x$\in$D]  \\  
\hline 
\end{tabularx}
\caption{\label{tab:famiglia-di-tipi}Famiglia di tipi.} 
\end{table}
\noindent
Il concetto di tipo dipendente \`e stato introdotto per la prima volta da \textit{Martin-L$\ddot{o}$f}. \textit{Russell} si era limitato a definire esclusivamente il concetto di funzione proposizionale dipendente da un tipo.
\newpage
\section{Esempi di tipi}
\label{sec:esempi-di-tipi}
\begin{table}[h]
\centering
\begin{tabularx}{\textwidth}{XXXX}
\hline 
\rowcolor{amethyst}
{\color[HTML]{FFFFFF}\textbf{A type}} & {\color[HTML]{FFFFFF}\textbf{A set}} & {\color[HTML]{FFFFFF} \textbf{A prop}} & {\color[HTML]{FFFFFF}\textbf{A data type}} \\
\hline\hline 
N$_1$ singoletto & l'insieme singoletto & \textit{tt} costante vero & tipo Unit  \\
\hline 
N$_0$ vuoto & l'insieme vuoto & \textit{$\bot$} costante falso & vuoto come \textit{datatype} \\
\hline
B$\times$C (tipo prodotto) & l'insieme prodotto cartesiano dell'insieme B con l'insieme C & B\&C congiunzione della proposizione B e della proposizione C & tipo prodotto cartesiano (come in \textit{set theory})\\ 
\hline 
B+C (tipo somma binaria) & l'insieme unione disgiunta dell'insieme B con l'insieme C & B$\lor$C disgiunta della  proposizione B e della proposizione C & tipo unione disgiunta con codifica \\
\hline 
B$\rightarrow$C & l'insieme delle funzioni dall'insieme B verso l'insieme C: \small{A$\rightarrow$B $\equiv$ \{f $|$ f: B$\rightarrow$C\}} & B$\rightarrow$C, implicazione della proposizione B e della proposizione C & insieme delle funzioni dal \textit{datatype} B al \textit{datatype} C \\
\hline
\end{tabularx}
\caption{\label{tab:famiglia-di-tipi}Esempi di tipi.} 
\end{table}
\noindent

\subsection{I tipi dipendenti}
\label{subsec:i-tipi-dipendenti}

\begin{table}[h]
\centering
\begin{tabular}{c}
\hline 
\rowcolor{amethyst}
{\color[HTML]{FFFFFF}\textbf{A(x)type[x$\in$B]}}\\
\hline\hline
\begin{tabular}[c]{cc}tipo prodotto dipendente\\ 
$\displaystyle\prod\limits_{x\in B} C(x)$\end{tabular}\\
\hline
\begin{tabular}[c]{cc}tipo somma dipendente disgiunta indiciata \\ $\displaystyle\sum\limits_{x\in B} C(x)$ \end{tabular}\\
\hline
\end{tabular}
\end{table}
\noindent

\begin{table}[H]
\centering
\begin{tabularx}{\textwidth}{p{3.8cm}XX}
\hline 
\rowcolor{amethyst}
{\color[HTML]{FFFFFF}\textbf{A(x)set[x$\in$B]}} & {\color[HTML]{FFFFFF} \textbf{A(x)prop[x$\in$B]}} & {\color[HTML]{FFFFFF}\textbf{A(x)datatype[x$\in$B]}}\\
\hline\hline
\centering \scriptsize{\{$\displaystyle f: B \rightarrow \displaystyle\coprod\limits_{x \in B} C(x)$\}} \scriptsize{$\displaystyle\coprod\limits_{x \in B} C(x) =$ \{$b,c | b\in B \quad c\in C(b)$\}}& \centering \small{$\displaystyle \forall {x \in B} \quad C(x)$} & tipo prodotto indiciato come in \textit{set theory} (non esiste un \textit{datatype} specifico)\\
\hline
\centering \scriptsize{$\displaystyle\bigcup\limits_{x \in B}^. C(x)$} \qquad \qquad \qquad \scriptsize{$\displaystyle\coprod\limits_{x \in B} C(x) =$ \{$b,c | b\in B \quad c\in C(b)$\}} & \centering \small{$\displaystyle \exists{x \in B} \quad C(x)$} & non \`e primitivo, deriva sempre dalla \textit{set theory}\\
\end{tabularx}
\caption{\label{tab:tipi-dipendenti}Tipi dipendenti.} 
\end{table}
\noindent
Lo \textit{slogan} principale della teoria dei tipi \`e quello di tipare le variabili in un linguaggio formale set teorico/computazionale.\\\\
Esiste anche il \textbf{Tipo uguaglianza}:
\begin{itemize}
\item intensionale: Id(B,c,d);
\item estensionale: Eq(B,c,d).
\end{itemize}
\noindent
Introdotte da \textit{Martin-L$\ddot{o}$f}.\\
E i costrutti degli \textbf{Universi}, in cui U \`e universo di proposizioni e di insiemi.
\section{Regole paradigmatiche per caratterizzare la teoria dei tipi}
\label{sec:prime-regole-teoria-dei-tipi}
La teoria dei tipi \`e stata formalizzata usando la nozione di \textbf{giudizio}, dove si asserisce qualcosa come vero.\\
Ci sono quattro forme di giudizio (nelle quali $\Gamma$ identifica il contesto):
\begin{itemize}
\item \textbf{A type[$\Gamma$]}: A \`e un tipo, possibilmente indicato da variabili nel contesto $\Gamma$, dipendente da $\Gamma$ stesso. Rappresenta il giudizio di tipo.
\item \textbf{A = B type[$\Gamma$]}: il tipo A dipendente da $\Gamma$ \`e uguale al tipo B dipendente da $\Gamma$. Rappresenta il giudizio di uguaglianza di tipo.
\item \textbf{a $\in$ A [$\Gamma$]}: a \`e un elemento del tipo A, possibilmente indiciato, ovvero dipendente da $\Gamma$ e dalle sue variabili di contesto. Un esempio di tipo dipendente \`e l'array, che ha termini di funzioni che dipendono da $\Gamma$. Invece il termine non \`e dipendente quando si parla di funzione costante senza variabili.
\item \textbf{a = b $\in$ A [$\Gamma$]}: a come elemento del tipo A dipende da $\Gamma$ ed \`e uguale in modo definizionale/computazionale al termine b. Quest'ultimo, difatti, \`e elemento del tipo A dipendente da $\Gamma$.
\end{itemize}
\noindent
All'interno di ogni singolo giudizio si lavora con la teoria dei tipi.\\
I giudizi sono esclusivamente asserzioni, dicono solo qualcosa quando \`e vero (non si usano i quantificatori). Essi limitano le frasi che si possono fare per codificare la Logica intuizionistica. 
\subsection{Simbolo $\in$}
\label{subsec:simbolo-appartiene}
Il significato di a$\in$A in teoria dei tipi \`e differente da quello insiemistico. Espongo il concetto con un esempio trattato a lezione:
\begin{center}
\textbf{1 $\in$ Nat}
\end{center}


\begin{itemize}
\item in \textbf{\textit{set theory}} usuale $\in$ \`e tra insiemi. Nell'equazione sopra, 1 rappresenta lui stesso un'insieme e Nat l'insieme dei numeri Naturali.
Risulta vero che 1$\equiv$\{$\varnothing$\}, poich\`e 0 $\equiv$ $\varnothing$.
\item invece in \textbf{teoria dei tipi} (di \textit{Martin-L$\ddot{o}$f} come di \textit{Russell})
1 rappresenta un elemento ma non un tipo e Nat il tipo dei Naturali. Vi \`e dunque la distinzione tra elemento e tipo (come esiste quella tra programmi e tipi).
\end{itemize}
\subsection{Uguaglianza definizionale vs uguaglianza proposizionale}
\label{subsec:uguaglianza-computazionale-proposizionale}
Specifico \textbf{a $=$ b$\in$ A[$\Gamma$]} come l'uguaglianza computazionale/definizionale, che viene data come primitiva e non va confusa con l'uguaglianza proposizionale/estensionale tra a e b.\\L'uguaglianza proposizionale a =$_A$ b \`e rappresentata non da un giudizio, che asserisce solo ci\`o che \`e vero, ma bens\`i da un tipo Eq(A,a,b) che pu\`o anche essere senza termini e/o essere falso, dal punto di vista logico.\\\\
Visti come programmi, a e b rappresentano lo stesso programma. In $\lambda$-calcolo a$\rightarrow$b oppure b$\rightarrow$a (si riducono). Inoltre a e b possono essere sia termini finali che trovarsi in mezzo alla computazione.
\subsection{Generazione di contesti}
\label{subsec:generazione-di-contesti}
Esiste anche un quinto giudizio ausiliario (\S\ref{subsec:formazione-contesti}) \textit{(F-c)}, che permette di generare i contesti. Tale giudizio, a differenza dei primi quattro, rimane immutato in ogni teoria dei tipi.

\section{Esercizi}
\label{subsec: lambda-calcolo-puro}
\paragraph{1)} 
\textbf{Dato il seguente termine, elencare quali sono le sue variabili libere e le sue variabili legate con i lambda termini relativi.}
\begin{center}$\lambda$z.((($\lambda$x.$\lambda$x.yx)x)(v$\lambda$z.$\lambda$w.v))\end{center}
\textbf{Soluzione}\\\\
$\lambda$z.((($\lambda$x.\textbf{\textcolor{red}{$\lambda$x}}.\textbf{\textcolor{green}{y}\textcolor{red}{x}})\textbf{\textcolor{green}{x}})(\textbf{\textcolor{green}{v}}$\lambda$z.$\lambda$w.\textbf{\textcolor{green}{v}}))
\begin{itemize}
\item le variabili libere (colorate in verde) sono \textit{y, x} e \textit{v}
\item le variabili legate con i relativi lambda termini (colorate in rosso) sono la \textit{x}
\end{itemize}

\paragraph{2)}
\textbf{Rinominare le variabili legate nel seguente termine in modo che non ci siano due variabili legate con lo stesso nome.}
\begin{center}x($\lambda$x.(($\lambda$x.x)x))\end{center}
\textbf{Soluzione}\\\\
x(\textbf{\textcolor{red}{$\lambda$x}}.((\textbf{\textcolor{red}{$\lambda$x}}.x)x))\\\\
Le variabili legate sono le x all'interno delle parentesi tonde. Una possibile rinomina, per evitare che queste variabili legate abbiano lo stesso nome, \`e x($\lambda$x.(($\lambda$y.y)x)), dove la x della parentesi pi\`u interna \`e stata sostituita con la y.

\paragraph{3)}
\textbf{Evidenziare di due colori diversi quali sono le variabili libere e quali quelle legate.
\begin{center}($\lambda$x.(z($\lambda$z.((xyz)x))zx))x($\lambda$x.(($\lambda$y.yy)($\lambda$z.zz)))\end{center}}
\noindent \textbf{Soluzione}\\\\
($\lambda$x.(\textbf{\textcolor{green}{z}}($\lambda$z.((\textbf{\textcolor{red}{x}\textcolor{green}{y}\textcolor{red}{z}})\textbf{\textcolor{red}{x}}))\textbf{\textcolor{green}{z}\textcolor{red}{x}}))\textbf{\textcolor{green}{x}}($\lambda$x.(($\lambda$y.\textbf{\textcolor{red}{yy}})($\lambda$z.\textbf{\textcolor{red}{zz}})))
\begin{itemize}
\item le variabili libere (colorate in verde) sono \textit{z, y} e \textit{x} 
\item le variabili legate (colorate in rosso) sono \textit{x, y} e \textit{z}
\end{itemize}

\paragraph{4)}
\textbf{Descrivere un termine del $\lambda$-calcolo, descritto in \S\ref{subsec:lambda-calcolo},  che \`e convergente con almeno un passo di riduzione rispetto a una specifica strategia di riduzione deterministica.}\\\\
\textbf{Soluzione}\\\\
Per definizione un termine t \`e convergente, rispetto a una strategia di
riduzione deterministica, se esiste un numero finito n $>=$ 1 di termini s$_1$,..., s$_n$ tale che s$_1$ $\equiv$ t e s$_i$ $\rightarrow_1$ s$_{i + 1}$ per i $=$ 1,..., n-1 e s$_n$ non \`e riducibile ad alcun termine.\\
Prendendo in considerazione una strategia di riduzione deterministica \textit{call-by value}, usata per la semantica nei linguaggi di programmazione, allora un esempio di termine t del $\lambda$-calcolo, convergente in almeno un passo di riduzione, \`e:
\begin{center}t $\equiv$ (($\lambda$x.x)z)(($\lambda$y.y)w) $\rightarrow_1$ z(($\lambda$y.y)w) $\rightarrow_1$ z(w) $\equiv$ s$_n$ $\equiv$ s\end{center}
\paragraph{5)} 
\textbf{Descrivere due termini diversi del $\lambda$-calcolo, descritto in \S\ref{subsec:lambda-calcolo}, che non sono convergenti, sempre rispetto a una strategia di riduzione deterministica.}\\\\
\textbf{Soluzione}\\\\
Per definizione un termine t diverge (\`e non convergente), rispetto a una strategia
di riduzione deterministica, se esiste una quantit\`a numerabile di termini s$_i$ al variare di i $\in$ Nat tale che s$_1$ $\equiv$ t e s$_i$ $\rightarrow_1$ s$_{i+1}$, per ogni i $\in$ Nat (ossia esiste una lista infinita di passi computazioni a partire da t).
Prendendo in considerazione una strategia di riduzione deterministica \textit{call-by value}, usata per la semantica nei linguaggi di programmazione, allora un esempio di due termine diversi del $\lambda$-calcolo che non sono convergenti \`e:
\begin{center}($\lambda$x x(x))($\lambda$y y(y)) $\rightarrow_1$ ($\lambda$y y(y))($\lambda$y y(y))  $\rightarrow_1$ ... $\rightarrow_1$ ($\lambda$y y(y))($\lambda$y y(y)) $\rightarrow_1$ ... \end{center}
Si riduce sempre a se stessa, a qualunque passo di computazione. Perci\`o ammette computazione infinita (diverge) non raggiungendo mai un valore finale.
\paragraph{6)} 
\textbf{Che relazione c'\`e tra il $\lambda$-calcolo puro con le regole di riduzione date in \S\ref{subsec:lambda-calcolo}, rispetto a quello in cui, adottando la stessa sintassi di termini, imponiamo la seguente definizione di riduzione $\rightarrow_1^\ast$.}
\begin{itemize}
\item per ogni termine t e b ($\lambda$x.t)(b) $\rightarrow_1^\ast$ t[$\frac{x}{b}$]
\item per ogni termine b, b$_1$, b$_2$ e a, a$_1$, a$_2$
\AxiomC{a$_1$ $\rightarrow_1^\ast$ a$_2$}
\AxiomC{b$_1$ $\rightarrow_1^\ast$ b$_2$}
\begin{center}
\LeftLabel{R$_I$}
\BinaryInfC{a$_1(b_1)$ $ \rightarrow_1^\ast$ a$_2(b_2)$}
\DisplayProof
\qquad
\AxiomC{t$_1$ $\rightarrow_1^\ast$ t$_2$}
\LeftLabel{R$_{II}$}
\UnaryInfC{$\lambda$x.t$_1$ $\rightarrow_1^\ast$ $\lambda$x.t$_2$}
\DisplayProof
\end{center}
\end{itemize}
\noindent
\\\\
\textbf{Soluzione}\\\\
\textit{Idea: devo provare che relazione esiste tra $\rightarrow_1^\ast$ e $\rightarrow$. Per cui verifico cosa accade per ($\rightarrow_1^\ast \subseteq \rightarrow_1$) e ($\rightarrow_1 \subseteq \rightarrow_1^\ast$)}\\\\
\noindent Sia L(\textit{T}) l'insieme dei $\lambda$ termini che \`e possibile ridurre in forma normale, con la strategia di riduzione \textit{T}. Dimostro che valgono le seguenti relazioni tra $\rightarrow_1^\ast e \rightarrow$:
\begin{enumerate}
\item L($\rightarrow_1^\ast$) $\nsubseteq$ L($\rightarrow_1$)
\item L($\rightarrow_1^\ast$) $\subset$ L($\rightarrow_1$)
\end{enumerate}
\noindent
\begin{enumerate}
\item Si ha il $\lambda$ termine t $\equiv$ (($\lambda$x.x)z)(($\lambda$y.y)w)\\
Allora applicando la strategia di riduzione $\rightarrow$, ottengo
\begin{prooftree}
\AxiomC{$\lambda$x.x $\rightarrow$ z}
\UnaryInfC{(($\lambda$x.x)z)(($\lambda$y.y)w) $\rightarrow$ z(($\lambda$y.y)w)}
\end{prooftree}
\begin{prooftree}
\AxiomC{($\lambda$y.y)w $\rightarrow$ w}
\UnaryInfC{z(($\lambda$y.y)w) $\rightarrow$ z(w)}
\end{prooftree}
\noindent Dunque riesco a giungere a una forma normale.\\
Cosa che non \`e possibile con la strategia $\rightarrow_1^\ast$. In quanto (($\lambda$x.x)z)(($\lambda$y.y)w) $\rightarrow_1^\ast$ z(($\lambda$y.y)w)  che non \`e in forma normale. Per cui ((($\lambda$x.x)z)(($\lambda$y.y)w) $\nrightarrow_1^\ast$.\\
In conclusione risulta vero che L($\rightarrow_1^\ast$) $\nsubseteq$ L($\rightarrow_1$).
\item Per provare l'inclusione di L($\rightarrow_1^\ast$) in L($\rightarrow_1$) basta che dimostro, per una valutazione che usa entrambe le strategie, il sempre possibile rimpiazzo della regola R$_I$ con la sua regola corrispondente in $\rightarrow_1$ (A$_I$ + A$_{II}$) . Pi\`u formalmente significa provare che per ogni valutazione di un termine M, che usa la regola R$_I$, si pu\`o sempre ottenere una valutazione che usa solo regole della strategia $\rightarrow_1$.\\ Per farlo procedo per induzione sul numero di volte n che la regola R$_I$ viene utilizzata durante la valutazione di M.
\begin{itemize}
\item (n=0): caso base. La regola R$_I$ non viene mai utilizzata nella valutazione di M, e dunque M \`e gi\`a implicitamente dimostrato, usando l'ipotesi induttiva, con la strategia $\rightarrow_1$.
\item (n $\rightarrow$ n+1): caso induttivo. Premesse:
\begin{enumerate}
\item per n risulta vero che L($\rightarrow_1^\ast$) $\subset$ L($\rightarrow_1$), devo provare che vale anche per n+1;
\item inoltre la valutazione di M utilizza almeno una volta la regola R$_I$;
\item per ipotesi induttuva, esiste almeno una valutazione di M con solo regole della strategia $\rightarrow_1$.
\end{enumerate}
\noindent
Dunque ho M $\rightarrow$...$\xrightarrow[]{R_I}_1^\ast$M$^I$ e voglio costruire una derivazione M $\rightarrow$...$\xrightarrow[]{A_I}_1M_1\xrightarrow[]{A_{II}}_1M^I$.\\
Le strategie di valutazione sono deterministiche, portano pertanto allo stesso risultato della sequenza di derivazione, per cui R$_I$ $=$ A$_I$ + A$_{II}$ risulta vero. Inoltre, per ipotesi induttiva, \`e sempre possibile avere una valutazione con l'utilizzo di solo regole $\rightarrow_1$ (A$_I$ + A$_{II}$ esistono). Perci\`o risulta corretto rimpiazzare $\rightarrow_1^\ast$ con $\rightarrow_1$.\\ In conclusione risulta vero che L($\rightarrow_1^\ast$) $\subset$ L($\rightarrow_1$).
\end{itemize}
\end{enumerate}






\newpage
\chapter{Regole della teoria dei tipi}
\label{cap:regole-teoria-dei-tipi}
%%dalla lezione 7 alla lezione 11
Lo scopo della teoria dei tipi \`e offrire un sistema formale in cui derivare, tramite regole e assiomi, giudizi nella forma:
\[ A \hspace{0.1cm} type[\Gamma] \qquad
A = B \hspace{0.1cm} type[\Gamma] \qquad
a \in A\hspace{0.1cm} [\Gamma] \qquad
a=b \in A \hspace{0.1cm}[\Gamma]
\]
\[
+ \hspace{0.1cm} ausiliaria \quad \Gamma \hspace{0.1cm} cont
\]
L'ultimo giudizio non \`e necessario, serve esclusivamente per imparare.\\
Quando si formula una nuova teoria dei tipi \`e bene impiegare il minor numero possibile di regole strutturali e di formazione di tipi e termini. Tali regole devono essere rivolte all'ottimizzazione e correttezza della teoria. Alcune di queste, come quelle di indebolimento e sostituzione in \S\ref{subsec:indebolimento-sostituzione}, sono irrinunciabili, la cui validit\`a \`e sempre garantita e utilizzate nella derivazione di ogni teoria.\\\\
Se la teoria dei tipi \`e dipendente si ha bisogno di tutti i giudizi. Invece in una teoria dei tipi non dipendente, come quella dei tipi semplici, il giudizio $A = B \hspace{0.1cm} type[\Gamma]$ pu\`o venire omesso.\\

\section{Regole strutturali}
\label{sec:regole-strutturali}
Assioma unico: [\hspace{0.1cm}] cont\\\\
Nel calcolo dei sequenti, in logica classica, le derivazioni di giudizio, valide in una teoria dei tipi con solo le regole singoletto, diventano derivazioni di sequenti nella forma $\Gamma$ $\dashv$ A e unico assioma $\varphi$ $\dashv$ $\varphi$.
\\\\
Di seguito illustro le principali regole di contesto comuni a tutte le teorie dei tipi.
\subsection{Regole di formazione dei contesti}
\label{subsec:formazione-contesti}
\begin{center} [\hspace{0.1cm}] cont \quad dove [\hspace{0.1cm}] = $\varnothing$ \end{center}
\begin{prooftree}
\AxiomC{A type[$\Gamma$]}
\LeftLabel{F-C)}\RightLabel{x $\in$ A $\notin$ $\Gamma$}
\UnaryInfC{$\Gamma$, x $\in$ A}
\end{prooftree}
\subsection{Regole di assunzione delle variabili}
\label{subsec:assunzione-variabili}
\begin{prooftree}
\AxiomC{$\Gamma$, x $\in$ A, $\Delta$}
\AxiomC{cont}
\LeftLabel{var-ass)}
\BinaryInfC{x $\in$ [$\Gamma$,x $\in$ A, $\Delta$]}
\end{prooftree}
\subsection{Regole strutturali addizionali sull'uguaglianza}
\label{subsec:uguaglianza}
L'uguaglianza, in una teoria dei tipi, consiste in una relazione di equivalenza sia fra tipi che fra termini. Sono perci\`o valide le seguenti regole di uguaglianza tra tipi:
\[ ref) \quad \frac{A\hspace{0.1cm}type[\Gamma]}{A=A\hspace{0.1cm}type[\Gamma]} \qquad sym) \quad \frac{A=B\hspace{0.1cm}type[\Gamma]}{B=A\hspace{0.1cm}type[\Gamma]} \]
\[ tra) \quad \frac{A=B\hspace{0.1cm}type[\Gamma] \qquad B=C\hspace{0.1cm}type[\Gamma]}{A=C\hspace{0.1cm}type[\Gamma]} \]
E allo stesso modo anche le regole di uguaglianza definizonale/computazionale tra termini:
\[ ref) \quad \frac{a \in A\hspace{0.1cm}[\Gamma]}{a=a \in A\hspace{0.1cm}[\Gamma]} \qquad sym) \quad \frac{a=b\in A \hspace{0.1cm}[\Gamma]}{b=a \in A\hspace{0.1cm}[\Gamma]} \]
\[ tra) \quad \frac{a=b \in A\hspace{0.1cm}[\Gamma] \qquad b=c \in A\hspace{0.1cm}[\Gamma]}{a=c \in A\hspace{0.1cm}[\Gamma]} \]
\subsection{Regole di conversione dell'uguaglianza per tipi uguali}
\label{subsec:conversione-uguaglianza}
L'appartenenza si conserva con l'uguaglianza di termini e tipi. Le regole da aggiungere, in una teoria dei tipi, per garantirlo sono:
\[ conv) \quad \frac{a \in A\hspace{0.1cm}[\Gamma] \quad A=B\hspace{0.1cm}type[\Gamma]}{a \in B\hspace{0.1cm}[\Gamma]} \]
\[ conv-eq) \quad \frac{a=b \in A\hspace{0.1cm}[\Gamma] \qquad A=B\hspace{0.1cm}type[\Gamma]}{a=b \in B\hspace{0.1cm}[\Gamma]} \]
\subsection{Regole di indebolimento e di sostituzione}
\label{subsec:indebolimento-sostituzione}
\subsubsection{Indebolimento}
\label{subsec:indebolimento}
\[ ind-ty) \quad \frac{A\hspace{0.1cm}type[\Gamma] \quad \Gamma,\Delta\hspace{0.1cm}cont}{A\hspace{0.1cm}type[\Gamma,\Delta]} \quad ind-ty-eq) \quad \frac{A=B\hspace{0.1cm}type[\Gamma] \quad \Gamma,\Delta\hspace{0.1cm}cont}{A=B\hspace{0.1cm}type[\Gamma,\Delta]} \]
\[ ind-te) \quad \frac{a \in A\hspace{0.1cm}[\Gamma] \quad \Gamma,\Delta\hspace{0.1cm}cont}{a \in A\hspace{0.1cm}[\Gamma,\Delta]} \quad ind-te) \quad \frac{a=b \in A\hspace{0.1cm}
[\Gamma] \quad \Gamma,\Delta\hspace{0.1cm}cont}{a=b \in A\hspace{0.1cm}[\Gamma,\Delta]} \]
\subsubsection{Sostituzione}
\label{subsec:sostituzione}
\begin{equation}
\begin{split}
C(x_1,...,x_n)\hspace{0.1cm}type[\Gamma, x_1 \in A_1,...,x_n \in A_n(x_1,...,x_{n-1})] \\ sub-typ) \quad \frac{a_1 \in A_1\hspace{0.1cm}[\Gamma]\hspace{0.1cm}...a_n \in A_n(a_1,...,a_{n-1})\hspace{0.1cm}[\Gamma]}{ C(a_1,...,a_n)\hspace{0.1cm}type[\Gamma]}
\end{split}
\end{equation}

\begin{equation}
\begin{split}
C(x_1,...,x_n)\hspace{0.1cm}type[\Gamma, x_1 \in A_1,...,x_n \in A_n(x_1,...,x_{n-1})] \\ sub-eq-typ) \quad \frac{a_1 = b_1  \in A_1\hspace{0.1cm}[\Gamma]\hspace{0.1cm}...a_n = b_n \in A_n(a_1,...,a_{n-1})\hspace{0.1cm}[\Gamma]}{C(a_1,...,a_n) = C(b_1,...,b_n) \hspace{0.1cm}type[\Gamma]}
\end{split}
\end{equation}

\begin{equation}
\begin{split}
C(x_1,...,x_n) = D(x_1,...,x_n) \hspace{0.1cm}type[\Gamma, x_1 \in A_1,...,x_n \in A_n(x_1,...,x_{n-1})] \\ sub-Eqtyp) \quad \frac{a_1 \in A_1\hspace{0.1cm}[\Gamma]\hspace{0.1cm}...a_n \in A_n(a_1,...,a_{n-1})\hspace{0.1cm}[\Gamma]}{C(a_1,...,a_n) = D(a_1,...,a_n)\hspace{0.1cm}type[\Gamma]}
\end{split}
\end{equation}

\begin{equation}
\begin{split}
C(x_1,...,x_n) = D(x_1,...,x_n) \hspace{0.1cm}type[\Gamma, x_1 \in A_1,...,x_n \in A_n(x_1,...,x_{n-1})] \\ sub-eq-Eqtyp) \quad \frac{a_1 = b_1\in A_1\hspace{0.1cm}[\Gamma]\hspace{0.1cm}...a_n = b_n \in A_n(a_1,...,a_{n-1})\hspace{0.1cm}[\Gamma]}{C(a_1,...,a_n) = D(a_1,...,a_n)\hspace{0.1cm}type[\Gamma]}
\end{split}
\end{equation}

\begin{equation}
\begin{split}
c(x_1,...,x_n) \in C(x_1,...,x_n) \hspace{0.1cm}[\Gamma, x_1 \in A_1,...,x_n \in A_n(x_1,...,x_{n-1})] \\ sub-ter) \quad \frac{a_1 in A_1\hspace{0.1cm}type[\Gamma]\hspace{0.1cm}...a_n \in A_n(a_1,...,a_{n-1})\hspace{0.1cm}[\Gamma]}{ c(a_1,...,a_n) \in C(a_1,...,a_n)\hspace{0.1cm}type[\Gamma]}
\end{split}
\end{equation}

\begin{equation}
\begin{split}
c(x_1,...,x_n) = d(x_1,...,x_n)  \in C(x_1,...,x_n) \hspace{0.1cm}[\Gamma, x_1 \in A_1,...,x_n \in A_n(x_1,...,x_{n-1})] \\ sub-eqter) \quad \frac{a_1 \in A_1\hspace{0.1cm}[\Gamma]\hspace{0.1cm}...a_n \in A_n(a_1,...,a_{n-1})\hspace{0.1cm}[\Gamma]}{ c(a_1,...,a_n) = d(a_1,...,a_n) \in C(a_1,...,a_n)\hspace{0.1cm}[\Gamma]}
\end{split}
\end{equation}

\begin{equation}
\begin{split}
c(x_1,...,x_n) \in C(x_1,...,x_n) \hspace{0.1cm}[\Gamma, x_1 \in A_1,...,x_n \in A_n(x_1,...,x_{n-1})] \\ sub-eq-ter) \quad \frac{a_1 = b_1 \in A_1\hspace{0.1cm}[\Gamma]\hspace{0.1cm}...a_n = b_n \in A_n(a_1,...,a_{n-1})\hspace{0.1cm}[\Gamma]}{ c(a_1,...,a_n) = c(b_1,...,b_n) \in C(a_1,...,a_n)\hspace{0.1cm}[\Gamma]}
\end{split}
\end{equation}

\begin{equation}
\begin{split}
c(x_1,...,x_n) = d(x_1,...,x_n) \in C(x_1,...,x_n) \hspace{0.1cm}[\Gamma, x_1 \in A_1,...,x_n \in A_n(x_1,...,x_{n-1})] \\ sub-eq-eqter) \quad \frac{a_1 = b_1 \in A_1\hspace{0.1cm}[\Gamma]\hspace{0.1cm}...a_n = b_n \in A_n(a_1,...,a_{n-1})\hspace{0.1cm}type[\Gamma]}{ c(a_1,...,a_n) = d(b_1,...,b_n) \in C(a_1,...,a_n)\hspace{0.1cm}[\Gamma]}
\end{split}
\end{equation}

\subsection{Regole proprie e derivabili}
\label{subsec:regole-proprie-derivabili}
In una teoria formale ci sono due tipi di regole:
\begin{itemize}
\item \textbf{regole proprie del calcolo}, come lo sono le regole strutturali e quelle del singoletto;
\item \textbf{regole derivabili}, come le regole di sostituzione, utili per abbreviare le derivazioni.
\end{itemize}
\noindent
Una regola r $\frac{J_1,...,J_n}{J}$ \`e ammissibile in un calcolo t sse i giudizi derivabili in $t+r$ sono gli stessi dei giudizi derivabili in t. Ci\`o comporta che l'aggiunta di una regola rt non cambia i giudizi che ne possono derivare.\\ Quando un assioma \`e ammissibili e derivabile questo coincide con un giudizio derivabile.

\subsection{Nozione di contesto telescopico}
\label{subsec:contesto-telescopico}
Un giudizio, in teoria dei tipi dipendenti si esprime nella forma
\[A(x_1,...,x_n)[x_1 \in B_1,...,x_n \in B_n]\]
e prende il nome di \textbf{contesto telescopico}.
Questi presenta una dipendenza continua, esemplificata nel seguente giudizio
\[A(x_1,x_2,x_3)\hspace{0.1cm}type[x \in C_1, x_2 \in C(x_1), x_3 \in C(x_1 x_2)..)\]
Inoltre si parla di contesti rigidi, ovvero senza possibilit\`a di scambio. Come appare dall'esempio sotto. \\
$[x \in Nat, y \in Nat, z \in Mat(x,y)]$ cont  $\Rightarrow$ \textbf{\`e derivabile}\\
$[y \in Nat, x \in Nat, z \in Mat(x,y)]$ cont $\Rightarrow$ \textbf{\`e derivabile}\\
$[y \in Nat, z \in Mat(x,y), x \in Nat]$ $\Rightarrow$ \textbf{non \`e un contesto}. \\\\ Percui non esiste lo scambio arbitrario, si deve porre attenzione alle dipendenza delle assunzioni, che provoca una sostituzione rigida.
\subsection{Esempi di applicazione}
\label{subsec:esempi-di-applicazione}
Attenzione all'ordine di sostituzione si deve partire sempre da quello con meno dipendenze.
\begin{prooftree}
\AxiomC{c $\in$ [C, $\Gamma$]}
\AxiomC{b $\in$ [B(c), $\Gamma]$}
\AxiomC{A(x,y) type[x $\in$ C, y $\in$ B(x)]}
\TrinaryInfC{A(c,b) type$[\Gamma]$}
\end{prooftree}

\begin{prooftree}
\AxiomC{c $\in$ [C,$\Gamma$]}
\AxiomC{b $\in$ [B(c),$\Gamma]$}
\AxiomC{A(x,y) type[x $\in$ C, y $\in$ B(x)]}
\TrinaryInfC{a(c,b) $\in$ A(c,b) $[\Gamma]$}
\end{prooftree}
Se si ha un tipo puo' venire usato il giudizio di uguaglianza tra termini e la sostituzione.
\begin{prooftree}
\AxiomC{A(x) type[$\Gamma$, x $\in$ C]}
\AxiomC{c = e $\in$ C[$\Gamma$]}
\BinaryInfC{A(c) = A(e) type[$\Gamma$]}
\end{prooftree}
\noindent
\`E dunque fondamentale il concetto di uguaglianza fra tipi. Se considero che ci sia un elemento

\begin{prooftree}
\AxiomC{a(x) $\in$ A(x) [$\Gamma$, x $\in$ C]}
\AxiomC{c = e $\in$ C[$\Gamma$]}
\BinaryInfC{a(c) = a(e) $\in$ A(c) [$\Gamma$]}
\end{prooftree}
\noindent
Per poter affermare che A(e) = A(c) devo poterlo dedurre. Per farlo mi sono indispensabili le \textbf{regole di conversione dell'uguaglianza del tipaggio (\S\ref{subsec:regole-tipaggio})}.

\subsection{Regole di tipaggio}
\label{subsec:regole-tipaggio}

\subsubsection{Regole di Conversione}
\label{subsubsec:regole-di-conversione}
\[\frac{c \in C[\Gamma] \qquad C = D\hspace{0.1cm}type[\Gamma]}{C \in D\hspace{0.1cm}[\Gamma]}\]
Se due tipi sono uguali allora hanno gli stessi termini: $C=D \Rightarrow (c \in C \Leftrightarrow c \in D)$. L'uguaglianza fra tipi \`e per questo simmetrica.\\
Tuttavia non sempre l'unicit\`a del tipaggio di un termine (il $\Leftrightarrow$) \`e garantito per ogni teoria. Nei casi trattati dal corso s\`i, in quanto verr\`a inteso che $C = D\hspace{0.1cm}type[\Gamma]$ sse due tipi hanno gli stessi elementi (come gi\`a accade in \textit{set theory}), ma pu\`o non essere sempre vero.

\subsubsection{Regole di Conversione dell'uguaglianza}
\label{subsubsec:regole-di-conversione-uguaglianza}
\[\frac{c=d \in C[\Gamma] \qquad C = D\hspace{0.1cm}type[\Gamma]}{c = d \in D \hspace{0.1cm} [\Gamma]}\]
Questa regola permette di convertire le uguaglianze nel tipaggio di un termine.


\section{Il tipo singoletto}
\label{sec:tipo-singoletto}
Il tipo singoletto risulta essere paradigmatico per gli altri tipi. Per definirlo impiegher\`o i giudizi nella forma $A\hspace{0.1cm}type[\Gamma]$, $a \in A\hspace{0.1cm}[\Gamma]$ e $a = b \in A\hspace{0.1cm}[\Gamma]$. L'uguaglianza, invece, non verr\`a coinvolta, in quanto non pu\`o essere impiegata per definire un nuovo tipo, \`e difatti usata solo nelle derivazioni.\\\\
Innanzitutto come gi\`a visto in \S\ref{subsec:formazione-contesti} ogni derivazione parte sempre dal contesto vuoto ($\varnothing$).
\begin{prooftree}
\AxiomC{A type[$\Gamma$]}
\LeftLabel{F-C)}\RightLabel{x $\in$ A $\notin$ $\Gamma$}
\UnaryInfC{$\Gamma$, x $\in$ A}
\end{prooftree}

\subsection{Regola di Formazione}
\label{subsec:formazione-singoletto}
\begin{prooftree}
\AxiomC{[$\Gamma$] cont}
\LeftLabel{F-S)}
\UnaryInfC{$N_1$ type[$\Gamma]$}
\end{prooftree}
\textit{La regola F-S Permette di derivare vari giudizi e di dire che cosa \`e un tipo.}\\\\
Con l'impiego solo delle regole \textit{F-C} e \textit{F-S} si possono derivare [$x_1$ $\in$ $N_1$... $x_n$ $\in$ $N_1$] cont e ottenere, cos\`i, contesti di una lista arbitraria di variabili diverse appartenenti a $N_1$, come si vede dall'esempio seguente.
\begin{prooftree}
\AxiomC{[ ] cont}
\LeftLabel{F-S)}
\LeftLabel{F-C)}\RightLabel{$x_1$ $\in$ $N_1$ $\notin$ $\varnothing$}
\UnaryInfC{[$x_1$ $\in$ $N_1$] cont}
\LeftLabel{F-S)}
\UnaryInfC{$N_1$ type [$x_1$ $\in$ $N_1$]}
\LeftLabel{F-C)}\RightLabel{$x_2$ $\in$ $N_1$ $\notin$ $x_1$ $\in$ $N_1$}
\UnaryInfC{[$x_1$ $\in$ $N_1$, $x_2$ $\in$ $N_1$] cont}
\end{prooftree}

\subsection{Regole di Introduzione}
\label{subsec:introduzione-singoletto}
\begin{prooftree}
\AxiomC{[$\Gamma$] cont}
\LeftLabel{I-S)}
\UnaryInfC{$\ast$ $\in$ $N_1$ [$\Gamma]$ cont}
\end{prooftree}
\textit{Sia $N_1$ in ogni contesto $\Gamma$, partendo da contesto $\varnothing$, la regola I-S permette di formare i termini, per mezzo dell'introduzione di un elemento costante $\ast$ in $N_1$.}\\\\
Un esempio diretto della sua applicazione \`e
\begin{prooftree}
\AxiomC{[ ] cont}
\LeftLabel{I-S)}
\UnaryInfC{$\ast$ $\in$ $N_1$ ($x_1$ $\in$ $N_1$...$x_n$ $\in$ $N_1$)}
\end{prooftree}

\subsection{Regole di Eliminazione}
\label{subsec:eliminazione-singoletto}
\begin{prooftree}
\AxiomC{t $\in$ $N_1$ [$\Gamma$]}
\AxiomC{M(z) type[$\Gamma$, z $\in$ $N_1$]}
\AxiomC{c  $\in$ M($\ast$)[$\Gamma$]}
\LeftLabel{E-S)}
\TrinaryInfC{$El_{N1}$(t, c) $\in$ M($\ast$)[$\Gamma$]}
\end{prooftree}
\noindent
\textit{El trattasi di costruttore di funzioni e $M[t] = M(z)[\frac{z}{t}]$.}
\subsection{Regole di Conversione}
\label{subsec:conversione-singoletto}
\begin{prooftree}
\AxiomC{M(z) type[$\Gamma$, z $\in$ $N_1$]}
\AxiomC{c $\in$ M($\ast$)[$\Gamma$]}
\LeftLabel{C-S)}
\BinaryInfC{$El_{N1}$($\ast$, c) = c $\in$ M($\ast$)[$\Gamma$]}
\end{prooftree}
\textit{La conversione rende possibile l'applicazione della regola di eliminazione introducendo delle uguaglianze.}
\\\\
Le regole \textit{(S), (I-S), (E-S)} e \textit{(C-S)}  hanno una spiegazione computazionale, e riguardano la compatibilit\`a tra tipi, ma non da caratterizzare il tipo dei tipi.\\ Inoltre il tipo singoletto non \`e dipendente.

\subsection{Eliminatore dipendente}
\label{subsec:eliminatore dipendente-singoletto}
La regola di eliminazione si pu\`o equivalentemente scrivere in un altro modo
\begin{prooftree}
\AxiomC{M(z) type[$\Gamma$, z $\in$ $N_1$]}
\AxiomC{c  $\in$ M($\ast$)[$\Gamma$]}
\LeftLabel{E-S)$_{dip}$}
\BinaryInfC{$El_{N1}$(z, c) $\in$ M(z)[$\Gamma$, z $\in$ $N_1$]}
\end{prooftree}
Le regole \textit{E-S)$_{dip}$} + \textit{la regole di sostituzione} + \textit{F-S} + \textit{I-S} permettono di verificare la validit\`a di \textit{E-S}.
\begin{prooftree}
\AxiomC{t $\in$ $N_1$[$\Gamma$]}
\AxiomC{M(z) type[$\Gamma$, z $\in$ $N_1$]}
\AxiomC{c $\in$ M($\ast$)[$\Gamma$]}
\LeftLabel{E-S)$_{dip}$}
\BinaryInfC{$El_{N1}$(z, c) $\in$ M(z)[$\Gamma$]}
\LeftLabel{sost)}
\BinaryInfC{$El_{N1}$(t, c) $\in$ M(t)[$\Gamma$]}
\end{prooftree}
\noindent
Inoltre vale anche il viceversa, da \textit{E-S} si riesce a ottenere \textit{E-S$_{dip}$}.
\subsection{Osservazioni sul tipo singoletto}
\label{subsec:osservazioni-singoletto}
L'eliminatore El$_{N1}$(z, c) rappresenta una funzione definita per ricorsione su N$_1$, difatti in \textit{C-S} si ha che El$_{N1}$(z, c)[$\frac{z}{\ast}$] = El$_{N1}$($\ast$, c).\\
Supposto che se $\ast$ $\in$ N$_1$[$\Gamma$] in \textit{E-S}, allora per la singola conversione vale che El$_{N1}$ = c $\in$ M($\ast$).
Dunque El$_{N1}$(z, c) rappresenta un programma funzionale per ricorsione. Questi \`e definito su N$_1$, a partire da c $\in$ M($\ast$), perci\`o El$_{N1}$($\ast$, c) = c.\\
La regola di eliminazione permette di definire un programma funzionale da N$_1$ a M(z) esclusivamente con c $\in$ M($\ast$), ovvero definendo $\ast$ come \textbf{elemento canonico}. Inoltre non svolge solo il compito di ricorsione, ma anche d'nduzione.
\\\\
In \begin{prooftree}
\AxiomC{t $\in$ N$_1$[$\Gamma$]}
\UnaryInfC{t = $\ast$ $\in$ N$_1$[$\Gamma$]}
\end{prooftree}
risulta vera l'uguaglianza definizionale?\\
No, non \`e vera. La regola di eliminazione consente di dare un valore al termine canonico, permettendo cos\`i di attribuire un valore a tutti i possibili termini del singoletto. Ma in generale questo non vale all'interno della teoria. Difatti l'uguaglianza definizionale \`e diversa da quella matematica e va intesa come computazionale e non proposizionale (come definito in \S \ref{subsec:uguaglianza-computazionale-proposizionale}).

\section{Sanitary checks rules}
\label{sec:sanitary-checks}
Le \textit{Sanitary checks} sono regole strutturali utili per abbreviare le derivazioni. Queste sono derivabili solo se la teoria dei tipi \`e corretta.\\\\
\noindent
Assumento \textit{T}, come una teoria dei tipi di riferimento, le \textit{sanitary checks} sono le seguenti:
\begin{prooftree}
\AxiomC{[$\Gamma$, $\Delta$] cont}
\UnaryInfC{$\Gamma$ cont}
\end{prooftree}
Se [$\Gamma$, $\Delta$] cont \`e derivabile in \textit{T} allora anche [$\Gamma$] cont \`e derivabile in \textit{T}.
\begin{prooftree}
\AxiomC{J$_1$,...,J$_n$}
\UnaryInfC{J}
\end{prooftree}
Se J$_i$ con i = 1,...,n in \textit{T} sono derivabili allora anche J \`e derivabile in \textit{T}.
\begin{prooftree}
\AxiomC{A type [$\Gamma$]}
\UnaryInfC{$\Gamma$ cont}
\end{prooftree}
Se A type [$\Gamma$] \`e derivabile in \textit{T} allora anche $\Gamma$ cont \`e derivabile in \textit{T}.
\begin{center}
\AxiomC{A = B type [$\Gamma$]}
\UnaryInfC{A type[$\Gamma$]}
\DisplayProof
\AxiomC{A = B type [$\Gamma$]}
\UnaryInfC{B type[$\Gamma$]}
\DisplayProof
\end{center}
Se A = B type [$\Gamma$] \`e derivabile in \textit{T} allora anche A type[$\Gamma$] e B type[$\Gamma$] sono derivabili in \textit{T}.
\begin{prooftree}
\AxiomC{a $\in$ A[$\Gamma$]}
\UnaryInfC{A type[$\Gamma$]}
\end{prooftree}
Se a $\in$ A[$\Gamma$] \`e derivabile in \textit{T} allora anche A type[$\Gamma$] \`e derivabile in \textit{T}.
\begin{center}
\AxiomC{a = b $\in$ A[$\Gamma$]}
\UnaryInfC{a $\in$ A[$\Gamma$]}
\DisplayProof
\AxiomC{a = b $\in$ A[$\Gamma$]}
\UnaryInfC{b $\in$ A[$\Gamma$]}
\DisplayProof
\end{center}
Se a = b $\in$ A[$\Gamma$] \`e derivabile in \textit{T} allora anche a $\in$ A[$\Gamma$] e b $\in$ A[$\Gamma$] sono  derivabili in \textit{T}.

\section{Schema generale}
\label{sec: schema-generale}
Di seguito illustro uno schema generale, valido per ogni teoria dei tipi, di produzione di regole definenti un tipo, i suoi termini e l'uguaglianza.
\begin{enumerate}
\item \textbf{Regole di Formazione}
Identificate con il preambolo F-\textit{T}, con \textit{T} teoria dei tipi in esame e t$\backprime$ elemento canonico.\\Tali regole sono del tipo k e rispettano la forma 
\AxiomC{[$\Gamma$] cont}
\UnaryInfC{\textit{T} type[$\Gamma$]}
\DisplayProof
\item \textbf{Regole di Introduzione}
identificate con il preambolo I-\textit{T}, con \textit{T} teoria dei tipi in esame e t$\backprime$ elemento canonico.\\Tali regole consistono nella forma introduttiva degli elementi canonici di \textit{T} e rispettono la forma
\AxiomC{[$\Gamma$] cont}
\UnaryInfC{t$\backprime$ $\in$ \textit{T}[$\Gamma$]}
\DisplayProof
\item \textbf{Regole di Eliminazione}
identificate con il preambolo E-\textit{T} con \textit{T} teoria dei tipi in esame e t$\backprime$ elemento canonico.\\Tali regole sono definenti E$_k$, a partire dagli elementi di k a valori in un tipo M(z) type[$\Gamma$, z $\in$k]. L'ipotesi valida \`e che siano dati degli elementi in M(z) sui valori canonici di k.
Tali regole sono del tipo k e rispettano la forma 
\AxiomC{t $\in$ \textit{T}[$\Gamma$]}
\AxiomC{M(z) type[$\Gamma$, z $\in$ \textit{T}]}
\AxiomC{c  $\in$ M(t$\backprime$)[$\Gamma$]}
\TrinaryInfC{$El_{\textit{T}}$(t, c) $\in$ M(t)[$\Gamma$]}
\DisplayProof
\item \textbf{Regole di Conversione}
identificate con il preambolo C-\textit{T}, con \textit{T} teoria dei tipi in esame e t$\backprime$ elemento canonico.\\, che stabilscono che gli eliminatori in (3) sono stabiliti per ricorsione a partire dalle ipotesi.
Tali regole sono del tipo k e rispettano la forma
\AxiomC{M(z) type[$\Gamma$, z $\in$ \textit{T}]}
\AxiomC{c $\in$ M(t$\backprime$)[$\Gamma$]}
\BinaryInfC{El$_{\textit{T}}$(t$\backprime$, c) = c $\in$ M(t$\backprime$)[$\Gamma$]}
\DisplayProof
\item \textbf{Regole di Uguaglianza}
identificate con il preambolo eq-E-\textit{T}, con \textit{T} teoria dei tipi in esame e t$\backprime$ elemento canonico.
Tali regole stabiliscono che i costruttori di k in (2) e (3) permettono l'uguaglianza definizionale dei termini da cui dipendono.
Tali regole sono del tipo k e rispettano la forma\\
\AxiomC{t = s $\in$ \textit{T}[$\Gamma$]}
\AxiomC{M(z) type[$\Gamma$, z $\in$ \textit{T}]}
\AxiomC{c = d $\in$ M(t$\backprime$)[$\Gamma$]}
\TrinaryInfC{El$_{\textit{T}}$(t, c) = El$_{\textit{T}}$(s, d) $\in$ M(t)[$\Gamma$]}
\DisplayProof
\end{enumerate}
\noindent
Le regole (5) sono implicite in \textit{T}, ma non ovvie dal punto di vista computazionale.

\section{Uguaglianza definizionale}
\label{sec: uguaglianza-definizionale}
\textit{Definizione}: se P$_1$ e P$_2$ programmi\\\\
\textit{$P_1$, $P_2$: Nat$^m$  $\rightarrow$ Nat \quad allora \quad
$P_1$ = $P_2$ sse $\forall$ n$_1$...n$_m$ e P$_1$(n$_1$...n$_m$) = P$_2$(n$_1$...n$_m$)} non \`e decidibile.\\\\
Ovvero le funzioni ricorsive per P$_1$ e P$_2$ non sono decidibili. A seguito di ci\`o non esiste un algoritmo in grado di decidere se due programmi P$_1$ e P$_2$ sono estensionalmente (proposizionalmente) uguali o meno.\\\\
Risulta per\`o vero il concetto di \textbf{ugualianza definizionale}/computazionale (in teoria dei tipi intensionali). Dati a $\in$ A[$\Gamma$] e b $\in$ A[$\Gamma$] derivabili, nella nostra teoria \textit{T} di \textit{Martin-L$\ddot{o}$f}, esiste un algoritmo \textit{H} (a $\in$ A[$\Gamma$], b $\in$ A[$\Gamma$]) =
$
\begin{cases}
\text{\textbf{sì} sse a = b} \in A[\Gamma]  \text{ è derivabile in \textit{T}} \\
\text{\textbf{no} sse a = b} \in A[\Gamma] \text{ non ha derivazione in \textit{T}}
\end{cases}
$
Il giudizio a $=$ b $\in$ A[$\Gamma$] \`e decidibile, anche con \textit{J} giuduzio, in teoria dei tipi di \textit{Martin-L$\ddot{o}$f}, derivabile. Percui con H si scrive: Giudizi di \textit{T} $\rightarrow$ \{0,1\}\\
H[\textit{J}] =
$
\begin{cases}
\textbf{1} \text{ sse \textit{J} è derivabile in \textit{T}}  \\
\textbf{0} \text{ sse \textit{J} non è derivabile in \textit{T}}
\end{cases}
$
\\
H lavora come un \textit{proof assistant} (esempio \textit{COQ}).\\\\
\noindent

\subsection{Applicazione dell'uguaglianza definizionale tra termini}
\label{subsec: applicazione-uguaglianza-definizionale-tra-termini}
\textit{Definzione}: i termini \textit{untyped} sono
\begin{center}t = x $|$ $\ast$ $|$ El$_{N1}$(t$_1$, t$_2$)\end{center}
\textit{Definizione}: relazione di riduzione
\begin{center}$\rightarrow$ $_1$ $\subseteq$ term x term\end{center}
t$_1$ $\rightarrow$ $_1$ t$_2$ $\equiv$ t$_1$ si riduce in un passo di computazione a t$_2$. Ecco che esiste una relazione che computa t$_1$ con t$_2$.\\
Le \textbf{relazioni di computazione}, dell'uguaglianza definizionale, le ho descritte in \S\ref{sec: semantica-operazionale-singoletto}.
\\\\
\textit{Definizione}: t termine \textit{untyped} \`e in forma normale (in NF) sse non esiste termine s tale che t $\rightarrow$ $_1$ s. Ovvero non \`e pi\`u riducibile a nulla. \\  Le forme normali sono i valori assumibili dai programmi.\\
\\\\
\textit{Definzione}:\textbf{Teoria di Validit\`a}\\ Dati t $\in$ A[$\Gamma$] e s $\in$ A[$\Gamma$] in \textit{T$_1$} \quad allora \quad $\rightarrow$ $_1$ s $\Rightarrow$ t = s $\in$ A[$\Gamma$] derivabile in \textit{T}.


\begin{table}[H]
\centering
\begin{tabularx}{\textwidth}{XXX}
\hline 
\rowcolor{orange}
{\color[HTML]{FFFFFF}\textbf{termini}} & {\color[HTML]{FFFFFF} \textbf{termini in forma canonica}} & {\color[HTML]{FFFFFF}\textbf{termini non in forma canonica o introduttiva}}\\
\hline\hline
termini in NF & $\ast$ & x, El$_{N1}$(x, $\ast$)\\
termini non in NF & $\varnothing$ & El$_{N1}$($\ast$, x)\\
\end{tabularx}
\caption{\label{tab:termini-NF-nonNF-N1}Termini a $\rightarrow$ $_1$ di N$_1$.} 
\end{table}
\noindent
I termini non in forma canonica  derivano dalle regole di introduzione; invece quelli non in forma canonica vengono introdotte dall'eliminatore.\\\\
La chiusura riflessiva, simmetrica e transitiva delle derivazioni \`e proprio l'uguaglianza definizionale. Tale propriet\`a non vale esclusivamente per $\rightarrow_1$ ma per qualsiasi combinazione delle riduzioni in \$\ref{sec: semantica-operazionale-singoletto}, con  variazione per\`o delle forma normali (che non sono altro che i risultati dei nostri programmi, derivanti dai diversi cammini di computazione).\\
Un esempio significativo di applicazione di strategie di computazione l'ho riportato in \S\ref{sec:esercizi-cap2}, esercizio 4. Utile per comprendere a cosa serve la relazione $\rightarrow_1$   e che ogni teoria \textit{T} $\rightarrow$ $_1$ non \`e deterministica.
\\\\
Le definizioni seguenti sono definite sulla regole di semantica operazionale.\\\\
\textit{Definizione} \textbf{Riducibilit\`a}\\
Dati i termini t e s allora \textbf{t \textit{Red$_{NF}$} s} sse s \`e in NF ed esistono h$_1$...h$_n$ (n>=1) tale che h$_1$ $\equiv$ t, h$_n$ $\equiv$ s e se n>1 h$_i$ $\rightarrow$ h$_{i+1}$ per i = 1 a n-1.\\\\
\textbf{t \textit{Rid$_{NF}$} s} =
$
\begin{cases}
\text{\textbf{t}} \equiv \text{s e t è in NF} \\
\text{esiste n>1, esistono } h_1\text{...}h_n \text{ termini tale che \textbf{t}} \equiv h_1 \rightarrow _1 h_2 \text{...}h_n
\end{cases}
$
\\\\
\noindent
\textit{Definizione} \textbf{Teorema di Confluenza}$\rightarrow _1$ per \textit{T} computabile\\
Dato t (termine) e s$_1$ e s$_2$ (in NF) tale che \textbf{t \textit{Red$_{NF}$ }s$_1$} e \textbf{t \textit{Red$_{NF}$} s$_2$} allora s$_1$ $\equiv$ s$_2$ (coincide a meno di rinomina di variabile  vincolante).\\
Quando t si riduce s$_1$ e in s$_2$ c'e' l'unicit\`a della forma normale.
\\\\
\noindent
\textit{Definizione} \textbf{Teorema della forma normale (debole)}\\
Dato t termine della grammatica esiste s termine in NF tale che \textbf{t \textit{Red$_{NF}$ }s}. Allora esistono t $\equiv$ h$_1$ $\rightarrow$ $_1$...$\rightarrow$ $_1$ h$_n$ con n>1 se t non \`e gi\`a in NF; oppure t $\equiv$ s se t \`e gi\`a in NF.\\
Questo significa che \`e sempre possibile rendere un programma convergente. Ma si pu\`o dire di pi\`u: $\nexists$ programmi che divergono. 
\\\\
\noindent
\textit{Definizione} \textbf{Teorema della forma normale (forte)}\\
Per ogni termine t, l'albero dei cammini di riduzione di t \`e ben formato (ovvero $\nexists$ un cammino di riduzioni $\rightarrow$ $_1$ infinito).\\
In questo modo ogni strategia deterministica \`e convergente.\\
Con quanto appena enunciato sopra possiamo denfinire quanto segue.\\
Dato t $\in$ A[$\Gamma$] derivabile in \textit{T}\\
NF(t$_1$) $\equiv$
$
\begin{cases}
\text{\textbf{t} se t è in NF} \\
\text{\textbf{s} se \textit{\textbf{t Red}}}_{NF} \text{\textit{\textbf{ s}}}
\end{cases}
$
\\
Dunque se t non \'e in NF per il teorema normale comunque esiste una riduzione in NF.\\
Sono cos\`i in grado di dimostrare che, dati a $\in$ A[$\Gamma$] e b $\in$ A[$\Gamma$], giudizi derivabili in \textit{T$_i$} allora a = b $\in$ A[$\Gamma$] sse NF(a) $\equiv$ NF(b) sse
\begin{enumerate}
\item a e b sono in NF e quindi a $\equiv$ b
\item a non in NF, b in NF e \textbf{a \textit{Red$_{NF}$ }b}
\item a in NF, b non in NF e \textbf{b \textit{Red$_{NF}$ }a}
\item n\'e a n\'e b sono in NF esiste s in NF tale che \textbf{a \textit{Red$_{NF}$ }s} e \textbf{b \textit{Red$_{NF}$ }s}
\end{enumerate}
Per i punti elencanti sopra trova validit\`a la relazione \textbf{a \textit{Red$_{NF}$ }NF(a)} $\Rightarrow$ a = NF(a) $\in$ A[$\Gamma$] \`e derivabile (la forma normale \`e uguale al termine stesso).\\ Questo rende l'uguaglianza computabile, si \`e difatti in grado di dimostrare che esiste P programma tale che P(a) = NF(a), per ogni a termine \textit{untyped} in \textit{T}(incluso \textit{T$_1$}).\\\\
In conclusione la computabilit\`a dell'uguaglianza (uguaglianza definizionale) tra due termini, si riduce a computare le forme normali del primo termine con quelle del secondo e a verificare se sono identicamente la stessa (a meno di rinomia di variabili).

\section{Semantica operazionale del singoletto}
\label{sec: semantica-operazionale-singoletto}
La relazione $\rightarrow_1$ viene definita all'interno dei termini con l'uso delle seguenti regole di riduzione:
\begin{itemize}
\item $\beta_{N1}$-red) El$_{N1}$($\ast$, t) $\rightarrow_1$ t
\item \AxiomC{t$_1$ $\rightarrow_1$ t$_2$}
\LeftLabel{red$_I$)}
\UnaryInfC{El$_{N1}$(t$_1$, c) $\rightarrow_1$ El$_{N1}$(t$_2$, c)}
\DisplayProof \qquad
\AxiomC{c$_1$ $\rightarrow_1$ c$_2$}
\LeftLabel{red$_{II}$)}
\UnaryInfC{El$_{N1}$(t, c$_1$) $\rightarrow_1$ El$_{N1}$(t, c$_2$)}
\DisplayProof 
\item red$_{I}$ e red$_{II}$ possono venire simultate da un'unica regola
\AxiomC{t$_1$ $\rightarrow_1$ t$_2$}
\AxiomC{c$_1$ $\rightarrow_1$ c$_2$}
\BinaryInfC{El$_{N1}$(t$_1$, c$_1$) $\rightarrow_1$ El$_{N1}$(t$_2$, c$_2$)}
\DisplayProof
\end{itemize}
\noindent
\textit{$\beta_{N1}$-red} risulta valida per \textit{C-S}, le regole di riduzione red$_{I}$ e red$_{II}$ per \textit{eq-E-S}.\\\\


\section{Esercizi}
\label{sec:esercizi-cap2}
\subsection{Tipo singoletto}
\label{subsec: tipo-singoletto}
\paragraph{1} 
\textbf{Data}
\begin{prooftree}
\AxiomC{M(w) type [$\Gamma$, w $\in$ N$_1$]}
\AxiomC{d $\in$ M($\ast$) type[$\Gamma$]}
\LeftLabel{E-N$_{1prog}$)}
\BinaryInfC{El$_{N1}$(w, d) $\in$ M(w) type[$\Gamma$, w $\in$ N$_1$]}
\end{prooftree}
\textbf{dimostrare che in \textit{T$_1$} la regola \textit{E-N$_1prog$} \`e derivabile. Al fine di ci\`o basta mostrare che se i giudizi premessa sono derivabili, allora lo \`e anche il giudizio di conclusione.}\\\\
\textbf{Soluzione}\\\\
\textit{Per una maggiore comprensione delle derivazioni, ho ritenuto opportuno, ove necessario, spezzare l'albero in pi\`u parti.}

\begin{prooftree}
\AxiomC{}
\UnaryInfC{M(w) type[$\Gamma$, w $\in$ N$_1$]}
\LeftLabel{s-checks}
\UnaryInfC{[$\Gamma$, w $\in$ N$_1$] cont}
\LeftLabel{var}
\UnaryInfC{w $\in$ N$_1$[$\Gamma$, w $\in$ N$_1$]}
\AxiomC{\textbf{1}}
\UnaryInfC{M(z) type[$\Gamma$, z $\in$ N$_1$, w $\in$ N$_1$]}
\AxiomC{}
\UnaryInfC{d $\in$ M($\ast$)[$\Gamma$, w $\in$ N$_1$]}
\LeftLabel{E-S}
\TrinaryInfC{El$_{N1}$(w, d) $\in$ M(w)[$\Gamma$, w $\in$ N$_1$]}
\end{prooftree}

\vspace{0.5cm}
\textbf{1}
\scriptsize{
\begin{adjustwidth}{-10em}{}
\begin{prooftree}
\AxiomC{}
\UnaryInfC{M(w) type[$\Gamma$, w $\in$ N$_1$]}
\AxiomC{}
\UnaryInfC{M(w) type[$\Gamma$, w $\in$ N$_1$]}
\LeftLabel{s-checks}
\UnaryInfC{[$\Gamma$, w $\in$ N$_1$] cont}
\LeftLabel{F-S}
\UnaryInfC{N$_1$ type[$\Gamma$, w $\in$ N$_1$]}
\LeftLabel{F-c}\RightLabel{(z $\in$ N$_1$) $\notin$ $\Gamma$}
\UnaryInfC{[$\Gamma$, z $\in$ N$_1$, w $\in$ $N_1$] cont}
\LeftLabel{ind-ty}
\BinaryInfC{M(w) type[$\Gamma$, z $\in$ N$_1$, w $\in$ $N_1$]}
\AxiomC{}
\UnaryInfC{M(w) type[$\Gamma$, w $\in$ N$_1$]}
\LeftLabel{s-checks}
\UnaryInfC{[$\Gamma$, w $\in$ N$_1$] cont}
\LeftLabel{F-S}
\UnaryInfC{N$_1$ type [$\Gamma$, w $\in$ N$_1$]}
\LeftLabel{F-c}\RightLabel{(z $\in$ N$_1$) $\notin$ $\Gamma$}
\UnaryInfC{[$\Gamma$, z $\in$ N$_1$, w $\in$ N$_1$] cont}
\LeftLabel{var}
\UnaryInfC{z $\in$ $N_1$[$\Gamma$, z $\in$ $N_1$, w $\in$ N$_1$]}
\LeftLabel{sub-typ}
\BinaryInfC{M(z) type[$\Gamma$, z $\in$ N$_1$, w $\in$ N$_1$]}
\end{prooftree}
\end{adjustwidth}
}
\noindent
\normalsize{Assumo che le premesse di \textit{E-N$_1prog$} (\textit{M(w) type [$\Gamma$,w $\in$ N$_1$]} e \textit{d $\in$ M($\ast$)[$\Gamma$]}) siano valide, p\`erci\`o \`e valido, dalla prova sopra, anche il giudizio di conclusione \textit{El$_{N1}$(w, d) $\in$ M(w)[$\Gamma$, w $\in$ N$_1$]}, di conseguenza derivabile in \textit{T$_1$}.}


\paragraph{2} 
\textbf{Dimostrare che la regola \textit{E-S} \`e derivabile in una teoria dei tipi \textit{T$_1$}, in cui si \`e rimpiazziata la regola di eliminazione \textit{E-S} con la regola \textit{E-N${_1prog}$}, aggiungendovi le regole di indebolimento, sostituzione e di \textit{sanitary checks}}.
\begin{prooftree}
\AxiomC{M(w) type [$\Gamma$, w $\in$ N$_1$]}
\AxiomC{d $\in$ M($\ast$)[$\Gamma$]}
\LeftLabel{E-N$_{1prog}$)}
\BinaryInfC{El$_{N1}$(w, d) $\in$ M(w)[$\Gamma$, w $\in$ N$_1$]}
\end{prooftree}
\begin{prooftree}
\AxiomC{t$\backprime$ $\in$ N$_1$[$\sum$] }
\AxiomC{M(w) type [$\Gamma$, w $\in$ N$_1$]}
\AxiomC{d$\backprime$ $\in$ M($\ast$)[$\sum$] }
\LeftLabel{E-S)}
\TrinaryInfC{El$_{N1}$(t$\backprime$, d$\backprime$) $\in$ M(t$\backprime$)[$\Gamma$]}
\end{prooftree}
\textbf{Soluzione}\\\\
\textit{Idea: parto dalla regola di eliminazione E-S, vi applico la regola di sostituzione sub-typ giungendo cos\`i alle premessi di E-N$_{1prog}$}
\begin{prooftree}
\AxiomC{}
\UnaryInfC{t$\backprime$ $\in$ N$_1$[$\Gamma$]}
\AxiomC{}
\UnaryInfC{M(w) type [$\Gamma$, w $\in$ N$_1$]}
\AxiomC{}
\UnaryInfC{d $\in$ D($\ast$)[$\Gamma$]}
\LeftLabel{E-N$_{1prog}$}
\BinaryInfC{El$_{N1}$(w, d$\backprime$) $\in$ M(w)[$\Gamma$, w $\in$ N$_1$]}
\LeftLabel{sub-typ}
\BinaryInfC{El$_{N1}$(t$\backprime$, d$\backprime$) $\in$ M(t$\backprime$)[$\Gamma$]}
\end{prooftree}
Assumo che siano valide per costruzione le premesse di \textit{E-N$_{1prog}$} (come dimostro nell'esercizio 2) e di \textit{E-S}.

\paragraph{3} \textbf{Sia \textit{T$_1$} la teoria dei tipi definita del tipo singoletto con le regole strutturali, definite in questo capitolo, incluse quelle di sostituzione e indebolimento. Allora stabilire se i seguenti termini sono tipabili come termini del tipo singoletto, secondo \textit{T$_1$} e quali sono uguali definizionalmente.}
\begin{itemize}
\item El$_{N1}$($\ast$, $\ast$)
\item El$_{N1}$(x, $\ast$)
\item El$_{N1}$($\ast$, y)
\item El$_{N1}$(x, y)
\item El$_{N1}$(El$_{N1}$($\ast$, y), El$_{N1}$(x, $\ast$))
\end{itemize}
\noindent
\\
\textbf{Soluzione}\\\\
\textit{Per una maggiore comprensione delle derivazioni, ho ritenuto opportuno, ove necessario, spezzare l'albero in pi\`u parti.}\\
\noindent
\textbf{1}\\
\small
\begin{prooftree}
\AxiomC{[ ] cont}
\LeftLabel{I-S}
\UnaryInfC{$\ast$ $\in$ N$1$[ ]}
\AxiomC{[ ] cont}
\LeftLabel{F-S}
\UnaryInfC{N$_1$ type[ ]}
\LeftLabel{F-c}\RightLabel{(z $\in$ N$_1$) $\notin$ [ ]}
\UnaryInfC{[z $\in$ N$_1$] cont}
\LeftLabel{I-S}
\UnaryInfC{N$_1$ type[z $\in$ N$_1$]}
\AxiomC{[ ] cont}
\LeftLabel{I-S}
\UnaryInfC{$\ast$ $\in$ N$_1$[ ]}
\LeftLabel{E-S}
\TrinaryInfC{El$_{N1}$($\ast$, $\ast$) $\in$ $N_1$[ ]}
\end{prooftree}
\normalsize
Applicando la \textit{$\beta$ N$_1$red} allora El$_{N1}$($\ast$, $\ast$) $\rightarrow_1$ $\ast$ \\
El$_{N1}$($\ast$, $\ast$) \`e uguale definizionalmente.
\\\\\\
\noindent
\textbf{2}\\
\scriptsize
\begin{prooftree}
\AxiomC{[ ] cont}
\LeftLabel{F-S}
\UnaryInfC{N$_1$ type[ ]}
\LeftLabel{F-c}\RightLabel{(x $\in$ N$_1$) $\notin$ [ ]}
\UnaryInfC{x $\in$ N$_1$ cont}
\LeftLabel{I-S}
\UnaryInfC{$\ast$ $\in$ N$_1$[x $\in$ N$_1$]}
\AxiomC{[ ] cont}
\LeftLabel{F-S}
\UnaryInfC{N$_1$ type [ ]}
\LeftLabel{F-c}\RightLabel{(x $\in$ N$_1$) $\notin$ [ ]}
\UnaryInfC{x $\in$ N$_1$ cont}
\LeftLabel{F-S}
\UnaryInfC{N$_1$ type[x $\in$ N$_1$]}
\LeftLabel{F-c}\RightLabel{(z $\in$ N$_1$) $\notin$ (x $\in$ N$_1$)}
\UnaryInfC{x $\in$ N$_1$, z $\in$ N$_1$ cont}
\LeftLabel{F-S}
\UnaryInfC{N$_1$ type[x $\in$ N$_1$, z $\in$ N$_1$]}
\AxiomC{[ ] cont}
\LeftLabel{F-S}
\UnaryInfC{N$_1$ type[ ]}
\LeftLabel{F-c}\RightLabel{(x $\in$ N$_1$) $\notin$ [ ]}
\UnaryInfC{x $\in$ N$_1$ cont}
\LeftLabel{var}
\UnaryInfC{x $\in$ N$_1$[x $\in$ N$_1$]}
\LeftLabel{E-S}
\TrinaryInfC{El$_{N1}$(x, $\ast$) $\in$ N$_1$[x $\in$ N$_1$]}
\end{prooftree}

\noindent
\normalsize
Applicando la \textit{$\beta$ N$_1$red} allora El$_{N1}$(x, $\ast$) $\nrightarrow_1$ \\
El$_{N1}$(x, $\ast$) non \`e uguale definizionalmente.
\\\\\\
\noindent
\textbf{3}\\
\scriptsize
\begin{prooftree}
\AxiomC{[ ] cont}
\LeftLabel{F-S}
\UnaryInfC{N$_1$ type[ ]}
\LeftLabel{F-c}\RightLabel{(y $\in$ N$_1$) $\notin$ [ ]}
\UnaryInfC{y $\in$ N$_1$ cont}
\LeftLabel{I-S}
\UnaryInfC{$\ast$ $\in$ N$_1$[y $\in$ N$_1$]}
\AxiomC{[ ] cont}
\LeftLabel{F-S}
\UnaryInfC{N$_1$ type[ ]}
\LeftLabel{F-c}\RightLabel{(y $\in$ N$_1$) $\notin$ [ ]}
\UnaryInfC{y $\in$ N$_1$  cont}
\LeftLabel{F-S}
\UnaryInfC{N$_1$ type[y $\in$ N$_1$]}
\LeftLabel{F-c}\RightLabel{(z $\in$ N$_1$) $\notin$ y $\in$ N$_1$} 
\UnaryInfC{y $\in$ N$_1$, z $\in$ N$_1$ cont}
\LeftLabel{F-S}
\UnaryInfC{N$_1$ type[y $\in$ N$_1$, z $\in$ N$_1$]}
\AxiomC{[ ] cont}
\LeftLabel{F-S}
\UnaryInfC{N$_1$ type[ ]}
\LeftLabel{F-c}\RightLabel{(y $\in$ N$_1$) $\notin$ [ ]}
\UnaryInfC{y $\in$ N$_1$ cont}
\LeftLabel{var}
\UnaryInfC{y $\in$ N$_1$[y $\in$ N$_1$]}
\LeftLabel{E-S}
\TrinaryInfC{El$_{N1}$($\ast$, y) $\in$ N$_1$[y $\in$ N$_1$]}
\end{prooftree}

\noindent
\normalsize
Applicando la \textit{$\beta$ N$_1$red} allora El$_{N1}$($\ast$, y) $\rightarrow_1$ y \\
El$_{N1}$($\ast$, y) \`e uguale definizionalmente.\\\\
\noindent
\textbf{4}\\\\
\scriptsize
\begin{adjustwidth}{-17em}{}
\begin{prooftree}
\AxiomC{[ ] cont}
\LeftLabel{F-S}
\UnaryInfC{N$_1$ type [ ]}
\LeftLabel{F-c}\RightLabel{(x $\in$ N$_1$) $\notin$ [ ]}
\UnaryInfC{x $\in$ N$_1$ cont}
\LeftLabel{F-S}
\UnaryInfC{N$_1$ type[x $\in$ N$_1$]}
\LeftLabel{F-c}\RightLabel{(y $\in$ N$_1$) $\notin$ x $\in$ N$_1$}
\UnaryInfC{x $\in$ N$_1$, y $\in$ N$_1$ cont}
\LeftLabel{var}
\UnaryInfC{x $\in$ N$_1$[x $\in$ N$_1$, y $\in$ N$_1$]}
\AxiomC{[ ] cont}
\LeftLabel{F-S}
\UnaryInfC{N$_1$ type[ ]}
\LeftLabel{F-c}\RightLabel{(x $\in$ N$_1$) $\notin$ [ ]}
\UnaryInfC{x $\in$ N$_1$ cont}
\LeftLabel{F-S}
\UnaryInfC{N$_1$ type[x $\in$ N$_1$]}
\LeftLabel{F-c}\RightLabel{(y $\in$ N$_1$) $\notin$ (x $\in$ N$_1$)}
\UnaryInfC{x $\in$ N$_1$, y $\in$ N$_1$ cont}
\LeftLabel{F-S}
\UnaryInfC{N$_1$ type[x $\in$ N$_1$, y $\in$ N$_1$]}
\LeftLabel{F-c}\RightLabel{\begin{tabular}[c]{cc}(z $\in$ N$_1$) $\notin$ (x $\in$  \\N$_1$, y $\in$ N$_1$)\end{tabular}}
\UnaryInfC{x $\in$ N$_1$, y $\in$ N$_1$, z $\in$ N$_1$ cont}
\LeftLabel{F-S}
\UnaryInfC{N$_1$ type[x $\in$ N$_1$, y $\in$ N$_1$, z $\in$ N$_1$]}
\AxiomC{[ ] cont}
\LeftLabel{F-S}
\UnaryInfC{N$_1$ type [ ]}
\LeftLabel{F-c}\RightLabel{(x $\in$ N$_1$) $\notin$ [ ]}
\UnaryInfC{x $\in$ N$_1$ cont}
\LeftLabel{F-S}
\UnaryInfC{N$_1$ type[x $\in$ N$_1$]}
\LeftLabel{F-c}\RightLabel{(y $\in$ N$_1$) $\notin$ x $\in$ N$_1$}
\UnaryInfC{x $\in$ N$_1$, y $\in$ N$_1$ cont}
\LeftLabel{var}
\UnaryInfC{y $\in$ N$_1$[x $\in$ N$_1$, y $\in$ N$_1$]}
\LeftLabel{E-S}
\TrinaryInfC{El$_{N1}$(x, y) $\in$ N$_1$[x $\in$ N$_1$, y $\in$ N$_1$]}
\end{prooftree}
\end{adjustwidth}
\noindent
\\
\normalsize
Applicando la \textit{$\beta$ N$_1$red} allora El$_{N1}$(x, y)$\nrightarrow_1$ \\
El$_{N1}$($\ast$, y) non \`e uguale definizionalmente.
\\\\\\
\noindent
\textbf{5}\\
\small
\begin{adjustwidth}{-7em}{}
\begin{prooftree}
\AxiomC{\textbf{5$_A$}}
\UnaryInfC{El$_{N1}$($\ast$, y) $\in$ N$_1$[y $\in$ N$_1$, x $\in$ N$_1$]}
\AxiomC{\textbf{5$_B$}}
\UnaryInfC{N$_1$ type[y $\in$ N$_1$, x $\in$ N$_1$, z $\in$ N$_1$]}
\AxiomC{\textbf{5$_C$}}
\UnaryInfC{El$_{N1}$(x, $\ast$) $\in$ N$_1$[y $\in$ N$_1$, x $\in$ N$_1$]}
\LeftLabel{E-S}
\TrinaryInfC{El$_{N1}$(El$_{N1}$($\ast$, y), El$_{N1}$(x, $\ast$)) $\in $ N$_1$[y $\in$ N$_1$, x $\in$ N$_1$]}
\end{prooftree}
\end{adjustwidth}

\vspace{0.5cm}
\textbf{5$_A$}}
\small
\begin{prooftree}
\AxiomC{}
\UnaryInfC{El$_{N1}$($\ast$, y) $\in$ N$_1$[y $\in$ N$_1$]}
\AxiomC{[ ] cont}
\LeftLabel{F-S}
\UnaryInfC{N$_1$ type[ ]}
\LeftLabel{F-c}\RightLabel{(y $\in$ N$_1$) $\notin$ [ ]}
\UnaryInfC{y $\in$ N$_1$ cont}
\LeftLabel{F-S}
\UnaryInfC{N$_1$ type[y $\in$ N$_1$]}
\LeftLabel{F-c}\RightLabel{(x $\in$ N$_1$) $\notin$ (y $\in$ N$_1$)}
\UnaryInfC{y $\in$ N$_1$, x $\in$ N$_1$ cont}
\LeftLabel{ind-te}
\BinaryInfC{El$_{N1}$($\ast$, y) $\in$ N$_1$[y $\in$ N$_1$, x $\in$ N$_1$]}
\end{prooftree}

\vspace{0.5cm}
\textbf{5$_B$}
\small
\begin{prooftree}
\AxiomC{[ ] cont}
\LeftLabel{F-S}
\UnaryInfC{N$_1$ type[ ]}
\LeftLabel{F-c}\RightLabel{(y $\in$ N$_1$) $\notin$ [ ]}
\UnaryInfC{y $\in$ N$_1$ cont}
\LeftLabel{F-S}
\UnaryInfC{N$_1$ type[y $\in$ N$_1$]}
\LeftLabel{F-c}\RightLabel{(x $\in$ N$_1$) $\notin$ (y $\in$ N$_1$)}
\UnaryInfC{y $\in$ N$_1$, x $\in$ N$_1$ cont}
\LeftLabel{F-S}
\UnaryInfC{N$_1$ type[y $\in$ N$_1$, x $\in$ N$_1$]}
\LeftLabel{F-c}\RightLabel{(z $\in$ N$_1$) $\notin$ (y $\in$ N$_1$, x $\in$ N$_1$)}
\UnaryInfC{y $\in$ N$_1$, x $\in$ N$_1$, z $\in$ N$_1$ cont}
\LeftLabel{F-S}
\UnaryInfC{N$_1$ type[y $\in$ N$_1$, x $\in$ N$_1$, z $\in$ N$_1$]}
\end{prooftree}

\newpage
\textbf{5$_C$}
\small
\begin{adjustwidth}{-11em}{}
\begin{prooftree}
\AxiomC{}
\UnaryInfC{El$_{N1}$(x, $\ast$) $\in$ N$_1$[x $\in$ N$_1$]}
\AxiomC{[ ] cont}
\LeftLabel{F-S}
\UnaryInfC{N$_1$ type [ ]}
\LeftLabel{F-c}\RightLabel{(x $\in$ N$_1$) $\notin$ [ ]}
\UnaryInfC{x $\in$ N$_1$ cont}
\LeftLabel{F-S}
\UnaryInfC{N$_1$ type[x $\in$ N$_1$]}
\LeftLabel{F-c}\RightLabel{(y $\in$ N$_1$) $\notin$ x $\in$ N$_1$}
\UnaryInfC{x $\in$ N$_1$, y $\in$ N$_1$ cont}
\LeftLabel{ind-te}
\BinaryInfC{El$_{N1}$(x, $\ast$) $\in$ N$_1$[x $\in$ N$_1$, y $\in$ N$_1$]}
\AxiomC{[ ] cont}
\LeftLabel{F-S}
\UnaryInfC{N$_1$ type [ ]}
\LeftLabel{F-c}\RightLabel{(y $\in$ N$_1$) $\notin$ y $\in$ [ ]}
\UnaryInfC{y $\in$ N$_1$ count}
\LeftLabel{F-S}
\UnaryInfC{N$_1$ type[y $\in$ N$_1$]}
\LeftLabel{F-c}\RightLabel{(x $\in$ N$_1$) $\notin$ y $\in$ N$_1$}
\UnaryInfC{y $\in$ N$_1$, x $\in$ N$_1$ cont}
\LeftLabel{ex-te}
\BinaryInfC{El$_{N1}$(x, $\ast$) $\in$ N$_1$[y $\in$ N$_1$, x $\in$ N$_1$]}
\end{prooftree}
\end{adjustwidth}
\noindent
\\
\normalsize Per i giudizi conclusione El$_{N1}$($\ast$, y) $\in$ N$_1$[y $\in$ N$_1$] e El$_{N1}$(x, $\ast$) $\in$ N$_1$[x $\in$ N$_1$] ho gi\`a dimostrato sopra (in 3 e 2) la loro tipabilit\`a per il tipo singoletto.

\noindent
\normalsize
Applicando la \textit{red$_I$} e \textit{$\beta$ N$_1$red} allora El$_{N1}$(El$_{N1}$($\ast$, y), El$_{N1}$(x, $\ast$)) $\rightarrow_1$ El$_{N1}$(y, El$_{N1}$(x, $\ast$)). \\
Pi\`u nel dettaglio la riduzione \`e la seguente
\small
\begin{prooftree}
\AxiomC{}
\LeftLabel{$\beta$ N$_1$red}
\UnaryInfC{El$_{N1}$($\ast$, y) $\rightarrow_1$ y}
\LeftLabel{red$_I$}
\UnaryInfC{El$_{N1}$(El$_{N1}$($\ast$, y), El$_{N1}$(x, $\ast$))$\rightarrow_1$ El$_{N1}$(y, El$_{N1}$(x, $\ast$))}
\end{prooftree}
\normalsize El$_{N1}$(El$_{N1}$($\ast$, y), El$_{N1}$(x, $\ast$)) \`e uguale definizionalmente.

\paragraph{4} \textbf{Dati i termini definiti dalla seguente grammatica relativa ai termini del tipo singoletto
\begin{center} t $\equiv$ v $|$ $\ast$ $|$ El$_{N1}$(t$_1$, t$_2$)\end{center}
con v $\in$ \{x,y,w,z\} $\bigcup$ \{x$_i$ $|$ i $\in$ Nat\}, ovvero considerando come variabili le ultime lettere dell'alfabeto inglese e poi tutte le variabili ottenute ponendo alla variabili x un indice che varia nei numeri naturali.\\
Sia $\rightarrow_1$ una relazione binaria tra questi termini \textit{untyped} definita a partire dalle seguenti regole
\begin{center} $\beta_{N1})$ red \qquad El$_{Nat}$(0,c) $\rightarrow_1$ c\end{center} 
}
\textbf{
\AxiomC{t$_1$ $\rightarrow$ t$_2$}
\LeftLabel{red$_I$)}
\UnaryInfC{El$_{N1}$(t$_1$, c) $\rightarrow$ $_1$ El$_{N1}$(t$_2$, c)}
\DisplayProof \qquad
\AxiomC{c$_1$ $\rightarrow$ c$_2$}
\LeftLabel{red$_{II}$)}
\UnaryInfC{El$_{N1}$(t, c$_1$) $\rightarrow$ $_1$ El$_{N1}$(t, c$_2$)}
\DisplayProof
\noindent
\begin{itemize}
\item Costruire l'albero dei cammini (ovvero sequenze) di passi di riduzione possibili fino a un termine in forma normale, ovvero non ulteriormente riducibile rispetto alla relazione $\rightarrow_1$ del termine
\begin{center}El${N_1}$(El$_{N1}$($\ast$, $\ast$), El$_{N1}$($\ast$, x)) \end{center}
\item Produrre un infinit\`a di termini del tipo singoletto che non sono riducibili secondo la relazione di un passo di riduzione$\rightarrow_1$.\\
Dati due di questi termini , si riesce a dire che sono definionalmente uguali secondo le regole del tipo singoletto?
\end{itemize}
}  


\textbf{Soluzione}\\\\
\textit{Idea: uso un albero di derivazione per mostrare ogni passo derivazione di ogni cammino.\\
Se w = El$_{N1}$(El$_{N1}$($\ast$, $\ast$), El$_{N1}$($\ast$, x)) combino il lambda termine w  con l'applicazione della strategia deterministica di riduzione ($\rightarrow_1$), con la quale il termine si riduce eventualmente a forma normale (implicando la definizione di riducibilit\`a).\\
$\beta$-red:\\
\textit{El$_{N1}$($\ast$, $\ast$) $\rightarrow_1$ $\ast$}\\
\textit{El$_{N1}$($\ast$, x) $\rightarrow_1$ x}
}

\begin{prooftree}
\AxiomC{x}
\LeftLabel{$\beta$-N$_1$ red}
\UnaryInfC{El$_{N1}$($\ast$, x)}
\AxiomC{x}
\RightLabel{$\beta$-N$_1$ red}
\UnaryInfC{El$_{N1}$($\ast$, x)}
\LeftLabel{$\beta$-N$_1$ red}\RightLabel{red$_{II}$}
\BinaryInfC{El$_{NI}$($\ast$, El$_{N1}$($\ast$, x))}
\AxiomC{x}
\RightLabel{$\beta$-N$_1$ red}
\UnaryInfC{El$_{N1}$($\ast$, x)}
\RightLabel{red$_I$}
\UnaryInfC{El$_{N1}$(El$_{N1}$($\ast$,$\ast$), x)}
\LeftLabel{red$_{I}$}\RightLabel{red$_{II}$}
\BinaryInfC{El$_{N1}$(El$_{N1}$($\ast$, $\ast$), El$_{N1}$($\ast$, x))}
\end{prooftree}
\noindent
 $\Rightarrow$ (w, (red$_II$, red$_I$, $\beta_{N1red}$)) rappresenta un programma.\\\\
\noindent
Termine t non pi\`u riducibile significa che \`e un termine \textit{untyped} che \`e in forma normale perch\`e non esiste alcun altro termine s tale che t $\rightarrow_1$ s. Dunque l'infinit\`a di termini singoletto, non pi\`u riducibili rispetterano la definizione data sopra\\
\begin{center}
t $\equiv$
$
\begin{cases}
v \\
\ast \\
El_{N1}(t_1, t_2)
\end{cases}
$
\end{center}
\noindent Dati due termini t$^I$ e t$^{II}$ termini \textit{untyped} non riesco a dire che sono definizionalmente uguali perch\`e gi\`a e in forma normale. Difatti  per il teorema della forma normale forte vale $\rightarrow_0$.






















\newpage
\chapter{Naturali, Somma disgiunta e Liste}
\label{naturali-somma-disgiunta-liste}
%%dalla lezione 12 alle lezione 15 inclusa
\newpage
%\chapter{Uguaglianza proposizionale}
%% lezione 16 e 17 inclusa

\newpage
%
\chapter{Tipo della somma disgiunta}
\label{cap: tipo-somma-disgiunta}
Il tipo somma disgiunta \`e un costruttore di tipo. Questi non \`e dipendente se da solo, lo diventa solo quando agisce su un tipo dipendente.\\
Anche con il tipo somma disgiunta si parla di tipo induttivo (accade gi\`a per N$_1$, Nat, List(A)).\\
Le regole del tipo della somma disgiunta sono le seguenti.

\section{Regole di Formazione}
\label{sec: formazione-disjointsum}
\begin{prooftree}
\AxiomC{B type [$\Gamma$]}
\AxiomC{C type [$\Gamma$]}
\LeftLabel{F-$+$)}
\BinaryInfC{B + C type [$\Gamma$]}
\end{prooftree}

\section{Regole di Introduzione}
\label{sec: introduzione-disjointsum}
\begin{center}
\AxiomC{b $\in$ B[$\Gamma$]}
\AxiomC{B + C type[$\Gamma$]}
\LeftLabel{I$_1$-$+$)}
\BinaryInfC{inl(b) $\in$ B + C[$\Gamma$]}
\DisplayProof \qquad
\AxiomC{c $\in$ C[$\Gamma$]}
\AxiomC{B + C type[$\Gamma$]}
\LeftLabel{I$_2$-$+$)}
\BinaryInfC{inr(c) $\in$ B + C[$\Gamma$]}
\DisplayProof
\end{center}

\section{Regole di Eliminazione}
\label{sec: eliminazione-disjointsum}
\small
\begin{adjustwidth}{-5em}{}
\begin{prooftree}
\AxiomC{\begin{tabular}[c]{cc}M(z) type[$\Gamma$, z $\in$ B + C]\\t $\in$ B + C[$\Gamma$]\end{tabular}}
\AxiomC{e$_B$(x$_1$) $\in$ M(inl(x$_1$))[$\Gamma$, x$_1$ $\in$ B]}
\AxiomC{e$_C$(x$_2$) $\in$ M(inr(x$_2$))[$\Gamma$, x$_2$ $\in$ C]}
\LeftLabel{E-$+$)}
\TrinaryInfC{El$_+$(t,e$_B$,e$_C$) $\in$ M(t)[$\Gamma$]}
\end{prooftree}
\end{adjustwidth}

\section{Regole di Conservazione}
\label{sec: conservazione-disjointsum}
\small
\begin{adjustwidth}{-5em}{}
\begin{prooftree}
\AxiomC{\begin{tabular}[c]{cc}M(z) type[$\Gamma$, z $\in$ B + C]\\b $\in$ B[$\Gamma$]\end{tabular}}
\AxiomC{e$_B$(x$_1$) $\in$ M(inl(x$_1$))[$\Gamma$, x$_1$ $\in$ B]}
\AxiomC{e$_C$(x$_2$) $\in$ M(inr(x$_2$))[$\Gamma$, x$_2$ $\in$ C]}
\LeftLabel{C$_1$-$+$)}
\TrinaryInfC{El$_+$(inl(b),e$_B$,e$_C$) = e$_B$(b) $\in$ M(inl(b))[$\Gamma$]}
\end{prooftree}
\end{adjustwidth}
\small
\begin{adjustwidth}{-5em}{}
\begin{prooftree}
\AxiomC{\begin{tabular}[c]{cc}M(z) type[$\Gamma$, z $\in$ B + C]\\c $\in$ C[$\Gamma$]\end{tabular}}
\AxiomC{e$_B$(x$_1$) $\in$ M(inl(x$_1$))[$\Gamma$, x$_1$ $\in$ B]}
\AxiomC{e$_C$(x$_2$) $\in$ M(inr(x$_2$))[$\Gamma$, x$_2$ $\in$ C]}
\LeftLabel{C$_1$-$+$)}
\TrinaryInfC{El$_+$(inr(c),e$_B$,e$_C$) $=$ e$_C$(c) $\in$ M(inr(c))[$\Gamma$]}
\end{prooftree}
\end{adjustwidth}


\section{Regole di Uguaglianza}
\label{sec: uguaglianza-disjointsum}
\normalsize
\begin{prooftree}
\AxiomC{B$_1$ $=$ B$_2$ $\in$ type[$\Gamma$]}
\AxiomC{C$_1$ $=$ C$_2$ $\in$ type[$\Gamma$]}
\LeftLabel{eq-F-$+$)}
\BinaryInfC{B$_1$ + C$_1$ $=$ B$_2$ + C$_2$ type($\Gamma$)}
\end{prooftree}

\section{Eliminatore dipendente}
\label{sec:eliminatore dipendente-disjointsum}
L'eliminatore ha, anche nel caso della somma disgiunta, la forma dipendente.
\small
\begin{adjustwidth}{-5em}{}
\begin{prooftree}
\AxiomC{M(z) type[$\Gamma$, z $\in$ B + C]}
\AxiomC{e$_B$(x$_1$) $\in$ M(inl(x$_1$))[$\Gamma$, x$_1$ $\in$ B]}
\AxiomC{e$_C$(x$_2$) $\in$ M(inr(x$_2$))[$\Gamma$, x$_2$ $\in$ C]}
\LeftLabel{E$_{dip}$-$+$)}
\TrinaryInfC{El$_+$(z,e$_B$,e$_C$) $\in$ M(z)[$\Gamma$,z $\in$ B + C]}
\end{prooftree}
\end{adjustwidth}
\noindent
\normalsize
Con questi non si definisce solo che si ricorre su B $+$ C, ma si da anche un principio di induzione.

\section{Semantica operazionale della somma disgiunta}
\label{sec: semantica-operazionale-somma-disgiunta}
La relazione $\rightarrow_1$ viene definita all'interno dei termini con l'uso delle seguenti regole di riduzione:
\begin{itemize}
\item $\beta_{1+}$-red) El$_+$(inl(b), e$_B$, e$_C$) $\rightarrow_1$ e$_B$(b)
\item $\beta_{2+}$-red) El$_+$(inl(b), e$_B$, e$_C$) $\rightarrow_1$ e$_C$(c)
\item \AxiomC{t$_1$ $\rightarrow_1$ t$_2$}
\LeftLabel{$+$-red)}
\UnaryInfC{El$_+$(t$_1$, e$_B$, e$_C$) $\rightarrow_1$ El$_+$(t$_2$, e$_B$, e$_C$)}
\DisplayProof \qquad
\item Novit\`a della somma disgiunta rispetto al tipo singoletto
\AxiomC{b$_1$ $\rightarrow_1$ b$_2$}
\LeftLabel{$+$-red$_I$)}
\UnaryInfC{inl(b$_1$) $\rightarrow_1$ inl(b$_2$)}
\DisplayProof
\item \AxiomC{c$_1$ $\rightarrow_1$ c$_2$}
\LeftLabel{$+$-red$_{II}$)}
\UnaryInfC{inr(c$_1$) $\rightarrow_1$ inr(c$_2$)}
\DisplayProof
\end{itemize}

\section{Osservazioni dal punto di vista logico}
\label{sec:osservazioni-dal-punto-di-vista-logico-disjointsum} 
Le regole di \textit{Eliminazione/Introduzione} partono dalle regole di congiunzione nella logica (nel caso in cui sia B che C siano proposizioni), inoltre nel caso dell'eliminazione si va a eliminare verso altre proposizioni.\\Tale concetto \`e stato voluto da \textit{Martin-L$\ddot{o}$f}, per riuscire a interpretare la logica con la teoria dei tipi.
\subsection{La regola di Formazione}
\label{subsec:la-regola-di-formazione-osservazioni-disjointsum}
Se si hanno le preposizioni $\beta$ prop[$\Gamma$] e $\gamma$ prop[$\Gamma$] e si interpreta la somma come disgiunzione, allora la regola di \textit{Formazione} diventa
\begin{prooftree}
\AxiomC{$\beta$ prop [$\Gamma$]}
\AxiomC{$\gamma$ prop [$\Gamma$]}
\LeftLabel{F-$+$)}
\BinaryInfC{B $\vee$ C prop [$\Gamma$]}
\end{prooftree}
Questi ha influenza sia sulla regola \textit{(E-$_{dip}+$)} che  \textit{(I-$+$)}. 

\subsubsection{La regola di Eliminazione}
\begin{prooftree}
\AxiomC{$\xi$ prop[$\Gamma$]}
\AxiomC{$\xi$ \`e vero[$\Gamma$,supponiamo $\beta$ vero]}
\AxiomC{$\xi$ \`e vero[$\Gamma$,$\gamma$ \`e vero]}
\LeftLabel{E$_{dip}$-$+$)}
\TrinaryInfC{\cancel{El$_+$(z,e$_B$,e$_C$)} $\in$ $\xi$ vero[$\Gamma$,z $\in$ $\beta$ $\vee$ $\gamma$ vero]}
\end{prooftree}
che, nel calcolo dei sequenti, equivale alla or a sinistra
\begin{prooftree}
\AxiomC{$\beta$ $\vdash_\Gamma$ $\xi$}
\AxiomC{$\gamma$ $\vdash_\Gamma$ $\xi$}
\LeftLabel{$\vee$-S)}
\BinaryInfC{$\beta$ $\vee$ $\gamma$ $\vdash_\Gamma$ $\xi$}
\end{prooftree}
\noindent
Allo stesso modo le regole di Introduzione si possono vedere in modo molto semplice.

\subsubsection{Le regole di Introduzione}
\begin{prooftree}
\AxiomC{b $\in$ $\beta$[$\Gamma$]}
\AxiomC{$\beta$ + $\gamma$ prop[$\Gamma$]}
\LeftLabel{I$_1$-$+$)}
\BinaryInfC{inl(b) $\in$ $\beta$ $\vee$ $\gamma$}
\end{prooftree}
b $\in$ $\beta$ $\equiv$ $\beta$ vero\\
$\beta$ $\vee$ $\gamma$ $\equiv$ $\beta$ $\vee$ $\gamma$ \`e vero\\
Questa, nel calcolo dei sequenti, equivale alla or a destra\\
$\Delta$ contesto, allora
\begin{prooftree}
\AxiomC{$\Delta$ $\vdash_\Gamma$ $\beta$}
\LeftLabel{$\vee$-D)}
\UnaryInfC{$\Delta$ $\vdash_\Gamma$ $\beta$ $\vee$ $\gamma$}
\end{prooftree}
\noindent
Lo stesso si pu\`o fare sulla seconda regola di Introduzione.
\begin{prooftree}
\AxiomC{c $\in$ $\gamma$[$\Gamma$]}
\AxiomC{$\beta$ + $\gamma$ prop[$\Gamma$]}
\LeftLabel{I$_2$-$+$)}
\BinaryInfC{inr(c) $\in$ $\beta$ $\vee$ $\gamma$}
\end{prooftree}
c $\in$ $\gamma$ $\equiv$ $\gamma$ vero\\
$\beta$ $\vee$ $\gamma$ $\equiv$ $\beta$ $\vee$ $\gamma$ \`e vero\\
Questa, nel calcolo dei sequenti, equivale alla regola di riduzione naturale nella logica\\
$\Delta$ contesto, allora
\begin{prooftree}
\AxiomC{$\Delta$ $\vdash_\Gamma$ $\gamma$}
\LeftLabel{I$_2$-$+$)}
\UnaryInfC{$\Delta$ $\vdash_\Gamma$ $\beta$ $\vee$ $\gamma$}
\end{prooftree}
\noindent
\vspace{0.5cm}


\section{Esercizi}
\label{sec: es-somma-disgiunta}
\paragraph{1)}
\textbf{Si scrivano le regole del tipo booleano come tipo semplice e si provi che \`e rappresentabile come N$_1$ + N$_2$.}\\\\
\textbf{Soluzione}\\\\
Definisco il tipo Bool nel modo seguente
Bool = {true || false}\\
inl($\ast$) $\equiv$ true\\
inr($\ast$) $\equiv$ false\\
\begin{itemize}
\item \textbf{Regole del tipo Bool:}

\begin{itemize}
\item\textit{Regole di Formazione}
\begin{prooftree}
\AxiomC{$\Gamma$ cont}
\LeftLabel{F-Bool)}
\UnaryInfC{Bool type[$\Gamma$]}
\end{prooftree}

\item\textit{Regole di Introduzione}
\begin{center}
\AxiomC{$\Gamma$ cont}
\LeftLabel{I$_1$-Bool)}
\UnaryInfC{true $\in$ Bool[$\Gamma$]}
\DisplayProof \qquad
\AxiomC{$\Gamma$ cont}
\LeftLabel{I$_2$-Bool)}
\UnaryInfC{false $\in$ Bool[$\Gamma$]}
\DisplayProof
\end{center}

\item\textit{Regole di Eliminazione}
\small
\begin{prooftree}
\AxiomC{M(z) type[$\Gamma$, z $\in$ Bool]}
\AxiomC{t $\in$ Bool[$\Gamma$]}
\AxiomC{\begin{tabular}[c]{cc}e$_B$ $\in$ M(true)[$\Gamma$] \\ e$_C$ $\in$ M(false)[$\Gamma$]\end{tabular}}
\LeftLabel{E-Bool)}
\TrinaryInfC{El$_{Bool}$(t,e$_B$,e$_C$) $\in$ M(t)[$\Gamma$]}
\end{prooftree}

\item\normalsize\textit{Regole di Conversione}
\small
\begin{prooftree}
\AxiomC{M(z) type[$\Gamma$, z $\in$ Bool]}
\AxiomC{e$_B$ $\in$ M(true)[$\Gamma$]}
\AxiomC{e$_C$ $\in$ M(false)[$\Gamma$]}
\LeftLabel{C$_1$-Bool)}
\TrinaryInfC{El$_{Bool}$(true,e$_B$,e$_C$) = e$_C$ $\in$ M(true)[$\Gamma$]}
\end{prooftree}
\begin{prooftree}
\AxiomC{M(z) type[$\Gamma$, z $\in$ Bool]}
\AxiomC{e$_B$ $\in$ M(true)[$\Gamma$]}
\AxiomC{e$_C$ $\in$ M(false)[$\Gamma$]}
\LeftLabel{C$_1$-Bool)}
\TrinaryInfC{El$_{Bool}$(false,e$_B$,e$_C$) = e$_C$ $\in$ M(false)[$\Gamma$]}
\end{prooftree}

\item\normalsize\textit{Regole di Uguaglianza}
\small
\begin{center}
\AxiomC{M(z) type [$\Gamma$, z $\in$ Bool]}
\AxiomC{\begin{tabular}[c]{cc}e$_B$ = s $\in$ M(true)[$\Gamma$]\\e$_C$ = t $\in$ M(false)[$\Gamma$]\end{tabular}}
\LeftLabel{eq$_1$-E-Bool)}
\BinaryInfC{El$_{Bool}$(true,e$_B$,e$_C$) = El$_{Bool}$(true,s,t) $\in$ M(true)[$\Gamma$]}
\DisplayProof \\
\vspace{0.3cm}
\AxiomC{M(z) type [$\Gamma$, z $\in$ Bool]}
\AxiomC{\begin{tabular}[c]{cc}e$_B$ = s $\in$ M(true)[$\Gamma$] \\ e$_C$ = t $\in$ M(false)[$\Gamma$]\end{tabular}}
\LeftLabel{eq$_2$-E-Bool)}
\BinaryInfC{El$_{Bool}$(false,e$_B$,e$_C$) = El$_{Bool}$(false,s,t) $\in$ M(false)[$\Gamma$]}
\DisplayProof  \\
\vspace{0.3cm}
\AxiomC{t = t$\backprime$ $\in$ Bool[$\Gamma$]]}
\AxiomC{e$_B$ $\in$ M(true)[$\Gamma$]}
\AxiomC{e$_C$ $\in$ M(false)[$\Gamma$]}
\LeftLabel{eq$_1$-Bool)}
\TrinaryInfC{M(t) = M(t$\backprime$) $\in$ M(true)[$\Gamma$]}
\DisplayProof \\
\vspace{0.3cm}
\AxiomC{t = t$\backprime$ $\in$ Bool[$\Gamma$]]}
\AxiomC{e$_B$ $\in$ M(true)[$\Gamma$]}
\AxiomC{e$_C$ $\in$ M(false)[$\Gamma$]}
\LeftLabel{eq$_2$-Bool)}
\TrinaryInfC{M(t) = M(t$\backprime$) $\in$ M(false)[$\Gamma$]}
\DisplayProof
\end{center}
\end{itemize}

\item \textbf{Semantica operazionale del tipo Bool:}
\begin{itemize}
\item $\beta_{1Bool}$-red) El$_{list}$(true, e$_B$, e$_C$) $\rightarrow_1$ e$_B$(x)
\item $\beta_{2Bool}$-red) El$_{list}$(true, e$_B$, e$_C$) $\rightarrow_1$ e$_C$(x)
\item IF-true) if true then M else N $\rightarrow_1$ M
\item IF-false) if false then M else N $\rightarrow_1$ N
\item \AxiomC{t$_1$ $\rightarrow_1$ t$_2$}
\LeftLabel{Bool-red)}
\UnaryInfC{El$_{list}$(t$_1$, e$_B$, e$_C$) $\rightarrow_1$ El$_{list}$(t$_2$, e$_B$, e$_C$)}
\DisplayProof
\item \AxiomC{M$_1$ $\rightarrow_1$ M$_1^\backprime$}
\LeftLabel{IF)}
\UnaryInfC{if M$_1$ then M$_2$ else M$_3$ $\rightarrow_1$ if M$_1^\backprime$ then M$_2$ else M$_3$}
\DisplayProof
\end{itemize}

\end{itemize}




\newpage
%\chapter{Uguaglianza proposizionale}
\label{cap: uguaglianza-proposizionale}
%% lezione 16 e 17 inclusa
Pi\`u modi permettono di descrivere il tipo dell'uguaglianza, uno di questi \`e il tipo dell'uguaglianza proposizionale, definito dalle regole seguenti.

\section{Regole di Formazione}
\label{subsec: formazione-id}
\begin{prooftree}
\AxiomC{A type [$\Gamma$]}
\AxiomC{a $\in$ A[$\Gamma$]}
\AxiomC{b $\in$ A[$\Gamma$]}
\LeftLabel{F-Id)}
\TrinaryInfC{Id(A,a,b) type[$\Gamma$]}
\end{prooftree}

\section{Regole di Introduzione}
\label{subsec: introduzione-id}
\begin{prooftree}
\AxiomC{a $\in$ A[$\Gamma$]}
\LeftLabel{I$_1$-Id)}
\UnaryInfC{id(a) $\in$ Id(A,a,a)[$\Gamma$]}
\end{prooftree}

\section{Regole di Eliminazione}
\label{subsec: eliminazione-id}
\small
\begin{adjustwidth}{-7em}{}
\begin{prooftree}
\AxiomC{\begin{tabular}[c]{cccc}M(z$_1$,z$_2$,z$_3$) type[$\Gamma$, z$_1$ $\in$ A, z$_2$ $\in$ A, z$_3$ $\in$ Id(A, z$_1$, z$_2$)]\\a $\in$ A[$\Gamma$] \\b $\in$ A[$\Gamma$] \\ t $\in$ Id(A,a,b)[$\Gamma$]\end{tabular}}
\AxiomC{e(x) $\in$ M(x,x,Id(x)))[$\Gamma$, x $\in$ A]}
\LeftLabel{E-List)}
\BinaryInfC{El$_{Id}$(t, (x).e(x) $\in$ M(a,b,t)[$\Gamma$]}
\end{prooftree}
\end{adjustwidth}

\section{Regole di Conservazione}
\label{subsec: conservazione-liste}
\small
\begin{adjustwidth}{-5em}{}
\begin{prooftree}
\AxiomC{\begin{tabular}[c]{cc}M(z) type[$\Gamma$, z $\in$ List(A)] \\ c $\in$ M(nil)[$\Gamma$]\end{tabular}}
\AxiomC{e(x,w,y) $\in$ M(cons(x,w))[$\Gamma$, x $\in$ List(A),  w $\in$ A, y $\in$ M(x)]}
\LeftLabel{C$_1$-list)}
\BinaryInfC{El$_{list}$(nil,c,e) = c $\in$ M(nil)[$\Gamma$]}
\end{prooftree}
\end{adjustwidth}
\small
\begin{adjustwidth}{-7em}{}
\begin{prooftree}
\AxiomC{M(z) type[$\Gamma$, z $\in$ List(A)]}
\AxiomC{\begin{tabular}[c]{ccc} s $\in$ List(A)[$\Gamma$] \\ a $\in$ A[$\Gamma$] \\ c $\in$ M(nil)[$\Gamma$]\end{tabular}}
\AxiomC{e(x,w,y) $\in$ M(cons(x,w))[$\Gamma$, x $\in$ List(A), w $\in$ A, y $\in$ M(x)]}
\LeftLabel{C$_2$-list)}
\TrinaryInfC{El$_{list}$(cons(s,a),c,e) = e(s,a, E$_{list}$(s,c,e) $\in$ M(cons(s,a))[$\Gamma$]}
\end{prooftree}
\end{adjustwidth}


\section{Regole di Uguaglianza}
\label{subsec: uguaglianza-liste}
\normalsize
\AxiomC{A$_1$ = A$_2$ $\in$ type[$\Gamma$]}
\LeftLabel{eq-I$_1$-List)}
\UnaryInfC{List(A$_1$) = List(A$_2$) type($\Gamma$)}
\DisplayProof  \\
\AxiomC{s$_1$ = s$_2$ $\in$ List(A)[$\Gamma$]}
\AxiomC{a$_1$ = a$_2$ $\in$ A[$\Gamma$]}
\LeftLabel{eq-I$_2$-List)}
\BinaryInfC{cons(s$_1$,a$_1$) = cons(s$_2$,a$_2$) $\in$ List(A)($\Gamma$)}
\DisplayProof
\small
\begin{adjustwidth}{-10em}{}
\begin{prooftree}
\AxiomC{\begin{tabular}[c]{ccc} M(z) type[$\Gamma$, z $\in$ List(A)] \\ t$_1$ = t$_2$ $\in$ list(A)[$\Gamma$] \\ c$_1$ = c$_2$ $\in$ M(nil)[$\Gamma$]\end{tabular}}
\AxiomC{e$_1$(x,w,y) = e$_2$(x,w,y) $\in$ M(cons(x,w))[$\Gamma$, x $\in$ List(A), w $\in$ A, y $\in$ M(x)]}
\LeftLabel{E-eq-List)}
\BinaryInfC{El$_{list}$(t$_1$, c$_1$, e$_1$) = El$_{list}$(t$_2$, c$_2$, e$_2$ $\in$ M(t$_1$)[$\Gamma$]}
\end{prooftree}
\end{adjustwidth}
\normalsize

\newpage
%\chapter{Somma indiciata forte}
\label{cap: indexet-sum-type}
%17 inclusa
La somma indiciata forte \`e il potenzialmento indiciato della somma disgiunta binaria (\S\ref{cap: tipo-somma-disgiunta}). Ttipo induttivo, ovvero generato con il principio d'induzione della regola di eliminazine, di tipi dipendente.\\\\
\noindent
\textit{Definizione set-teorica}\\
\noindent
\textbf{
\begin{center}$\bigcup\limits_{x \epsilon B\hspace{0.1cm}set}^.$C(x) \qquad (C(x) set) x $\in$ B \end{center}
$\neq$ \quad $\bigcup$ \{y : $\exists$ x $\in$ B \quad y $\in$ C(x)\}\\
\begin{center}$\equiv$ \{(b,c) b $\in$ B e c $\in$ C(b)\}\end{center}
}
\`E l'unione disgiunta di una famiglia di insiemi, definito dalle regole seguenti.

\newpage
%\appendix
\chapter{Tipi principali e regole}

\section{Il tipo singoletto}
\subsection{Regola di Formazione}
\begin{prooftree}
\AxiomC{[$\Gamma$] cont}
\LeftLabel{F-S)}
\UnaryInfC{$N_1$ type[$\Gamma]$}
\end{prooftree}

\subsection{Regole di Introduzione}
\begin{prooftree}
\AxiomC{[$\Gamma$] cont}
\LeftLabel{I-S)}
\UnaryInfC{$\ast$ $\in$ $N_1$ [$\Gamma]$ cont}
\end{prooftree}

\subsection{Regole di Eliminazione}
\begin{prooftree}
\AxiomC{t $\in$ $N_1$ [$\Gamma$]}
\AxiomC{M(z) type[$\Gamma$, z $\in$ $N_1$]}
\AxiomC{c  $\in$ M($\ast$)[$\Gamma$]}
\LeftLabel{E-S)}
\TrinaryInfC{$El_{N1}$(t, c) $\in$ M($\ast$)[$\Gamma$]}
\end{prooftree}
\noindent
\subsection{Regole di Conversione}
\begin{prooftree}
\AxiomC{M(z) type[$\Gamma$, z $\in$ $N_1$]}
\AxiomC{c $\in$ M($\ast$)[$\Gamma$]}
\LeftLabel{C-S)}
\BinaryInfC{$El_{N1}$($\ast$, c) $=$ c $\in$ M($\ast$)[$\Gamma$]}
\end{prooftree}
\subsection{Eliminatore dipendente}
\begin{prooftree}
\AxiomC{M(z) type[$\Gamma$, z $\in$ $N_1$]}
\AxiomC{c  $\in$ M($\ast$)[$\Gamma$]}
\LeftLabel{E-S)$_{dip}$}
\BinaryInfC{$El_{N1}$(z, c) $\in$ M(z)[$\Gamma$, z $\in$ $N_1$]}
\end{prooftree}
\newpage

\section{Tipo dei numeri Naturali}

\subsection{Regole di Formazione}
\begin{prooftree}
\AxiomC{$\Gamma$ cont}
\LeftLabel{F-Nat)}
\UnaryInfC{Nat type[$\Gamma$]}
\end{prooftree}

\subsection{Regole di Introduzione}
\begin{center}
\AxiomC{$\Gamma$ cont}
\LeftLabel{I$_1$-Nat)}
\UnaryInfC{0 $\in$ Nat[$\Gamma$]}
\DisplayProof \qquad
\AxiomC{m $\in$ Nat[$\Gamma$]}
\LeftLabel{I$_2$-Nat)}
\UnaryInfC{succ(m) $\in$ Nat[$\Gamma$]}
\DisplayProof
\end{center}

\subsection{Regole di Eliminazione}
\small
\begin{adjustwidth}{-5em}{}
\begin{prooftree}
\AxiomC{t $\in$ Nat[$\Gamma$]}
\AxiomC{M(z) type[$\Gamma$, z $\in$ Nat]}
\AxiomC{\begin{tabular}[c]{cc}c $\in$ M(0)[$\Gamma$] \\ e(x,y) $\in$ M(succ(x))[$\Gamma$, x $\in$ Nat, y $\in$ M(x)]\end{tabular}}
\LeftLabel{E-Nat)}
\TrinaryInfC{El$_{Nat}$(t,c,e) $\in$ M(t)[$\Gamma$]}
\end{prooftree}
\end{adjustwidth}

\subsection{Regole di Conversione}
\small
\begin{adjustwidth}{-5em}{}
\begin{prooftree}
\AxiomC{M(z) type[$\Gamma$, z $\in$ Nat]}
\AxiomC{c $\in$ M(0)[$\Gamma$]}
\AxiomC{e(x,y) $\in$ M(succ(x))[$\Gamma$, x $\in$ Nat, y $\in$ M(x)]}
\LeftLabel{C$_1$-Nat)}
\TrinaryInfC{El$_{Nat}$(0,c,e) = c $\in$ M(0)[$\Gamma$]}
\end{prooftree}
\end{adjustwidth}

\begin{adjustwidth}{-5em}{}
\begin{prooftree}
\AxiomC{m  $\in$ Nat[$\Gamma$]}
\AxiomC{M(z) type[$\Gamma$, z $\in$ Nat]}
\AxiomC{\begin{tabular}[c]{cc}c $\in$ M(0)[$\Gamma$] \\ e(x,y) $\in$ M(succ(x))[$\Gamma$, x $\in$ Nat, y $\in$ M(x)]\end{tabular}}
\LeftLabel{C$_2$-Nat)}
\TrinaryInfC{El$_{Nat}$(succ(m),c,e) = e(m, El$_{Nat}$(m,c,e)) $\in$ M(succ(m))[$\Gamma$]}
\end{prooftree}
\end{adjustwidth}

\subsection{Regole di Uguaglianza}
\normalsize
\begin{center}
\begin{prooftree}
\AxiomC{t$_1$ $=$ t$_2$ $\in$ Nat[$\Gamma$]}
\LeftLabel{eq-Nat)}
\UnaryInfC{succ(t$_1$) $=$ succ(t$_2$) $\in$ Nat[$\Gamma$]}
\end{prooftree}
\end{center}

\subsection{Introduzione ed Eliminatore dipendente}
\begin{prooftree}
\AxiomC{$\Gamma$ cont}
\LeftLabel{I$_2$-Nat$_{prog}$)}
\UnaryInfC{succ(x) $\in$ Nat[$\Gamma$, x $\in$ Nat]}
\end{prooftree}

\begin{adjustwidth}{-5em}{}
\begin{prooftree}
\AxiomC{M(z) type[$\Gamma$, z $\in$ Nat]}
\AxiomC{c $\in$ M(0)[$\Gamma$]}
\AxiomC{e(x,y) $\in$ M(succ(x))[$\Gamma$, x $\in$ Nat, y $\in$ M(x)]}
\LeftLabel{E-Nat$_{dip}$)}
\TrinaryInfC{El$_{Nat}$(z,c,e) $\in$ M(z)[$\Gamma$, z $\in$ Nat]}
\end{prooftree}
\end{adjustwidth}
\noindent
\newpage
\section{Tipo delle liste di un tipo}
\begin{prooftree}
\AxiomC{A type [$\Gamma$]}
\LeftLabel{F-cost)}
\UnaryInfC{List(A)type [$\Gamma$]}
\end{prooftree}

\subsection{Regole di Introduzione}
\begin{center}
\AxiomC{list(A) type [$\Gamma$]}
\LeftLabel{I$_1$-list)}
\UnaryInfC{nil $\in$ List(A)[$\Gamma$]}
\DisplayProof \qquad
\AxiomC{s $\in$ List(A)[$\Gamma$]}
\AxiomC{a $\in$ A[$\Gamma$]}
\LeftLabel{I$_2$-list)}
\BinaryInfC{cons(s,a) $\in$ List(A)[$\Gamma$]}
\DisplayProof
\end{center}

\subsection{Regole di Eliminazione}
\small
\begin{adjustwidth}{-5em}{}
\begin{prooftree}
\AxiomC{\begin{tabular}[c]{ccc}M(z) type[$\Gamma$, z $\in$ List(A)]\\t $\in$ List(A)[$\Gamma$] \\c $\in$ M(nil)[$\Gamma$]\end{tabular}}
\AxiomC{e(x,w,y) $\in$ M(cons(x,w))[$\Gamma$, x $\in$ List(A),  w $\in$ A, y $\in$ M(x)]}
\LeftLabel{E-List)}
\BinaryInfC{El$_{list}$(t,c,e) $\in$ M(t)[$\Gamma$]}
\end{prooftree}
\end{adjustwidth}

\subsection{Regole di Conservazione}
\small
\begin{adjustwidth}{-4em}{}
\begin{prooftree}
\AxiomC{\begin{tabular}[c]{cc}M(z) type[$\Gamma$, z $\in$ List(A)] \\ c $\in$ M(nil)[$\Gamma$]\end{tabular}}
\AxiomC{e(x,w,y) $\in$ M(cons(x,w))[$\Gamma$, x $\in$ List(A),  w $\in$ A, y $\in$ M(x)]}
\LeftLabel{C$_1$-list)}
\BinaryInfC{El$_{list}$(nil,c,e) = c $\in$ M(nil)[$\Gamma$]}
\end{prooftree}
\end{adjustwidth}
\small
\begin{adjustwidth}{-6em}{}
\begin{prooftree}
\AxiomC{M(z) type[$\Gamma$, z $\in$ List(A)]}
\AxiomC{\begin{tabular}[c]{ccc} s $\in$ List(A)[$\Gamma$] \\ a $\in$ A[$\Gamma$] \\ c $\in$ M(nil)[$\Gamma$]\end{tabular}}
\AxiomC{e(x,w,y) $\in$ M(cons(x,w))[$\Gamma$, x $\in$ List(A), w $\in$ A, y $\in$ M(x)]}
\LeftLabel{C$_2$-list)}
\TrinaryInfC{El$_{list}$(cons(s,a),c,e) $=$ e(s,a, E$_{list}$(s,c,e)) $\in$ M(cons(s,a))[$\Gamma$]}
\end{prooftree}
\end{adjustwidth}


\subsection{Regole di Uguaglianza}
\normalsize
\begin{prooftree}
\AxiomC{A$_1$ $=$ A$_2$ $\in$ type[$\Gamma$]}
\LeftLabel{eq-I$_1$-List)}
\UnaryInfC{List(A$_1$) $=$ List(A$_2$) type($\Gamma$)}
\end{prooftree}
\begin{prooftree}
\AxiomC{s$_1$ $=$ s$_2$ $\in$ List(A)[$\Gamma$]}
\AxiomC{a$_1$ $=$ a$_2$ $\in$ A[$\Gamma$]}
\LeftLabel{eq-I$_2$-List)}
\BinaryInfC{cons(s$_1$,a$_1$) = cons(s$_2$,a$_2$) $\in$ List(A)($\Gamma$)}
\end{prooftree}
\small
\begin{adjustwidth}{-6em}{}
\begin{prooftree}
\AxiomC{\begin{tabular}[c]{ccc} M(z) type[$\Gamma$, z $\in$ List(A)] \\ t$_1$ = t$_2$ $\in$ list(A)[$\Gamma$] \\ c$_1$ $=$ c$_2$ $\in$ M(nil)[$\Gamma$]\end{tabular}}
\AxiomC{e$_1$(x,w,y) $=$ e$_2$(x,w,y) $\in$ M(cons(x,w))[$\Gamma$, x $\in$ List(A), w $\in$ A, y $\in$ M(x)]}
\LeftLabel{E-eq-List)}
\BinaryInfC{El$_{list}$(t$_1$, c$_1$, e$_1$) $=$ El$_{list}$(t$_2$, c$_2$, e$_2$ $\in$ M(t$_1$)[$\Gamma$]}
\end{prooftree}
\end{adjustwidth}
\normalsize
\subsection{Eliminatore dipendente}
\small
\begin{adjustwidth}{-6em}{}
\begin{prooftree}
\AxiomC{\begin{tabular}[c]{cc}M(z) type[$\Gamma$, z $\in$ List(A)] \\ c $\in$ M(nil)[$\Gamma$]\end{tabular}}
\AxiomC{e(x,w,y) $\in$ M(cons(x,w))[$\Gamma$, x $\in$ List(A),  w $\in$ A, y $\in$ M(x)]}
\LeftLabel{E-List$_{dip}$)}
\BinaryInfC{El$_{list}$(z,c,e) $\in$ M(z)[$\Gamma$, z $\in$ List(A)]}
\end{prooftree}
\end{adjustwidth}
\newpage

\section{Tipo della somma disgiunta}
\subsection{Regole di Formazione}
\begin{prooftree}
\AxiomC{B type [$\Gamma$]}
\AxiomC{C type [$\Gamma$]}
\LeftLabel{F-$+$)}
\BinaryInfC{B + C type [$\Gamma$]}
\end{prooftree}

\subsection{Regole di Introduzione}
\begin{center}
\AxiomC{b $\in$ B[$\Gamma$]}
\AxiomC{B + C type[$\Gamma$]}
\LeftLabel{I$_1$-$+$)}
\BinaryInfC{inl(b) $\in$ B + C[$\Gamma$]}
\DisplayProof \qquad
\AxiomC{c $\in$ C[$\Gamma$]}
\AxiomC{B + C type[$\Gamma$]}
\LeftLabel{I$_2$-$+$)}
\BinaryInfC{inr(c) $\in$ B + C[$\Gamma$]}
\DisplayProof
\end{center}
\subsection{Regole di Eliminazione}
\small
\begin{adjustwidth}{-5em}{}
\begin{prooftree}
\AxiomC{\begin{tabular}[c]{cc}M(z) type[$\Gamma$, z $\in$ B + C]\\t $\in$ B + C[$\Gamma$]\end{tabular}}
\AxiomC{e$_B$(x$_1$) $\in$ M(inl(x$_1$))[$\Gamma$, x$_1$ $\in$ B]}
\AxiomC{e$_C$(x$_2$) $\in$ M(inr(x$_2$))[$\Gamma$, x$_2$ $\in$ C]}
\LeftLabel{E-$+$)}
\TrinaryInfC{El$_+$(t,e$_B$,e$_C$) $\in$ M(t)[$\Gamma$]}
\end{prooftree}
\end{adjustwidth}

\subsection{Regole di Conservazione}
\small
\begin{adjustwidth}{-5em}{}
\begin{prooftree}
\AxiomC{\begin{tabular}[c]{cc}M(z) type[$\Gamma$, z $\in$ B + C]\\b $\in$ B[$\Gamma$]\end{tabular}}
\AxiomC{e$_B$(x$_1$) $\in$ M(inl(x$_1$))[$\Gamma$, x$_1$ $\in$ B]}
\AxiomC{e$_C$(x$_2$) $\in$ M(inr(x$_2$))[$\Gamma$, x$_2$ $\in$ C]}
\LeftLabel{C$_1$-$+$)}
\TrinaryInfC{El$_+$(inl(b),e$_B$,e$_C$) = e$_B$(b) $\in$ M(inl(b))[$\Gamma$]}
\end{prooftree}
\end{adjustwidth}
\small
\begin{adjustwidth}{-5em}{}
\begin{prooftree}
\AxiomC{\begin{tabular}[c]{cc}M(z) type[$\Gamma$, z $\in$ B + C]\\c $\in$ C[$\Gamma$]\end{tabular}}
\AxiomC{e$_B$(x$_1$) $\in$ M(inl(x$_1$))[$\Gamma$, x$_1$ $\in$ B]}
\AxiomC{e$_C$(x$_2$) $\in$ M(inr(x$_2$))[$\Gamma$, x$_2$ $\in$ C]}
\LeftLabel{C$_1$-$+$)}
\TrinaryInfC{El$_+$(inr(c),e$_B$,e$_C$) $=$ e$_C$(c) $\in$ M(inr(c))[$\Gamma$]}
\end{prooftree}
\end{adjustwidth}


\subsection{Regole di Uguaglianza}
\normalsize
\begin{prooftree}
\AxiomC{B$_1$ $=$ B$_2$ $\in$ type[$\Gamma$]}
\AxiomC{C$_1$ $=$ C$_2$ $\in$ type[$\Gamma$]}
\LeftLabel{eq-F-$+$)}
\BinaryInfC{B$_1$ + C$_1$ $=$ B$_2$ + C$_2$ type($\Gamma$)}
\end{prooftree}

\subsection{Eliminatore dipendente}
\small
\begin{adjustwidth}{-5em}{}
\begin{prooftree}
\AxiomC{M(z) type[$\Gamma$, z $\in$ B + C]}
\AxiomC{e$_B$(x$_1$) $\in$ M(inl(x$_1$))[$\Gamma$, x$_1$ $\in$ B]}
\AxiomC{e$_C$(x$_2$) $\in$ M(inr(x$_2$))[$\Gamma$, x$_2$ $\in$ C]}
\LeftLabel{E$_{dip}$-$+$)}
\TrinaryInfC{El$_+$(z,e$_B$,e$_C$) $\in$ M(z)[z $\in$ B + C]}
\end{prooftree}
\end{adjustwidth}
\normalsize
\newpage

\section{Tipo dell'uguaglianza proposizionale}
\subsection{Regole di Formazione}
\begin{prooftree}
\AxiomC{A type [$\Gamma$]}
\AxiomC{a $\in$ A[$\Gamma$]}
\AxiomC{b $\in$ A[$\Gamma$]}
\LeftLabel{F-Id)}
\TrinaryInfC{Id(A,a,b) type[$\Gamma$]}
\end{prooftree}

\subsection{Regole di Introduzione}
\begin{prooftree}
\AxiomC{a $\in$ A[$\Gamma$]}
\LeftLabel{I-Id)}
\UnaryInfC{id(a) $\in$ Id(A,a,a)[$\Gamma$]}
\end{prooftree}

\subsection{Regole di Eliminazione}
\small
\begin{prooftree}
\AxiomC{\begin{tabular}[c]{cccc}M(z$_1$,z$_2$,z$_3$) type[$\Gamma$, z$_1$ $\in$ A, z$_2$ $\in$ A, z$_3$ $\in$ Id(A, z$_1$, z$_2$)]\\a $\in$ A[$\Gamma$] \\b $\in$ A[$\Gamma$] \\ t $\in$ Id(A,a,b)[$\Gamma$]\end{tabular}}
\AxiomC{e(x) $\in$ M(x,x,Id(x))[$\Gamma$, x $\in$ A]}
\LeftLabel{E-Id)}
\BinaryInfC{El$_{Id}$(t, (x).e(x)) $\in$ M(a,b,t)[$\Gamma$]}
\end{prooftree}

\subsection{Regole di Conservazione}
\small
\begin{prooftree}
\AxiomC{\begin{tabular}[c]{cc}M(z$_1$,z$_2$,z$_3$) type[$\Gamma$, z$_1$ $\in$ A, z$_2$ $\in$ A, z$_3$ $\in$ Id(A, z$_1$, z$_2$)] \\ a $\in$ A[$\Gamma$]\end{tabular}}
\AxiomC{e(x) $\in$ M(x,x,Id(x))[$\Gamma$, x $\in$ A]}
\LeftLabel{E-Id)}
\BinaryInfC{El$_{Id}$(Id(a), (x).e(x)) $=$ e(a) $\in$ M(a,a,Id(a))[$\Gamma$]}
\end{prooftree}


\subsection{Regole di Uguaglianza}
\normalsize
\AxiomC{A$_1$ = A$_2$ type[$\Gamma$]}
\AxiomC{a$_1$ = a$_2$ $\in$ A[$\Gamma$]}
\AxiomC{b$_1$ = b$_2$ $\in$ A[$\Gamma$]}
\LeftLabel{eq-F-Id)}
\TrinaryInfC{Id(A$_1$,a$_1$,b$_1$) = Id(A$_2$,a$_2$,b$_2$) type[$\Gamma$]}
\DisplayProof  \\
\vspace{0.5cm}\\
\AxiomC{a$_1$ = a$_2$ $\in$ A[$\Gamma$]}
\LeftLabel{eq-I-Id)}
\UnaryInfC{Id(a$_1$) = Id(a$_2$) $\in$ Id(A,a$_1$,a$_1$)[$\Gamma$]}
\DisplayProof

\subsection{Eliminatore dipendente}
\small
\begin{prooftree}
\AxiomC{M(z$_1$,z$_2$,z$_3$) type[$\Gamma$, z$_1$ $\in$ A, z$_2$ $\in$ A, z$_3$ $\in$ Id(A, z$_1$, z$_2$)]}
\AxiomC{e(x) $\in$ M(x,x,Id(x))[$\Gamma$, x $\in$ A]}
\LeftLabel{E-Id$_{dip}$)}
\BinaryInfC{El$_{Id}$(z$_3$, (x).e(x)) $\in$ M(z$_1$, z$_2$, z$_3$)[$\Gamma$,z$_1$ $\in$ A, z$_2$ $\in$ A, z$_3$ $\in$ Id(A,z$_1$, z$_2$)]}
\end{prooftree}
\normalsize
\newpage

\section{Tipo somma indiciata forte}
\subsection{Regole di Formazione}
\begin{prooftree}
\AxiomC{C(x) type [$\Gamma$, x $\in$ B]}
\LeftLabel{F-$\Sigma$)}
\UnaryInfC{$\sum\limits_{x \in B}$ C(x) type[$\Gamma$]}
\end{prooftree}

\subsection{Regole di Introduzione}
\begin{prooftree}
\AxiomC{b $\in$ B[$\Gamma$]}
\AxiomC{c $\in$ C(b)[$\Gamma$]}
\AxiomC{$\sum\limits_{x \in B}$ C(x) type[$\Gamma$]}
\LeftLabel{I-$\Sigma$)}
\TrinaryInfC{$<$b,c$>$ $\in$ $\sum\limits_{x\in B}$ C(x) type[$\Gamma$]}
\end{prooftree}

\subsection{Regole di Eliminazione}
\begin{adjustwidth}{-4em}{}
\begin{prooftree}
\AxiomC{M(z) type[$\Gamma$, z $\in$ $\sum\limits_{x \in B}$ C(x)]}
\AxiomC{\begin{tabular}[c]{cc}t $\in$ $\sum\limits_{x \in B}$ C(x)[$\Gamma$] \\ e(x,y) $\in$ M($<$x,y$>$)[$\Gamma$, x $\in$ B, y $\in$ C(x)]\end{tabular}}
\LeftLabel{E-$\Sigma$)}
\BinaryInfC{El$_\Sigma$(t,(x,y).e(x,y)) $\in$ M(t)[$\Gamma$]}
\end{prooftree}
\end{adjustwidth}

\subsection{Regole di Conservazione}
\small
\begin{adjustwidth}{-4em}{}
\begin{prooftree}
\AxiomC{M(z) type[$\Gamma$, z $\in$ $\sum\limits_{x \in B}$ C(x)]}
\AxiomC{\begin{tabular}[c]{cc}b $\in$ B[$\Gamma$] \\ c $\in$ C(b)\end{tabular}}
\AxiomC{e(x,y) $\in$ M($<$x,y$>$)[$\Gamma$, x $\in$ B, y $\in$ C(x)]}
\LeftLabel{C-$\Sigma$)}
\TrinaryInfC{El$_\Sigma$($<$b,c$>$,e) $=$ e(b,c) $\in$ M($<$b,c$>$)[$\Gamma$]}
\end{prooftree}
\end{adjustwidth}

\subsection{Regole di Uguaglianza}
\small
\begin{prooftree}
\AxiomC{B$_1$ $=$ B$_2$ type[$\Gamma$]}
\AxiomC{C$_1$(x) $=$ C$_2$(x) type[$\Gamma$, x $\in$ B$_1$]}
\LeftLabel{eq-F-$\Sigma$)}
\BinaryInfC{$\sum\limits_{x \in B_1}$ C$_1$(x) $=$ $\sum\limits_{x \in B_2}$ C$_2$(x) type[$\Gamma$]}
\end{prooftree}
\begin{prooftree}
\AxiomC{$\sum\limits_{x \in B}$ C(x) type[$\Gamma$]}
\AxiomC{b$_1$ $=$ b$_2$ $\in$ B[$\Gamma$]}
\AxiomC{c$_1$ $=$ c$_2$ $\in$ C(b$_1$)[$\Gamma$]}
\LeftLabel{eq-I-$\Sigma$)}
\TrinaryInfC{$<$b$_1$,c$_1$$>$ $=$ $<$b$_2$,c$_2$$>$ $\in$ $\sum\limits_{x \in B}$ C(x) type[$\Gamma$]}
\end{prooftree}
\begin{adjustwidth}{-8em}{}
\begin{prooftree}
\AxiomC{M(z) type[$\Gamma$, z $\in$ $\sum\limits_{x \in B}$ C(x)]}
\AxiomC{\begin{tabular}[c]{cc}t$_1$ $=$ t$_2$ $\in$ $\sum\limits_{x \in B}$ C(x)[$\Gamma$] \\ e$_1$(x,y) $=$ e$_2$(x,y) $\in$ M($<$x,y$>$)[$\Gamma$, x $\in$ B, y $\in$ C(x)]\end{tabular}}
\LeftLabel{eq-E-$\Sigma$)}
\BinaryInfC{El$_\Sigma$(t$_1$,e$_1$) $=$ El$_\Sigma$(t$_2$,e$_2$) $\in$ M(t$_1$)[$\Gamma$]}
\end{prooftree}
\end{adjustwidth}

\subsection{Eliminatore dipendente}
\small
\begin{prooftree}
\AxiomC{M(z) type[$\Gamma$, z $\in$ $\sum\limits_{x \in B}$ C(x)]}
\AxiomC{e(x,y) $\in$ M($<$x,y$>$)[$\Gamma$, x $\in$ B, y $\in$ C(x)]}
\LeftLabel{E-$\Sigma_{dip}$)}
\BinaryInfC{El$_\Sigma$(z,(x,y).e(x,y)) $\in$ M(z)[$\Gamma$, z $\in$ $\sum\limits_{x \in B}$ C(x)]}
\end{prooftree}
\normalsize
\newpage

\section{Tipo prodotto cartesiano}

\subsection{Regole di Formazione}
\begin{prooftree}
\AxiomC{B type [$\Gamma$]}
\AxiomC{C type [$\Gamma$]}
\LeftLabel{F-x)}
\BinaryInfC{B $\times$ C type[$\Gamma$]}
\end{prooftree}

\subsection{Regole di Introduzione}
\begin{prooftree}
\AxiomC{b $\in$ B[$\Gamma$]}
\AxiomC{c $\in$ C[$\Gamma$]}
\LeftLabel{I-x)}
\BinaryInfC{$<$b,c$>$ $\in$ B $\times$ C[$\Gamma$]}
\end{prooftree}

\subsection{Regole di Proiezione}
\begin{prooftree}
\AxiomC{d $\in$ B $\times$ C[$\Gamma$]}
\LeftLabel{PJ$_1$-x)}
\UnaryInfC{$\pi_1$d $\in$ B[$\Gamma$]}
\end{prooftree}
\begin{prooftree}
\AxiomC{d $\in$ B $\times$ C[$\Gamma$]}
\LeftLabel{PJ$_2$-x)}
\UnaryInfC{$\pi_2$d $\in$ C[$\Gamma$]}
\end{prooftree}

\subsection{Regole di Uguaglianza delle proiezioni}
\begin{prooftree}
\AxiomC{b $\in$ B[$\Gamma$]}
\AxiomC{c $\in$ C[$\Gamma$]}
\LeftLabel{PJ$_1$-eq)}
\BinaryInfC{$\pi_1$($<$b,c$>$) $=$ b $\in$ B[$\Gamma$]}
\end{prooftree}
\begin{prooftree}
\AxiomC{b $\in$ B[$\Gamma$]}
\AxiomC{c $\in$ C[$\Gamma$]}
\LeftLabel{PJ$_2$-eq)}
\BinaryInfC{$\pi_2$($<$b,c$>$) $=$ c $\in$ B[$\Gamma$]}
\end{prooftree}
\newpage

\section{Tipo delle funzioni}

\subsection{Regole di Formazione}
\begin{prooftree}
\AxiomC{B type [$\Gamma$]}
\AxiomC{C type [$\Gamma$]}
\LeftLabel{F-$\rightarrow$)}
\BinaryInfC{B $\rightarrow$ C type[$\Gamma$]}
\end{prooftree}

\subsection{Regole di Introduzione}
\begin{prooftree}
\AxiomC{c(x) $\in$ C[$\Gamma$,x $\in$ B]}
\LeftLabel{I-$\rightarrow$)}
\UnaryInfC{$\lambda$x$^B$.c(x) $\in$ B $\rightarrow$ C[$\Gamma$]}
\end{prooftree}

\subsection{Regole di Eliminazione}
\begin{prooftree}
\AxiomC{f $\in$ B $\rightarrow$ C[$\Gamma$]}
\AxiomC{b $\in$ B[$\Gamma$]}
\LeftLabel{E-$\rightarrow$)}
\BinaryInfC{Ap(f,b) $\in$ C[$\Gamma$]}
\end{prooftree}

\subsection{Regole di Conservazione}
\begin{prooftree}
\AxiomC{c(x) $\in$ C[$\Gamma$, x $\in$ B]}
\AxiomC{b $\in$ B[$\Gamma$]}
\LeftLabel{C-$\rightarrow$)}
\BinaryInfC{Ap($\lambda$x$^B$.c(x),b] $=$ c(b)$\in$ C[$\Gamma$]}
\end{prooftree}

\subsection{Regole di Uguaglianza}
\begin{prooftree}
\AxiomC{f$_1$ $=$ f$_2$ $\in$ B $\rightarrow$ C[$\Gamma$]}
\AxiomC{b$_1$ $=$ b$_2$ $\in$ B[$\Gamma$]}
\LeftLabel{eq-E-$\rightarrow$)}
\BinaryInfC{Ap(f$_1$,b$_1$) $=$ Ap(f$_2$,b$_2$) $\in$ C[$\Gamma$]}
\end{prooftree}

\begin{prooftree}
\AxiomC{c$_1$(x) $=$ c$_2$(x) $\in$ C[$\Gamma$,x $\in$ B]}
\LeftLabel{eq-I-$\rightarrow$)}
\UnaryInfC{$\lambda$x$^B$.c$_1$(x) $=$ $\lambda$x$^B$.c$_2$(x) $\in$ B $\rightarrow$ C[$\Gamma$]}
\end{prooftree}
\newpage

\section{Tipo prodotto dipendente}
\label{sec: tipo-prodotto-dipendente}

\subsection{Regole di Formazione}
\begin{prooftree}
\AxiomC{B type [$\Gamma$]}
\AxiomC{C(x) type [$\Gamma$,x $\in$ B]}
\LeftLabel{F-{\scriptsize$\prod$})}
\BinaryInfC{$\prod\limits_{x \in B}$ C(x) type[$\Gamma$]}
\end{prooftree}

\subsection{Regole di Introduzione}
\begin{prooftree}
\AxiomC{c(x) $\in$ C(x)[$\Gamma$,x $\in$ B]}
\LeftLabel{I-{\scriptsize$\prod$})}
\UnaryInfC{$\lambda$x$^B$.c(x) $\in$ $\prod\limits_{x \in B}$ C(x)[$\Gamma$]}
\end{prooftree}

\subsection{Regole di Eliminazione}
\begin{prooftree}
\AxiomC{f $\in$ $\prod\limits_{x \in B}$ C(x)[$\Gamma$]}
\AxiomC{b $\in$ B[$\Gamma$]}
\LeftLabel{E-{\scriptsize$\prod$})}
\BinaryInfC{Ap(f,b) $\in$ C(b)[$\Gamma$]}
\end{prooftree}
\noindent

\subsection{Regole di Conservazione}
\begin{prooftree}
\AxiomC{c(x) $\in$ C(x)[$\Gamma$, x $\in$ B]}
\AxiomC{b $\in$ B[$\Gamma$]}
\LeftLabel{C-{\scriptsize$\prod$})}
\BinaryInfC{Ap($\lambda$x$^B$.c(x),b] $=$ c(b)$\in$ C(b)[$\Gamma$]}
\end{prooftree}

\subsection{Regole di Uguaglianza}
\begin{prooftree}
\AxiomC{f$_1$ $=$ f$_2$ $\in$  $\prod\limits_{x \in B}$ C(x)[$\Gamma$]}
\AxiomC{b$_1$ $=$ b$_2$ $\in$ B[$\Gamma$]}
\LeftLabel{eq-E-{\scriptsize$\prod$})}
\BinaryInfC{Ap(f$_1$,b$_1$) $=$ Ap(f$_2$,b$_2$) $\in$ C(b$_1$)[$\Gamma$]}
\end{prooftree}

\begin{prooftree}
\AxiomC{c$_1$(x) $=$ c$_2$(x) $\in$ C(x)[$\Gamma$,x $\in$ B]}
\LeftLabel{eq-I-{\scriptsize$\prod$})}
\UnaryInfC{$\lambda$x$^B$.c$_1$(x) $=$ $\lambda$x$^B$.c$_2$(x) $\in$  $\prod\limits_{x \in B}$ C(x)[$\Gamma$]}
\end{prooftree}

\begin{prooftree}
\AxiomC{B$_1$ $=$ B$_2$ $\in$ type[$\Gamma$]}
\AxiomC{c$_1$(x) $=$ c$_2$(x) type[$\Gamma$,x $\in$ B$_1$]}
\LeftLabel{eq-F-{\scriptsize$\prod$})}
\BinaryInfC{$\prod\limits_{x \in B}$ C$_1$(x)[$\Gamma$] $=$ $\prod\limits_{x \in B}$ C$_2$(x)[$\Gamma$]}
\end{prooftree}
\normalsize
\newpage

\section{Tipo vuoto}

\subsection{Regole di Formazione}
\begin{prooftree}
\AxiomC{$\Gamma$ cont}
\LeftLabel{F-N$_0$)}
\UnaryInfC{N$_0$ type [$\Gamma$]}
\end{prooftree}

\subsection{Regole di Eliminazione}
\begin{prooftree}
\AxiomC{t $\in$ N$_0$[$\Gamma$]}
\AxiomC{M(z) type[$\Gamma$,z $\in$ N$_0$]}
\LeftLabel{E-N$_0$)}
\BinaryInfC{El$_{N0}$(t) $\in$ M(t)[$\Gamma$]}
\end{prooftree}

\subsection{Regole di Uguaglianza}
\begin{prooftree}
\AxiomC{t$_1$ $=$ t$_2$ $\in$ N$_0$[$\Gamma$]}
\AxiomC{M(z) type[$\Gamma$,z $\in$ N$_0$]}
\LeftLabel{eq-E-N$_0$)}
\BinaryInfC{El$_{N0}$(t$_1$) $=$ El$_{N0}$(t$_2$) $\in$ M(t$_1$)[$\Gamma$]}
\end{prooftree}

\subsection{Eliminatore dipendente}
\begin{prooftree}
\AxiomC{M(z) type[$\Gamma$,z $\in$ N$_0$]}
\LeftLabel{E-N$_{0dip}$)}
\UnaryInfC{El$_{N0}$(z) $\in$ M(z)[$\Gamma$,z $\in$ N$_0$]}
\end{prooftree}
\newpage

\section{Tipi Primo Universo}

\subsection{Regole di Formazione}
\begin{prooftree}
\AxiomC{$\Gamma$ cont}
\LeftLabel{F-Uno)}
\UnaryInfC{U$_0$ type[$\Gamma$]}
\end{prooftree}

\subsection{Regole di Introduzione}
\begin{center}
\AxiomC{$\Gamma$ cont}
\LeftLabel{I$_1$-Uno)}
\UnaryInfC{$\hat{N_0}$ $\in$ U$_0$[$\Gamma$]}
\DisplayProof \quad
\AxiomC{$\Gamma$ cont}
\LeftLabel{I$_2$-Uno)}
\UnaryInfC{$\hat{N_1}$ $\in$ U$_0$[$\Gamma$]}
\DisplayProof \quad
\vspace{0.5cm}
\AxiomC{$\Gamma$ cont}
\LeftLabel{I$_3$-Uno)}
\UnaryInfC{$\hat{Nat}$ $\in$ U$_0$[$\Gamma$]}
\DisplayProof
\vspace{0.5cm}
\AxiomC{c(x) $\in$ U$_0$[$\Gamma$,x $\in$ \textbf{T}(b)]}
\AxiomC{b $\in$ U$_0$[$\Gamma$]}
\LeftLabel{I$_4$-Uno)}
\BinaryInfC{$\hat{\sum}_{x \in b}$ c(x) $\in$ U$_0$[$\Gamma$]}
\DisplayProof \quad
\AxiomC{c(x) $\in$ U$_0$[$\Gamma$,x $\in$ \textbf{T}(b)]}
\AxiomC{b $\in$ U$_0$[$\Gamma$]}
\LeftLabel{I$_5$-Uno)}
\BinaryInfC{$\hat{\prod}_{x \in b}$ c(x) $\in$ U$_0$[$\Gamma$]}
\DisplayProof \\
\vspace{0.5cm}
\AxiomC{b $\in$ U$_0$[$\Gamma$]}
\AxiomC{c $\in$ U$_0$[$\Gamma$]}
\LeftLabel{I$_6$-Uno)}
\BinaryInfC{b $\hat{+}$ c $\in$ U$_0$[$\Gamma$]}
\DisplayProof \quad
\AxiomC{b $\in$ U$_0$[$\Gamma$]}
\LeftLabel{I$_7$-Uno)}
\UnaryInfC{$\hat{List}(b)$ $\in$ U$_0$[$\Gamma$]}
\DisplayProof
\begin{prooftree}
\AxiomC{b $\in$ U$_0$[$\Gamma$]}
\AxiomC{c $\in$ \textbf{T}(b)[$\Gamma$]}
\AxiomC{d $\in$ \textbf{T}(b)[$\Gamma$]}
\LeftLabel{I$_8$-Uno)}
\TrinaryInfC{$\hat{Id}$(b,c,d) $\in$ U$_0$[$\Gamma$]}
\end{prooftree}
\end{center}


\subsection{Regole di Eliminazione}
\begin{prooftree}
\AxiomC{a $\in$ U$_0$[$\Gamma$]}
\LeftLabel{E-Uno)}
\UnaryInfC{\textbf{T}(a) type[$\Gamma$]}
\end{prooftree}

\subsection{Regole di Conversione}
\begin{center}
\AxiomC{$\Gamma$ cont}
\LeftLabel{C$_1$-Uno)}
\UnaryInfC{\textbf{T}($\hat{N_0}$) $=$ N$_0$ type[$\Gamma$]}
\DisplayProof \quad
\AxiomC{$\Gamma$ cont}
\LeftLabel{C$_2$-Uno)}
\UnaryInfC{\textbf{T}($\hat{N_1}$) $=$ N$_1$ type[$\Gamma$]}
\DisplayProof\\
\vspace{0.5cm}
\AxiomC{$\Gamma$ cont}
\LeftLabel{C$_3$-Uno)}
\UnaryInfC{\textbf{T}(Nat) $=$ Nat type[$\Gamma$]}
\DisplayProof\\
\vspace{0.5cm}
\AxiomC{c(x) $\in$ U$_0$[$\Gamma$,x $\in$ \textbf{T}(b)]}
\AxiomC{b $\in$ U$_0$[$\Gamma$]}
\LeftLabel{C$_4$-Uno)}
\BinaryInfC{\textbf{T}($\hat{\sum}_{x \in b} c(x)$) $=$ $\sum_{x \in \textbf{T}(b)}$T(c(x)) type[$\Gamma$]}
\DisplayProof\\
\vspace{0.5cm}
\AxiomC{c(x) $\in$ U$_0$[$\Gamma$,x $\in$ \textbf{T}(b)]}
\AxiomC{b $\in$ U$_0$[$\Gamma$]}
\LeftLabel{C$_5$-Uno)}
\BinaryInfC{\textbf{T}($\hat{\prod}_{x \in b} c(x)$) $=$ $\prod_{x \in \textbf{T}(b)}$T(c(x)) type[$\Gamma$]}
\DisplayProof\\
\end{center}
\vspace{0.5cm}
\AxiomC{b $\in$ U$_0$[$\Gamma$]}
\AxiomC{c $\in$ U$_0$[$\Gamma$]}
\LeftLabel{C$_6$-Uno)}
\BinaryInfC{\textbf{T}(b $\hat{+}$ c) $=$ \textbf{T}(b) $+$ \textbf{T}(c) type[$\Gamma$]}
\DisplayProof \quad
\AxiomC{b $\in$ U$_0$[$\Gamma$]}
\LeftLabel{C$_7$-Uno)}
\UnaryInfC{\textbf{T}($\hat{List}$(b)) $=$ List(\textbf{T}(b)) type[$\Gamma$]}
\DisplayProof
\begin{center}
\vspace{0.5cm}
\AxiomC{b $\in$ U$_0$[$\Gamma$]}
\AxiomC{c $\in$ \textbf{T}(b)[$\Gamma$]}
\AxiomC{d $\in$ \textbf{T}(b)[$\Gamma$]}
\LeftLabel{C$_8$-Uno)}
\TrinaryInfC{\textbf{T}($\hat{Id}$(b,c,d)) $=$ Id(\textbf{T}(b),c,d) type[$\Gamma$]}
\DisplayProof
\end{center}

\subsection{Regole di Uguaglianza}
\begin{center}
\vspace{0.5cm}
\AxiomC{c$_1$(x) $=$ c$_2$(x) $\in$ U$_0$[$\Gamma$,x $\in$ \textbf{T}(b)]}
\AxiomC{b$_1$ $=$ b$_2$ $\in$ U$_0$[$\Gamma$]}
\LeftLabel{eq-I$_4$-Uno)}
\BinaryInfC{$\hat{\sum}_{x \in b_1}$ c$_1$(x) $=$ $\hat{\sum}_{x \in b_2}$ c$_2$(x) $\in$ U$_0$[$\Gamma$]}
\DisplayProof \\ \vspace{0.5cm}
\AxiomC{c$_1$(x) $=$ c$_2$(x) $\in$ U$_0$[$\Gamma$,x $\in$ \textbf{T}(b)]}
\AxiomC{b$_1$ $=$ b$_2$ $\in$ U$_0$[$\Gamma$]}
\LeftLabel{eq-I$_5$-Uno)}
\BinaryInfC{$\hat{\prod}_{x \in b_1}$ c$_1$(x) $=$ $\hat{\prod}_{x \in b_2}$ c$_2$(x) $\in$ U$_0$[$\Gamma$]}
\DisplayProof \\
\vspace{0.5cm}
\AxiomC{b$_1$ $=$ b$_2$ $\in$ U$_0$[$\Gamma$]}
\AxiomC{c$_1$ $=$ c$_2$ $\in$ U$_0$[$\Gamma$]}
\LeftLabel{eq-I$_6$-Uno)}
\BinaryInfC{b$_1$ $\hat{+}$ c$_1$ $=$ b$_2$ $\hat{+}$ c$_2$ $\in$ U$_0$[$\Gamma$]}
\DisplayProof \\ \vspace{0.5cm}
\AxiomC{b$_1$ $=$ b$_2$ $\in$ U$_0$[$\Gamma$]}
\LeftLabel{eq-I$_7$-Uno)}
\UnaryInfC{$\hat{List}$(b$_1$) $=$ $\hat{List}$(b$_2$) $\in$ U$_0$[$\Gamma$]}
\DisplayProof\\
\vspace{0.5cm}
\begin{prooftree}
\AxiomC{b$_1$ $=$ b$_2$ $\in$ U$_0$[$\Gamma$]}
\AxiomC{c$_1$ $=$ c$_2$ $\in$ \textbf{T}(b)[$\Gamma$]}
\AxiomC{d$_1$ $=$ d$_2$ $\in$ \textbf{T}(b)[$\Gamma$]}
\LeftLabel{eq-I$_8$-Uno)}
\TrinaryInfC{$\hat{Id}$(b$_1$,c$_1$,d$_1$) $=$ $\hat{Id}$(b$_2$,c$_2$,d$_2$) $\in$ U$_0$[$\Gamma$]}
\end{prooftree}
\end{center}
\newpage\newpage
\end{document}