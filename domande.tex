\documentclass[10pt,a4paper, italian]{book}
\usepackage[italian]{babel}
\usepackage[T1]{fontenc} % riconosce gli acenti
\usepackage[utf8]{inputenc} % riconosce gli acenti
\usepackage{amssymb}
\usepackage[table]{xcolor}
\usepackage{tabularx}
\usepackage{float}
\usepackage{hyperref}
\hypersetup{
	colorlinks=true,
	linkcolor=blue,
	urlcolor=linkcolor
}	
\usepackage{amsmath}
\usepackage{amsfonts}
\usepackage{bussproofs} % per alberi di deduzione naturale
\usepackage{graphicx} % per le immagini
\usepackage[strict]{changepage}
\usepackage[ddmmyyyy]{datetime}
\usepackage{cancel} % barra il testo
\definecolor{amethyst}{rgb}{0.6, 0.4, 0.8}
\begin{document}
\noindent
\textbf{domanda 1.\\
Se non si hanno regole di tipi, ma solo quelle di \textit{pagina 13}, che contesti posso derivare, partendo dall'unico assioma [ ] cont (ovvero che il contesto vuoto \`e un contesto)?}\\
Esclusivamente [ ] cont, in quanto \`e a sua volta un giudizio di contesto e il solo ausilio delle regole di formazione dei contesti \textit{(F-c)}, su di esso, non mi \`e possibile e perci\`o non mi permette di generarne di nuovi.
\\\\
\noindent
\textbf{domanda 2.\\
Se si aggiunge  alla regola dei contesti la sola regola 
\begin{prooftree}
\AxiomC{$\Gamma$ cont}
\LeftLabel{F-Nat)}
\UnaryInfC{Nat type[$\Gamma$]}
\end{prooftree}
\noindent 
che contesti posso scrivere? quali giudizi posso derivare?}\\\\
\noindent
Riempio $\Gamma$ cont con un "vero" contesto. Dunque \`e lecito che sia [ ] cont.
\begin{prooftree}
\AxiomC{[ ] cont}
\LeftLabel{F-Nat}
\UnaryInfC{Nat type[ ]}
\LeftLabel{F-c}\RightLabel{(x$_1$ $\in$ Nat) $\notin$ [ ]}
\UnaryInfC{x$_1$ $\in$ Nat cont}
\LeftLabel{F-Nat}
\UnaryInfC{Nat type[x$_1$ $\in$ Nat]}
\LeftLabel{F-c}\RightLabel{(x$_2$ $\in$ Nat) $\notin$ x$_1$ $\in$ Nat}
\UnaryInfC{x$_1$ $\in$ Nat, x$_2$ $\in$ Nat cont}
\end{prooftree}
\noindent
Partendo dal giudizio di contesto [ ] cont e applicando alternativamente  \textit{(F-Nat)} e \textit{(F-c)}, ottengo i contesti di giudizio [x$_1$ $\in$ Nat] cont e [x$_1$ $\in$ Nat, x$_2$ $\in$ Nat] cont. Questi sono a loro volta derivabili e mi permettono di ottenere sempre nuovi contesti, con una nuova variabile che non appartiene al contesto precedente. Nella derivazione sopra, una nuova applicazione di \textit{(F-Nat)} e \textit{(F-c)} mi permette di ottenere [x$_1$ $\in$ Nat, x$_2$ $\in$ Nat, x$_3$ $\in$ Nat] cont, dove difatti (x$_3$ $\in$ Nat) $\notin$ x$_1$ $\in$ Nat, x$_2$ $\in$ Nat.\\
Sono invece giudizi di tipo i giudizi che rispettano la forma Nat type[$\Gamma$] con $\Gamma$ meta-variabile riempita.

\end{document}